\documentclass{echw}

\title{Timed Practice 2\\A2}
\author{Eytan Chong}
\date{2024-04-11}

\begin{document}
    \problem{}
        \begin{enumerate}
            \item The fixed-point iteration $g(x_n) = x_{n+1}$ is used for iterating to a root $\a$ of the equation $f(x) = 0$. Explain, with the aid of a diagram, when this procedure might not converge to $\a$ even when the initial approximation $x_0$ is close to $\a$.
            \item Let $f(x) = \ln{3x} - \dfrac{k}{x}$, where $x > 0$ and $k$ is a real positive constant.
            \begin{enumerate}
                \item Show that the equation $f(x) = 0$ has exactly one root $\a$.
                \item Find the exact range of values of $k$ such that $\a$ lies in the interval $(2, 3)$.
            \end{enumerate}

            Let $k = 5$ for the rest of the question.

            \begin{enumerate}
                \setcounter{enumii}{2}
                \item If $\a$ lies in the interval $(2, 3)$, obtain, by linear interpolation over the interval $(2, 3)$, a first approximation $\a_1$ to $\a$. Give your answer correct to 3 decimal places.
                \item Explain, with a sketch of $y = f(x)$ for $2 < x < 3$, why $\a_1$ is greater than $\a$.
                \item Using the Newton-Raphson method, and $\a_1 = 2.5$ as the first estimate, find an approximation to the root $\a$. Give your answer correct to 3 decimal places.
                \item Without further calculation, determine, with justification, if Newton-Raphson can be used to estimate the root if the first approximation used is $\a_1 = 7$.
            \end{enumerate}
        \end{enumerate}

    \solution
        \part
            \begin{center}
                \begin{tikzpicture}[trim axis left, trim axis right, scale=0.86]
                    \begin{axis}[
                        domain = 0:5,
                        samples = 101,
                        axis y line=middle,
                        axis x line=middle,
                        xtick = {2.678, 3},
                        xticklabels = {$\a$, $x_0$},
                        ytick = \empty,
                        xlabel = {$x$},
                        ylabel = {$y$},
                        ymax=20,
                        legend cell align={left},
                        legend pos=outer north east,
                        after end axis/.code={
                            \path (axis cs:0,0) 
                                node [anchor=east] {$O$};
                            }
                        ]
                        \addplot[plotRed] {e^(x-1)};

                        \addlegendentry{$y=g(x)$};

                        \addplot[plotBlue] {2*x};

                        \addlegendentry{$y=x$};

                        \draw[dotted, thick] (2.678, 0) -- (2.678, 5.357);

                        \draw[dotted, thick] (3, 0) -- (3, 7.389);

                        \begin{scope}[decoration={
                            markings,
                            mark=at position 0.5 with {\arrow{>}}}
                            ] 

                            \draw[postaction={decorate}](3, 7.389) -- (3.6945, 7.389);

                            \draw[postaction={decorate}] (3.6945, 7.389) -- (3.6945, 14.798);

                            \draw[postaction={decorate}](3.6945, 14.798) -- (5, 14.798);
                        \end{scope}
                    \end{axis}
                \end{tikzpicture}
            \end{center}

             From the above diagram, although the initial approximation $x_0$ is close to $\a$, if the gradient of $y=g(x)$ is too steep compared to $y=x$ near $\a$, the procedure will not converge to $\a$.

        \part
            \subpart
                Observe that $f$ is a continuous function. Note that $f\bp{\dfrac13} = \ln{3 \cdot \dfrac13} - \dfrac{k}{\tfrac13} = -3k < 0$. Furthermore, as $x$ approaches infinity, $f(x)$ also approaches infinity. Thus, there is at least one root to the equation $f(x) = 0$.

                Note that $f'(x) = \dfrac1x + \dfrac{k}{x^2}$, which is greater than 0 for all real $x$. Thus, $f$ is an increasing function. Hence, there is only one root to the equation $f(x) = 0$.

            \subpart
                Since $f(x)$ is strictly increasing and continuous, if $\a \in (2, 3)$, we need $f(2) < 0$ and $f(3) > 0$.

                \smallskip

                \case{1}{$f(2) < 0 \implies \ln 6 - \dfrac{k}2 < 0 \implies k > 2\ln 6$}

                \case{2}{$f(3) > 0 \implies \ln 9 - \dfrac{k}3 > 0 \implies k < 3\ln 9$}

                \boxt{$2\ln6 < k < 3\ln9$}

            \subpart
                Using linear interpolation on $(2, 3)$,

                \begin{align*}
                    \a_1 &= \dfrac{2f(3) - 3f(2)}{f(3) - f(2)}\\
                    &= 2.572 \todp{3}
                \end{align*}

                \boxt{$\a_1 = 2.572 \todp{3}$}

            \subpart
                \begin{center}
                    \begin{tikzpicture}[trim axis left, trim axis right, scale=0.86]
                        \begin{axis}[
                            domain = 1.8:3.1,
                            samples = 150,
                            axis y line=middle,
                            axis x line=middle,
                            xtick = {2.488, 2.571, 2, 3},
                            xticklabels = {$\a$, $\a_1$, 2, 3},
                            ytick = \empty,
                            xlabel = {$x$},
                            ylabel = {$y$},
                            ymin=-0.8,
                            legend cell align={left},
                            legend pos=outer north east,
                            after end axis/.code={
                                \path (axis cs:1.949,0) 
                                    node [anchor=east] {$O$};
                                }
                            ]
                            \addplot[plotRed] {ln(3*x) - 5/x};

                            \addlegendentry{$y=f(x)$};

                            \draw[dotted, thick] (2, 0) -- (2, -0.708);

                            \draw[dotted, thick] (3, 0) -- (3, 0.531);

                            \draw[dotted, thick] (2, -0.708) -- (3, 0.531);
                        \end{axis}
                    \end{tikzpicture}
                \end{center}
            
                 The graph of $y = f(x)$ is increasing and concave downwards. Thus, $\a_1 > \a$.

            \subpart
                \begin{alignat*}{2}
                    && \a_1 &= 2.5 \\
                    \implies&&\a_2 &= \nrm{\a_1}{f} = 2.48758\\
                    \implies&&\a_3 &= \nrm{\a_2}{f} = 2.48763
                \end{alignat*}

                Since $f(2.4875) < 0$ and $f(2.4885) > 0$, we know $\a \in (2.4875, 2.4885)$. Hence, $\a = 2.488 \todp{3}$.

                \boxt{$\a = 2.488 \todp{3}$}

                \newpage

            \subpart
                \begin{center}
                    \begin{tikzpicture}[trim axis left, trim axis right, scale=0.86]
                        \begin{axis}[
                            domain = 0:10,
                            samples = 150,
                            axis y line=middle,
                            axis x line=middle,
                            xtick = {7},
                            ytick = \empty,
                            xlabel = {$x$},
                            ylabel = {$y$},
                            ymin=-2,
                            xmin=-4,
                            legend cell align={left},
                            legend pos=outer north east,
                            after end axis/.code={
                                \path (axis cs:0,0) 
                                    node [anchor=north east] {$O$};
                                }
                            ]
                            \addplot[plotRed] {ln(3*x) - 5/x};

                            \addlegendentry{$y=f(x)$};

                            \draw (10, 3.065) -- (-2.514, 0);
                        \end{axis}
                    \end{tikzpicture}
                \end{center}

                Newton-Raphson cannot be used to estimate the root if the first approximation used is $\a_1 = 7$. From the diagram, the gradient at $x=7$ is very small. As the iteration continues, the second approximation $\a_2$ will be a negative number, causing it to be impossible to continue with the Newton-Raphson iteration as $f(x)$ will not be defined.
\end{document}