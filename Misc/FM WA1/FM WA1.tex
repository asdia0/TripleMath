\documentclass{jhwhw}

\title{Further Mathematics\\Weighted Assignment 1}
\author{Eytan Chong}
\date{2024-04-18}

\begin{document}
    \problem{}
        Show that 
        
        \begin{equation*}
            \sum_{r=1}^{(n-1)^2 + 3} (3^{2r} - n + 1) = \dfrac{a}{b} \left(729 \cdot 9^{(n-1)^2} - 1\right) - c(n-1)^3 - d(n-1)
        \end{equation*}

        \noindent where $a$, $b$, $c$ and $d$ are constants to be determined.
    
    \solution
        \begin{align*}
            \sum_{r=1}^{(n-1)^2 + 3} (3^{2r} - n + 1) &= \sum_{r=1}^{(n-1)^2 + 3} 9^r  - \sum_{r=1}^{(n-1)^2 + 3}(n - 1)\\
            &= \dfrac{9\left(9^{(n-1)^2 + 3} - 1\right)}{9-1} - (n-1)\left((n-1)^2 + 3\right)\\
            &= \dfrac98 \left(729 \cdot 9^{(n-1)^2} - 1\right) - (n-1)^3 - 3(n-1)
        \end{align*}
        
    \problem{}
        \textbf{Do not use a calculator in answering this question.}

        \medskip

        \noindent The sequence of positive numbers, $u_n$, satisfies the recurrence relation:

        \begin{equation*}
            u_{n+1} = \sqrt{2u_n + 3}, \qquad n = 1, 2, 3, \ldots
        \end{equation*}

        \begin{enumerate}
            \item If the sequence converges to $m$, find the value of $m$.
            \item By using a graphical approach, explain why $m < u_{n_1} < u_n$ when $u_n > u_m$. Hence determine the behaviour of the sequence when $u_1 > m$.
        \end{enumerate}

    \solution
        \part
            Observe that
            \begin{align*}
                \lim_{n \to \infty} u_n &= \lim_{n \to \infty} \sqrt{2u_{n-1} + 3}\\
                &= \sqrt{2\lim_{n \to \infty} u_{n-1} + 3}\\
                &= \sqrt{2\lim_{n \to \infty} u_n + 3}
            \end{align*}

            Since the sequence converges to $m$, we have $\lim_{n \to \infty} u_n = m$. Thus,
            \begin{alignat*}{2}
                &&m &= \sqrt{2m + 3}\\
                \implies&&m^2 &= 2m+3\\
                \implies&&m^2-2m-3&=0\\
                \implies&&(m-3)(m+1) &= 0
            \end{alignat*}

            Thus, $m = 3$ or $m = -1$. Since $u_n$ is always positive, we take $m = 3$.
            
            \boxt{
                $m = 3$
            }

        \part
            \begin{center}
                \begin{tikzpicture}[trim axis left, trim axis right]
                    \begin{axis}[
                        domain = 0:7,
                        samples = 101,
                        axis y line=middle,
                        axis x line=middle,
                        xtick = {3, 5},
                        ytick = {3},
                        xticklabels = {3, $u_n > 3$},
                        xlabel = {$u_n$},
                        ylabel = {$u_{n+1}$},
                        ymax=6,
                        legend cell align={left},
                        legend pos=outer north east,
                        after end axis/.code={
                            \path (axis cs:0,0) 
                                node [anchor=north east] {$O$};
                            }
                        ]
                        \addplot[plotRed] {sqrt(2 * x + 3)};
            
                        \addlegendentry{$u_{n+1} = \sqrt{2u_n + 3}$};

                        \addplot[plotBlue] {x};

                        \addlegendentry{$u_{n+1} = u_n$};
                        
                        \addplot[dotted, thick] {3};

                        \draw[dotted, thick] (3, 0) -- (3,3);

                        \draw[dotted, thick] (5, 0) -- (5, 5);
                        
                        \fill (5, 3) circle[radius=2.5pt] node[anchor = north west] {3};

                        \fill (5, 3.61) circle[radius=2.5pt]node[anchor=north west] {$u_{n+1}$};

                        \fill (5, 5) circle[radius=2.5pt]node[anchor=north west] {$u_{n}$};
                    \end{axis}
                \end{tikzpicture}
            \end{center}

            From the graph, if $u_n > 3$, then $3 < u_{n+1} < u_n$.

            \boxt{
                The sequence decreases and converges to 3.
            }

    \problem{}
        Two expedition teams are to climb a vertical distance of 8100 m from the foot to the peak of a mountain. Team $A$ plans to cover a vertical distance of 400 m on the first day. On each subsequent day, the vertical distance covered is 5 m less than the vertical distance covered in the previous day. Team $B$ plans to cover a vertical distance of 800 m on the first day. On each subsequent day, the vertical distance covered is 90\% of the vertical distance covered in the previous day.

        \begin{enumerate}
            \item Find the number of days required for Team $A$ to reach the peak.
            \item Explain why Team $B$ will never be able to reach the peak.
            \item At the end of the 15th day, Team $B$ decided to modify their plan, such that on each subsequent day, the vertical distance covered is 95\% of the vertical distance covered in the previous day. Which team will be the first to reach the peak of the mountain? Justify your answer.
        \end{enumerate}

    \solution
        \part
            The vertical distance Team $A$ plans to cover in a day can be described as a sequence in arithmetic progression with first term 400 and common difference $-5$. In order to reach the peak, the total vertical distance covered by Team $A$ has to be greater than 8100 m. Hence,

            \begin{equation*}
                \dfrac{n}2 \left(2 (400) + (n-1)(-5)\right) \geq 8100
            \end{equation*}

            From the graphing calculator, $n \geq 24$. Hence, Team $A$ requires 24 days to reach the peak.

            \boxt{
                24 days.
            }

        \part
            The vertical distance Team $B$ plans to cover in the $n$th day can be described by the sequence $U_n$ in geometric progression with first term 800 and common ratio $r = 0.9$. Let $S^U_n$ be the $n$th partial sum of $U_n$. Since $\abs{r} < 1$, the sum to infinity of exists and is equal to

            \begin{equation*}
                S^U_\infty = \dfrac{800}{1 - 0.9} = 8000
            \end{equation*}

            Hence, Team $B$ will never be able to surpass 8 km in height. Thus, they will ot reach the peak no matter how long they take.

        \part
            The new vertical distance covered by Team $B$ after Day 15 can be described by the sequence $V_n$ in geometric progression with first term $U_{15}$ and common ratio $r = 0.95$. Let $S^V_n$ be the $n$th partial sum of $V_n$. Then,

            \begin{equation*}
                S^V_n = \frac{U_{15}\cdot 0.95 \left(1 - (0.95)^n\right)}{1 - 0.95}
            \end{equation*}

            Note that 
            
            \begin{equation*}
                S^U_n = \frac{800\left(1 - (0.9)^n\right)}{1 - 0.9}
            \end{equation*}

            Hence, after Day 15, Team $B$ has to climb another $8000 - S^U_{15} = 1747.13$ metres. Since $U_{15} = 183.01$, we have the inequality

            \begin{equation*}
                \frac{183.01\cdot0.95\left(1 - (0.95)^n\right)}{1 - 0.95} \geq 1747.13
            \end{equation*}

            Using the graphing calculator, $n \geq 14$. Hence, Team $B$ will need at least $15 + 14 = 29$ days to reach the peak.

            \boxt{
                Team $A$ will reach the peak first.
            }

    \problem{}
        The function $f$ is given by $f(x) = x^2 - 3x + 2 - e^{-x}$. It is known from graphical work that this equation has 2 roots $x = \alpha$ and $x = \beta$, where $\alpha < \beta$.

        \begin{enumerate}
            \item Show that $f(x) = 0$ has at least one root in the interval $[0, 1]$.
        \end{enumerate}

        \noindent It is known that there is exactly one root in $[0, 1]$ where $x = \alpha$.

        \begin{enumerate}
            \setcounter{enumi}{1}
            \item Starting with $x_0 = 0.5$, use an iterative method based on the form 
            \begin{equation*}
                x_{n+1} = p\left(x_n^2 + q - e^{-x_n}\right)
            \end{equation*}            
            where $p$ and $q$ are real numbers to be determined, to find the value of $\alpha$ correct to 3 decimal places. You should demonstrate that your answer has the required accuracy.
        \end{enumerate}

        \noindent It is known that the other root $x = \beta$ lies in the interval $[2, 3]$.

        \begin{enumerate}
            \setcounter{enumi}{2}
            \item With the aid of a clearly labelled diagram, explain why the method in (b) will fail to obtain any reasonable approximation to $\beta$, where $x_0$ is chosen such that $x_0 \in [2, 3]$, $x_0 \neq \beta$.
        \end{enumerate}

        \noindent To obtain an approximation to $\beta$, another approach is used.

        \begin{enumerate}
            \setcounter{enumi}{3}
            \item Use linear interpolation once in the interval $[2, 3]$ to find a first approximation to $\beta$, giving your answer to 2 decimal places. Explain whether this approximate is an overestimate or underestimate.
            \item With your answer in (d) as the initial approximate, use the Newton-Raphson method to obtain $\beta$ correct to 3 decimal places. Your process should terminate when you have two sucessive iterates that are equal when rounded to 3 decimal places.
        \end{enumerate}

    \solution
        \part
            Observe that $f(0) = 1 > 0$ and $f(1) = -e^{-1} < 0$. Since $f$ is continuous and $f(0)f(1) < 0$, there must be at least one root to $f(x) = 0$ in the interval $[0, 1]$.

        \part
            Let $f(x) = 0$. Then,
            \begin{alignat*}{2}
                &&x^2 - 3x + 2 - e^{-x} &= 0\\
                \implies&&x^2 + 2 - e^{-x} &= 3x\\
                \implies&&x &= \dfrac13 \left(x^2 + 2 - e^{-x}\right)
            \end{alignat*}

            Hence, we should use an iterative method based on the form
            \begin{equation*}
                x_{n+1} = \dfrac13 \left( x^2_n + 2 - e^{-x_n}\right)
            \end{equation*}

            Starting with $x_0 = 0.5$,
            \begin{alignat*}{3}
                && x_1 &= \dfrac13 \left( x^2_0 + 2 - e^{-x_0}\right) &= 0.54782\\
                \implies&&x_2 &= \dfrac13 \left( x^2_1 + 2 - e^{-x_1}\right) &= 0.57396\\
                \implies&&x_3 &= \dfrac13 \left( x^2_2 + 2 - e^{-x_2}\right) &= 0.58871\\
                \implies&&x_4 &= \dfrac13 \left( x^2_3 + 2 - e^{-x_3}\right) &= 0.59718\\
                \implies&&x_5 &= \dfrac13 \left( x^2_4 + 2 - e^{-x_4}\right) &= 0.60208\\
                \implies&&x_6 &= \dfrac13 \left( x^2_5 + 2 - e^{-x_5}\right) &= 0.60494\\
                \implies&&x_7 &= \dfrac13 \left( x^2_6 + 2 - e^{-x_6}\right) &= 0.60662\\
                \implies&&x_8 &= \dfrac13 \left( x^2_7 + 2 - e^{-x_7}\right) &= 0.60759\\
                \implies&&x_9 &= \dfrac13 \left( x^2_8 + 2 - e^{-x_8}\right) &= 0.60817\\
                \implies&&x_{10} &= \dfrac13 \left( x^2_9 + 2 - e^{-x_9}\right) &= 0.60851\\
                \implies&&x_{11} &= \dfrac13 \left( x^2_{10} + 2 - e^{-x_{10}}\right) &= 0.60870
            \end{alignat*}

            Since $f(0.6085) = 0.000606 > 0$ and $f(0.6095) = -0.000632 < 0$, we have that $\alpha \in (0.6085, 0.6095)$. Hence,

            \boxt{
                $\alpha = 0.609 \todp{3}$
            }

        \part
            \begin{center}
                \begin{tikzpicture}[trim axis left, trim axis right, scale=0.86]
                    \begin{axis}[
                        domain = 0:3,
                        samples = 101,
                        axis y line=middle,
                        axis x line=middle,
                        xtick = {0.609, 2.109, 2.5, 1.6},
                        xticklabels = {$\alpha$, $\beta$, $x_0 > \beta$, $x_0 < \beta$},
                        ytick = \empty,
                        xlabel = {$x$},
                        ylabel = {$y$},
                        legend cell align={left},
                        legend pos=outer north east,
                        after end axis/.code={
                            \path (axis cs:0,0) 
                                node [anchor=east] {$O$};
                            }
                        ]
                        \addplot[plotRed] {1/3 * (x^2 + 2 - e^(-x))};

                        \addlegendentry{$y=\frac13 (x^2 + 2 - e^{-x})$};

                        \addplot[plotBlue] {x};

                        \addlegendentry{$y=x$};

                        \draw[dotted, thick] (0.609, 0) -- (0.609, 0.609);

                        \draw[dotted, thick] (2.109, 0) -- (2.109, 2.109);

                        \draw[dotted, thick] (2.5, 0) -- (2.5, 2.723);

                        \draw[dotted, thick] (1.6, 0) -- (1.6, 1.453);

                        \begin{scope}[decoration={
                            markings,
                            mark=at position 0.5 with {\arrow{>}}}
                            ] 

                            \draw[postaction={decorate}]  (2.5, 2.723) --  (2.723, 2.723);

                            \draw[postaction={decorate}] (2.723, 2.723) -- (2.723, 3.116);

                            \draw[postaction={decorate}] (2.723, 3.116) -- (3, 3.116);

                            \draw[postaction={decorate}] (1.6, 1.453) -- (1.453, 1.453);

                            \draw[postaction={decorate}] (1.453, 1.453) -- (1.453, 1.292);

                            \draw[postaction={decorate}] (1.453, 1.292) -- (1.292, 1.292);

                            \draw[postaction={decorate}] (1.292, 1.292) -- (1.292, 1.132);

                            \draw[postaction={decorate}] (1.292, 1.132) -- (1.132, 1.132);
                        \end{scope}
                    \end{axis}
                \end{tikzpicture}
            \end{center}

            From the diagram, we see that whether we chose $x_0 < \beta$ or $x_0 > \beta$, we the approximates move away from the root $\beta$. In fact, if we choose $x_0 < \beta$, the approximates converge to the root $\alpha$ instead.

        \part
            Using linear interpolation on the interval $[2, 3]$,
            \begin{equation*}
                x_1 = \dfrac{2f(3)-3f(2)}{f(3)-f(2)} = 2.06 \todp{2}
            \end{equation*}

            \boxt{
                $\beta = 2.06 \todp{2}$
            }

            Observe that $f(2.06) = -0.039 < 0$ and $f(3) = 1.950 > 0$. Hence, $\beta \in (2.06, 3)$. Thus,

            \boxt{
                $\beta = 2.06$ is an underestimate.
            }

        \part
            Observe that $f^\prime(x) = 2x x- 3 + e^{-x}$.Using the Newton-Raphson method with $x_1 = 2.06$,
            \begin{alignat*}{2}
                &&x_2 &= \nrm{x_1}{f} = 2.11118 = 2.111 \todp{3}\\
                \implies&&x_3 &= \nrm{x_2}{f} = 2.10935 = 2.109\todp{3}\\
                \implies&&x_4 &= \nrm{x_3}{f} = 2.10935 = 2.109\todp{3}
            \end{alignat*}

            Hence,
            \boxt{
                $\beta = 2.109 \todp{3}$
            }
\end{document}