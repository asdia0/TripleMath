\documentclass{echw}

\title{Timed Practice 1\\A4, B1}
\author{Eytan Chong}
\date{2024-04-04}

\begin{document}
    \problem{}
        Find $\displaystyle\sum\limits_{r = 0}^n \bp{n^2 + 1 - 3r}$ in terms of $n$, giving your answer in factorized form.

    \solution
        \begin{align*}
            \sum_{r = 0}^n \bp{n^2 + 1 - 3r} &= (n+1)(n^2 + 1) - 3 \cdot \dfrac{n(n+1)}2\\
            &= (n+1)\bp{n^2 - \dfrac32 n + 1}\\
            &= \dfrac12 (n+1)(2n^2 - 3n + 2)
        \end{align*}

        \boxt{$\displaystyle\sum\limits_{r = 0}^n \bp{n^2 + 1 - 3r} = \dfrac12 (n+1)(2n^2 - 3n + 2)$}

    \problem{}
        Given that $\displaystyle\sum\limits_{k=1}^n k!\bp{k^2 + 1} = (n+1)!\,n$, find $\displaystyle\sum\limits_{k=1}^{n-1} (k+1)!\bp{k^2 + 2k + 2}$.

    \solution
        \begin{align*}
            \sum_{k=1}^{n-1} (k+1)!\bp{k^2 + 2k + 2} &= \sum_{k-1=1}^{k-1=n-1} \bp{(k-1)+1}!\bp{(k-1)^2 + 2(k-1) + 2}\\
            &= \sum_{k=2}^{n} k! \bp{k^2 + 1}\\
            &= \sum_{k=1}^{n} k! \bp{k^2 + 1} - 1! \bp{1^2 + 1}\\
            &= (n+1)! \, n - 2
        \end{align*}

        \boxt{$\displaystyle\sum\limits_{k=1}^{n-1} (k+1)!\bp{k^2 + 2k + 2} = (n+1)! \, n - 2$}

    \problem{}
        Given that $\displaystyle\sum\limits_{r=1}^n = \dfrac16 n(n+1)(2n+1)$, find $\displaystyle\sum\limits_{r=N+1}^{2N} \bp{7^{r+1} + 3r^2}$ in terms of $N$, simplifying your answer.

    \solution
        \begin{align*}
            \sum_{r=N+1}^{2N} \bp{7^{r+1} + 3r^2} &= \sum_{r=N+1}^{2N} 7^{r+1} + 3\sum_{r=N+1}^{2N} r^2 \\
            &= \dfrac{7^{(N+1)+1}(7^N - 1)}{7-1} + 3\bp{\sum_{r=1}^{2N} r^2 - \sum_{r=1}^{N} r^2}\\
            &= \dfrac{7^{N+2}(7^N - 1)}{6} + 3\bp{\dfrac16 (2N)(2N+1)(2 \cdot 2N + 1) - \dfrac16 N(N+1)(2N+1)}\\
            &= \dfrac{7^{N+2}(7^N - 1)}{6} + \dfrac12\bp{2N(2N+1)(4N + 1) - N(N+1)(2N+1)}\\
            &= \dfrac{7^{N+2}(7^N - 1)}{6} + \dfrac12 N(2N+1)(2 \cdot (4N + 1) - (N+1))\\
            &= \dfrac{7^{N+2}(7^N - 1)}{6} + \dfrac12 N(2N+1)(7N+1)
        \end{align*}

        \boxt{$\displaystyle\sum\limits_{r=N+1}^{2N} \bp{7^{r+1} + 3r^2} = \dfrac{7^{N+2}(7^N - 1)}{6} + \dfrac12 N(2N+1)(7N+1)$}

    \problem{}
        Let $f(r) = \dfrac3{r-1}$.

        \begin{enumerate}
            \item Show that $f(r+1) - f(r) = -\dfrac3{r(r-1)}$.
            \item Hence, find in terms of $N$, the sum of the series $S_N = \displaystyle\sum\limits_{r=2}^N \dfrac1{r(r-1)}$.
            \item Explain why $\displaystyle\sum\limits_{r=2}^\infty \dfrac1{r(r-1)}$ is a convergent series, and find the value of the sum to infinity.
            \item Using the result from part (b), find $\displaystyle\sum\limits_{r=2}^N \dfrac1{r(r+1)}$.
        \end{enumerate}

    \solution
        \part
            \begin{align*}
                f(r+1) - f(r) &= \dfrac3{(r+1) - 1} - \dfrac3{r - 1}\\
                &= \dfrac{3(r-1) - 3r}{r(r-1)}\\
                &= -\dfrac3{r(r-1)}
            \end{align*}

        \part
            \begin{align*}
                S_N = \sum_{r=2}^N \dfrac1{r(r-1)}
                &= -\dfrac13 \sum_{r=2}^N -\dfrac{3}{r(r-1)}\\
                &= -\dfrac13 \bp{\sum_{r=2}^N f(r+1) - \sum_{r=2}^N f(r) }\\
                &= -\dfrac13 \bp{\sum_{r=3}^{N+1} f(r) - \sum_{r=2}^N f(r) }\\
                &= -\dfrac13 \bp{\bp{\sum_{r=3}^{N} f(r) + f(N+1)} - \bp{f(2) + \sum_{r=3}^N f(r) }}\\
                &= -\dfrac13 \bp{f(N+1) - f(2)}\\
                &= -\dfrac13 \bp{\dfrac3{N+1-1} - \dfrac3{2-1}}\\
                &= 1 - \dfrac1N
            \end{align*}

            \boxt{$S_N = 1 - \dfrac1N$}

        \part
            Consider $\lim\limits_{n \to \infty} S_n$.
            \begin{align*}
                \lim_{n \to \infty} S_n &= \lim_{n \to \infty} \bp{1 - \dfrac1N}\\
                &= 1 - 0\\
                &= 1
            \end{align*}

            Since 1 is a constant, $S_N$ is a convergent series.

            \boxt{The value of the sum to infinity is 1.}

        \part
            \begin{align*}
                \sum_{r=2}^N \dfrac1{r(r+1)} &= \sum_{r=3}^{N+1} \dfrac1{(r-1)r}\\
                &= \sum_{r=2}^{N} \dfrac1{r(r-1)} - \dfrac1{2(2-1)} + \dfrac1{(N+1)N} \\
                &= 1 - \dfrac1N - \dfrac12 + 1\dfrac1{N(N+1)}\\
                &= \dfrac12 + \dfrac{1 - (N+1)}{N(N+1)}\\
                &= \dfrac12 - \dfrac{N}{N(N+1)}\\
                &= \dfrac12 - \dfrac1{N+1}
            \end{align*}

            \boxt{$\displaystyle\sum\limits_{r=2}^N \dfrac1{r(r+1)} = \dfrac12 - \dfrac1{N+1}$}

    \problem{}
        Sketch the graph of $y = \dfrac{4\l - x^2}{x^2 + \l}$ in each of the following two cases:

        \begin{enumerate}
            \item $\l > 0$,
            \item $\l < 0$.
        \end{enumerate}

         By using your graph in part (b) and considering a suitable graph whose Cartesian equation is to be stated, find the positive value of $h$ such that the equation

        \[
            \bp{\dfrac{4\l - x^2}{hx^2 + h\l}}^2 = 1 + \dfrac{x^2}{\l}
        \]
        where $\l < 0$, has only one real root. State the value of the real root for this value of $h$.

    \solution
        \part
            \begin{center}
                \begin{tikzpicture}[trim axis left, trim axis right]
                    \begin{axis}[
                        domain = -5:5,
                        samples = 101,
                        axis y line=middle,
                        axis x line=middle,
                        xtick = {-2, 2},
                        xticklabels = {$-2\sqrt\l$, $2\sqrt\l$},
                        ytick = {4},
                        xlabel = {$x$},
                        ymax=5,
                        ymin=-2,
                        ylabel = {$y$},
                        legend cell align={left},
                        legend pos=outer north east,
                        after end axis/.code={
                            \path (axis cs:0,0) 
                                node [anchor=north east] {$O$};
                            }
                        ]
                        \addplot[plotRed] {(4 - x^2)/(x^2 + 1)};
            
                        \addlegendentry{$y = \tfrac{4\l - x^2}{x^2 + \l}$, $\l > 0$};

                        \draw[dotted, thick] (-5, -1) -- (5, -1) node[anchor=north east] {$y = -1$};
                    \end{axis}
                \end{tikzpicture}
            \end{center}

        \part
            \begin{center}
                \begin{tikzpicture}[trim axis left, trim axis right]
                    \begin{axis}[
                        domain = -5:5,
                        samples = 101,
                        axis y line=middle,
                        axis x line=middle,
                        xtick = \empty,
                        ytick = {4},
                        xlabel = {$x$},
                        ymax=10,
                        ymin=-10,
                        ylabel = {$y$},
                        legend cell align={left},
                        legend pos=outer north east,
                        after end axis/.code={
                            \path (axis cs:0,0) 
                                node [anchor=north east] {$O$};
                            }
                        ]
                        \addplot[plotRed, unbounded coords = jump] {(-4 - x^2)/(x^2 - 1)};
            
                        \addlegendentry{$y = \tfrac{4\l - x^2}{x^2 + \l}$, $\l < 0$};

                        \draw[dotted, thick] (-5, -1) -- (5, -1) node[anchor=north east] {$y = -1$};
                        
                        \draw[dotted, thick] (-1, -10) -- (-1, 10) node[anchor=north east] {$x = -\sqrt{-\l}$};

                        \draw[dotted, thick] (1, -10) -- (1, 10) node[anchor=north west] {$x = \sqrt{-\l}$};
                    \end{axis}
                \end{tikzpicture}
            \end{center}

            \newpage

            Consider $\bp{\dfrac{4\l - x^2}{hx^2 + h\l}}^2 = 1 + \dfrac{x^2}{\l}$.

            \begin{alignat*}{2}
                &&\bp{\dfrac{4\l - x^2}{hx^2 + h\l}}^2 &= 1 + \dfrac1{x^2}{\l}\\
                \implies&&\bp{\dfrac1h \cdot \dfrac{4\l - x^2}{x^2 + \l}}^2 &= 1 + \dfrac{x^2}\l\\
                \implies&&\bp{\dfrac1h \cdot y}^2 &= 1 + \dfrac{x^2}\l\\
                \implies&&\dfrac{y^2}{h^2} &= 1 + \dfrac{x^2}\l\\
                \implies&&- \dfrac{x^2}\l + \dfrac{y^2}{h^2} &= 1\\
                \implies&&- \dfrac{x^2}{\sqrt{\l}^2} + \dfrac{y^2}{h^2} &= 1
            \end{alignat*}

            We hence plot an ellipse centred at the origin with horizontal radius $\sqrt{\l}$ and vertical radius $h$.
            
            For only one real root, the ellipse must meet the above graph at exactly one point. Hence, $h = 4$ and the corresponding root is $x = 0$.

            \begin{center}
                \begin{tikzpicture}[trim axis left, trim axis right]
                    \begin{axis}[
                        domain = -5:5,
                        samples = 101,
                        axis y line=middle,
                        axis x line=middle,
                        xtick = \empty,
                        ytick = {4},
                        xlabel = {$x$},
                        ymax=10,
                        ymin=-10,
                        ylabel = {$y$},
                        legend cell align={left},
                        legend pos=outer north east,
                        after end axis/.code={
                            \path (axis cs:0,0) 
                                node [anchor=north east] {$O$};
                            }
                        ]
                        \addplot[plotRed, unbounded coords = jump] {(-4 - x^2)/(x^2 - 1)};
            
                        \addlegendentry{$y = \tfrac{4\l - x^2}{x^2 + \l}$, $\l < 0$};

                        \draw[dotted, thick] (-5, -1) -- (5, -1) node[anchor=north east] {$y = -1$};
                        
                        \draw[dotted, thick] (-1, -10) -- (-1, 10) node[anchor=north east] {$x = -\sqrt{-\l}$};

                        \draw[dotted, thick] (1, -10) -- (1, 10) node[anchor=north west] {$x = \sqrt{-\l}$};

                        \addplot[plotBlue] (0, 0) ellipse[x radius = 1, y radius = 4];

                        \addlegendentry{$-\frac{x^2}{\l} + \frac{y^2}{h^2} = 1$}
                    \end{axis}
                \end{tikzpicture}
            \end{center}

            \boxt{$h = 4$, $x = 0$}
\end{document}