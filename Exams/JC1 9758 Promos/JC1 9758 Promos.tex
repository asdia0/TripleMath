\documentclass{echw}

\title{H2 Mathematics\\Promotional Examination}
\author{Eytan Chong}
\date{2024-10-09}

\begin{document}
    \problem{}
        A dietitian in a hospital is to arrange a special diet meal composed of Food A, Food B and Food C. The diet is to include exactly 7800 units of calcium, 80 units of iron and 7.5 units of vitamin A. The number of units of calcium, iron and vitamin A for each gram of the foods is indicated in the table.

        \begin{table}[H]
            \centering
            \begin{tabular}{|l|l|l|l|}
            \hline
             & \multicolumn{3}{|c|}{Units Per Gram} \\\hline
             & Food A & Food B & Food C \\\hline
             Calcium & 15 & 20 & 24 \\\hline
             Iron & 0.2 & 0.15 & 0.28 \\\hline
             Vitamin A & 0.015 & 0.02 & 0.02\\\hline
            \end{tabular}
        \end{table}

        Find the total weight of the foods, in grams, of this special diet.
        
    \solution
        Let $a$, $b$, and $c$ represent the weight of Food A, B, and C respectively, in grams. We have the following system of equations:
        \[\systeme{15a + 20b + 24c = 7800,0.2a+0.15b+0.28c=80,0.015a+0.02b+0.02c = 7.5}\]
        Using G.C., $a=160$, $b=180$ and $c=75$. Hence, the total weight of the foods is $160 + 180 + 75 = \boxed{415}$ grams.

    \problem{}
        By expressing $\dfrac{3x^2 + 2x - 12}{x-1} - (x+2)$ as a single simplified fraction, solve the inequality \[\frac{3x^2 + 2x - 12}{x-1} \geq x + 2,\] without using a calculator.

    \solution
        \begin{align*}
            \dfrac{3x^2+2x-12}{x-1} - (x+2) &= \dfrac{3x^2 + 2x - 12 - (x+2)(x-1)}{x-1}\\
            &= \dfrac{3x^2 + 2x - 12 - (x^2 + x - 2)}{x-1}\\
            &= \dfrac{2x^2 + x - 10}{x-1}\\
            &= \dfrac{(x-2)(2x+5)}{x-1}
        \end{align*}
        Consider the inequality.
        \begin{alignat*}{2}
            &&\frac{3x^2 + 2x - 12}{x-1} &\geq x + 2\\
            \implies&&\frac{3x^2 + 2x - 12}{x-1} - (x+1) &\geq 0\\
            \implies&&\dfrac{(x-2)(2x+5)}{x-1} &\geq 0\\
            \implies&&(x-2)(2x+5)(x-1) &\geq 0
        \end{alignat*}

        Hence, $\boxed{-\dfrac52 \leq x < 1 \lor x \geq 2}$.

    \problem{}
        \begin{enumerate}
            \item Given that $\displaystyle \sum_{r=1}^n r^2 = \frac{n}6 (n+1)(2n+1)$, evaluate $\displaystyle \sum_{r=-n}^n (r+1)(r+3)$ in terms of $n$.
            \item Using standard series from the List of Formulae (MF27), find the range of values of $x$ for which the series $\displaystyle \sum_{r=1}^\infty \frac{(-1)^{r+1} x^r}{r 2^r}$ converges. State the sum to infinity in terms of $x$.
        \end{enumerate}

    \solution
        \part
            \begin{align*}
                &\sum_{r=-n}^n (r+1)(r+3) \\
                &= \sum_{r=0}^{2n} (r-n+1)(r-n+3)\\
                &= \sum_{r=0}^{2n} \bs{r^2 + r(4-2n) + (n^2 - 4n + 3)}\\
                &= \dfrac{(2n)(2n+1)(4n+1)}{6} + (4-2n) \, \dfrac{(2n)(2n+1)}2 + (n^2 - 4n + 3) (2n+1)\\
                &= (2n+1)\bs{\dfrac{n(4n+1)}{3} + (4-2n)(n) + (n^2 - 4n + 3)}\\
                &= \dfrac{2n+1}3 \bs{4n^2 + n + 12n - 6n^2 + 3n^2 - 12 n + 9}\\
                &= \boxed{\dfrac{2n+1}{3} \bp{n^2 + n +9}}
            \end{align*}

        \part
            \begin{align*}
                \sum_{r=1}^\infty \frac{(-1)^{r+1} x^r}{r 2^r} &= \sum_{r=1}^\infty \frac{(-1)^{r+1} (x/2)^r}{r}\\
                &= \boxed{\ln{1 + \dfrac{x}2}}
            \end{align*}
            Range of convergence: $-1 < x/2 \leq 1 \implies \boxed{-2 < x \leq 2}$.

    \problem{}
        The curve $C$ has parametric equations \[x = -\frac2{t-1}, \quad y = \frac{4}{t+1}, \quad t < 1, \quad t \neq \pm 1.\]
        
        \begin{enumerate}
            \item Sketch a clearly labelled diagram of $C$, indicating any axial intercepts and asymptotes (if any) of this curve.
            \item Find also its Cartesian equation, stating any restrictions where applicable.
        \end{enumerate}

    \solution
        \part
            \begin{center}
                \begin{tikzpicture}[trim axis left, trim axis right]
                    \begin{axis}[
                        domain = 0:5,
                        restrict y to domain =-10:10,
                        samples = 101,
                        axis y line=middle,
                        axis x line=middle,
                        xtick = \empty,
                        ytick = \empty,
                        xlabel = {$x$},
                        ylabel = {$y$},
                        legend cell align={left},
                        legend pos=outer north east,
                        after end axis/.code={
                            \path (axis cs:0,0) 
                                node [anchor=north east] {$O$};
                            }
                        ]
                        \addplot[plotRed, unbounded coords = jump] {2*x/(x-1)};
            
                        \addlegendentry{$C$};

                        \draw (0, 0) circle[radius=2.5pt] node[anchor=south west] {$(0, 0)$};

                        \addplot[dotted] {2};
                        \draw[dotted] (1, 10) -- (1, -8) node[anchor=south west] {$x=1$};

                        \node[anchor=north] at (3, 2) {$y = 2$};
                    \end{axis}
                \end{tikzpicture}
            \end{center}

        \part
            Note that $x = -\dfrac{2}{t - 1} \implies t = -\dfrac2x + 1$. Hence,
            \begin{align*}
                y &= \dfrac{4}{(-2/x + 1) + 1}\\
                &= \dfrac{4x}{-2+2x}\\
                &= \dfrac{2x}{x-1}
            \end{align*}

            Hence, $\boxed{y = \dfrac{2x}{x-1}, x \neq 1, x > 0}$.

    \problem{}
        \begin{enumerate}
            \item Find, using an algebraic method, the exact roots of the equation $\abs{3x^2 + 5x - 8} = 4-x$.
            \item On the same axes, sketch the curves with equations $y = \abs{3x^2 + 5x - 8}$ and $y = 4-x$. Hence, solve exactly the inequality $\abs{3x^2 + 5x - 8} < 4-x$.
        \end{enumerate}

    \solution
        \part
            \textbf{Case 1: $3x^2 + 5x - 8 = 4-x$.}
            \begin{alignat*}{2}
                &&3x^2 + 5x - 8 &= 4-x\\
                \implies&&3x^2 + 6x - 12 &= 0\\
                \implies&&x^2 + 2x - 4 &= 0\\
                \implies&&x^2 + 2x + 1 &= 5\\
                \implies&&(x+1)^2 &= 5\\
                \implies&& x &= -1\pm\sqrt5
            \end{alignat*}

            \textbf{Case 2: $3x^2 + 5x - 8 = -(4-x)$.}
            \begin{alignat*}{2}
                &&3x^2 + 5x - 8 &= -4 + x\\
                \implies&&3x^2 + 4x - 4 &= 0\\
                \implies&&(3x-2)(x+2) &= 0\\
                \implies&& x &= \dfrac23, -2
            \end{alignat*}

            Hence, the roots are $x = -1\pm\sqrt5$, $\dfrac23$, $-2$.

        \part
            \begin{center}
                \begin{tikzpicture}[trim axis left, trim axis right]
                    \begin{axis}[
                        domain = -4:2.5,
                        restrict y to domain =0:15,
                        samples = 150,
                        axis y line=middle,
                        axis x line=middle,
                        xtick = {-1-sqrt(5), -1+sqrt(5), 2/3, -2},
                        xticklabels = {$-1-\sqrt5$, $-1+\sqrt5$, $\frac23$, $-2$},
                        ytick = {8, 4},
                        xlabel = {$x$},
                        ylabel = {$y$},
                        legend cell align={left},
                        legend pos=outer north east,
                        after end axis/.code={
                            \path (axis cs:0,0) 
                                node [anchor=north] {$O$};
                            }
                        ]
                        \addplot[plotRed] {abs(3*x^2 + 5*x - 8)};
            
                        \addlegendentry{$y = \abs{3x^2 + 5x - 8}$};

                        \addplot[plotBlue] {4-x};

                        \addlegendentry{$y = 4-x$};

                        \draw[dotted] (-3.236, 0) -- (-3.236, 7.236);
                        
                        \draw[dotted] (-2, 0) -- (-2, 6);
                        
                        \draw[dotted] (2/3, 0) -- (2/3, 10/3);

                        \draw[dotted] (1.236, 0) -- (1.236, 2.764);
                    \end{axis}
                \end{tikzpicture}
            \end{center}

            From the graph, $\boxed{-1-\sqrt5 < x < -2 \lor \dfrac23 < x < -1 + \sqrt5}$.

    \problem{}
        \begin{enumerate}
            \item The transformations $A$, $B$ and $C$ are given as follows:

            $A$: A reflection in the $x$-axis.

            $B$: A translation of 1 unit in the positive $y$-direction.

            $C$: A translation of 2 units in the negative $x$-direction.

            A curve undergoes in succession, the transformations $A$, $B$ and $C$, and the equation of the resulting curve is $y = \dfrac{2x+1}{2x+2}$. Determine the equation of the curve before the transformations, expressing your answer as a single fraction.
            \item The diagram shows the curve $y = f(x)$. The lines $x = -2$, $x = 2$ and $y = \dfrac32 x$ are asymptotes to the curve. The curve has turning points at $(-3, -5)$ and $(3, 5)$. It also has a stationary point of inflexion at the origin $O$.
            \begin{center}
                \begin{tikzpicture}[trim axis left, trim axis right]
                    \begin{axis}[
                        domain = -10:10,
                        restrict y to domain =-15:15,
                        samples = 150,
                        axis y line=middle,
                        axis x line=middle,
                        xtick = \empty,
                        ytick = \empty,
                        xlabel = {$x$},
                        ylabel = {$y$},
                        legend cell align={left},
                        legend pos=outer north east,
                        after end axis/.code={
                            \path (axis cs:0,0) 
                                node [anchor=north east] {$O$};
                            }
                        ]
                        \addplot[plotRed, unbounded coords = jump] {(3/2 * x^3)/(x^2 - 4)};
            
                        \addlegendentry{$y = f(x)$};

                        \fill (-3.46, -7.79) circle[radius=2.5pt] node[anchor=south] {$(-3, -5)$};
                        \fill (3.46, 7.79) circle[radius=2.5pt] node[anchor=north] {$(3, 5)$};

                        \draw[dotted] (-2, 15) -- (-2, -15) node[anchor=south east] {$x=-2$};
                        \draw[dotted] (2, 15) -- (2, -15) node[anchor=south west] {$x=2$};
                        \addplot[dotted] {3/2 * x};
                        \node[rotate=40] at (8, 10) {$y = \frac32 x$};
                    \end{axis}
                \end{tikzpicture}
            \end{center}
            \begin{enumerate}
                \item State the range of values of $x$ for which the graph is concave downwards.
                \item Sketch the graph of $y = \dfrac{1}{f(x)}$.
                \item Sketch the graph of $y = f'(x)$.
            \end{enumerate}
        \end{enumerate}

    \solution
        \part
            \begin{align*}
                A : y \mapsto -y &\implies \inv A : y \mapsto -y\\
                B : y \mapsto y-1 &\implies \inv B : y \mapsto y+1\\
                C : x \mapsto x + 2 &\implies \inv C : x \mapsto x-2
            \end{align*}

            \begin{align*}
                y = \dfrac{2x+1}{2x+2} \xmapsto{\inv C} y = \dfrac{2(x-2)+1}{2(x-2)+2} = \dfrac{2x-3}{2x-2} \xmapsto{\inv B} y + 1 = \dfrac{2x-3}{2x-2} \xmapsto{\inv A} -y + 1 = \dfrac{2x-3}{2x-2}
            \end{align*}

            Hence, $y = 1 - \dfrac{2x-3}{2x-2}$, whence $\boxed{y = \dfrac1{2x-2}}$.

        \part
            \subpart
                \[\boxed{x < -2 \lor 0 < x < 2}\]

            \subpart
                \begin{center}
                    \begin{tikzpicture}[trim axis left, trim axis right]
                        \begin{axis}[
                            domain = -10:10,
                            restrict y to domain =-0.8:0.8,
                            samples = 150,
                            axis y line=middle,
                            axis x line=middle,
                            xtick = {2, -2},
                            ytick = \empty,
                            xlabel = {$x$},
                            ylabel = {$y$},
                            legend cell align={left},
                            legend pos=outer north east,
                            after end axis/.code={
                                \path (axis cs:0,0) 
                                    node [anchor=south east] {$O$};
                                }
                            ]
                            \addplot[plotRed, unbounded coords = jump] {(x^2 - 4)/(3/2 * x^3)};
                
                            \addlegendentry{$y = 1/f(x)$};

                            \fill (-3.46, -1/7.79) circle[radius=2.5pt] node[anchor=north east] {$(-3, -1/5)$};
                            \fill (3.46, 1/7.79) circle[radius=2.5pt] node[anchor=south west] {$(3, 1/5)$};
                        \end{axis}
                    \end{tikzpicture}
                \end{center}

            \subpart
            \begin{center}
                \begin{tikzpicture}[trim axis left, trim axis right]
                    \begin{axis}[
                        domain = -5:5,
                        restrict y to domain =-4:4,
                        ymax = 4,
                        samples = 150,
                        axis y line=middle,
                        axis x line=middle,
                        xtick = {-3.46, 3.46},
                        xticklabels={-3, 3},
                        ytick = \empty,
                        xlabel = {$x$},
                        ylabel = {$y$},
                        legend cell align={left},
                        legend pos=outer north east,
                        after end axis/.code={
                            \path (axis cs:0,0) 
                                node [anchor=north east] {$O$};
                            }
                        ]
                        \addplot[plotRed, unbounded coords = jump] {(x^2 * (1.5*x^2 - 18))/(x^2 - 4)^2};
            
                        \addlegendentry{$y = f(x)$};

                        \draw[dotted] (-2, 15) -- (-2, -4) node[anchor=south east] {$x=-2$};
                        \draw[dotted] (2, 15) -- (2, -4) node[anchor=south west] {$x=2$};

                        \addplot[dotted] {1.5};
                        \node[anchor=south east] at (5, 1.5) {$y = 3/2$};
                    \end{axis}
                \end{tikzpicture}
            \end{center}

    \problem{}
        A curve has parametric equations \[x = 3u^2, \quad y = 6u.\]
        \begin{enumerate}
            \item Find the equations of the normal to the curve at the point $P\bp{3p^2, 6p}$, where $p$ is a non-zero constant.
            \item The normal meets the $x$-axis at $Q$ and the $y$-axis at $R$. Find the coordinates of $Q$ and of $R$.
            \item Find two possible expressions for the area bounded by the normal and the axes in terms of $p$, stating the range of values of $p$ in each case.
            \item Given that $p$ is positive and increasing at a rate of 2 units/s, find the rate of change of the area of the triangle in terms of $p$.
        \end{enumerate}

    \solution
        \part
            Note that $\der{y}{x} = \dfrac{\derx{y}{u}}{\derx{x}{u}} = \dfrac{6}{6u} = \dfrac1u$. At $u = p$, the gradient of the normal is hence $-\dfrac1{1/p} = -p$. Thus, the equation of the normal at $P$ is $\boxed{y - 6p = -p(x - 3p^2)}$.

        \part
            At $Q$, $y = 0$. Hence, $x = 6 + 3p^2$, whence $\boxed{Q(6 + 3p^2, 0)}$.

            At $R$, $x = 0$. Hence, $y = 6p + 3p^2$, whence $\boxed{R(0, 6p+3p^2)}$.

        \part
            When $p > 0$, the area of the triangle is $\boxed{\dfrac12 (6p + 3p^3)(6 + 3p^2)}$ units$^2$.

            When $p < 0$, the area of the triangle is $\boxed{-\dfrac12 (6p + 3p^3)(6 + 3p^2)}$ units$^2$.

        \part
            Let the area of the triangle be $A$ unit$^2$. Since $p>0$, we have $A = \dfrac12 (6p + 3p^3)(6 + 3p^2)$. Thus,
            \begin{align*}
                \der{A}{p} &= \dfrac12 \bs{(6p + 3p^3)(6p) + (6+3p^2)(6 + 9p^2)}\\
                &= \dfrac92 \bp{5p^4 + 12p^2 + 4}
            \end{align*}
            Hence, the rate of change of area of the triangle is $\der{A}{t} = \der{A}{p} \cdot \der{p}{t} = \boxed{9\bp{5p^4 + 12p^2 + 4}}$ units$^2$/s.

    \problem{}
        The shaded region $R$ is bounded by the lines $y = 2x$, $y = \dfrac32$, the $x$-axis and the curve $y = \sqrt{\dfrac{3x^2 - 1}{x^2}}$.

        \begin{center}
            \begin{tikzpicture}[trim axis left, trim axis right]
                \begin{axis}[
                    domain = 0:1.3,
                    restrict y to domain =0:2,
                    samples = 150,
                    axis y line=middle,
                    axis x line=middle,
                    ytick = {3/2},
                    yticklabels = {$\frac32$},
                    xtick = {1/sqrt(3), 3/4, 2/sqrt(3)},
                    xticklabels = {$\frac1{\sqrt3}$, $\frac34$, $\frac2{\sqrt3}$},
                    xlabel = {$x$},
                    ylabel = {$y$},
                    legend cell align={left},
                    legend pos=outer north east,
                    after end axis/.code={
                        \path (axis cs:0,0) 
                            node [anchor=north east] {$O$};
                        }
                    ]
                    \addplot[plotRed, domain=1/sqrt(3):1.3] {sqrt((3*x^2 - 1)/x^2)};
        
                    \addlegendentry{$y = \sqrt{(3x^2-1)/(x^2)}$};
        
                    \addplot[plotBlue] {2*x};
        
                    \addlegendentry{$y=2x$};

                    \draw[dotted] (3/4, 0) -- (3/4, 3/2);
                    \draw[dotted] (1.1547, 0) -- (1.1547, 3/2);
                    \draw[dotted] (0, 3/2) -- (3/4, 3/2);
                    \draw[black] (3/4, 3/2) -- (1.1547, 3/2);

                    \node at (0.4, 0.4) {$R$};
                \end{axis}
            \end{tikzpicture}
        \end{center}

        \begin{enumerate}
            \item By using the substitution $x = \dfrac1{\sqrt3} \sec \t$, find the exact value of $\displaystyle \int_{1/\sqrt{3}}^{2/\sqrt3} \sqrt{\dfrac{3x^2-1}{x^2}} \d x$.
            \item Hence, find the exact area of the shaded region $R$.
            \item Find the volume of the solid generated when $R$ is rotated through $2\pi$ radians about the $x$-axis, giving your answer in 3 decimal places.
        \end{enumerate}

    \solution
        \part
            Note that $x = \dfrac1{\sqrt3} \sec \t \implies \d x = \dfrac1{\sqrt3} \sec\t\tan\t \d \t$, with the bounds of integration going from $\t = 0$ to $\pi/3$.
            \begin{align*}
                \int_{1/\sqrt{3}}^{2/\sqrt3} \sqrt{\dfrac{3x^2-1}{x^2}} \d x &= \int_0^{\pi/3} \sqrt{\dfrac{\tan^2 \t}{\frac13 \sec^2 \t}} \dfrac1{\sqrt3} \sec \t \tan \t \d \t\\
                &= \int_0^{\pi/3} \tan^2 \t \d \t\\
                &= \int_0^{\pi/3} \bp{\sec^2 \t - 1} \d \t\\
                &= \evalint{\tan \t - t}{0}{\pi/3}\\
                &= \boxed{\sqrt3 - \dfrac\pi3}
            \end{align*}

        \part
            \begin{align*}
                \area R &= \underbrace{\dfrac12 \bs{\dfrac2{\sqrt3} + \bp{\dfrac2{\sqrt3} - \dfrac34}}\dfrac32}_{\text{area of trapezium}} - \int_{1/\sqrt3}^{2/\sqrt3} \sqrt{\dfrac{3x^2 - 1}{x^2}} \d x\\
                &= \boxed{\dfrac\pi3 - \dfrac{9}{16}} \text{ units$^2$}
            \end{align*}
        
        \part
            \begin{align*}
                \volume &= \underbrace{\dfrac13\pi \bp{\dfrac32}^2 \bp{\dfrac34}}_{\text{volume of cone}} + \underbrace{\pi\bp{\dfrac2{\sqrt3} - \dfrac34}\bp{\dfrac32}^2}_{\text{volume of cylinder}} - \pi\int_{1/\sqrt3}^{2/\sqrt3} \bp{\sqrt{\dfrac{3x^2-1}{x^2}}}^2 \d x\\
                &= \boxed{1.907} \text{ units$^3$}
            \end{align*}

    \problem{}
        It is given that $y = \arccos{2x} \arcsin{2x}$, where $-0.5 \leq x \leq 0.5$, and $\arccos{2x}$ and $\arcsin{2x}$ denote their principal values.
        \begin{enumerate}
            \item Show that $\bp{1-4x^2} \der[2]{y}{x} - 4x \bp{\der{y}{x}} + 8 = 0$. Hence, find the MacLaurin series for $y$ up to and including the term in $x^3$, giving the coefficients in exact form.
            \item Use your expansion from part (a) and integration to find an approximate value for $\displaystyle\int_{0.1}^{0.2} \bp{\dfrac2x}^3 \arccos{2x}\arcsin{2x} \d x$, correct to 4 decimal places.
            \item A student, Adam, claims that the approximation in part (b) is accurate. Without performing any further calculations, justify whether Adam's claim is valid.
            \item Suggest one way to improve the accuracy of the approximated value obtained in part (b).
        \end{enumerate}

    \solution
        \part
            \begin{alignat*}{2}
                && y &= \arccos{2x}\arcsin{2x}\\
                \implies&& y' &= 2 \cdot \dfrac{\arccos{2x} - \arcsin{2x}}{\sqrt{1-4x^2}}\\
                \implies&& y'' &= \dfrac{2}{\sqrt{1-4x^2}} \bs{-\dfrac{4}{\sqrt{1-4x^2}}} + \dfrac{2\bs{\arccos{2x} - \arcsin{2x}} \cdot -8x}{-2 \bp{1-4x^2}^{3/2}}\\
                && &= \dfrac1{1 - 4x^2} \bs{-8 + 4x \cdot \dfrac{2\bs{\arccos{2x} - \arcsin{2x}}}{\sqrt{1-4x^2}}}\\
                && &= \dfrac1{1-4x^2} \bp{-8 + 4xy'}
            \end{alignat*}

            Hence, $\bp{1-4x^2} y'' - 4xy' + 8 = 0$.

            Differentiating with respect to $x$, we get \[\bp{1-4x^2} y''' - 12xy'' - 4y' = 0.\]

            Evaluating $y$, $y'$, $y''$ and $y'''$ at $x = 0$, we get
            \begin{align*}
                y(0) &= 0\\
                y'(0) &= \pi\\
                y''(0) &= -8\\
                y'''(0) &= 4\pi
            \end{align*}
            Hence, $\boxed{y = \pi x - 4x^2 + \dfrac23 \pi x^3 + \cdots}$.

        \part
            \begin{align*}
                &\int_{0.1}^{0.2} \bp{\dfrac2x}^3 \arccos{2x}\arcsin{2x} \d x \\
                &= 8 \int_{0.1}^{0.2} x^{-3} \arccos{2x}\arcsin{2x} \d x\\
                &\approx 8 \int_{0.1}^{0.2} x^{-3} \bs{\pi x - 4x^2 + \dfrac23 \pi x^3} \d x\\
                &= 8 \int_{0.1}^{0.2} \bp{\pi x^{-2} - 4x^{-1} + \dfrac23 \pi x} \d x\\
                &= 8 \evalint{-\dfrac\pi{x} - 4\ln\abs{x} + \dfrac23 \pi x}{0.1}{0.2}\\
                &= \boxed{105.1585} \todp{4}
            \end{align*}

        \part
            Adam's claim is valid. Since the approximation for $\arccos{2x}\arcsin{2x}$ is accurate for $x$ near 0, and we are integrating over $(0.1, 0.2)$ (which is close to 0), the integral approximation should also be accurate.
        
        \part
            Consider more terms in the MacLaurin series of $y = \arccos{2x}\arcsin{2x}$ and use the improved series in the approximation for the integral.

    \problem{}
        The function $f$ is defined by \[f : x \mapsto 3\sin{2x - \dfrac16 \pi}, \quad 0 \leq x \leq k.\]
        \begin{enumerate}
            \item Show that the largest exact value of $k$ such that $\inv f$ exists is $\dfrac13 \pi$. Find $\inv f(x)$.
            \item It is given that $k = \dfrac13 \pi$. In a single diagram, sketch the graphs of $y = f(x)$ and $y = \inv f(x)$, labelling your graphs clearly.
        \end{enumerate}

        The function $h$ is defined by $h : x \mapsto 3\sin{2x - \dfrac16 \pi}$, $0 \leq x \leq \dfrac13 \pi$, $x = \dfrac1{12} \pi$.

        Another function $g$ is defined by $g : x \mapsto \abs{3 - \dfrac1{x}}$, $-3 \leq x \leq 3$, $x \neq 0$.

        \begin{enumerate}
            \setcounter{enumi}{2}
            \item Explain clearly why $gh$ exists. Find $gh(x)$ and its range.
            \item Supposing $\inv{(gh)}$ exists for a restriction, find the exact value of $x$ for which $\inv{(gh)} (x) = 0$. Show your working clearly.
        \end{enumerate}

    \solution
        \part
            For $\inv f$ to exist, $f$ must be one-one. Since $\dfrac13 \pi$ is the first maximum point of $f$, it is the largest value of $k$. 

            Let $y = f(x)$.
            \begin{alignat*}{2}
                && 3\sin{2x - \dfrac16 \pi} &= y \\
                \implies&& \sin{2x - \dfrac16 \pi} &= \dfrac{y}3 \\
                \implies&& 2x - \dfrac16 \pi &= \arcsin{\dfrac{y}3}\\
                \implies&& x &= \dfrac12 \arcsin{\dfrac{y}3} + \dfrac1{12} \pi
            \end{alignat*}

            Hence, $\boxed{\inv f(x) = \dfrac12 \arcsin{\dfrac{x}3} + \dfrac1{12} \pi}$.

        \part
            \begin{center}
                \begin{tikzpicture}[trim axis left, trim axis right]
                    \begin{axis}[
                        xmin = -2,
                        ymin=-1.7,
                        ymax=3.2,
                        xmax=3.2,
                        samples = 101,
                        axis y line=middle,
                        axis x line=middle,
                        xtick = {pi/12},
                        xticklabels = {$\frac\pi{12}$},
                        ytick = {pi/12},
                        yticklabels = {$\frac{\pi}{12}$},
                        xlabel = {$x$},
                        ylabel = {$y$},
                        legend cell align={left},
                        legend pos=outer north east,
                        after end axis/.code={
                            \path (axis cs:0,0) 
                                node [anchor=north east] {$O$};
                            }
                        ]
                        \addplot[plotRed, domain=0:pi/3] {3*sin(2*x*180/pi - 30)};
            
                        \addlegendentry{$y = f(x)$};
            
                        \addplot[plotBlue, domain=-3/2:3] {0.5*asin(x/3)/180*pi + pi/12};
            
                        \addlegendentry{$y=\inv f(x)$};

                        \addplot[dotted, domain=-3/2:3] {x};

                        \fill (-3/2, 0) circle[radius=2.5pt] node[anchor=south] {$\bp{-\frac32, 0}$};
                        \fill (0, -3/2) circle[radius=2.5pt] node[anchor=south west] {$\bp{0, -\frac32}$};

                        \fill (pi/3, 3) circle[radius=2.5pt] node[anchor=north west] {$\bp{\frac\pi3, 3}$};

                        \fill (3, pi/3) circle[radius=2.5pt] node[anchor=east] {$\bp{3, \frac\pi3}$};
                        
                        \node[rotate=45] at (2.7, 2.4) {$y = x$};

                    \end{axis}
                \end{tikzpicture}
            \end{center}
        
        \part
            Since $\ran h = [-\frac32, 0) \cup (0, 3]$ and $\dom g = [-3, 0) \cup (0, 3]$, we have $\ran h \subseteq \dom g$, whence $gh$ exists.

            \begin{align*}
                gh(x) &= g\bp{3\sin{2x - \dfrac16\pi}}\\
                &= \boxed{\abs{3 - \dfrac1{3\sin{2x - \frac16\pi}}}}
            \end{align*}

            When $h(x) = \dfrac13$, $gh(x) = 0$. When $x \to \dfrac1{12} \pi$, $gh(x) \to \infty$. Hence, $\boxed{\ran {gh} = [0, \infty)}$.

        \part
            \begin{alignat*}{2}
                && \inv {gh}(x) &= 0\\
                \implies&& x &= gh(0)\\
                && &= \abs{3 - \dfrac1{3\sin{-\frac16 \pi}}}\\
                && &= \boxed{\dfrac{11}3}
            \end{alignat*}

    \problem{}
        \begin{enumerate}
            \item \begin{enumerate}
                \item Express $\dfrac{2x}{(x+1)(x^2 + 1)} = \dfrac{A}{x+1} + \dfrac{Bx + c}{x^2 + 1}$, where $A$, $B$ and $C$ are constants to be found.
                \item Evaluate $\displaystyle\int_0^1 \dfrac{\ln{1+x^2}}{(x+1)^2} \d x$, giving your answer in the form $a\pi - \ln b$, where $a$ and $b$ are positive constants to be found.
            \end{enumerate}
            \item Find $\displaystyle \int \sin[3]{kx} \d x$, where $k$ is a constant.
        \end{enumerate}

    \solution
        \part
            \subpart
                Clearing denominators, we have \[2x = A(x^2 + 1) + (Bx + C)(x+1) = (A+B)x^2 + (B+C)x + (A + C).\]
                Comparing coefficients of $x^2$, $x$ and constant terms, we have \[\left\{\begin{aligned}
                    A + B = 0\\
                    B + C = 2\\
                    A + C = 0
                \end{aligned}\right.\] Hence, $A = -1$, $B = 1$ and $C = 1$, giving \[\boxed{\dfrac{2x}{(x+1)(x^2 + 1)} = -\dfrac1{x+1} + \dfrac{x+1}{x^2 + 1}}.\]

            \subpart
                \begin{align*}
                    \int_0^1 \dfrac{\ln{1+x^2}}{(x+1)^2} \d x &= \evalint{-\dfrac{\ln{1+x^2}}{x+1}}{0}{1} + \int_0^1 \dfrac{2x}{(x+1)(x^2 + 1)} \d x\\
                    &= -\dfrac{\ln2}2 + \int_0^1 \bp{-\dfrac1{x+1} + \dfrac{x+1}{x^2 + 1}} \d x\\
                    &= -\dfrac12 \ln 2 + \evalint{-\ln\abs{x+1}}{0}{1} + \dfrac12 \int_0^1 \dfrac{2x+2}{x^2 + 1} \d x\\
                    &= -\dfrac32 \ln 2 + \dfrac12 \int_0^1 \bp{\dfrac{2x}{x^2 + 1} + \dfrac2{x^2 + 1}} \d x\\
                    &= -\dfrac32 \ln 2 + \dfrac12 \evalint{\ln{x^2 + 1} + 2\arctan x}{0}{1}\\
                    &= -\dfrac32 \ln 2 + \dfrac12 \bp{\ln2 + 2 \cdot \dfrac\pi4}\\
                    &= \boxed{-\ln2 + \dfrac\pi4}
                \end{align*}
                Hence, $a = \dfrac14$ and $b = 2$.

        \part
            Note that $\sin 3u = 3\sin u - 4\sin^3 u$, whence $\sin^3 u = \dfrac{3\sin u - \sin 3u}{4}$.
            \begin{align*}
                \int \sin[3]{kx} \d x &= \dfrac1k \int \sin^3 u \d u \usub{u = kx}\\
                &= \dfrac1k \int \dfrac{3\sin u - \sin 3u}{4} \d u\\
                &= \dfrac1{4k} \bs{-3\cos u + \dfrac13 \cos 3u} + C\\
                &= \boxed{-\dfrac3{4k} \cos kx + \dfrac1{12k} \cos 3kx + C}
            \end{align*}

    \problem{}
        Alan wants to sign up for a triathlon competition which requires him to swim for 1.5 km, cycle for 30 km and run for 10 km. He plans a training programme as follows: In the first week, Alan is to swim 400 m, cycle 1 km and run 400 m. Each subsequent week, he will increase his swimming distance by 50 m, his cycling distance by 15\% and his running distance by $r$\%.
        
        \begin{enumerate}
            \item Given that Alan will run 829.44 m in Week 5, show that $r = 20$. Hence, determine the distance that Alan will run in Week 20, giving your answer to the nearest km.
            \item Determine the week when Alan first achieves the distances required for all three categories of the competition.
        \end{enumerate}

        During a particular running practice, Alan plans to run $q$ metres in the first minute. The distance he covers per minute will increase by 80 m for the next four minutes. Subsequently, he will cover 6\% less distance in a minute than that in the previous minute.

        \begin{enumerate}
            \setcounter{enumi}{2}
            \item Show that the distance, in metres, Alan will cover in the sixth minute is $0.94q + 300.8$, and hence find the minimum value of $q$, to the nearest metre, such that he can eventually complete 10 km.
        \end{enumerate}

        While training, Alan suffers from inflammation and needs medication. The concentration of the medicine in the bloodstream after administration can be modelled by the recurrence relation \[C_{n+1} = C_n e^{-\bp{p + \frac{1}{n+100}}},\] where $n$ represents the number of complete hours from which the medicine is first taken and $p$ is the decay constant.

        \begin{enumerate}
            \setcounter{enumi}{3}
            \item The dosage of the medicine prescribed for Alan is 200 mg and the concentration of the medicine dropped to approximately 168 mg one hour later. It is given that his pain will be significant once the concentration falls below 60 mg. Determine after which complete hour he would feel significant pain and should take his medicine again.
        \end{enumerate}
        
    \solution
        \part
            Let $R_n$ m be the distance ran in the $n$th week. We have $R_1 = 400$ and $R_{n+1} = \bp{1 + \dfrac{r}{100}}R_n$, whence $R_n = 400 \bp{1 + \dfrac{r}{100}}^{n-1}$.

            \begin{alignat*}{2}
                && R_5 &+ 829.44\\
                \implies&& 400 \bp{1 + \dfrac{r}{100}}^4 &= 829.44\\
                \implies&& \bp{1 + \dfrac{r}{100}}^4 &= 2.0736\\
                \implies&& 1 + \dfrac{r}{100} &= 1.2\\
                \implies&& r &= 20
            \end{alignat*}

            Hence, $R_20 = 400 \cdot 1.2^19 = 12779.2 = 13000$ (to the nearest thousand). Hence, Alan will run $\boxed{13 \text{ km}}$ in week 20.

        \part
            Let $S_n$, $C_n$ be the distance swam and cycled in week $n$, respectively, in metres.

            We have $S_1 =400$ and $S_{n+1} = S_n + 50$, whence $S_n = 400 + (n-1)50$. Consider $S_n \geq 1500$. Then $n \geq 23$.

            We have $C_1 = 1000$ and $C_{n+1} = 1.15 C_n$, whence $C_n = 1000 \cdot 1.15^{n-1}$. Consider $C_n \geq 30000$. Then $n \geq \log_{1.15} 30 = 25.3$.

            Consider $R_n \geq 10000$. Then $n \geq 1 + \log_{1.2} 25 = 18.7$.

            Hence, the minimum $n$ required is 26. Thus, in $\boxed{\text{week } 26}$, Alan will achieve all distances required.

        \part
            The distance Alan will run in the 6th minute is $(q + 4 \cdot 80) \cdot (1 - 0.06) = 0.94q + 300.8$.

            Let $d_n$ be the distance travelled in the $n$th minute, where $n \geq 6$. We have $d_6 = 0.94 q + 300.8$ and $d_{n+1} = (1-0.06)d_n$, whence $d_n = 0.94^{n-6} d_6$. The total distance Alan will eventually travel is thus given by 
            \[5q + 80(4 + 3 + 2 + 1) + \sum_{n=6}^\infty (0.94)^{n-6} d_6 = 5q + 800 + \dfrac{0.94q + 300.8}{1 - 0.94}.\] Let the above expression be greater than 10000. Then $q \geq 202.581$. Hence, $\boxed{\min q = 203}$.

        \part
            We have $C_0 = 200$ and $C_1 = 168$. We hence have
            \[168 = 200 e^{-(p + \frac1{100})},\] whence $p = 0.16435$. Using G.C., the first time $C_n \leq 60$ occurs when $n = 7$. Thus, after $\boxed{7}$ complete hours, Alan will feel significant pain and should take his medicine again.

\end{document}