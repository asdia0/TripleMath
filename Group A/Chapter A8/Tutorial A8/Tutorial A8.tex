\documentclass{echw}

\title{Tutorial A8\\Vectors II}
\author{Eytan Chong}
\date{2024-05-09}

\begin{document}
    \problem{}
        For each of the following, write down a vector equivalent of the line $l$ and convert it to parametric and Cartesian forms.

        \begin{enumerate}
            \item $l$ passes through the point with position vector $-\vec i + \vec k$ and is parallel to the vector $\vec i + \vec j$.
            \item $l$ passes through the points $P(1, -1, 3)$ and $Q(2, 1, -2)$.
            \item $l$ passes through the origin and is parallel to the line $m: \vec r = \cveciii1{-1}3 + \l \cveciii123, \, \l \in \R$.
            \item $l$ is the $x$-axis.
            \item $l$ passes through the point $C(4,-1, 2)$ and is parallel to the $z$-axis.
        \end{enumerate}

    \solution
        \part
            \[
                \begin{array}{r @{\hspace*{1.0cm}} l}\toprule
                    \textbf{Form} & \textbf{Expression} \\\cmidrule{1-2}
                    \textbf{Vector} & \vec r = \cveciii{-1}01 + \l \cveciii110, \, \l \in \R \\
                    \textbf{Parametric} & \begin{cases}
                        x = \l - 1\\
                        y = \l\\
                        z = 1
                    \end{cases} \\
                    \textbf{Cartesian} & x + 1 = y, \, z = 1 \\\bottomrule
                \end{array}
            \]

        \part
            \[
                \begin{array}{r @{\hspace*{1.0cm}} l}\toprule
                    \textbf{Form} & \textbf{Expression} \\\cmidrule{1-2}
                    \textbf{Vector} & \vec r = \cveciii1{-1}3 + \l \cveciii12{-5}, \, \l \in \R \\
                    \textbf{Parametric} & \begin{cases}
                        x = \l + 1\\
                        y = 2\l - 1\\
                        z = -5\l + 3
                    \end{cases} \\
                    \textbf{Cartesian} & x-1 = \dfrac{y+1}2 = \dfrac{3-z}{5}\\\bottomrule
                \end{array}
            \]

        \part
        \[
            \begin{array}{r @{\hspace*{1.0cm}} l}\toprule
                \textbf{Form} & \textbf{Expression} \\\cmidrule{1-2}
                \textbf{Vector} & \vec r = \l \cveciii123, \, \l \in \R \\
                \textbf{Parametric} & \begin{cases}
                    x = \l\\
                    y = 2\l\\
                    z = 3\l
                \end{cases} \\
                \textbf{Cartesian} & x = \dfrac{y}2 = \dfrac{z}3\\\bottomrule
            \end{array}
        \]

        \part
            \[
                \begin{array}{r @{\hspace*{1.0cm}} l}\toprule
                    \textbf{Form} & \textbf{Expression} \\\cmidrule{1-2}
                    \textbf{Vector} & \vec r = \l \cveciii100, \, \l \in \R \\
                    \textbf{Parametric} & \begin{cases}
                        x = \l\\
                        y = 0\\
                        z = 0
                    \end{cases} \\
                    \textbf{Cartesian} & x \in \R, \, y = 0, \, z = 0 \\\bottomrule
                \end{array}
            \]

        \part
            \[
                \begin{array}{r @{\hspace*{1.0cm}} l}\toprule
                    \textbf{Form} & \textbf{Expression} \\\cmidrule{1-2}
                    \textbf{Vector} & \vec r = \cveciii4{-1}2 + \l \cveciii001, \, \l \in \R \\
                    \textbf{Parametric} & \begin{cases}
                        x = 4\\
                        y = -1\\
                        z = \l + 2
                    \end{cases} \\
                    \textbf{Cartesian} & x = 4, \, y = -1, \, z \in \R \\\bottomrule
                \end{array}
            \]

    \problem{}
        For each of the following, determine if $l_1$ and $l_2$ are parallel, intersecting or skew. In the case of intersecting lines, find the position vector of the point of intersection. In addition, find the acute angle between the lines $l_1$ and $l_2$.

        \begin{enumerate}
            \item $l_1 : x-1 = -y = z-2$ and $l_2 : \dfrac{x-2}2 = -\dfrac{y+1}2 = \dfrac{z-4}{2}$
            \item $l_1 : \vec r = \cveciii100 + \a\cveciii4{-2}{-3}, \, \a \in \R$ and $l_2 : \vec r = \cveciii0{10}1 + \b \cveciii381$
            \item $l_1 : \vec r = (\vec i - 5 \vec k) + \l(\vec i - \vec j + \vec k), \, \l \in \R$ and $l_2 : \vec r = (\vec i - \vec j + \vec k) + \m(5 \vec i - 4 \vec j - \vec k), \, \m \in \R$
        \end{enumerate}

    \solution
        \part
            Note that $l_1$ and $l_2$ have vector form
            \[
                l_1 : \vec r = \cveciii102 + \l\cveciii1{-1}1, \, \l \in \R \text{ and } l_2 : \vec r = \cveciii214 + \m\cveciii2{-2}{2}, \, \m \in \R
            \]
            Since $\cveciii2{-2}2 = 2\cveciii1{-1}1$, $l_1$ and $l_2$ are parallel. Since $\cveciii102 \neq \cveciii214 + \m\cveciii2{-2}2$ for all real $\m$, $l_1$ and $l_2$ are distinct.

            \boxt{Distinct parallel lines. $\t = 0$.}

        \part
            Since $\cveciii4{-2}3 \neq \b \cveciii381$ for all real $\b$, $l_1$ and $l_2$ are not parallel.

            Consider $l_1 = l_2$.
            \begin{alignat*}{2}
                &&l_1 &= l_2\\
                \implies&&\cveciii100 + \a\cveciii4{-2}{-3} &= \cveciii0{10}1 + \b \cveciii381\\
                \implies&&\a\cveciii4{-2}{-3} - \b\cveciii381 &= \cveciii{-1}{10}1
            \end{alignat*}
            This gives the following system:
            \[
                \systeme[\a\b]{
                    4\a - 3\b = -1,
                    -2\a - 8\b = 10,
                    -3\a - \b = 1
                }
            \]
            There are no solutions to the above system. Hence, $l_1$ and $l_2$ do not intersect and are hence skew.

            Let $\t$ be the acute angle between $l_1$ and $l_2$.
            \begin{alignat*}{2}
                &&\cos\t &= \dfrac{\abs{\cveciii4{-2}{-3} \dotp \cveciii381}}{\abs{\cveciii4{-2}{-3}} \abs{\cveciii381}}\\
                && &= \dfrac{7}{\sqrt{2146}}\\
                \implies&&\t &= 81.3 \deg \todp{1}
            \end{alignat*}

            \boxt{Skew lines. $\t = 81.3 \deg$.}

        \part
            Note that $l_1$ and $l_2$ have vector form
            \[
                l_1 : \vec r = \cveciii10{-5} + \l \cveciii1{-1}1 \text{ and } l_2 : \vec r = \cveciii1{-1}1 + \m \cveciii5{-4}{-1}
            \]

            Since $\cveciii1{-1}1 \neq \m \cveciii5{-4}{-1}$ for all real $\m$, $l_1$ and $l_2$ are not parallel.

            Consider $l_1 = l_2$.
            \begin{alignat*}{2}
                &&l_1 &= l_2\\
                \implies&&\cveciii10{-5} + \l \cveciii1{-1}1 &= \cveciii1{-1}1 + \m \cveciii5{-4}{-1}\\
                \implies&&\l\cveciii1{-1}1 - \m\cveciii5{-4}{-1} &= \cveciii0{-1}6
            \end{alignat*}
            This gives the following system:
            \[
                \systeme[\l\m]{
                    \l - 5\m = 0,
                    -\l + 4\m = -1,
                    \l + \m = 6
                }
            \]
            The above system has the unique solution $\l = 5$ and $\m = 1$. Hence, $l_1$ and $l_2$ intersect at $\cveciii10{-5} + 5\cveciii1{-1}1 = \cveciii6{-5}0$.

            Let $\t$ be the acute angle between $l_1$ and $l_2$.
            \begin{alignat*}{2}
                &&\cos\t &= \dfrac{\abs{\cveciii1{-1}1 \dotp \cveciii5{-4}{-1}}}{\abs{\cveciii1{-1}1} \abs{\cveciii5{-4}{-1}}}\\
                && &= \dfrac8{3\sqrt{14}}\\
                \implies&&\t &= 44.5 \deg \todp{1}
            \end{alignat*}

            \boxt{Intersecting lines. $\cveciii6{-5}0$. $\t = 44.5 \deg \todp{1}$.}

    \problem{}
        \begin{enumerate}
            \item Find the shortest distance from the point $(1, 2, 3)$ to the line with equation $\vec r = 3\vec i + 2\vec j + 4\vec k + \l(\vec i + 2\vec j + 2\vec k), \, \l \in \R$.
            \item Find the length of projection of $4\vec i - 5 \vec j + 6 \vec k$ onto the line with equation $\dfrac{x+5}4 = \dfrac{y-5}3 = 10-2z$.
            \item Find the projection of $4\vec i - 5 \vec j + 6 \vec k$ onto the line with equation $\dfrac{x+5}4 = \dfrac{y-5}3 = 10-2z$.
        \end{enumerate}

    \solution
        \part
            Let $\oa{OP} = \cveciii123$ and $\oa{OA} = \cveciii324$. We have that $A$ is on the line with equation

            \[
                \vec r = \cveciii324 + \l\cveciii122,\, \l \in \R
            \]

            Note that $\oa{AP} = \oa{OP} - \oa{OA} = \cveciii123 - \cveciii324 = \cveciii{-2}0{-1}$.

            \begin{align*}
                \text{Shortest distance} &= \abs{\cveciii{-2}0{-1} \crossp \cveciii122} \Bigg/ \abs{\cveciii122}\\
                &= \dfrac1{\sqrt{1^2 + 2^2 + 2^2}} \abs{\cveciii2{-3}{-4}}\\
                &= \dfrac{\sqrt{2^2 + (-3)^2 + (-4)^2}}3\\
                &= \dfrac{\sqrt{29}}3
            \end{align*}

            \boxt{The shortest distance is $\dfrac{\sqrt{29}}3$ units.}

        \part
            Note that the line has vector form
            \begin{align*}
                \vec r &= \cveciii{-5}55 + \l \cveciii43{-1/2}\\
                &= \cveciii{-5}55 + \l \cveciii86{-1}, \, \l \in \R
            \end{align*}

            \begin{align*}
                \text{Length of projection} &= \abs{\cveciii4{-5}6 \dotp \cveciii86{-1}} \Bigg/ \abs{\cveciii86{-1}}\\
                &= \dfrac{4}{\sqrt{101}}
            \end{align*}

            \boxt{The length of projection is $\dfrac4{\sqrt{101}}$ units.}

        \part
            \begin{align*}
                \text{Projection} &= \bs{\cveciii4{-5}6 \dotp \cveciii86{-1} \Bigg/ \abs{\cveciii86{-1}}} \dotp \bs{\cveciii86{-1} \Bigg/ \abs{\cveciii86{-1}}}\\
                &= \dfrac{-4}{101} \cveciii86{-1}
            \end{align*}

    \problem{}
        The points $P$ and $Q$ have coordinates $(0, -1, -1)$ and $(3, 0, 1)$ respectively, and the equations of the lines $l_1$ and $l_2$ are given by

        \[
            l_1 : \vec r = \cveciii01{-3} + \l\cveciii01{-1}, \, \l \in \R \text{ and } l_2 : \vec r = \cveciii{-3}31 + \m \cveciii2{-1}0, \, \m \in \R 
        \]

        \begin{enumerate}
            \item Show that $P$ lies on $l_1$ but not on $l_2$.
            \item Determine if $l_2$ passes through $Q$.
            \item Find the coordinates of the foot of the perpendicular from $P$ to $l_2$. Hence, or otherwise, find the perpendicular distance from $P$ to $l_2$.
            \item Find the length of projection of $\oa{PQ}$ onto $l_2$.
        \end{enumerate}

    \solution
        We have that $\oa{OP} = \cveciii0{-1}{-1}$ and $\oa{OQ} = \cveciii301$.

        \part
            When $\l = -2$, we have $\cveciii01{-3} - 2\cveciii01{-1} = \cveciii0{-1}{-1} = \oa{OP}$. Hence, $P$ lies on $l_1$.

            Observe that all points on $l_2$ have a $z$-coordinate of 1. Since $P$ has a $z$-coordinate of $-1$, $P$ does not lie on $l_2$.

        \part
            When $\m = 3$, we have $\cveciii{-3}31 + 3\cveciii2{-1}0 = \cveciii301 = \oa{OQ}$. Hence, $l_2$ passes through $Q$.

            \boxt{$l_2$ passes through $Q$.}

        \part
            Let the foot of the perpendicular from $P$ to $l_2$ be $F$. Since $F$ is on $l_2$, we have that $\oa{OF} = \cveciii{-3}31 + \m\cveciii2{-1}0$ for some real $\m$. We also have that $\oa{PF} \cdot \cveciii2{-1}0 = 0$.

            \begin{alignat*}{2}
                &&\oa{PF} \dotp \cveciii2{-1}0 &= 0\\
                \implies&&\bp{\oa{OF} - \oa{OP}} \dotp \cveciii2{-1}0 &= 0\\
                \implies&&\bs{\cveciii{-3}31 + \m\cveciii2{-1}0 - \cveciii0{-1}{-1}} \dotp \cveciii2{-1}0 &= 0\\
                \implies&&\bs{\cveciii{-3}42 + \m\cveciii2{-1}0} \dotp \cveciii2{-1}0 &= 0\\
                \implies&&\cveciii{-3}42 \dotp \cveciii2{-1}0 + \m\cveciii2{-1}0 \dotp \cveciii2{-1}0 &= 0\\
                \implies&&-10 + 5m &= 0\\
                \implies&&m &= 2
            \end{alignat*}

            Hence, $\oa{OF} = \cveciii{-3}31 + 2\cveciii3{-1}0 = \cveciii111$.

            \boxt{$F(1, 1, 1)$}
            \begin{align*}
                \text{Perpendicular distance} &= \abs{\oa{PF}}\\
                &= \abs{\oa{OF} - \oa{OP}}\\
                &= \abs{\cveciii111 - \cveciii0{-1}{-1}}\\
                &= \abs{\cveciii122}\\
                &= \sqrt{1^2 + 2^2 + 2^2}\\
                &= 3
            \end{align*}

            \boxt{The perpendicular distance from $P$ to $l_2$ is 3 units.}

        \part
            Note that $\oa{PQ} = \oa{OQ} - \oa{OP} = \cveciii301 - \cveciii0{-1}{-1} = \cveciii312$.

            {\allowdisplaybreaks
            \begin{align*}
                \text{Length of projection} &= \abs{\cveciii312 \dotp \cveciii2{-1}0} \Bigg/ \abs{\cveciii2{-1}0}\\
                &= \dfrac{\abs{6 - 1 + 0}}{\sqrt{2^2 + (-1)^2 + 0^2}}\\
                &= \dfrac5{\sqrt 5}\\
                &= \sqrt{5}
            \end{align*}}

            \boxt{The length of projection of $\oa{PQ}$ onto $l_2$ is $\sqrt5$ units.}

    \problem{}
        The lines $l_1$ and $l_2$ have equations

        \[
            \vec r = \cveciii012 + s\cveciii103 \text{ and } \vec r = \cveciii{-2}31 + t\cveciii210
        \]
        respectively. Find the position vectors of the points $P$ on $l_1$ and $Q$ on $l_2$ such that $O$, $P$ and $Q$ are collinear, where $O$ is the origin.

    \solution
        We have that $\oa{OP} = \cveciii012 + s\cveciii103$ and $\oa{OQ} = \cveciii{-2}31 + t\cveciii210$ for some reals $s$ and $t$. For $O$, $P$ and $Q$ to be collinear, we need $\oa{OP} = \l \oa{OQ}$ for some real $\l$.
        \begin{alignat*}{2}
            &&\cveciii012 + s\cveciii103 &= \l\bs{\cveciii{-2}31 + t\cveciii210}\\
            \implies&&\cveciii{s}1{2 + 3s} &= \l\cveciii{-2+2t}{3+t}1
        \end{alignat*}
        This gives use the system:
        \[
            \begin{cases}
                \begin{aligned}
                    s &= \l (-2 + 2t)\\
                    1 &= \l(3 + t)\\
                    2 + 3s &= \l 
                \end{aligned}
            \end{cases}
        \]
        Substituting the third equation into the first two gives the reduced system:
        \[
            \begin{cases}
                \begin{aligned}
                    s &= (2+3s)(-2+2t)\\
                    1 &= (2+3s)(3+t)
                \end{aligned}
            \end{cases}
        \]
        Subtracting twice of the second equation from the first yields
        \begin{alignat*}{2}
            &&s - 2 &= (2+3s)(-2+2t) - 2(2+3s)(3+t)\\
            && &= (2+3s)(-2+2t) - (2+3s)(6+2t)\\
            && &= (2+3s)(-2+2t - (6+2t))\\
            && &= -8(2+3s)\\
            && &= -16 - 24s\\
            \implies&&25s &= -14\\
            \implies&&s &= -\dfrac{14}{25}
        \end{alignat*}
        It quickly follows that $t = \dfrac18$. Hence,
        \begin{align*}
            \oa{OP} &= \cveciii012 - \dfrac{14}{25}\cveciii103\\
            &= \dfrac1{25}\cveciii0{25}{50} - \dfrac1{25}\cveciii{14}0{42}\\
            &= \dfrac1{25}\cveciii{-14}{25}{8}\\
            \oa{OQ} &= \cveciii{-2}31 + \dfrac18\cveciii210\\
            &= \dfrac18 \cveciii{-16}{24}{8} + \dfrac18 \cveciii210\\
            &= \dfrac18 \cveciii{-14}{25}8
        \end{align*}

        \boxt{$\oa{OP} = \dfrac1{25}\cveciii{-14}{25}{8}$, $\oa{OQ} = \dfrac18 \cveciii{-14}{25}8$}
        
    \problem{}
        Relative to the origin $O$, the points $A$, $B$ and $C$ have position vectors $5\vec i + 4\vec j + 10\vec k$, $-4\vec i + 4\vec j - 2\vec k$ and $-5\vec i + 9\vec j + 5\vec k$ respectively.

        \begin{enumerate}
            \item Find the Cartesian equation of the line $AB$.
            \item Find the length of projection of $\oa{AC}$ onto the line $AB$. Hence, find the perpendicular distance from $C$ to the line $AB$.
            \item Find the position vector of the foot $N$ of the perpendicular from $C$ to the line $AB$.
            \item The point $D$ is such that it is a reflection of point $C$ about the line $AB$. Find the position vector of $D$.
        \end{enumerate}

    \solution
        We have that $\oa{OA} = \cveciii54{10}$, $\oa{OB} = \cveciii{-4}4{-2}$ and $\oa{OC} = \cveciii{-5}95$.

        \part
            Note that $\oa{AB} = \oa{OB} - \oa{OA} = \cveciii{-4}4{-2} - \cveciii54{10} = \cveciii{-9}0{-12} = -3\cveciii304$. The line $AB$ hence has the vector form
            \[
                \vec r = \cveciii54{10} + \l\cveciii304, \, \l \in \R
            \]
            The line $AB$ thus has the Cartesian form

            \boxt{$\dfrac{x-5}{3} = \dfrac{z-10}4,\,y=4$}

        \part
            Note that $\oa{AC} = \oa{OC} - \oa{OA} = \cveciii{-5}95 - \cveciii54{10} = \cveciii{-10}5{-5} = -5\cveciii2{-1}1$.
            \begin{align*}
                \text{Length of projection} &= \abs{\oa{AC} \dotp \oa{AB}} \Big/ \abs{\oa{AB}}\\
                &= \dfrac1{-3\sqrt{3^2 + 0^2 + 4^2}} \abs{-5\cveciii2{-1}1 \dotp -3\cveciii304}\\
                &= 10
            \end{align*}

            \boxt{The perpendicular distance from $C$ to the line $AB$ is 10 units.}

        \part
            Let $\oa{AN} = \l \cveciii{-9}0{-12}$ for some real $\l$ such that $\abs{\oa{AN}} = 10$.
            \begin{alignat*}{2}
                &&\oa{AN} &= 10\\
                \implies&&\l \cdot -3\sqrt{3^2 + 0^2 + 4^2} &= 10\\
                \implies&&\l &= \dfrac23
            \end{alignat*}
            Hence, $\oa{AN} = \dfrac23 \cveciii{-9}0{-12} = \cveciii{-6}0{-8}$. Thus, $\oa{ON} = \oa{OA} + \oa{AN} = \cveciii54{10}+\cveciii{-6}0{-8} = \cveciii{-1}42$.

            \boxt{$\oa{ON} = \cveciii{-1}42$}

        \part
            Note that $\oa{NC} = \oa{OC} - \oa{ON} = \cveciii{-5}95 - \cveciii{-1}42 = \cveciii{-4}53$. Since $D$ is the reflection of $C$ about $AB$, we have that $\oa{ND} = -\oa{NC}$.

            \begin{align*}
                \oa{OD} &= \oa{ON} + \oa{ND}\\
                &= \oa{ON} - \oa{NC}\\
                &= \cveciii{-1}42 - \cveciii{-4}53\\
                &= \cveciii3{-1}{-1}
            \end{align*}

            \boxt{$\oa{OD} = \cveciii3{-1}{-1}$}

    \problem{}
        The points $A$ and $B$ have coordinates $(0, 9, c)$ and $(d, 5, -2)$ respectively, where $c$ and $d$ are constants. The line $l$ has equation $\dfrac{x+3}{-1} = \dfrac{y-1}4 = \dfrac{z-5}3$.

        \begin{enumerate}
            \item Given that $d = \dfrac{22}7$ and the line $AB$ intersects $l$, find the value of $c$. Find also the coordinates of the foot of the perpendicular from $A$ to $l$.
            \item Given instead that the lines $AB$ and $l$ are parallel, state the value of $c$ and $d$ and find the shortest distance between the lines $AB$ and $l$.
        \end{enumerate}

    \solution
        We have that $\oa{OA} = \cveciii09c$ and $\oa{OB} = \cveciii{d}5{-2}$. We also have that $l$ is given by the vector $\vec r = \cveciii{-3}15 + \l \cveciii{-1}43$ for $\l \in \R$.

        Note that $\oa{AB} = \oa{OB} - \oa{OA} = \cveciii{d}{-4}{-2-c}$. Hence, the line $AB$ is given by the vector $\vec r_{AB} = \cveciii{d}5{-2} + \m \cveciii{d}{-4}{-2-c}$ for $\m \in \R$.

        \part
            Consider the direction vectors of $AB$ and $l$. Since $\cveciii{22/7}{-4}{-2-c} \neq \l\cveciii{-1}43$ for all real $\l$ and $c$, the lines $AB$ and $l$ are not parallel. Hence, $AB$ and $l$ intersect at only one point. Thus, there must be a unique solution to $\vec r = \vec r_{AB}$.
            \begin{alignat*}{2}
                &&\vec r &= \vec r_{AB}\\
                \implies&&\cveciii{-3}15 + \l \cveciii{-1}43 &= \cveciii{22/7}5{-2} + \m \cveciii{22/7}{-4}{-2-c}\\
                \implies&&\cveciii{-21}{7}{35} + \l \cveciii{-1}43 &= \cveciii{22}{35}{-14} + \m \cveciii{22}{-28}{-14-7c}\\
                \implies&&\l \cveciii{-1}43 - \m\cveciii{22}{-28}{-14-7c} &= \cveciii{43}{28}{-49}
            \end{alignat*}
            This gives the following system:
            \[
                \systeme[\l\m]{
                    {-}\l - 22\m = 43,
                    4\l + 28\m = 28,
                    3\l + \bp{14 + 7c}\m = -49
                }
            \]

            Solving the first two equations gives $\l = \dfrac{91}3$ and $\m = -\dfrac{10}3$. It follows from the third equation that $c = 4$.

            \boxt{$c = 4$}

            Let $F$ be the foot of the perpendicular from $A$ to $l$. We have that $\oa{OF} = \cveciii{-3}15 + \l\cveciii{-1}43$ for some real $\l$. We also have that $\oa{AF} \dotp \cveciii{-1}43 = 0$.
            \begin{alignat*}{2}
                &&\oa{AF} \dotp \cveciii{-1}43 &= 0\\
                \implies&&\bp{\oa{OF} - \oa{OA}} \dotp \cveciii{-1}43 &= 0\\
                \implies&&\bs{\cveciii{-3}15 + \l\cveciii{-1}43 - \cveciii094} \dotp \cveciii{-1}43 &= 0\\
                \implies&&\bs{\cveciii{-3}{-8}1 + \l\cveciii{-1}43} \dotp \cveciii{-1}43 &= 0\\
                \implies&&\cveciii{-3}{-8}1 \dotp \cveciii{-1}43 + \l\cveciii{-1}43 \dotp \cveciii{-1}43 &= 0\\
                \implies&&-26 + 26\l&= 0\\
                \implies&&\l &= 1
            \end{alignat*}

            Hence, $\oa{OF} = \cveciii{-3}15 + \l\cveciii{-1}43 = \cveciii{-4}58$.

            \boxt{The foot of the perpendicular from $A$ to $l$ has coordinates $(-4, 5, 8)$.}

        \part
            Given that $AB$ is parallel to $l$, one of their direction vectors must be a scalar multiple of the other. Hence, for some real $\l$,
            \[
                \cveciii{-1}4{3} = \l\cveciii{d}{-4}{-2-c}
            \]
            It is obvious that $\l = -1$, whence $c = 1$ and $d = 1$.

            \boxt{$c = 1$, $d = 1$}

            Note that the direction vector of $l$ and $AB$ is $\cveciii{-1}43$. Also note that $(-3, 1, 5)$ is on $l$ and $(1, 5, -2)$ is on $AB$.

            \begin{align*}
                \text{Shortest distance between $AB$ and $l$} &= \abs{\cveciii{-1}43 \crossp \bp{\cveciii15{-2} - \cveciii{-3}15}} \Bigg/ \abs{\cveciii{-1}43}\\
                &= \dfrac{1}{\sqrt{(-1)^2 + 4^2 + 3^2}} \abs{\cveciii{-1}43 \crossp \cveciii44{-7}}\\
                &= \dfrac1{\sqrt{26}} \abs{\cveciii{-40}{-5}{-20}}\\
                &= \dfrac1{\sqrt{26}} \abs{-5\cveciii814}\\
                &= \dfrac{5 \sqrt{8^2 + 1^2 + 4^2}}{\sqrt{26}}\\
                &= \dfrac{45}{\sqrt{26}}
            \end{align*}

            \boxt{The shortest distance between $AB$ and $l$ is $\dfrac{45}{\sqrt{26}}$ units.}
    
    \problem{}
        The equation of the line $L$ is $\vec r = \cveciii137 + t\cveciii2{-1}5$, $t \in \R$. The points $A$ and $B$ have position vectors $\cveciii93{26}$ and $\cveciii{13}9\a$ respectively. The line $L$ intersects the line through $A$ and $B$ at $P$.

        \begin{enumerate}
            \item Find $\a$ and the acute angle between line $L$ and $AB$.
        \end{enumerate}

         The point $C$ has position vector $\cveciii251$ and the foot of the perpendicular from $C$ to $L$ is $Q$.

        \begin{enumerate}
            \setcounter{enumi}{1}
            \item Find the position vector of $Q$. Hence, find the shortest distance from $C$ to $L$.
            \item Find the position vector of the point of reflection of the point $C$ about the line $L$. Hence, find the reflection of the line passing through $C$ and the point $(1, 3, 7)$ about the line $L$.
        \end{enumerate}

    \solution
        \part
            We have that $\oa{OA} = \cveciii93{26}$ and $\oa{OB} = \cveciii{13}9\a$. Hence, $\oa{AB} = \oa{OB} - \oa{OA} = \cveciii46{\a - 26}$. The line $AB$ is thus given by $\vec r_{AB} = \cveciii93{26} + u\cveciii46{\a - 26}$ for $u \in \R$. Note that $AB$ is not parallel to $L$. Hence, $\oa{OP}$ is the only solution to the equation $\vec r = \vec r_{AB}$.
            \begin{alignat*}{2}
                &&\cveciii137 + t\cveciii2{-1}5 &= \cveciii93{26} + u\cveciii46{\a - 26}\\
                \implies&&t\cveciii2{-1}5 - u\cveciii46{\a - 26} &= \cveciii80{19}
            \end{alignat*}
            This gives the following system:
            \[
                \systeme[tu]{
                    2t - 4u = 8,
                    -t - 6u = 0,
                    5t - \bp{\a - 26}u = 19
                }
            \]
            Solving the first two equations gives $t = 3$ and $u = -\dfrac12$. It follows from the third equation that $\a = 34$.

            \boxt{$\a = 34$}

            Let the acute angle between $L$ and $AB$ be $\t$.
            \begin{alignat*}{2}
                &&\cos\t &= \dfrac{\abs{\cveciii2{-1}5 \dotp \cveciii468}}{\abs{\cveciii2{-1}5} \abs{\cveciii468}}\\
                && &= \dfrac{42}{\sqrt{30} \sqrt{116}}\\
                \implies&&\t &= \arccos \dfrac{42}{\sqrt{30} \sqrt{116}}\\
                && &= 44.6\deg \todp{1}
            \end{alignat*}

            \boxt{$\t = 44.6\deg \todp{1}$}
        
        \part
            Since $Q$ is on $L$, we have that $\oa{OQ} = \cveciii137 + t\cveciii2{-1}5$ for some real $t$. Further, since $\oa{CQ} \perp L$, we have that $\oa{CQ} \dotp \cveciii2{-1}5 = 0$.
            \begin{alignat*}{2}
                &&\oa{CQ} \dotp \cveciii2{-1}5 &= 0\\
                \implies&&\bp{\oa{OQ} - \oa{OC}} \dotp \cveciii2{-1}5 &= 0\\
                \implies&&\bs{\cveciii137 + t\cveciii2{-1}5 - \cveciii251} \dotp \cveciii2{-1}5 &= 0\\
                \implies&&\bs{\cveciii{-1}{-2}6 + t\cveciii2{-1}5} \dotp \cveciii2{-1}5 &= 0\\
                \implies&&\cveciii{-1}{-2}6 \dotp \cveciii2{-1}5 + t\cveciii2{-1}5 \dotp \cveciii2{-1}5 &= 0\\
                \implies&&30 + 30t &= 0\\
                \implies&&t &= 1
            \end{alignat*}
            Hence, $\oa{OQ} = \cveciii137 + \cveciii2{-1}5 = \cveciii{-1}42$.

            \boxt{$\oa{OQ} = \cveciii{-1}42$}

            \begin{align*}
                \text{Shortest distance from $C$ to $L$} &= \abs{\oa{CQ}}\\
                &= \abs{\cveciii{-1}42 - \cveciii251}\\
                &= \abs{\cveciii{-3}{-1}1}\\
                &= \sqrt{(-3)^2 + (-1)^2 + 1^2}\\
                &= \sqrt{11}
            \end{align*}

            \boxt{The shortest distance from $C$ to $L$ is $\sqrt{11}$ units.}

        \part
            Let $C'$ be the reflection of $C$ about $L$.
            \begin{align*}
                \oa{OC'} &= \oa{OQ} - \oa{QC}\\
                &= \oa{OQ} + \oa{CQ}\\
                &= \cveciii{-1}42 + \cveciii{-3}{-1}1\\
                &= \cveciii{-4}33
            \end{align*}

            \boxt{$\oa{OC'} = \cveciii{-4}33$}

            Note that $(1, 3, 7)$ is on $L$ and is hence invariant under a reflection about $L$. Let the reflection about $L$ of the line passing through $C$ and $(1, 3, 7)$ be $L'$. Since $\cveciii{-4}33 - \cveciii137 = \cveciii{-5}0{-4} = -\cveciii504$, $L'$ has direction vector $\cveciii504$. Thus, $L'$ is given by $\vec r' = \cveciii137 + \l \cveciii504$ for $\l \in \R$.

            \boxt{$L' : \vec r' = \cveciii137 + \l \cveciii504, \, \l \in \R$}

    \problem{}
        \begin{center}
            \begin{tikzpicture}
                \draw (0, 0) -- (6, 0);
                \draw (0, 0) -- (4, 3);
                \draw (6, 0) -- (10, 3);
                \draw (4, 3) -- (10, 3);

                \draw (0, 0) -- (5, 6);
                \draw (6, 0) -- (5, 6);
                \draw (4, 3) -- (5, 6);
                \draw (10, 3) -- (5, 6);

                \node[anchor=north east] at (0, 0) {$A$};
                \node[anchor=north west] at (6, 0) {$B$};
                \node[anchor=west] at (10, 3) {$C$};
                \node[anchor=south east] at (4, 3) {$D$};
                \node[anchor=south] at (5, 6) {$V$};

                \node[anchor=north east] at (5, 1.5) {$O$};
                \draw[very thick, ->] (5, 1.5) -- (6, 1.5);
                \draw[very thick, ->] (5, 1.5) -- (5.8, 2.1);
                \draw[very thick, ->] (5, 1.5) -- (5, 2.5);

                \node[anchor=west] at (6, 1.5) {$\vec i$};
                \node[anchor=south west] at (5.8, 2.1) {$\vec j$};
                \node[anchor=south] at (5, 2.5) {$\vec k$};

                \draw[dotted] (5, 1.5) -- (5, 6);
            \end{tikzpicture}
        \end{center}
        In the diagram, $O$ is the origin of the square base $ABCD$ of a right pyramid with vertex $V$. The perpendicular unit vectors $\vec i$, $\vec j$ and $\vec k$ are parallel to $AB$, $AD$ and $OV$ respectively. The length of $AB$ is 4 units and the length of $OV$ is $2h$ units. $P$, $Q$, $M$ and $N$ are the mid-points of $AB$, $BC$, $CV$ and $VA$ respectively. The point $O$ is taken as the origin for position vectors.

        Show that the equation of the line $PM$ may be expressed as $\vec r = \cveciii0{-2}0 + t\cveciii13h$, where $t$ is a parameter.

        \begin{enumerate}
            \item Find an equation for the line $QN$.
            \item Show that the lines $PM$ and $QN$ intersect and that the position vector $\oa{OX}$ of their point of intersection is $\vec r = \dfrac12 \cveciii1{-1}h$.
            \item Given that $OX$ is perpendicular to $VB$, find the value of $h$ and calculate the acute angle between $PM$ and $QN$, giving your answer correct to the nearest $0.1\deg$.
        \end{enumerate}

    \solution
        We are given that $\oa{OP} = \cveciii0{-2}0$, $\oa{OC} = \cveciii220$ and $\oa{OV} = \cveciii00{2h}$. Hence, $\oa{CV} = \oa{OV} - \oa{OC} = \cveciii{-2}{-2}{2h}$. Thus, $\oa{CM} = \dfrac12 \oa{CV} = \cveciii{-1}{-1}h$. Since $\oa{OM} = \oa{OC} + \oa{CM} = \cveciii11h$, we have that $\oa{PM} = \oa{OM} - \oa{OP} = \cveciii13h$. Thus, $PM$ is given by

        \[
            \vec r = \cveciii0{-2}0 + t\cveciii13h, \, t \in \R
        \]

        \part
            Since $\oa{OM} = \cveciii11h$, by symmetry, $\oa{ON} = \cveciii{-1}{-1}h$. Given that $\oa{OQ} = \cveciii200$, we have that $\oa{QN} = \oa{ON} - \oa{OQ} = \cveciii{-3}{-1}h$. Thus, $QN$ is given by

            \boxt{$\vec r = \cveciii200 + u\cveciii{-3}{-1}h, \, u \in \R$}

        \part
            Consider $PM = QN$.
            \begin{alignat*}{2}
                &&PM &= QN\\
                \implies&&\cveciii0{-2}0 + t\cveciii13h &= \cveciii200 + u\cveciii{-3}{-1}h\\
                \implies&&t\cveciii13h - u\cveciii{-3}{-1}h &= \cveciii220
            \end{alignat*}
            This gives the following system:
            \[
                \systeme[tu]{
                    t + 3u = 2,
                    3t + u = 2,
                    ht - hu = 0
                }
            \]
            From the first two equations, we see that $t = \dfrac12$ and $u = \dfrac12$, which is consistent with the third equation. Hence, $\oa{OX} = \cveciii0{-2}0 + \dfrac12 \cveciii13h = \dfrac12 \cveciii1{-1}h$.

        \part
            Note that $\oa{OB} = \cveciii2{-2}6$, whence $\oa{VB} = \oa{OB} - \oa{OV} = \cveciii2{-2}{-2h}$. Since $OX$ is perpendicular to $VB$, we have that $\oa{OX} \dotp \oa{VB} = 0$.
            \begin{alignat*}{2}
                &&\oa{OX} \dotp \oa{VB} &= 0\\
                \implies&&\dfrac12 \cveciii1{-1}h \dotp 2 \cveciii1{-1}{-h} &= 0\\
                \implies&&1 + 1 - h^2 &= 0\\
                \implies&&h^2 &= 2
            \end{alignat*}
            We hence have that $h = \sqrt2$. Note that we reject $h = -\sqrt2$ since $h > 0$.

            \boxt{$h = \sqrt2$}

            Let the acute angle between $PM$ and $QN$ be $\t$.
            \begin{alignat*}{2}
                &&\cos\t &= \dfrac{\abs{\oa{PM} \dotp \oa{QN}}}{\abs{\oa{PM}}\abs{\oa{QN}}}\\
                && &= \dfrac{1}{\sqrt{1^2 + 3^2 + \sqrt{2}^2}} \cdot \frac1{\sqrt{(-3)^2 + (-1)^2 + \sqrt{2}^2}} \cdot \abs{\cveciii13{\sqrt2} \dotp \cveciii{-3}{-1}{\sqrt2}}\\
                && &= \dfrac1{\sqrt{12}}\cdot\dfrac1{\sqrt{12}}\cdot\abs{-3 - 3 + 2}\\
                && &= \dfrac13\\
                \implies&&\t &= \arccos \dfrac13\\
                && &= 70.5\deg \todp{1}
            \end{alignat*}

            \boxt{$\t = 70.5\deg \todp{1}$}
\end{document}