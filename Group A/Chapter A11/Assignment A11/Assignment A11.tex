\documentclass{echw}

\title{Assignment A11\\Permutations and Combinations}
\author{Eytan Chong}
\date{2024-08-12}

\begin{document}
    \problem{}
        Find the number of different arrangements of seven letters in the word ADVANCE. Find the number of these arrangements which begin and end with ``A'' and in which ``C'' and ``D'' are always together.

        Find the number of 4-letter code words that can be made from the letters of the word ADVANCE, using
        \begin{enumerate}
            \item neither of the ``A''s,
            \item both of the ``A''s.
        \end{enumerate}

    \solution
        Tally of letters: 2 ``A''s, 1 ``D'', 1 ``V'', 1 ``N'', 1 ``C'', 1 ``E'' (7 total, 6 distinct)
        \begin{align*}
            \text{Number of different arrangements} &= \dfrac{7!}{2!}\\
            &= \boxed{2520}
        \end{align*}

        Since both ``A''s are at the extreme ends, we are effectively finding the number of arrangements of the word ``DVNCE'' such that ``C'' and ``D'' are always together.

        Let ``C'' and ``D'' be one unit. Altogether, there are 4 units. 
        \begin{align*}
            \text{Required number of arrangements} &= 4! \cdot 2\\
            &= \boxed{48}
        \end{align*}

        \part
            Without both ``A''s, there are only 5 available letters to form the code words. This gives $\comb54$ ways to select the 4 letters of the code word. Since each of the 5 remaining letters are distinct, there are $4!$ possible ways to arrange each word. This gives $\comb54 \cdot 4! = \boxed{120}$ such code words.

        \part
            With both ``A''s included, we need another 2 letters from the 5 non-``A'' letters. This gives $\comb52$ ways to select the 4 letters of the code word. Since the 2 non-``A'' letters are distinct, but the ``A''s are repeated, there are $\dfrac{4!}{2!}$ possible ways to arrange each code word. This gives $\comb52 \cdot \dfrac{4!}{2!} = \boxed{120}$ such code words.

    \problem{}
        A box contains 8 balls, of which 3 are identical (and so are indistinguishable from one anoter) and the other 5 are different from each other. 3 balls are to be picked out of the box; the order in which they are picked out does not matter. Find the number of different possible selections of 3 balls.

    \solution
        Note that there are 6 distinct balls in the box.

        \case{1}{No indentical balls chosen.} $\text{No. of selections} = \comb63$

        \case{2}{2 identical balls chosen.} $\text{No. of selections} = \comb51$

        \case{3}{3 identical balls chosen.} $\text{No. of selections} = \comb33$

        Hence, the total number of selections is given by $\comb63 + \comb51 + \comb33 = \boxed{26}$.

    \problem{}
        The management board of a company conists of 6 men and 4 women. A chairperson, a secretary and a treasurer are chosen from the 10 members of the board. Find the number of ways the chairperson, the secretary and the treasurer can be chosen so that
        \begin{enumerate}
            \item they are all women,
            \item at least one is a woman and at least one is a man.
        \end{enumerate}

        The 10 members of the board sit at random around a round table. Find the number of ways that
        \begin{enumerate}
            \setcounter{enumi}{2}
            \item the chairperson, the secretary and the treasurer sit in three adjacent places.
            \item the chairperson, the secretary and the treasurer are all separated from each other by at least one other person.
        \end{enumerate}

        \textbf{Extension.} What if the seats around the table are numbered? Try parts (c) and (d) again.

    \solution
        \part
            Since there are 4 women and 3 distinct roles, the required number of ways is given by $\perm43 = \boxed{24}$.

        \part
            Note that the number of ways that all three positions are men is given by $\perm63$, while the number of ways to choose without restriction is given by $\perm{10}3$. Hence, the required number of ways is given by $\perm{10}3 - \perm63 - 24 = \boxed{576}$.

        \part
            Consider the three positions as one unit. This gives 8 units altogether. There are hence $(8-1)! \cdot 3! = \boxed{30240}$ ways.

        \part
            Seat the seven other people first. There are $(7-1)!$ ways to do so. Then, slot in the three positions in the 7 slots. There are $\comb73 \cdot 3!$ ways to do so. Hence, the required number of ways is given by $(7-1)! \cdot \comb73 \cdot 3! = \boxed{151200}$.

        \dash

        \textbf{Extension.} Since the seats are numbered, the number of ways scales up by the number of seats, i.e. 10. Hence, the number of ways becomes $\boxed{302400}$ and $\boxed{1512000}$.

\end{document}