\documentclass{echw}

\title{Assignment A2\\Numerical Methods of Finding Roots}
\author{Eytan Chong}
\date{2024-03-18}

\begin{document}
    \problem{}
        By considering the graphs of $y = \cos x$ and $y = -\dfrac14 x$, or otherwise, show that the equation $x + 4\cos x = 0$ has one negative root and two positive roots.

        Use linear interpolation, once only, on the interval $[-1.5, 1]$ to find an approximation to the negative root of the equation $x + 4\cos x = 0$ correct to 2 decimal places.

        \begin{center}
            \begin{tikzpicture}[trim axis left, trim axis right]
                \begin{axis}[
                    axis y line=middle,
                    axis x line=middle,
                    xtick = {3.595},
                    xticklabels = {$\a$},
                    ytick = \empty,
                    xlabel = {$x$},
                    ylabel = {$y$},
                    xmin=2.3,
                    legend cell align={left},
                    legend pos=outer north east,
                    after end axis/.code={
                        \path (axis cs:2.3,0) 
                            node [anchor=east] {$O$};
                        }
                    ]

                    \draw[thick,black!70,TwoMarks=0.1] (2.4,0) -- (2.6,0);
                    
                    \addplot[plotRed, domain=2.5:4] {x + 4*cos(\x r)};
        
                    \addlegendentry{$y = x+4\cos x$};
                \end{axis}
            \end{tikzpicture}
        \end{center}

        The diagram shows part of the graph of $y = x + 4\cos x$ near the larger positive root, $\a$, of the equation $x + 4\cos x = 0$. Explain why, when using the Newton-Raphson method to find $\a$, an initial approximation which is smaller than $\a$ may not be satisfactory.

        Use the Newton-Raphson method to find $\a$ correct to 2 significant figures. You should demonstrate that your answer has the required accuracy.

    \solution
        \begin{center}
            \begin{tikzpicture}[trim axis left, trim axis right, scale=0.86]
                \begin{axis}[
                    domain = -2:5,
                    samples = 101,
                    axis y line=middle,
                    axis x line=middle,
                    xtick = \empty,
                    ytick = \empty,
                    xlabel = {$x$},
                    ylabel = {$y$},
                    ymax=1.2,
                    ymin=-1.2,
                    legend cell align={left},
                    legend pos=outer north east,
                    after end axis/.code={
                        \path (axis cs:0,0) 
                            node [anchor=north east] {$O$};
                        }
                    ]
                    \addplot[plotRed, name path=f2] {cos(\x r)};
        
                    \addlegendentry{$y=\cos x$};

                    \addplot[plotBlue, name path=f1] {-0.25 * x};
        
                    \addlegendentry{$y = -\tfrac14 x$};
        
                    \fill [name intersections={of=f1 and f2,by={E1, E2, E3}}] (E1) circle[radius=2.5pt];

                    \fill (E2) circle[radius=2.5pt];

                    \fill (E3) circle[radius=2.5pt];
                \end{axis}
            \end{tikzpicture}
        \end{center}

        Note that $x + 4\cos x = 0 \implies \cos x = -\dfrac14 x$. Plotting the graphs of $y = \cos x$ and $y = -\dfrac14 x$, we see that there is one negative root and two positive roots. Hence, the equation $x + 4\cos x = 0$ has one negative root and two positive roots.

        Let $f(x) = x+4\cos x$. Let $\b$ be the negative root of the equation $f(x) = 0$. Using linear interpolation on the interval $[-1.5. -1]$,
        \begin{align*}
            \b &= \dfrac{-1.5f(-1) - (-1)f(-1.5)}{f(-1)-f(1.5)}\\
            &= -1.24 \todp{2}
        \end{align*}

        \boxt{$\b = -1.24 \todp{2}$}

        There is a minimum at $x=m$ such that $m$ is between the two positive roots. Hence, when using the Newton-Raphson method, an initial approximation which is smaller than $m$ would result in subsequent approximations being further away from the desired root $\a$. Hence, an initial approximation that is smaller than $\a$ may not be satisfactory.

        We know from the above graph that $\a \in \bp{\pi, \dfrac32 \pi}$. Following the above discussion, we pick $\dfrac32 \pi$ as our initial approximation.
        \begin{alignat*}{2}
            && x_1 &= \dfrac32 \pi \\
            \implies&&x_2 &= \nrm{x_1}{f} = 3.7699\\
            \implies&&x_3 &= \nrm{x_2}{f} = 3.6106\\
            \implies&&x_4 &= \nrm{x_3}{f} = 3.5955\\
            \implies&&x_5 &= \nrm{x_4}{f} = 3.5953
        \end{alignat*}

        Since $f(3.55) = -0.1 < 0$ and $f(3.65) = 0.2 > 0$, $\a \in (3.55, 3.65)$. Hence, $\a = 3.6 \tosf{2}$.

        \boxt{$\a = 3.6 \tosf{2}$}

    \problem{}
        Find the coordinates of the stationary points on the graph $y = x^3 + x^2$. Sketch the graph and hence write down the set of values of the constant $k$ for which the equation $x^3 + x^2 = k$ has three distinct real roots.

        The positive root of the equation $x^3 + x^2 = 0.1$ is denoted by $\a$.

        \begin{enumerate}
            \item Find a first approximation to $\a$ by linear interpolation on the interval $0 \leq x \leq 1$.
            \item With the aid of a suitable figure, indicate why, in this case, linear interpolation does not give a good approximation to $\a$.
            \item Find an alternative first approximation to $\a$ by using the fact that if $x$ is small then $x^3$ is negligible when compared to $x^2$.
        \end{enumerate}

    \solution
        For stationary points, $y' = 0$.
        \begin{alignat*}{2}
            &&y' &= 0\\
            \implies&&3x^2 + 2x &= 0\\
            \implies&&x(3x+2) &= 0
        \end{alignat*}

        Hence, $x = 0$ or $x = -\dfrac23$. When $x = 0$, $y = 0$. When $x = -\dfrac23$, $y = \dfrac4{27}$. Thus, the coordinates of the stationary points of $y = x^3 + x^2$ are $(0, 0)$ and $\bp{-\dfrac23, \dfrac4{27}}$.

        \boxt{$(0, 0)$, $\bp{-\dfrac23, \dfrac4{27}}$}

        \begin{center}
            \begin{tikzpicture}[trim axis left, trim axis right, scale=0.86]
                \begin{axis}[
                    domain = -1.1:0.4,
                    samples = 101,
                    axis y line=middle,
                    axis x line=middle,
                    xtick = {-1},
                    ytick = \empty,
                    xlabel = {$x$},
                    ylabel = {$y$},
                    legend cell align={left},
                    legend pos=outer north east,
                    after end axis/.code={
                        \path (axis cs:0,0) 
                            node [anchor=north east] {$O$};
                        }
                    ]
                    \addplot[plotRed] {x^3 + x^2};
        
                    \addlegendentry{$y=x^3 + x^2$};

                    \fill (-2/3, 4/27) circle[radius=2.5pt] node[anchor=south] {$\bp{-\dfrac23, \dfrac4{27}}$};
                \end{axis}
            \end{tikzpicture}
        \end{center}

        Therefore, $k \in \bp{0, \dfrac4{27}}$.

        \boxt{$\bc{k \in \R \colon 0 < k < \dfrac4{27}}$}

        \part
            Let $f(x) = x^2 + x^2 - 0.1$. Using linear interpolation on the interval $[0, 1]$,
            \begin{align*}
                \a &= \dfrac{0f(1) - 1f(0)}{f(1)-f(0)}\\
                &= \dfrac1{20}
            \end{align*}

            \boxt{$\a = \dfrac1{20}$}

        \part
            \begin{center}
                \begin{tikzpicture}[trim axis left, trim axis right, scale=0.86]
                    \begin{axis}[
                        domain = 0:1,
                        samples = 101,
                        axis y line=middle,
                        axis x line=middle,
                        xtick = {0.279},
                        xticklabels={$\a$},
                        ytick = {-0.1},
                        xlabel = {$x$},
                        ylabel = {$y$},
                        ymin=-0.2,
                        legend cell align={left},
                        legend pos=outer north east,
                        after end axis/.code={
                            \path (axis cs:0,0) 
                                node [anchor=south east] {$O$};
                            }
                        ]
                        \addplot[plotRed] {x^3 + x^2 -0.1};
            
                        \addlegendentry{$y=x^3 + x^2-0.1$};

                        \fill (-2/3, 4/27) circle[radius=2.5pt] node[anchor=south] {$\bp{-\dfrac23, \dfrac4{27}}$};

                        \draw[dotted, thick] (0, -0.1) -- (1, 1.9);
                    \end{axis}
                \end{tikzpicture}
            \end{center}

            On the interval $[0, 1]$, the gradient of $y = x^3 + x^2 - 0.1$ changes considerably. Hence, linear interpolation gives an approximation much less than the actual value.

        \part
            For small $x$, $x^3$ is negligible when compared to $x^2$. Consider $g(x) = x^2-0.1$. Then the positive root of $g(x)=0$ is approximately $\a$. Hence, an alternative approximation to $\a$ is $\sqrt{0.1} = 0.316 \tosf{3}$.

            \boxt{$\a = 0.316 \tosf{3}$}

    \problem{}
        The equation $2\cos x - x =0$ has a root $\a$ in the interval $[1, 1.2]$. Iterations of the form $x_{n+1} = F(x_n)$ are based on each of the following rearrangements of the equation:

        \begin{enumerate}
            \item $x = 2\cos x$
            \item $x = \cos x + \dfrac12 x$
            \item $x = \dfrac23 (\cos x + x)$
        \end{enumerate}

         Determine which iteration will converge to $\a$ and illustrate your answer by a `staircase' or `cobweb' diagram. Use the most appropriate iteration with $x_1 = 1$, to find $\a$ to 4 significant figures. You should demonstrate that your answer has the required accuracy.

    \solution
        \part
            Consider $f(x) = 2\cos x$. Then $f'(x) = -2\sin x$.

            Observe that $\sin x$ is increasing on $[1, 1.2]$. Since $\sin 1 > \dfrac12$, $\abs{f'(x)} > 1$ for all $x \in [1, 1.2]$. Thus, fixed-point iteration fails and will not converge to $\a$.

            \begin{center}
                \begin{tikzpicture}[trim axis left, trim axis right, scale=0.86]
                    \begin{axis}[
                        domain = 0:pi/2,
                        samples = 101,
                        axis y line=middle,
                        axis x line=middle,
                        xtick = {1.2},
                        ytick = \empty,
                        xlabel = {$x$},
                        ylabel = {$y$},
                        ymax=2.2,
                        xmax=1.7,
                        legend cell align={left},
                        legend pos=outer north east,
                        after end axis/.code={
                            \path (axis cs:0,0) 
                                node [anchor=east] {$O$};
                            }
                        ]
                        \addplot[plotRed] {2 *cos(\x r)};

                        \addlegendentry{$y=2\cos x$};

                        \addplot[plotBlue] {x};
                        
                        \addlegendentry{$y=x$};
            
                        \draw[dotted, thick] (1.2, 0) -- (1.2, 0.725);

                        \begin{scope}[decoration={
                            markings,
                            mark=at position 0.5 with {\arrow{>}}}
                            ] 

                            \draw[postaction={decorate}] (1.2, 0.725) -- (0.725, 0.725);

                            \draw[postaction={decorate}] (0.725, 0.725) -- (0.725, 1.497);

                            \draw[postaction={decorate}] (0.725, 1.497) -- (1.497, 1.497);

                            \draw[postaction={decorate}] (1.497, 1.497) -- (1.497, 0.147);
                        \end{scope}
                    \end{axis}
                \end{tikzpicture}
            \end{center}

        \part
            Consider $f(x) = \cos x + \dfrac12 x$. Then $f'(x) = -\sin x + \dfrac12 -\bp{\sin x - \dfrac12}$. Since $0 \leq \sin x \leq 1$ for $x \in \bs{0, \dfrac{\pi}2}$, and $[1, 1.2] \subset \bs{0, \dfrac{\pi}2}$, we know $-\dfrac12 \leq \sin x - \dfrac12 \leq \dfrac12$ for $x \in [1, 1.2]$. Thus, $0 \leq \abs{\sin x - \dfrac12} \leq \dfrac12$ for $x \in [1, 1.2]$. Hence, fixed-point iteration will work and converge to $\a$.

            \begin{center}
                \begin{tikzpicture}[trim axis left, trim axis right, scale=0.86]
                    \begin{axis}[
                        domain = 0:2,
                        samples = 101,
                        axis y line=middle,
                        axis x line=middle,
                        xtick = {1.7},
                        xticklabels = {1.2},
                        ytick = \empty,
                        xlabel = {$x$},
                        ylabel = {$y$},
                        ymax=1.4,
                        xmax=2,
                        legend cell align={left},
                        legend pos=outer north east,
                        after end axis/.code={
                            \path (axis cs:0,0) 
                                node [anchor=east] {$O$};
                            }
                        ]
                        \addplot[plotRed] {cos(\x r) + 0.5 * x};

                        \addlegendentry{$y=\cos x + \tfrac12 x$};

                        \addplot[plotBlue] {x};

                        \addlegendentry{$y=x$};

                        \draw[dotted, thick] (1.7, 0) -- (1.7, 0.721);

                        \begin{scope}[decoration={
                            markings,
                            mark=at position 0.5 with {\arrow{>}}}
                            ] 

                            \draw[postaction={decorate}] (1.7, 0.721) -- (0.721, 0.721);

                            \draw[postaction={decorate}] (0.721, 0.721) -- (0.721, 1.112);

                            \draw[postaction={decorate}] (0.721, 1.112) -- (1.112, 1.112);

                            \draw[postaction={decorate}] (1.112, 1.112) -- (1.112, 0.999);
                        \end{scope}
                    \end{axis}
                \end{tikzpicture}
            \end{center}

        \part
            Consider $f(x) = \dfrac23 (\cos x + x)$. Then $f'(x) = \dfrac23(-\sin x + 1)$.

            For fixed-point iteration to converge to $\a$, we need $\abs{f'(x)} < 1$ for $x$ near $\a$. It thus suffices to show that $\abs{-\sin x + 1} < \dfrac32$ for all $x \in [1, 1.2]$. Observe that $1 - \sin x$ is strictly decreasing and positive for $x \in \bs{0, \dfrac{\pi}2}$. Since $1 - \sin 1 < \dfrac32$, and $[1, 1.2] \subset \bs{0, \dfrac{\pi}2}$, we have that $\abs{-\sin x + 1} < \dfrac32$ for all $x \in [1, 1.2]$. Thus, $\abs{f'(x)} < 1$ for $x$ near $\a$. Hence, fixed-point iteration will work and converge to $\a$.

            \begin{center}
                \begin{tikzpicture}[trim axis left, trim axis right, scale=0.86]
                    \begin{axis}[
                        domain = 0:3.3,
                        samples = 101,
                        axis y line=middle,
                        axis x line=middle,
                        xtick = {3},
                        xticklabels = {1.2},
                        ytick = \empty,
                        xlabel = {$x$},
                        ylabel = {$y$},
                        ymax=1.6,
                        legend cell align={left},
                        legend pos=outer north east,
                        after end axis/.code={
                            \path (axis cs:0,0) 
                                node [anchor=east] {$O$};
                            }
                        ]
                        \addplot[plotRed] {2/3 * (cos(\x r) + x)};

                        \addlegendentry{$y=\tfrac23 (\cos x + x)$};

                        \addplot[plotBlue] {x};

                        \addlegendentry{$y=x$};

                        \draw[dotted, thick] (3, 0) -- (3, 1.34);

                        \begin{scope}[decoration={
                            markings,
                            mark=at position 0.5 with {\arrow{>}}}
                            ] 

                            \draw[postaction={decorate}] (3, 1.34) -- (1.34,1.34);

                            \draw[postaction={decorate}] (1.34,1.34) -- (1.34, 1.046);

                            \draw[postaction={decorate}] (1.34, 1.046) -- (1.046, 1.046);
                        \end{scope}
                    \end{axis}
                \end{tikzpicture}
            \end{center}

            For $x \in [1, 1.2]$, $\abs{\dfrac23 (-\sin x + 1)} < \abs{-\sin x + \dfrac12} < 1$. Thus, $x_{n+1} = \dfrac23 (\cos x_n + x_n)$ is the most suitable iteration as it will converge to $\a$ the quickest. Using $F(x) = \dfrac23 (\cos x + x)$ with $x_1 = 1$,
            \begin{alignat*}{2}
                && x_1 &= 1 \\
                \implies&&x_2 &= F(x_1) = 1.02687\\
                \implies&&x_3 &= F(x_2) = 1.02958\\
                \implies&&x_4 &= F(x_3) = 1.02984\\
                \implies&&x_5 &= F(x_4) = 1.02986\\
            \end{alignat*}

            Since $F(1.0295) > 1.0295$ and $F(1.0305) < 1.0305$, $\a \in (1.0295, 1.0305)$. Hence, $\a = 1.030 \tosf{4}$.

            \boxt{$\a = 1.030 \tosf{4}$}
\end{document}