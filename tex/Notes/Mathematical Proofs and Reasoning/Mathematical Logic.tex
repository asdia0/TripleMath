\chapter{Mathematical Logic}

Mathematics is a deductive science, where from a set of basic axioms, we prove more complex results. To do so, we often restate a sentence into \vocab{statements}, which are mathematical expressions. One important axiom that all statements obey is the law of the excluded middle.

\begin{axiom}[Law of the Excluded Middle]
    The \vocab{law of the excluded middle} states that either a statement or its negation is true. Equivalently, a statement cannot be both true and false, nor can it be neither true nor false.
\end{axiom}

\section{Statements}

\subsection{Forming Statements}

We call a sentence such as ``$x$ is even'' that depends on the value of $x$ a ``statement about $x$''. We can denote this statement more compactly as $P(x)$. For instance, $P(5)$ is the statement ``5 is even'', while $P(72)$ is the statement ``72 is even'', and so forth. We can also write $P(x)$ more compactly as $P$.

We now introduce some operations of statements, namely the negation, conjunction and disjunction operations.

\begin{definition}
    The \vocab{negation} of a statement $P$, denoted $\lnot P$, is false when $P$ is true, and true when $P$ is false. In a truth table,
    \begin{table}[H]
        \centering
        \begin{tabular}{|c|c|}
        \hline
        $P$ & $\lnot P$ \\ \hline\hline
        T & F \\ \hline
        F & T \\ \hline
        \end{tabular}
    \end{table}
\end{definition}

\begin{example}
    If $P(x)$ is the statement ``$x$ is even'', then $\lnot P(x)$ is the statement ``$x$ is odd''.
\end{example}

\begin{definition}
    The \vocab{conjunction} of two statements $P$ and $Q$, denoted $P \land Q$, has truth table
    \begin{table}[H]
        \centering
        \begin{tabular}{|c|c|c|}
        \hline
        $P$ & $Q$ & $P \land Q$ \\ \hline\hline
        T & T & T \\ \hline
        T & F & F \\ \hline
        F & T & F \\ \hline
        F & F & F \\ \hline
        \end{tabular}
    \end{table}
\end{definition}

\begin{example}
    If $P$ is the statement ``I like cats'', and $Q$ is the statement ``I like dogs'', then $P \land Q$ is the statement ``I like cats and dogs''.
\end{example}

\begin{definition}
    The \vocab{disjunction} of two statements $P$ and $Q$, denoted $P \lor Q$, has truth table
    \begin{table}[H]
        \centering
        \begin{tabular}{|c|c|c|}
        \hline
        $P$ & $Q$ & $P \lor Q$ \\ \hline\hline
        T & T & T \\ \hline
        T & F & T \\ \hline
        F & T & T \\ \hline
        F & F & F \\ \hline
        \end{tabular}
    \end{table}
\end{definition}

\begin{example}
    If $P$ is the statement ``I like cats'', and $Q$ is the statement ``I like dogs'', then $P \lor Q$ is the statement ``I like cats or dogs or both''.
\end{example}

\begin{proposition}[De Morgan's Law]
    Let $P$ and $Q$ be statements. Then \[\lnot \bp{P \land Q} \quad \iff \quad (\lnot P) \lor (\lnot Q)\] and \[\lnot \bp{P \lor Q} \quad \iff \quad (\lnot P) \land (\lnot Q).\]
\end{proposition}
\begin{proof}
    Consider the following truth tables:

    \begin{table}[H]
        \centering
        \begin{tabular}{|c|c|c|c|c|c|c|c|c|c|}
        \hline
        $P$ & $Q$ & $P \land Q$ & $P \lor Q$ & $\lnot (P \land Q)$ & $\lnot (P \lor Q)$ & $\lnot P$ & $\lnot Q$ & $(\lnot P) \land (\lnot Q)$ & $(\lnot P) \lor (\lnot Q)$ \\ \hline
        T & T & T & T & F & F & F & F & F & F \\ \hline
        T & F & F & T & T & F & F & T & F & T \\ \hline
        F & T & F & T & T & F & T & F & F & T \\ \hline
        F & F & F & F & T & T & T & T & T & T \\ \hline
        \end{tabular}
    \end{table}

    We see that the truth table of $\lnot \bp{P \land Q}$ is equivalent to that of $(\lnot P) \lor (\lnot Q)$, thus the statements are equivalent.
    
    Similarly, the truth table of $\lnot \bp{P \lor Q}$ is equivalent to that of $(\lnot P) \land (\lnot Q)$, thus the statements are equivalent.
\end{proof}

\begin{example}
    Let $P$ be the statement ``I like cats'', and $Q$ be the statement ``I like dogs''. Then $\lnot (P \land Q)$ is ``It is not the case that I like both cats and dogs'', while $(\lnot P) \lor (\lnot Q)$ is ``I do not like cats, or I do not like dogs, or I do not like both''. Clearly, the two statements are equivalent.
\end{example}

\subsection{Conditional and Biconditional Statements}

In this section, we examine how statements are linked together to form more complicated statements. The first type of statement we will examine is the conditional statement.

\begin{definition}
    A \vocab{conditional statement} has the form ``if $P$ then $Q$''. Here, $P$ is the \vocab{hypothesis} and $Q$ is the \vocab{conclusion}, denoted by $P \implies Q$. This statement is defined to have the truth table
    \begin{table}[H]
        \centering
        \begin{tabular}{|c|c|c|}
        \hline
        $P$ & $Q$ & $P \implies Q$ \\ \hline\hline
        T & T & T \\ \hline
        T & F & F \\ \hline
        F & T & T \\ \hline
        F & F & T \\ \hline
        \end{tabular}
    \end{table}
    In words, the statement $P \implies Q$ also reads:
    \begin{itemize}
        \item $P$ \vocab{implies} $Q$.
        \item $P$ is a \vocab{sufficient condition} for $Q$.
        \item $Q$ is a \vocab{necessary condition} for $P$.
        \item $P$ \vocab{only if} $Q$.
    \end{itemize}
\end{definition}

To justify the truth table of $P \implies Q$, consider the following example:
\begin{example}[Conditional Statement]
    Suppose I say \[\text{``If it is raining, then the floor is wet.''}\] We can write this as $P \implies Q$, where $P$ is the statement ``it is raining'' and $Q$ is the statement ``the floor is wet''.

    \begin{itemize}
        \item Suppose both $P$ and $Q$ are true, i.e. it is raining, and the floor is wet. It is reasonable to say that I am telling the truth, whence $P \implies Q$ is true.
        \item Suppose $P$ is true but $Q$ is false, i.e. it is raining, and the floor is not wet. Clearly, I am not telling the truth; the floor would be wet if I was. Hence, $P \implies Q$ is false.
        \item Suppose $P$ is false, i.e. it is not raining. Notice that the hypothesis of my claim is not fulfilled; I did not say anything about the floor when it is not raining. Hence, I am not lying, so $P \implies Q$ is true whenever $P$ is false.
    \end{itemize}
\end{example}

Examples of conditional statements in mathematics include
\begin{itemize}
    \item If $\abs{x - 1} < 4$, then $-3 < x < 5$.
    \item If a function $f$ is differentiable, then $f$ is continuous.
\end{itemize}

We now look at biconditional statements. As the name suggests, a biconditional statement comprises two conditional statements: $P \implies Q$ and $Q \implies P$. The conditional statement is much stronger than the conditional statement.

\begin{definition}
    A \vocab{biconditional statement} has the form ``$P$ if and only if'', denoted $P \iff Q$. This statement is defined to have the truth table
    \begin{table}[H]
        \centering
        \begin{tabular}{|c|c|c|}
        \hline
        $P$ & $Q$ & $P \iff Q$ \\ \hline\hline
        T & T & T \\ \hline
        T & F & F \\ \hline
        F & T & F \\ \hline
        F & F & T \\ \hline
        \end{tabular}
    \end{table}

    When $P \iff Q$ is true, we say that $P$ and $Q$ are \vocab{equivalent}, i.e. $P \equiv Q$.
\end{definition}

An equivalent definition of $P \iff Q$ is the statement \[(P \implies Q) \quad \tand \quad (Q \implies P).\] This allows us to easily justify the truth table of $P \iff Q$:
\begin{table}[H]
    \centering
    \begin{tabular}{|c|c|c|c|c|}
    \hline
    $P$ & $Q$ & $P \implies Q$ & $Q \implies P$ & $P \iff Q$ \\ \hline\hline
    T & T & T & T & T \\ \hline
    T & F & F & T & F \\ \hline
    F & T & T & F & F \\ \hline
    F & F & T & T & T \\ \hline
    \end{tabular}
\end{table}

Examples of conditional statements in mathematics include
\begin{itemize}
    \item A triangle $ABC$ is equilateral if and only if its three angles are congruent.
    \item $a$ is a rational number if and only if $2a + 4$ is rational.
\end{itemize}

\subsection{Quantifiers}

We now introduce two important symbols, namely the universal quantifier ($\forall$) and the existential quantifier ($\exists$)

\begin{definition}
    Let $P(x)$ be a statement about $x$, where $x$ is a member of some set $S$ (i.e. $S$ is the \vocab{domain} of $x$). Then the notation \[\forall x \in S, \, P(x)\] means that $P(x)$ is true for every $x$ in the set $S$. The notation \[\exists x \in S, \, P(x)\] means that there exists at least one element of $x$ of $S$ for which $P(x)$ is true.
\end{definition}

\begin{example}
    Let $P(x)$ be the statement ``$x$ is even''. Clearly, the statement \[\forall x \in \ZZ, \, P(x)\] is not true; not all integers are even. However, the statement \[\exists x \in \ZZ, \, P(x)\] is true, because we can find an integer that is even (e.g. $x = 8$).
\end{example}

Note that a statement $P(x)$ does not necessarily have to mention $x$. For instance, we could define $P(x)$ as the statement ``5 is even''. Compare this with how a function $f(x)$ does not necessarily have to ``mention'' $x$, e.g. we could have $f(x) = 5$.

\begin{proposition}
    The negation of a universal statement is an existential statement, and vice versa. \[\lnot \bp{\forall x \in D, \, P(x)} \quad \iff \quad \exists x \in D, \, \lnot P(x).\]
\end{proposition}
\begin{proof}
    We prove that the negation a universal statement is an existential statement. Observe that a universal statement is equivalent to a conjunction of many statements: \[\forall x \in D, \, P(x) \quad \iff \quad P(x_1) \land P(x_2) \land \dots,\] where $D = \bc{x_1, x_2, \dots}$. Using De Morgan's laws, we can easily negate the above statements: \[\lnot \bp{\forall x \in D, \, P(x)} \quad \iff \lnot P(x_1) \lor \lnot P(x_2) \lor \dots.\] However, the last statement is equivalent to the existential statement \[\exists x \in D, \, \lnot P(x).\] Thus, \[\lnot \bp{\forall x \in D, \, P(x)} \quad \iff \quad \exists x \in D, \, \lnot P(x).\]

    Using a similar argument, one can prove that the negation of an existential statement is a universal statement, i.e. \[\lnot \bp{\exists x \in D, \, P(x)} \quad \iff \quad \forall x \in D, \, \lnot P(x).\]
\end{proof}

\begin{example}
    Let $D$ be the set of all students in a class, and let $P(x)$ be ``$x$ likes durian''. Then the statement $\forall x \in D, \, P(x)$ reads as ``everyone in the class likes durian''. Intuitively, its negation would be ``someone in the class does not like durian'', which we can write as $\exists x \in D, \, \lnot P(x)$.
\end{example}

\subsection{Types of Statements}

Most of the statements we will encounter can be grouped into three classes, namely axioms, definitions and theorems.

\begin{definition}
    \phantom{.}
    \begin{itemize}
        \item An \vocab{axiom} is a mathematical statement that does not require proof.
        \item A \vocab{definition} is a true mathematical statement that gives the precise meaning of a word or phrase that represents some object, property or other concepts.
        \item A \vocab{theorem} is a true mathematical statement that can be proven mathematically.
    \end{itemize}
\end{definition}

\section{Proofs}

Mathematical proofs are convincing arguments expressed in mathematical language, i.e. a sequence of statements leading logically to the conclusion, where each statement is either an accepted truth, or an assumption, or a statement derived from previous statements. Occasionally there will be the clarifying remark, but this is just for the reader and has no logical bearing on the structure of the proof.

\begin{definition}
    A \vocab{proof} is a deductive argument for a mathematical statement, showing that the stated assumptions logically guarantee the conclusion.
\end{definition}

There are three main types of proofs: direct proof, proof by contrapositive and proof by contradiction.

\subsection{Direct Proof}

A \vocab{direct proof} is an approach to prove a conditional statement $P \implies Q$. It is a series of valid arguments that starts with the hypothesis $P$, and ends with the conclusion $Q$.

As an example, we will prove the following statement:

\begin{statement}
    For all $n \in \ZZ^+$, both $n$ and $n^2$ have the same parity.
\end{statement}
\begin{proof}
    Since $n$ can only be either odd or even, we just need to consider the following cases:

    \case{1} Suppose $n$ is even. By definition, there exists some $k \in \ZZ$ such that $n = 2k$. Then \[n^2 = (2k)^2 = 4k^2 = 2\bp{2k^2} = 2a,\] where $a = 2k^2$. Since $a$ is an integer, it follows from our definition that $n^2$ is even. Hence, $n$ and $n^2$ have the same parity.

    \case{2} Suppose $n$ is odd. By definition, there exists some $h \in \ZZ$ such that $n = 2h + 1$. Then \[n^2 = (2h+1)^2 = 4h^2 + 4h + 1 = 2\bp{2h^2 + 2h} + 1 = 2b + 1,\] where $b = 2h^2 + 2h$. Since $b$ is an integer, it follows from our definition that $n^2$ is odd. Hence, $n$ and $n^2$ have the same parity.
\end{proof}

\subsection{Proof by Contrapositive}

Suppose we wish to prove $P \implies Q$. Occasionally, the hypothesis $P$ is more complicated than the conclusion $Q$, which is not desirable. In such a scenario, we can choose to prove the statement via the \vocab{contrapositive}, i.e. prove that $\lnot Q \implies \lnot P$. This typically simplifies the proof, since our hypothesis $\lnot Q$ is now simpler.

We now show the equivalence between $P \implies Q$ and $\lnot Q \implies \lnot P$.

\begin{proposition}
    Let $P$ and $Q$ be statements. Then \[P \implies Q \quad \iff \quad \lnot Q \implies \lnot P.\]
\end{proposition}
\begin{proof}
    Consider the following truth table:

    \begin{table}[H]
        \centering
        \begin{tabular}{|c|c|c|c|c|c|}
        \hline
        $P$ & $Q$ & $P \implies Q$ & $\lnot Q$ & $\lnot P$ & $\lnot Q \implies \lnot P$ \\ \hline
        T & T & T & F & F & T \\ \hline
        T & F & F & T & F & F \\ \hline
        F & T & T & F & T & T \\ \hline
        F & F & T & T & T & T \\ \hline
        \end{tabular}
    \end{table}

    Since $P \implies Q$ and $\lnot Q \implies \lnot P$ have the same truth table, they are equivalent.
\end{proof}

As an example, we will prove the following statement using the contrapositive.

\begin{statement}
    For any real numbers $x$ and $y$, if $x^2 y + x y^2 < 30$, then $x < 2$ or $y < 3$.
\end{statement}
\begin{proof}
    Since the hypothesis is much more complicated than the conclusion, we are motivated to use the contrapositive.

    Suppose $x > 2$ and $y > 3$ (this is the negation of $x < 2$ or $y < 3$). Then $x^2 y > (2)^2 (3) = 12$ and $xy^2 > (2)(3)^2 = 18$. Thus, $x^2 y + xy^2 > 12 + 18 = 30$. (this is the negation of $x^2y + xy^2 < 30$). Thus, by the contrapositive, the statement is true.
\end{proof}

\subsection{Proof by Contradiction}

A \vocab{proof by contradiction} is a proving technique where we want to prove that a statement is true by assuming that it is false, and arrive at a contradiction. That is, to prove a statement $P$, we can 
\renewcommand{\theenumi}{\arabic{enumi}.}
\begin{enumerate}
    \item Assume $\lnot P$.
    \item Derive a contradiction, or absurdity.
    \item Conclude that $\lnot P$ is false, which implies $P$ is true.
\end{enumerate}
\renewcommand{\theenumi}{(\alph{enumi})}

A classic example of a proof by contradiction is the irrationality of $\sqrt 2$.

\begin{statement}
    $\sqrt2$ is irrational.
\end{statement}
\begin{proof}
    Seeking a contradiction, suppose $\sqrt2$ is rational. Write $\sqrt2 = a/b$, where $a$ and $b$ are coprime integers with $b \neq 0$. Squaring, we get \[2 = \frac{a^2}{b^2} \implies a^2 = 2b^2. \tag{1}\] Thus, $a^2$ is even, which implies $a$ is even. Hence, $a = 2k$ for some integer $k$. Substituting this back into (1), we get \[(2k)^2 = 2b^2 \implies b^2 = 2k^2,\] whence $b^2$ is even, which implies $b$ is also even. Thus, both $a$ and $b$ have a factor of 2, contradicting our assumption that $a$ and $b$ are coprime. Thus, our assumption that $\sqrt2$ is rational is false, whence $\sqrt2$ is irrational.
\end{proof}

\subsection{Induction}

Induction is typically used to prove statements of the form ``$P(n)$ is true for all $n \in \ZZ^+$''. There are several variants of induction.

\subsubsection{Principle of Mathematical Induction}

The basic form of mathematical induction requires two steps:

\begin{itemize}
    \item Showing that $P(0)$ is true, and
    \item Proving that $P(k) \implies P(k+1)$ for some $k \in \ZZ^+$.
\end{itemize}

With these two statements, we see that \[P(0) \implies P(1) \implies P(2) \implies P(3) \implies \dots,\] i.e. $P(n)$ is true for all $n \in \ZZ^+$.

Of course, the base case need not always be $n = 0$. If we wish to prove that $P(n)$ holds for $n = m, m+1, m+2, \dots$ for some integer $m$, our base case becomes $n = m$, so we have to verify that $P(m)$ holds.

Intuitively, we can think of induction as a ladder. The base case acts as the first rung, while the statement $P(k) \implies P(k+1)$ enables us to climb the ladder rung by rung.

A classic example of an inductive proof is to verify that the first $n$ natural numbers sum to $n(n+1)/2$.

\begin{statement}
    For $n$ a natural number, $1 + 2 + \dots + n = n(n+1)/2$.
\end{statement}
\begin{proof}
    Let $P(n)$ be the statement $1 + 2 + \dots + n = n(n+1)/2$. We induct on $n$.

    The base case $P(1)$ is trivial, since $1 = (1)(2)/2$. Suppose that $P(k)$ holds for some natural number $k$. Consider the sum of the first $k+1$ natural numbers. By our \vocab{induction hypothesis}, we see that \[1 + 2 + \dots + k + (k+1) = \frac{k(k+1)}{2} + (k + 1) = \frac{(k+1)((k+1) + 1)}{2},\] so $P(k+1)$ also holds. By the principle of mathematical induction, it follows that $P(n)$ holds for all natural numbers $n$.
\end{proof}

\subsubsection{Principle of Strong Induction}

Another common variant of induction is \emph{strong} induction. Like before, it involves showing two steps:

\begin{itemize}
    \item Showing that $P(0)$ is true, and
    \item If $P(0)$, $P(1)$, $\dots$, $P(k)$ are true, then so is $P(k+1)$.
\end{itemize}

Here, the inductive step is replaced with a \emph{stronger} hypothesis that requires all the terms before $P(k+1)$ to be true, as demonstrated in the following example:

\begin{statement}
    All integers greater than 1 are either a prime or a product of primes.
\end{statement}
\begin{proof}
    Let $P(n)$ be the statement ``$n$ is either a prime or a product of primes''. We induct on $n$. The base case $n = 2$ is trivial (2 itself is a prime). Now suppose $P(2)$ to $P(k)$ are true for some integer $k \geq 2$. If $k+1$ is prime, then $P(k+1)$ is trivially true. Else, $k+1$ must be composite, so we can write $k+1 = ab$, for some $2 \leq a, b \leq k$. But by our induction hypothesis, both $a$ and $b$ are either primes or a product of primes, hence $ab$ itself is a product of primes, so $P(k+1)$ is true. This closes the induction.
\end{proof}

We can also use multiple base cases for strong induction:

\begin{itemize}
    \item Showing that the base cases $P(0)$, $P(1)$, $\dots$, $P(m)$ are true, and
    \item Proving that if $P(k)$, $P(k+1)$, $\dots$, $P(k+m)$ are true, then $P(k+m+1)$ is true.
\end{itemize}

\subsubsection{All Horses are the Same Colour}

Caution must be exercised when proving a statement inductively. Consider now the following ``proof'' that purports to show that all horses share the same colour.

\begin{statement}
    All horses are the same colour.
\end{statement}
\begin{proof}
    Let $P(n)$ be the statement ``A group of $n$ horses have the same colour''. We induct on $n$. $P(1)$ is trivial. Suppose that $P(k)$ is true for some integer $k \geq 1$. Consider now a group of $k+1$ horses.
    \begin{itemize}
        \item First, exclude horse $k+1$. Horses 1 to $k$ are a group of $k$ horses, so by our induction hypothesis, they must all be of the same colour.
        \item Next, exclude horse 1. Horses 2 to $k+1$ form another group of $k$ horses, so they must also all be of the same colour.
    \end{itemize}
    Hence, horse $k+1$ must have been the same colour as the non-excluded horses, i.e. all $k+1$ horses share the same colour, so $P(k+1)$ holds. Thus, by the principle of mathematical induction, $P(n)$ is true for all integers $n \geq 1$, so all horses are the same colour.
\end{proof}

Of course, we know that the claim is wrong, so we must have made an error somewhere in the proof. As an exercise, find the flaw in the proof. (Hint: consider the inductive step $P(1) \implies P(2)$.)

\subsection{Counter-Example}

In the case where we wish to prove a statement false, we can find a counter-example. In providing a counter-example, it must fulfil the hypothesis, but not the conclusion. That is, to show that $P \implies Q$ is false, we must show that $P$ is true but $Q$ is false.

\begin{example}[Counter-Example]
    Consider the statement $c \mid ab$, then $c \mid a$ or $c \mid b$, where $a, b, c \in \ZZ^+$. We can easily find a counter-example to this statement, e.g. $a = 3 \times 37$, $b = 7 \times 37$, $c = 3 \times 7$.
\end{example}