\chapter{Principle of Inclusion and Exclusion}

\begin{theorem}[Principle of Inclusion and Exclusion]
    Let $A_1$, $A_2$, $\dots$, $A_n$ be finite sets. Then \[\abs{\bigcup_{k = 1}^n A_k} = \sum_{\substack{I \subseteq [n]\\I \neq \varnothing}} (-1)^{\abs{I} + 1} \abs{\bigcap_{i \in I} A_i}.\]
\end{theorem}
\begin{proof}
    Let $A = \bigcup_{k = 1}^n A_k$ be the union of all $n$ sets. Define the indicator function of a set $A_i$ to be $\mathbf{1}_{i} : A \to \bc{0, 1}$ such that \[\mathbf{1}_{i}(x) = \begin{cases} 1, & x \in A_i, \\ 0, & x \notin A_i.\end{cases}\] Consider now the function \[F(x) = \prod_{i = 1}^n \bs{1 - \mathbf{1}_i(x)}.\] Observe that for all $x \in A$, we must have $x \in A_i$ for some $1 \leq i \leq n$, thus $F(x)$ is identically zero. We now expand $F(x)$: \[F(x) = 1 + \sum_{\substack{I \subseteq [n]\\I \neq \varnothing}} (-1)^{\abs{I}} \prod_{i \in I} \mathbf{1}_i(x).\] It is not too hard to see that $\prod_{i \in I} \mathbf{1}_i(x)$ is the indicator function of $\bigcap_{i \in I} A_i$. Summing over all $x \in A$, we hence obtain
    \begin{align*}
        \sum_{x \in A} F(x) &= \sum_{x \in A} \bs{1 + \sum_{\substack{I \subseteq [n]\\I \neq \varnothing}} (-1)^{\abs{I}} \prod_{i \in I} \mathbf{1}_i(x)}\\
        &= \abs{A} + \sum_{\substack{I \subseteq [n]\\I \neq \varnothing}} (-1)^{\abs{I}} \bp{\sum_{x \in A} \prod_{i \in I} \mathbf{1}_i(x)}\\
        &= \abs{\bigcup_{k = 1}^n A_k} + \sum_{I \subseteq [n]} (-1)^{\abs{I}} \abs{\bigcap_{i \in I} A_i}.
    \end{align*}
    Since $F(x)$ is identically zero, we immediately obtain the desired result: \[\abs{\bigcup_{k = 1}^n A_k} = \sum_{\substack{I \subseteq [n]\\I \neq \varnothing}} (-1)^{\abs{I} + 1} \abs{\bigcap_{i \in I} A_i}.\]
\end{proof}

A classic application of the Principle of Inclusion and Exclusion is counting the number of surjections between two finite sets.

\begin{proposition}
    Let $X$ and $Y$ be finite sets with cardinality $\abs{X} = m$ and $\abs{Y} = n$, where $m \geq n$. Then the number of surjections from $X$ to $Y$ is given by \[\sum_{k = 0}^{n-1} (-1)^k \binom{n}{k} (n-k)^m.\]
\end{proposition}
\begin{proof}
    For convenience, we number the elements of $X$ and $Y$ such that $X = [m]$ and $Y = [n]$. Let $S$ be the set of mappings from $X$ to $Y$, and $A_i$ be the set of mappings from $X$ to $Y\setminus\bc{i}$, where $1 \leq i \leq n$. We see that for an arbitrary non-empty set of indices $I \subseteq [n]$ of size $k$, \[\abs{\bigcap_{i \in I} A_i} = \text{\# (mappings from $m$ elements to $n-k$ elements)} = (n-k)^m.\] Since there are $\binom{n}{k}$ possible sets of indices of size $k$, by the Principle of Inclusion and Exclusion,
    \begin{align*}
        \abs{\bigcup_{k = 1}^n A_k} &= \sum_{\substack{I \subseteq [n]\\I \neq \varnothing}} (-1)^{\abs{I} + 1} \abs{\bigcap_{i \in I} A_i}.\\
        &= \sum_{k = 1}^{n} (-1)^{k+1} \binom{n}{k} (n-k)^m.
    \end{align*}
    This counts the number of mappings that are not surjective. For the number of mappings that are surjective, we simply take
    \begin{align*}
        \abs{S} - \abs{\bigcup_{k = 1}^n A_k} &= n^m - \sum_{k = 1}^{n} (-1)^{k+1} \binom{n}{k} (n-k)^m\\
        &= \sum_{k = 0}^{n-1} (-1)^k \binom{n}{k} (n-k)^m.
    \end{align*}
\end{proof}

\begin{corollary}
    The Stirling numbers of the second kind are given by \[S(m, n) = \frac1{n!} \sum_{k = 0}^{n-1} (-1)^k \binom{n}{k} (n-k)^m.\]
\end{corollary}
\begin{proof}
    There are $S(m,n)$ ways to partition $[m]$ into $n$ non-empty subsets. The number of ways to assign these $n$ parts to a distinct value in $[n]$ is $n!$. Thus, the number of surjective functions from $[m]$ to $[n]$ is $n! S(m,n)$. Using the above result, we obtain \[S(m, n) = \frac1{n!} \sum_{k = 0}^{n-1} (-1)^k \binom{n}{k} (n-k)^m.\]
\end{proof}

Yet another famous application of the Principle of Inclusion and Exclusion is counting the number of derangements.

\begin{definition}
    A \vocab{derangement} is a permutation $\pi : [n] \to [n]$ with no fixes point, i.e. for all $1 \leq i \leq n$, we have $\pi(i) \neq i$.
\end{definition}

\begin{proposition}
    The number of derangements $\pi : [n] \to [n]$ is given by \[\sum_{k = 0}^n (-1)^k \frac{n!}{k!}.\]
\end{proposition}
\begin{proof}
    Let $S$ be the set of all permutations of $[n]$, and let $A_i$ be the set of all permutations that fix $i$. Note that $\abs{S} = n!$, and for an arbitrary non-empty set of indices $I \subseteq [n]$ of size $k$, \[\abs{\bigcap_{i \in I} A_i} = \text{\#(permutations of $n-k$ elements)} = (n-k)!.\] Since there are $\binom{n}{k}$ possible sets of indices of size $k$, by the Principle of Inclusion and Exclusion,
    \begin{align*}
        \abs{\bigcup_{k = 1}^n A_k} &= \sum_{\substack{I \subseteq [n]\\I \neq \varnothing}} (-1)^{\abs{I} + 1} \abs{\bigcap_{i \in I} A_i}\\
        &= \sum_{k = 1}^{n} (-1)^{k+1} \binom{n}{k} (n-k)!\\
        &= \sum_{k = 1}^{n} (-1)^{k+1} \frac{n!}{k!}.
    \end{align*}
    This counts the number of permutations with fixed points. For the number of derangements, we simply take
    \begin{align*}
        \abs{S} - \abs{\bigcup_{k = 1}^n A_k} &= n! - \sum_{k = 1}^{n} (-1)^{k+1} \frac{n!}{k!}\\
        &= \sum_{k = 0}^n (-1)^k \frac{n!}{k!}.
    \end{align*}
\end{proof}