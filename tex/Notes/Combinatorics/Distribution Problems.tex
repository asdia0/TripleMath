\chapter{Distribution Problems}

In the previous chapter, we learnt how to count the number of ways to distribute distinct objects into distinct boxes. In this chapter, we focus mainly on counting the number of ways to distribute identical objects into distinct boxes.

\section{The Bijection Principle}

\begin{theorem}[Bijection Principle]
    Let $A$ and $B$ be finite sets. If there exists a bijection $f: A \to B$, then $\abs{A} = \abs{B}$.
\end{theorem}

The bijection principle is particularly useful when enumerating $A$ is hard, but enumerating $B$ is easy.

One common application of the bijection principle is counting the number of divisors of an integer.

\begin{proposition}
    Let $n = \prod_{i = 1}^k p_i^{e_i}$ where $p_i$ are distinct primes and $e_i$ are non-negative integers. Then $n$ has $\prod_{i = 1}^k (e_i + 1)$ positive divisors.
\end{proposition}
\begin{proof}
    Let $A$ be the set of divisors of $n$, and let $B$ be the set \[B = \bc{(x_1, \dots, x_k) \in \ZZ^k \mid 0 \leq x_i \leq e_i \text{ for } i = 1, \dots, k}.\] Define the function $f : B \to A$ such that $f(x_1, \dots, x_k) = \prod_{i = 1}^k p_i^{x_i}$. By the Fundamental Theorem of Algebra, it is clear that $f$ is bijective; every divisor $d$ of $n$ is uniquely expressible as a product of prime powers of $p_1, \dots, p_k$. Thus, by the bijection principle, we have $\abs{A} = \abs{B} = \prod_{i = 1}^k \bp{e_i + 1}$.
\end{proof}

\begin{sample}
    Determine the number of positive divisors of 12600.
\end{sample}
\begin{sampans}
    Since $12600 = 2^3 \times 3^2 \times 5^2 \times 7^1$, by the above result, we have that 12600 has $(3+1)(2+1)(2+1)(1+1) = 72$ positive divisors.
\end{sampans}

\section{Identical Objects into Distinct Boxes}

We first prove a standard result:

\begin{proposition}[Stars and Bars]
    The number of non-negative integer solutions to the equation $x_1 + \dots + x_n = r$ is $\binom{r + n - 1}{n - 1}$.
\end{proposition}
\begin{proof}
    Let $A = \bc{(x_1, \dots, x_n) \in \NN_0 \mid x_1 + \dots + x_n = r}$ be the set of all non-negative integer solutions to the equation in question. Consider a row of $r + n - 1$ objects. Let $B$ be the set of all possible ways to colour $n-1$ of these $r + n - 1$ objects red, and the remaining $r$ objects blue. It is easy to see that $\abs{B} = \binom{r + n - 1}{n-1}$.

    \begin{figure}[H]\tikzsetnextfilename{518}
        \centering
        \begin{tikzpicture}
            \fill[plotBlue] (0, 0) circle[radius=0.35];
            \fill[plotBlue] (1, 0) circle[radius=0.35];
            \fill[plotRed] (2, 0) circle[radius=0.35];
            \fill[plotBlue] (3, 0) circle[radius=0.35];
            \fill[plotBlue] (4, 0) circle[radius=0.35];
            \fill[plotBlue] (5, 0) circle[radius=0.35];
            \fill[plotRed] (6, 0) circle[radius=0.35];
            \fill[plotRed] (7, 0) circle[radius=0.35];
            \fill[plotBlue] (8, 0) circle[radius=0.35];
        \end{tikzpicture}
        \caption{An example colouring, where $r = 2 + 3 + 1 = 6$ and $n = 4$.}
    \end{figure}
    
    We now establish a bijection between $A$ and $B$. Consider the following procedure, starting with a solution $(x_1, \dots, x_n) \in A$:
    \begin{itemize}
        \item Colour the first $x_1$ balls blue, and the next ball red.
        \item Colour the next $x_2$ balls blue, and the next ball red.
        \item[] $\vdots$
        \item Colour the next $x_n$ balls blue.
    \end{itemize}
    It is easy to see that all $r + n - 1$ balls will be coloured, and exactly $n-1$ balls will be red. Further, each solution $(x_1, \dots, x_n) \in A$ uniquely determines a colouring in $B$ and vice versa, i.e. the procedure is a bijection between $A$ and $B$. By the bijection principle, $\abs{A} = \abs{B} = \binom{r + n - 1}{n - 1}$.
\end{proof}

The method of counting is commonly known as ``stars and bars''. We can think of the blue objects as ``stars'' (the objects we wish to distribute), and the red objects as ``bars'' (the dividers separating the objects).

\begin{proposition}[Identical Objects into Distinct Boxes (Part I)]
    The number of ways of distributing $r$ identical objects into $n$ distinct boxes is given by $\binom{r + n - 1}{n - 1}$.
\end{proposition}
\begin{proof}
    A distribution of $r$ identical objects into $n$ distinct boxes is completely determined by the number of objects placed in each box. Thus, each such distribution corresponds uniquely to an $n$-tuple of non-negative integers $(x_1, \dots, x_n)$ satisfying $x_1 + \dots + x_n = r$, where $x_i$ denotes the number of objects in the $i$th box.

    Conversely, any $n$-tuple $(x_1, \dots, x_n)$ of non-negative integers summing to $r$ determines a unique distribution by placing $x_i$ objects in the $i$th box.

    Thus, by the bijection principle, the number of ways to distribute $r$ identical objects into $n$ distinct boxes is equal to the number of non-negative solutions to $x_1 + \dots + x_n = r$, which we know to be $\binom{r+n-1}{n-1}$ by stars-and-bars.
\end{proof}

\begin{proposition}[Identical Objects into Distinct Boxes (Part II)]
    The number of ways of distributing $r$ identical objects into $n$ distinct boxes, such that each box has at least $k$ objects, is given by $\binom{r-nk + n - 1}{n - 1}$.
\end{proposition}
\begin{proof}
    To satisfy the condition that each box have at least $k$ objects, we first place $k$ objects in each of the $n$ boxes. Using the above result with the remaining $r - nk$ objects, we see that the desired count is $\binom{r-nk+n-1}{n-1}$.
\end{proof}

\begin{corollary}
    In the case where we require each box to be non-empty ($k = 1$), the number of distributions is given by $\binom{r - 1}{n - 1}$.
\end{corollary}

\section{Distinct Objects into Identical Boxes}

\begin{definition}
    A \vocab{Stirling number of the second kind} is defined to be the number of ways of distributing $r$ distinct objects into $n$ identical boxes such that no box is empty. It is denoted $S(r, n)$.
\end{definition}

\begin{proposition}
    We have the recurrence relation $S(r+1, n) = S(r,n-1) + nS(r,n)$ for $0 < n < r$, with initial conditions $S(r, r) = 1$ for $r \geq 0$ and $S(r, 0) = S(0, r) = 0$ for $r > 0$.
\end{proposition}
\begin{proof}
    Let $A$ be an arbitrary object.

    \case{1}[$A$ is alone in a box] There remains $r$ distinct objects to be distributed into $n-1$ identical boxes with no empty boxes. The number of ways to do so is $S(r, n-1)$.

    \case{2}[$A$ is not alone in a box] We first distribute the other $r$ distinct objects into $n$ identical boxes such that no box is empty. This can be done in $S(r, n)$ ways. Then, we place $A$ into one box. There are $n$ boxes, thus by the multiplicative principle, the total number of ways in this case is $n S(r,n)$.

    Altogether, the total number of ways to distribute $r+1$ distinct objects into $n$ identical boxes such that no box is empty is given by $S(r+1, n) = S(r, n-1) + nS(r,n)$.

    The initial conditions can be easily verified.
\end{proof}

\begin{sample}
    Find the number of ways to express 2730 as a product $ab$ of two integers $a$ and $b$, where $2 \geq a \geq b$.
\end{sample}
\begin{sampans}
    Note that $2730 = 2 \times 3 \times 5 \times 7 \times 13$. The number of ways to express 2730 as a product $ab$ is hence given by $S(5,2) = 15$, as we have 5 distinct prime factors and 2 identical boxes ($a$ and $b$).
\end{sampans}

\begin{proposition}
    The number of ways to distribute $r$ distinct objects into $n$ identical boxes with empty boxes allowed is given by $\sum_{k = 1}^n S(r,k)$.
\end{proposition}
\begin{proof}
    There are $S(r,k)$ ways to distribute the $r$ objects to fill exactly $k$ boxes. Enumerating over $k$, we get the desired result.
\end{proof}

\section{Identical Objects into Identical Boxes}

\begin{definition}
    The \vocab{partition} of a positive integer $r$ into $n$ parts is a set of $n$ positive integers whose sum is $r$. We denote the number of different partitions of $r$ into $n$ parts with $P(r, n)$.
\end{definition}

\begin{proposition}
    We have the recurrence relation $P(r, n) = P(r-1, n-1) + P(r-n, n)$, with conditions $P(r, 1) = 1$ for all $r \geq 1$, and $P(r, n) = 0$ if $n > r$.
\end{proposition}
\begin{proof}
    \case{1}[At least one box has exactly one object] We place one object in one box. We then distribute the remaining $r-1$ objects into the remaining $n-1$ boxes such that no boxes are empty. The number of ways this can be done is $P(r-1, n-1)$.

    \case{2}[All the boxes have more than one object] We place one object into each of the $n$ boxes. We then distribute the remaining $r-n$ objects into the $n$ boxes so that no boxes are empty. The number of ways this can be done is $P(r-n, n)$.

    Altogether, we have $P(r, n) = P(r-1, n-1) + P(r-n, n)$ as desired.
\end{proof}