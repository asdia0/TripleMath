\chapter{Vectors}

\paragraph{Prerequisites.} None.
\paragraph{Exercises.} \nameref{sec:A7}, \nameref{sec:A8}, \nameref{sec:A9}

\section{Basic Properties and Vector Algebra}

\subsection{Basic Definitions and Notations}

\begin{definition}
    A \vocab{vector} is an object that has both magnitude and direction. Geometrically, we can represent a vector by a \textbf{directed} line segment $\oa{PQ}$, where the length of the line segment represents the magnitude of the vector, and the direction of the line segment represents the direction of the vector. Vectors are typically denoted by bold print (e.g. $\vec a$) or by $\oa{PQ}$.
\end{definition}

\begin{definition}
    The \vocab{magnitude} of a vector $\vec a$ is the length of the line representing $\vec a$, and is denoted by $\abs{\vec a}$.
\end{definition}

\begin{definition}
    Two vectors $\vec a$ and $\vec b$ are said to be \vocab{equal vectors} if they both have the same magnitude and direction. $\vec a$ and $\vec b$ are said to be \vocab{negative vectors} if they have the same magnitude but opposite directions.
\end{definition}

\begin{definition}[Multiplication of a Vector by a Scalar]
    Let $\l$ be a scalar. If $\l > 0$, then $\l \vec a$ is a vector of magnitude $\l \abs{\vec a}$ and has the same direction as $\vec a$. If $\l < 0$, then $\l \vec a$ is a vector of magnitude $-\l \abs{\vec a}$ and is in the opposite direction of $\vec a$.
\end{definition}

\begin{definition}
    The \vocab{zero vector} is the vector with a magnitude of 0 and is denoted $\vec 0$.
\end{definition}

\begin{definition}
    Let $\vec a$ and $\vec b$ be non-zero vectors. Then $\vec a$ and $\vec b$ are said to be \vocab{parallel} if and only if $\vec b$ can be expressed as a non-zero scalar multiple of $\vec a$. Mathematically, \[\vec a \parallel \vec b \iff (\exists \l \in \RR \setminus \bc{0}): \quad \vec b = \l \vec a.\]
\end{definition}

\begin{definition}
    A \vocab{unit vector} is a vector with a magnitude of 1. Unit vectors are typically denoted with a hat (e.g. $\hat{\vec a}$).
\end{definition}
Observe that for any non-zero vector $\vec a$, the unit vector parallel to $\vec a$ is given by \[\hat{\vec a} = \frac{\vec a}{\abs{\vec a}}.\]

\begin{definition}
    The \vocab{Triangle Law of Vector Addition} states that \[\oa{AB} + \oa{BC} = \oa{AC}.\] Geometrically, we add two vectors $\vec a$ and $\vec b$ by placing them head to tail, taking the resultant vector as their sum.

    \begin{center}\tikzsetnextfilename{335}
        \begin{tikzpicture}
            \coordinate[label=left:$A$] (A) at (0, 0);
            \coordinate[label=above:$B$] (B) at (2, 1);
            \coordinate[label=right:$C$] (C) at (5, 1);

            \draw[->-=0.5] (A) -- (B);
            \draw[->-=0.5] (B) -- (C);
            \draw[->-=0.5] (A) -- (C);

            \node[anchor=south east] at (1, 0.5) {$\vec a$};
            \node[anchor=south] at (3.5, 1) {$\vec b$};
            \node[anchor=north west, rotate=11.3] at (2, 0.2) {$\vec a + \vec b$};
        \end{tikzpicture}
    \end{center}

    We subtract vectors by adding $\vec a + -(\vec b)$.
\end{definition}

\begin{definition}
    The \vocab{angle between two vectors} refers to the angle between their directions when the arrows representing them \textit{both converge} or \textit{both diverge}.
\end{definition}

\begin{definition}
    A \vocab{free vector} is a vector that has no specific location in space. The \vocab{position vector} of some point $A$ relative to the origin $O$ is unique and is denoted $\oa{OA}$. A \vocab{displacement vector} is a vector that joins its initial position to its final position. For instance, $\oa{OA}$ is the displacement vector from $O$ to $A$.
\end{definition}

\begin{definition}
    A set of vectors are said to be \vocab{coplanar} if their directions are all parallel to the same plane.
\end{definition}

\begin{fact}
    Any vector $\vec c$ that is coplanar with $\vec a$ and $\vec b$ can be expressed as a \textbf{unique linear combination} of $\vec a$ and $\vec b$, i.e. \[(\exists! \, \l, \m \in \RR): \quad \vec c = \l \vec a + \m \vec b.\]
\end{fact}

\begin{theorem}[Ratio Theorem]
    If $P$ divides $AB$ in the ratio $\l : \m$, then \[\oa{OP} = \frac{\m \vec a + \l \vec b}{\l + \m}.\]
\end{theorem}
\begin{proof}
    Since $P$ divides $AB$ in the ratio $\l : \m$, we have \[\oa{AP} = \frac{\l}{\l + \m} \oa{AB} = \frac{\l}{\l + \m} (\vec b - \vec a).\] Thus, \[\oa{OP} = \oa{OA} + \oa{AP} = \vec a + \frac{\l}{\l + \m} (\vec b - \vec a) = \frac{\m \vec a + \l \vec b}{\l + \m}.\]
\end{proof}

\begin{corollary}[Mid-Point Theorem]
    If $P$ is the mid-point of $AB$, then \[\oa{OP} = \frac{\vec a + \vec b}{2}.\]
\end{corollary}

\subsection{Vector Representation using Cartesian Unit Vectors}

\subsubsection{2-D Cartesian Unit Vectors}

\begin{definition}[2-D Cartesian Unit Vectors]
    In the 2-D Cartesian plane, $\vec i = \cveciix10$ is defined to be the unit vector in the positive direction of the $x$-axis, while $\vec j = \cveciix01$ is defined to be the unit vector in the positive direction of the $y$-axis.
\end{definition}

Thus, if $P$ is the point with coordinates $(a, b)$, then we can express $\oa{OP}$ in terms of the unit vectors $\vec i$ and $\vec j$. In particular, $\oa{OP} = a\vec i + b \vec j$.

\begin{proposition}[Magnitude in 2-D]
    \[\abs{\cvecii{a}{b}} = \sqrt{a^2 + b^2}.\]
\end{proposition}
\begin{proof}
    Follows immediately from Pythagoras' theorem.
\end{proof}

\subsubsection{3-D Cartesian Unit Vectors}

\begin{definition}[3-D Cartesian Unit Vectors]
    In the 3-D Cartesian plane, $\vec i = \cveciiix100$, $\vec j = \cveciiix010$ and $\vec k = \cveciiix001$ denote the unit vectors in the positive direction of the $x$, $y$ and $z$-axes respectively.
\end{definition}

\begin{proposition}[Magnitude in 3-D]
    \[\abs{\cveciii{a}{b}{c}} = \sqrt{a^2 + b^2 + c^2}.\]
\end{proposition}
\begin{proof}
    Use Pythagoras' theorem twice.
\end{proof}

\begin{fact}[Operations on Cartesian Vectors]
    To add vectors given in Cartesian unit vector form, the coefficients of $\vec i$, $\vec j$ and $\vec k$ are added separately. \[\cveciii{x_1}{y_1}{z_1} + \cveciii{x_2}{y_2}{z_2} = \cveciii{x_1+x_2}{y_1+y_2}{z_1+z_2}.\] Subtraction and scalar multiplication follows immediately.
\end{fact}

\clearpage
\subsection{Scalar Product}

\begin{definition}
    The \vocab{scalar product} (or dot product) of two vectors $\vec a$ and $\vec b$ is defined by \[\vec a \dotp \vec b = \abs{\vec a} \abs{\vec b} \cos \t,\] where $\t$ is the angle between the two vectors (note that $0 \leq \t \leq \pi$).
\end{definition}
\begin{remark}
    $\vec a \dotp \vec b$ is called the scalar product as the result is a real number (a scalar). It is also called the dot product because of the notation.
\end{remark}

\begin{fact}[Algebraic Properties of Scalar Product]
    Let $\vec a$, $\vec b$ and $\vec c$ be vectors and let $\l \in \RR$. Then
    \begin{itemize}
        \item (commutative) $\vec a \dotp \vec b = \vec b \dotp \vec a$.
        \item (distributive over addition) $\vec a \dotp (\vec b + \vec c) = \vec a \dotp \vec b + \vec a \dotp \vec c$.
        \item $\vec a \dotp \vec a = \abs{\vec a}^2$.
        \item $(\l \vec a) \dotp \vec b = \vec a \dotp (\l \vec b) = \l (\vec a \dotp \vec b)$.
    \end{itemize}
\end{fact}

\begin{proposition}[Geometric Properties of Scalar Product]
    Let $\vec a$ and $\vec b$ be non-zero vectors, and let $\t$ be the angle between them.
    \begin{itemize}
        \item $\vec a \dotp \vec b = 0$ if and only if $\t = \frac\pi2$, i.e. $\vec a \perp \vec b$.
        \item $\vec a \dotp \vec b > 0$ if and only if $\t$ is acute.
        \item $\vec a \dotp \vec b < 0$ if and only if $\t$ is obtuse.
    \end{itemize}
\end{proposition}
\begin{proof}
    The sign of $\vec a \dotp \vec b$ is determined solely by $\cos \t$.
\end{proof}

\begin{proposition}[Scalar Product in Cartesian Unit Vector Form]
    \[\cveciii{x_1}{y_1}{z_1} \dotp \cveciii{x_2}{y_2}{z_2} = x_1 x_2 + y_1 y_2 + z_1 z_2.\]
\end{proposition}
\begin{proof}
    Since $\vec i$, $\vec j$ and $\vec k$ are pairwise perpendicular, their pairwise scalar products are 0. That is, \[\vec i \dotp \vec j = \vec j \dotp \vec k = \vec k \dotp \vec i = 0.\] Hence, by the distributive property of the scalar product, \[(x_1 \vec i + y_1 \vec j + z_1 \vec k) \dotp (x_2 \vec i + y_2 \vec j + z_2 \vec k) = x_1 x_2 \vec i \dotp \vec i + y_1 y_2 \vec j \dotp \vec j + z_1 z_2 \vec k \dotp \vec k.\] Lastly, since $\vec i$, $\vec j$ and $\vec k$ are all unit vectors, \[\vec i \dotp \vec i = \vec j \dotp \vec j = \vec k \dotp \vec k = 1.\] Thus, \[\cveciii{x_1}{y_1}{z_1} \dotp \cveciii{x_2}{y_2}{z_2} = x_1 x_2 + y_1 y_2 + z_1 z_2.\]
\end{proof}

\subsubsection{Applications of Scalar Product}

\begin{proposition}[Angle between Two Vectors]
    Let $\t$ be the angle between two non-zero vectors $\vec a$ and $\vec b$. Then \[\cos \t = \frac{\vec a \dotp \vec b}{\abs{\vec a} \abs{\vec b}}.\]
\end{proposition}
\begin{proof}
    Follows immediately from the definition of the scalar product.
\end{proof}

\begin{definition}
    Let $\vec a$ and $\vec b$ denote the position vectors of $A$ and $B$ respectively, relative to the origin $O$. Let $\t$ be the angle between $\vec a$ and $\vec b$, and let $N$ be the foot of the perpendicular from the point $A$ to the line passing through $O$ and $B$.

    Then, the length $ON$ is defined to be the \vocab{length of projection} of the vector $\vec a$ onto the vector $\vec b$. Also, $\oa{ON}$ is the \vocab{vector projection} of $\vec a$ onto $\vec b$.
\end{definition}

\begin{proposition}[Length of Projection]
    The length of projection of $\vec a$ onto $\vec b$ is $\abs{\vec a \dotp \hat{\vec b}}$.
\end{proposition}
\begin{proof}
    Consider the case where $\t$ is acute.

    \begin{center}\tikzsetnextfilename{336}
        \begin{tikzpicture}
            \coordinate[label=right:$A$] (A) at (2, 2);
            \coordinate[label=right:$B$] (B) at (3, 0);
            \coordinate[label=below:$N$] (N) at (2, 0);
            \coordinate[label=below left:$O$] (O) at (0, 0);
    
            \draw[->] (O) -- (A);
            \draw[->] (O) -- (B);
            \draw[dotted] (A) -- (N);

            \node[anchor=south east] at (1, 1) {$\vec a$};
            \node[anchor=north] at (0.5, 0) {$\vec b$};
    
            \draw pic [draw, angle radius=3mm] {right angle = B--N--A};
            \draw pic [draw, angle radius=10mm, "$\t$"] {angle = B--O--A};
        \end{tikzpicture}
    \end{center}

    From the diagram, \[ON = OA \cos \t = \abs{\vec a} \frac{\vec a \dotp \vec b}{\abs{\vec a} \abs{\vec b}} = \frac{\vec a \dotp \vec b}{\abs{\vec b}} = \vec a \dotp \hat{\vec b}.\] A similar argument shows that when $\t$ is obtuse, $ON = -\vec a \dotp \hat{\vec b}$. Hence, in any case, $ON = \abs{\vec a \dotp \hat{\vec b}}$.
\end{proof}

\begin{proposition}[Vector Projection]
    The vector projection of $\vec a$ onto $\vec b$ is $(\vec a \dotp \hat{\vec b}) \hat{\vec b}$.
\end{proposition}
\begin{proof}
    \case{1}[$\t$ is acute] Then $\oa{ON}$ is in the same direction as $\vec b$. Hence, \[\oa{ON} = \abs{ON} \hat{\vec b} = (\vec a \dotp \hat{\vec b}) \hat{\vec b}.\]

    \case{2}[$\t$ is obtuse] Then $\oa{ON}$ is in the opposite direction as $\vec b$. Hence, \[\oa{ON} = \abs{ON} \bp{-\hat{\vec b}} = -(\vec a \dotp \hat{\vec b}) (-\hat{\vec b}) = (\vec a \dotp \hat{\vec b}) \hat{\vec b}.\]
\end{proof}

\subsection{Vector Product}

\begin{definition}
    The \vocab{vector product} (or cross product) of two vectors $\vec a$ and $\vec b$ is denoted by $\vec a \crossp \vec b$ and is defined by \[\vec a \crossp \vec b = \abs{\vec a}\abs{\vec b} \sin \t \hat{\vec n},\] where $\t$ is the angle between $\vec a$ and $\vec b$ and $\hat{\vec n}$ is the unit vector perpendicular to both $\vec a$ and $\vec b$, in the direction determined by teh right-hand grip rule.
\end{definition}
\begin{remark}
    $\vec a \crossp \vec b$ is called the vector product as the result is a vector. It is also called the cross product due to its notation.
\end{remark}

\begin{fact}[Algebraic Properties of Vector Product]
    Let $\vec a$, $\vec b$ and $\vec c$ be three vectors, and $\t$ be the angle between $\vec a$ and $\vec b$.
    \begin{itemize}
        \item (anti-commutative) $\vec a \crossp \vec b = -\vec b \crossp \vec a$.
        \item (distributive over addition) $\vec a \crossp (\vec b + \vec c) = (\vec a \crossp \vec b) + (\vec a \crossp \vec c)$.
        \item $\abs{\vec a \crossp \vec b} = \abs{a} \abs{b} \sin \t$.
        \item $(\l \vec a) \crossp \vec b = \vec a \crossp (\l \vec b) = \l (\vec a \crossp \vec b)$, where $\l \in \RR$.
    \end{itemize}
\end{fact}

\begin{proposition}[Geometric Properties of Vector Product]
    Let $\vec a$ and $\vec b$ be non-zero vectors and $\t$ be the angle between them.
    \begin{itemize}
        \item $\abs{\vec a \crossp \vec b} = 0$ if and only if $\vec a \parallel \vec b$.
        \item $\abs{\vec a \crossp \vec b} = \abs{\vec a} \abs{\vec b}$ if and only if $\vec a \perp \vec b$.
    \end{itemize}
\end{proposition}
\begin{proof}
    Follows from the definition of the vector product (consider $\t = 0, \frac\pi2, \pi$).
\end{proof}

\begin{proposition}[Vector Product in Cartesian Unit Vector Form]
    \[\cveciii{x_1}{y_1}{z_1} \dotp \cveciii{x_2}{y_2}{z_2} = \cveciii{y_1 z_2 - z_1 y_2}{z_1 x_2 - x_1 z_2}{x_1 y_2 - y_1 x_2}.\]
\end{proposition}
\begin{proof}
    From the geometric properties of the vector product, we have \[\vec i \crossp \vec i = \vec j \crossp \vec j = \vec k \crossp \vec k = \vec 0.\] Furthermore, since $\vec i$, $\vec j$ and $\vec k$ are pairwise perpendicular, by the right-hand grip rule, one has \[\vec i \crossp \vec j = \vec k, \quad \vec j \crossp \vec k = \vec i, \quad \vec k \crossp \vec i = \vec j.\] Hence, by the distributive property of the vector product,
    \begin{align*}
        &(x_1 \vec i + y_1 \vec j + z_1 \vec k) \crossp (x_2 \vec i + y_2 \vec j + z_2 \vec k) \\
        &\hspace{2em}= x_1 y_2 \vec k + x_1 z_2 (-\vec j) + y_1 x_2 (-\vec k) + y_1 z_2 \vec i + z_1 x_2 \vec j + z_1 y_2 (-\vec i) \\
        &\hspace{2em}=(y_1 z_2 - z_1 y_2) \vec i + (z_1 x_2 - x_1 z_2) \vec j + (x_1 y_2 - y_1 x_2) \vec k.
    \end{align*}
\end{proof}

\subsubsection{Applications of Vector Product}

\begin{proposition}[Length of Side of Right-Angled Triangle]
    Let $\vec a$ and $\vec b$ denote the position vectors of $A$ and $B$ respectively, relative to the origin $O$. Let $\t$ be the angle between $\vec a$ and $\vec b$, and let $N$ be the foot of the perpendicular from $A$ to $OB$. Then \[AN = \abs{\vec a \crossp \hat{\vec b}}.\]
\end{proposition}
\begin{proof}
    Consider the following diagram.
    \begin{center}
        \includegraphics{figures/336}
    \end{center}
    We hence see that \[AN = OA \sin \t = \abs{\vec a} \frac{\abs{\vec a \crossp \vec b}}{\abs{\vec a}\abs{\vec b}} = \frac{\abs{\vec a \crossp \vec b}}{\vec b} = \abs{\vec a \crossp \hat{\vec b}}.\]
\end{proof}

\begin{proposition}[Area of Triangles and Parallelogram]
    Let $ABCD$ be a parallelogram. Let $\vec a = \oa{AB}$ and $\vec b = \oa{AC}$. Let $\t$ be the angle between $\vec a$ and $\vec b$. Then \[[\triangle ABC] = \frac12 \abs{\vec a \crossp \vec b}\] and \[[ABCD] = \abs{\vec a \crossp \vec b}.\]
\end{proposition}
\begin{proof}
    Recall that the formula for the area of a triangle is \[[\triangle ABC] = \frac12 (AB)(AC) \sin \t = \frac12 \abs{\vec a} \abs{\vec b} \sin \t = \frac12 \abs{\vec a \crossp \vec b}.\] Since the area of parallelogram $ABCD$ is twice that of $\triangle ABC$, we immediately have \[[ABCD] = \abs{\vec a \crossp \vec b}.\]
\end{proof}

\section{Lines}

\subsection{Equation of a Line}

\begin{definition}
    The \vocab{vector equation} of the line $l$ passing through point $A$ with position vector $\vec a$ and parallel to $\vec b$ is given by \[\vec r = \vec a + \l \vec b, \quad \l \in \RR,\] where $\vec r$ is the position vector of any point on the line, and $\l$ is a real, scalar parameter. The vector $\vec b$ is also called the \vocab{direction vector} of the line.
\end{definition}
\begin{remark}
    Note that $\vec a$ can be any position vector on the line and $\vec b$ can be any vector parallel to the line. Hence, the vector equation of a line is not unique.
\end{remark}

\begin{definition}
    Let $l : \vec r = \vec a + \l \vec b$, $\l \in \RR$. By writing $\vec r = \cveciiix{x}{y}{z}$, $\vec a = \cveciiix{a_1}{a_2}{a_3}$ and $\vec b = \cveciiix{b_1}{b_2}{b_3}$, we have \[\left\{ \begin{aligned}
        x = a_1 + \l b_1\\
        y = a_2 + \l b_2\\
        z = a_3 + \l b_3
    \end{aligned}, \quad \l \in \RR. \right.\] This set of three equations is known as the \vocab{parametric equations} of the line $l$.
\end{definition}

\begin{definition}
    From the parametric form of the line $l$, by making $\l$ the subject, we have \[\l = \frac{x-a_1}{b_1} = \frac{y - a_2}{b_2} = \frac{z - a_3}{b_3}.\] This equation is known as the \vocab{Cartesian equation} of the line $l$.
\end{definition}
\begin{remark}
    If $b_1 = 0$, we simply have $x = a_1$. A similar result arises when $b_2 = 0$ or $b_3 = 0$.
\end{remark}

\subsection{Point and Line}

\begin{proposition}[Relationship between Point and Line]
    A point $C$ lies on a line $l : \vec r = \vec a + \l \vec b$, $\l \in \RR$, if and only if \[(\exists \l \in \RR): \quad \oa{OC} = \vec a + \l \vec b.\]
\end{proposition}
\begin{proof}
    Trivial.
\end{proof}

\begin{proposition}[Perpendicular Distance between Point and Line]
    Let $C$ be a point not on the line $l : \vec r = \vec a + \l \vec b$, $\l \in \RR$. Let $F$ be the foot of perpendicular from $C$ to $l$. Then \[CF = \abs{\oa{AC} \crossp \hat{\vec b}}.\]
\end{proposition}
\begin{proof}
    Trivial (recall the application of the vector product in finding side lengths of right-angled triangles).
\end{proof}

\begin{method}[Finding Foot of Perpendicular from Point to Line]
    Let $F$ be the foot of perpendicular from $C$ to the line $l : \vec r = \vec a + \l \vec b$, $\l \in \RR$. To find $\oa{OF}$, we use the fact that
    \begin{itemize}
        \item $F$ lies on $l$, i.e. $\oa{OF} = \vec a + \l \vec b$ for some $\l \in \RR$.
        \item $\oa{CF}$ is perpendicular to $l$, i.e. $\oa{CF} \dotp \vec b = 0$.
    \end{itemize}
\end{method}

\subsection{Two Lines}

\begin{definition}
    The relationship between two lines in 3-D space can be classified as follows:
    \begin{itemize}
        \item \vocab{Parallel lines}: The lines are parallel and non-intersecting;
        \item \vocab{Intersecting lines}: The lines are non-parallel and intersecting;
        \item \vocab{Skew lines}: The lines are non-parallel and non-intersecting.
    \end{itemize}
\end{definition}
\begin{remark}
    Note that parallel and intersecting lines are coplanar, while skew lines are non-coplanar.
\end{remark}

\begin{method}[Relationship between Two Lines]
    Consider two distinct lines, $l_1 : \vec r = \vec a + \l \vec b$, $\l \in \RR$ and $l_2 : \vec r = \vec c + \m \vec d$, $\m \in \RR$.
    \begin{itemize}
        \item $l_1$ and $l_2$ are parallel lines if their direction vectors are parallel.
        \item $l_1$ and $l_2$ are intersecting lines if there are unique values of $\l$ and $\m$ such that $\vec a + \l \vec b = \vec c + \m \vec d$.
        \item $l_2$ and $l_2$ are skew lines if their direction vectors are not parallel and there are no values of $\l$ and $\m$ such that $\vec a + \l \vec b = \vec c + \m \vec d$.
    \end{itemize}
\end{method}

\begin{proposition}[Acute Angle between Two Lines]
    Let the acute angle between two lines with direction vectors $\vec b_1$ and $\vec b_2$ be $\t$. Then \[\cos\t = \frac{\abs{\vec b_1 \dotp \vec b_2}}{\abs{\vec b_1} \abs{\vec b_2}}.\]
\end{proposition}
\begin{proof}
    Observe that we are essentially finding the angle between the direction vectors of the two lines, which is given by \[\cos\t = \frac{\vec b_1 \dotp \vec b_2}{\abs{\vec b_1} \abs{\vec b_2}}.\] However, to ensure that $\t$ is acute (i.e. $\cos \t \geq 0$), we introduce a modulus sign in the numberator. Hence, \[\cos\t = \frac{\abs{\vec b_1 \dotp \vec b_2}}{\abs{\vec b_1} \abs{\vec b_2}}.\]
\end{proof}

\section{Planes}


\subsection{Equation of a Plane}

\begin{definition}
    Suppose the plane $\pi$ passes through a fixed point $A$ with position vector $\vec a$, and $\pi$ is parallel to two vectors $\vec b_1$ and $\vec b_2$, where $\vec b_1$ and $\vec b_2$ are not parallel to each other. Then the vector equation (in \vocab{parametric form}) of $\pi$ is given by \[\pi : \vec r = \vec a + \l \vec b_1 + \m \vec b_2,\] where $\vec r$ is the position vector of any point $P$ on $\pi$, and $\l$ and $\m$ are real parameters.
\end{definition}

\begin{definition}
    Suppose the plane $\pi$ passes through a fixed point $A$ with position vector $\vec a$, and $\pi$ has normal vector $\vec n$. Let $P$ be an arbitrary point on $\pi$. Then $\oa{AP}$ is perpendicular to the normal vector $\vec n$, i.e. $\oa{AP} \dotp \vec n = 0$. Since $\oa{AP} = \vec r - \vec a$, by the distributivity of the scalar product, one has \[\vec r \dotp \vec n = \vec a \dotp \vec n.\] This is the \vocab{scalar product form} of the vector equation of $\pi$, which is more commonly written as \[\vec r \dotp \vec n = d.\]
\end{definition}

\begin{definition}
    Let the plane $\pi$ have scalar product form \[\pi : \vec r \dotp \vec n = \vec a \dotp \vec n.\] Let $\vec r = \cveciiix{x}{y}{z}$, $\vec a = \cveciiix{a_1}{a_2}{a_3}$ and $\vec n = \cveciiix{n_1}{n_2}{n_3}$. Then \[\pi : n_1 x + n_2 y + n_3 z = a_1 n_1 + a_2 n_2 + a_3 n_3\] is the \vocab{Cartesian equation} of $\pi$, which is more commonly written as \[\pi : n_1 x + n_2 y + n_3 z = d.\]
\end{definition}

\begin{method}[Converting between Forms]
    To convert from parametric form to scalar product form, take $\vec n = \vec b_1 \crossp \vec b_2$. To convert from the Cartesian equation to parametric form, express $x$ in terms of $y$ and $z$, then replace $y$ and $z$ with $\l$ and $\m$ respectively.
\end{method}

\begin{example}[Parametric to Scalar Product Form]
    Let the plane $\pi$ have parametric form $\vec r = \cveciiix123 + \l \cveciiix456 + \m \cveciiix789$. Then the normal vector to $\pi$ is given by \[\vec n = \cveciii456 \crossp \cveciii789 = \cveciii{-3}{6}{-3} \parallel \cveciii{1}{-2}{1}.\] Hence, \[d = \cveciii123 \dotp \cveciii1{-2}1 = 0,\] whence $\pi$ has scalar product form \[\vec r \dotp \cveciii1{-2}1 = 0.\]
\end{example}

\begin{example}[Cartesian to Parametric Form]
    Let the plane $\pi$ have Cartesian equation \[x + y + z = 10.\] Solving for $x$ and replacing $y$ and $z$ with $\l$ and $\m$ respectively, we get \[x = 10 - \l - \m, \quad y = \l, \quad z = \m.\] Hence, $\pi$ has parametric form \[\vec r = \cveciii{x}{y}{z} = \cveciii{10 - \l - \m}{\l}{\m} = \cveciii{10}00 + \l \cveciii{-1}{1}{0} + \m \cveciii{-1}0{1}, \quad \l, \m \in \RR.\]
\end{example}

\subsection{Point and Plane}

\begin{proposition}[Relationship between Point and Plane]
    A point lies on a plane if and only if its position vector (or its equivalent coordinates) satisfies the equation of the plane.
\end{proposition}
\begin{proof}
    Trivial.
\end{proof}

\begin{proposition}[Perpendicular Distance between Point and Plane]
    Let $F$ be the foot of perpendicular from a point $Q$ to the plane $\pi$ with vector equation $\pi : \vec r \dotp \vec n = d$. Let $A$ be a point on $\pi$. Then $QF$, the perpendicular distance from $Q$ to $\pi$, is given by \[QF = \abs{\oa{QA} \dotp \hat{\vec n}} = \frac{\abs{d - \vec q \dotp \vec n}}{\abs{\vec n}}.\]
\end{proposition}
\begin{proof}
    Note that $QF$ is the length of projection of $\oa{QA}$ onto the normal vector $\vec n$. Hence, \[QF = \abs{\oa{QA} \dotp \hat{\vec n}}\] follows directly from the formula for the length of projection. Now, observe that \[\oa{QA} \dotp \vec n = \oa{OA} \dotp \vec n - \oa{OQ} \dotp \vec n = d - \vec q \dotp \vec n.\] Hence, \[QF = \frac{\abs{\oa{QA} \dotp \vec n}}{\abs{\vec n}} = \frac{\abs{d - \vec q \dotp \vec n}}{\abs{\vec n}}.\]
\end{proof}

\begin{corollary}
    $OF$, the perpendicular distance from the plane $\pi$ to the origin $O$, is \[OF = \frac{\abs{d}}{\abs{\vec n}}.\]
\end{corollary}
\begin{proof}
    Take $\vec q = \vec 0$.
\end{proof}

\begin{method}[Foot of Perpendicular from Point to Plane]
    Let $F$ be the foot of perpendicular from a point $Q$ to the plane $\pi$ with vector equation $\pi : \vec r \dotp \vec n = d$. To find the position vector $\oa{OF}$, we use the fact that
    \begin{itemize}
        \item $QF$ is perpendicular to $\pi$, i.e. $\oa{QF} = \l \vec n$ for some $\l \in \RR$, and
        \item $F$ lies on $\pi$, i.e. $\oa{OF} \dotp \vec n = d$.
    \end{itemize}
\end{method}

\begin{example}[Foot of Perpendicular from Point to Plane]
    Let the plane $\pi$ have equation $\pi : \vec r \dotp \cveciiix123 = 10$. Let $Q(4, 5, 6)$. Let $F$ be the foot of perpendicular from $Q$ to $\pi$. We wish to find $\oa{OF}$.

    Since $QF$ is perpendicular to $\pi$, we have \[\oa{QF} = \l \cveciii123, \quad \l \in \RR.\] Hence, \[\oa{OF} = \oa{OQ} + \oa{QF} = \cveciii456 + \l\cveciii123.\] Taking the scalar product on both sides, we get \[10 = \oa{OF} \dotp \cveciii123 = \bs{\cveciii456 + \l\cveciii123} \dotp \cveciii123 = 32 + 14\l.\] Thus, $\l = -11/7$, whence \[\oa{OF} = \cveciii456 - \frac{11}{7}\cveciii123 = \frac17 \cveciii{17}{13}{9}.\]
\end{example}

\subsection{Line and Plane}

\begin{fact}[Relationship between Line and Plane]
    Given a line $l : \vec r = \vec a + \l \vec b$, $\l \in \RR$, and a plane $\pi : \vec r \dotp \vec n = d$, there are three possible cases:
    \begin{itemize}
        \item \textbf{$l$ and $\pi$ do not intersect.} $l$ and $\pi$ are parallel and have no common point.
        \item \textbf{$l$ lies on $\pi$.} $l$ and $\pi$ are parallel and any point on $l$ is also a point on $\pi$.
        \item \textbf{$l$ and $\pi$ intersect once.} $l$ and $\pi$ are not parallel.
    \end{itemize}
\end{fact}

There are two methods to determine the relationship between a line and a plane.

\begin{method}[Using Normal Vector]
    \phantom{.}
    \begin{itemize}
        \item If $l$ and $\pi$ do not intersect, then $\vec b \dotp \vec n = 0$ and $\vec a \dotp vec n \neq d$.
        \item If $l$ lies on $\pi$, then $\vec b \dotp \vec n = 0$ and $\vec a \dotp \vec n =  d$.
        \item If $l$ and $\pi$ intersect once, then $\vec b \dotp \vec n \neq 0$.
    \end{itemize}
\end{method}

\begin{method}[Solving Simultaneous Equations]
    Solve $l : \vec r = \vec a + \l \vec b$, $\l \in \RR$ and $\pi : \vec r \dotp \vec n = d$ simultaneously.
    \begin{itemize}
        \item If there are no solutions, then $l$ and $\pi$ do not intersect.
        \item If there are infinitely many solutions, then $l$ lies on $\pi$.
        \item If there is a unique solution, then $l$ and $\pi$ intersect once.
    \end{itemize}
\end{method}

\begin{proposition}[Acute Angle between Line and Plane]
    Let $\t$ be the acute angle between the line $l : \vec r = \vec a + \l \vec b$, $\l \in \RR$ and the plane $\pi : \vec r \dotp \vec n = d$. Then \[\sin \t = \frac{\abs{\vec b \dotp \vec n}}{\abs{\vec b} \abs{\vec n}}.\]
\end{proposition}
\begin{proof}
    We first find $\f$, the acute angle between $l$ and the normal. Recall that \[\cos \f = \frac{\abs{\vec b \dotp \vec n}}{\abs{\vec b} \abs{\vec n}}.\] Since $\f = \frac\pi2 - \t$, we have \[\cos{\frac\pi2 - \t} = \sin \t = \frac{\abs{\vec b \dotp \vec n}}{\abs{\vec b} \abs{\vec n}}.\]
\end{proof}

\subsection{Two Planes}

\begin{proposition}[Acute Angle between Two Planes]
    The acute angle $\t$ between two planes $\pi_1 : \vec r \dotp \vec n_1 = d_1$ and $\pi_2 : \vec r \dotp \vec n_2 = d_2$ is given by \[\cos \t = \frac{\abs{\vec n_1 \dotp \vec n_2}}{\abs{\vec n_1} \abs{\vec n_2}}.\]
\end{proposition}
\begin{proof}
    Consider the following diagram.

    \begin{center}\tikzsetnextfilename{337}
        \begin{tikzpicture}
            \coordinate (A) at (0, 0);
            \coordinate (B) at (6, 0);
            \coordinate (C) at (4, -1);
            \coordinate (D) at (1, 2);
            \coordinate (E) at (3, 0);

            \coordinate (F) at (2, 1);
            \coordinate (G) at (4, 0);
            \coordinate (H) at (3, 2);
            \coordinate (I) at (4, 1);
            \coordinate (J) at (4, 3);
            \coordinate (K) at (5, 4);
            \coordinate (L) at (4, 4.5);

            \draw[very thick] (A) -- (B);
            \draw[very thick] (C) -- (D);

            \node[anchor=south west] at (A) {$\pi_1$};
            \node[anchor=west] at (C) {$\pi_2$};

            \draw[-Latex] (F) -- (H);
            \draw[-Latex] (G) -- (I);
            \draw[dotted] (H) -- (K);
            \draw[dotted] (I) -- (L);

            \node[anchor=south east] at (H) {$\vec n_2$};
            \node[anchor=west] at (I) {$\vec n_1$};

            \draw pic [draw, angle radius=9mm, "$\t$"] {angle = D--E--A};
            \draw pic [draw, angle radius=9mm, "$\t$"] {angle = K--J--L};

            \draw pic [draw, angle radius=4mm] {right angle = J--F--C};
            \draw pic [draw, angle radius=4mm] {right angle = A--G--L};
        \end{tikzpicture}
    \end{center}

    It is hence clear that the acute angle between the two planes is equal to the acute angle between the two normal vectors. Thus, \[\cos \t = \frac{\abs{\vec n_1 \dotp \vec n_2}}{\abs{\vec n_1} \abs{\vec n_2}}.\]
\end{proof}

\begin{fact}[Relationship between Two Planes]
    Given two distinct planes $\pi_1 : \vec r \dotp \vec n_1 = d_1$ and $\pi_2 : \vec r \dotp \vec n_2 = d_2$, there are two possible cases:
    \begin{itemize}
        \item \textbf{$\pi_1$ and $\pi_2$ do not intersect.} The two planes are parallel ($\vec n_1 \parallel \vec n_2$).
        \item \textbf{$\pi_1$ and $\pi_2$ intersect at a line.} The two planes are not parallel ($\vec n_1 \nparallel \vec n_2$).
    \end{itemize}
\end{fact}

Suppose the two planes are not parallel to each other. There are two methods to obtain the equation of the line of intersection.

\begin{method}[Via Cartesian Form]
    Write the equations of the two planes in Cartesian form and solve the two equations simultaneously.
\end{method}
\begin{method}[Via Normal Vectors]
    Observe that as the line of intersection $l$ lies on both planes, $l$ is perpendicular to both the normal vectors $\vec n_1$ and $\vec n_2$. Hence, $l$ is parallel to their cross product, $\vec n_1 \crossp \vec n_2$. Thus, if we know a point on the line of intersection $l$ (say point $A$ with position vector $\vec a$), then the vector equation of $l$ is given by \[l : \vec r = \vec a + \l \vec b, \quad \l \in \RR,\] where $\vec b$ is any scalar multiple of $\vec n_1 \crossp \vec n_2$.
\end{method}