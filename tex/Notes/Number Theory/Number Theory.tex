\chapter{Number Theory}

\section{Modular Arithmetic}

\begin{definition}
    Let two integers $a$ and $b$ (with $b \neq 0$). If there exists some integer $n$ such that $a = bn$, we say
    \begin{itemize}
        \item $b$ divides $a$, and
        \item $a$ is divisible by $b$.
    \end{itemize}
    We write this as $b \mid a$.
\end{definition}

\begin{proposition}
    For $a, b, c \in \ZZ$, if $a \mid b$ and $a \mid c$, then $a \mid (b \pm c)$.
\end{proposition}
\begin{proof}
    By definition, there exist integers $x$ and $y$ such that $b = ax$ and $c = ay$. Since $b \pm c = ax \pm ay = a\bp{x \pm y}$, by definition, $a \mid (b \pm c)$.
\end{proof}

\begin{definition}[Congruence Modulo]
    Let $a, b, n \in \ZZ$ with $n > 0$. We say that $a$ is \vocab{congruent} to $b$ \vocab{modulo} $n$ if $n$ divides $a - b$. We write $a \equiv b \pmod{n}$.
\end{definition}

\begin{example}[Congruence Modulo]
    $25 \equiv 7 \pmod{3}$ since $25 - 7 = 18$ is a multiple of 3.
\end{example}

\begin{fact}[Congruence is an Equivalence Relation]
    Let $a, b, n \in \ZZ$.
    \begin{itemize}
        \item (reflexivity) $a \equiv a$ modulo $n$.
        \item (symmetry) If $a \equiv b$ then $b \equiv a$ (modulo n).
        \item (transitivity) If $a \equiv b$ and $b \equiv c$, then $a \equiv c$ (all modulo $n$).
    \end{itemize}
\end{fact}

\begin{fact}[Modular Arithmetic]
    For all integers $a, b, c, d, k, n$, with $n > 1$, suppose $a \equiv b \pmod{n}$ and $c \equiv d \pmod{n}$. Then
    \begin{itemize}
        \item $a \pm c \equiv b \pm d \pmod{n}$.
        \item $a \cdot c \equiv b \cdot d \pmod{n}$.
        \item $a + k \equiv b + k \pmod{n}$.
        \item $ka \equiv kb \pmod{n}$.
        \item $a^m \equiv b^m \pmod{n}$ for all non-negative integers $m$.
    \end{itemize}
\end{fact}

In other words, congruence modulo preserves addition, subtraction, multiplication, and exponentiation. Take not that congruence modulo does \emph{not} always preserve division. That is, if $c \mid a$ and $d \mid b$, it is not always true that \[\frac{a}{c} \equiv \frac{b}{d} \pmod{n}.\]

\section{Divisibility and Primes}

\begin{lemma}[Euclid's Division Lemma]
    Let $n$ be a positive integer. Then for any integer $m$, there exists a unique integer $r$ with $0 \leq r < n$ such that $m \equiv r \pmod{n}$. Equivalently, there exists an integer $q$ such that $m = qn + r$.
\end{lemma}
We will prove this statement for $m, n > 0$. We can take $m > n$ since if $0 < m < n$, we can simply take $q = 0$ and $r = m$.
\begin{proof}
    We prove that such an $r$ exists, and show that it must be unique.

    \item\bd{Existence.} Let $q$ be the largest number such that $m \geq nq$ and let $r = m - nq \geq 0$. Seeking a contradiction, suppose $r \geq n$, i.e. $r = n + d$ for $d \geq 0$. Then \[m = nq + r = nq + \bp{n + d} = n(q+1) + d \geq n(q+1),\] contradicting the maximality of $q$. Hence, $0 \leq r < n$, i.e. $r$ exists.

    \item\bd{Uniqueness.} Suppose there exist $r_1$, $r_2$, with $0 \leq r_1, r_2 < n$ such that \[m = q_1 n + r_1 = q_2 n + r_2.\] Then $r_1 = (q_2 - q_1) n + r_2$. Since $0 \leq r_1, r_2 < n$, we must have $r_1 = r_2$. Hence, $r$ must be unique. This concludes the proof.
\end{proof}

\begin{lemma}[Euclid's Lemma]
    Let $p$ be prime. If $p$ divides $ab$, then $p$ divides $a$ or $p$ divides $b$.
\end{lemma}
\begin{proof}
    Let \[a = \prod_{i = 1}^{k} p_i^{n_i} \quad \tand \quad b = \prod_{j = 1}^{l} q_j^{m_j},\] where $p_i$ and $q_j$ are primes, while $n_i$ and $m_j$ are positive integers. Then \[p \mid ab = \prod_{i = 1}^{k} p_i^{n_i} \prod_{j = 1}^{l} q_j^{m_j}.\] By the uniqueness of prime decomposition, either $p = p_i$ for some $i = 1, \dots, k$ (in which case $p \mid a$), or $p = q_j$ for some $j = 1, \dots, l$ (in which case $p \mid b$). Hence, either $p \mid a$ or $p \mid b$.
\end{proof}

\begin{theorem}
    There are infinitely many primes.
\end{theorem}
\begin{proof}
    Seeking a contradiction, suppose there are finitely many primes $p_1, \dots, p_n$. Consider $a = p_1 \dots p_n + 1$. Since $a > p_1, \dots, p_n$, by our hypothesis, $a$ cannot be a prime, so it is composite and must hence have a prime factorization. Without loss of generality, suppose $p_1$ is a prime factor of $a$. Then $p_1 \mid a$ and $p_1 \mid a - 1 = p_1 \dots p_n$. Hence, by divisibility rules, $p_1$ must divide the difference between $a$ and $a-1$, so $p_1 \mid 1$, implying that $p_1 = 1$. But this is a contradiction as 1 is not a prime. Thus, there are infinitely many primes.
\end{proof}