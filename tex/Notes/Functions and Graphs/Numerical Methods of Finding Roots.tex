\chapter{Numerical Methods of Finding Roots}

\section{Bolzano's Theorem}\label{S:Bolzano}

The following theorem forms the basis for finding roots numerically.

\begin{theorem}[Bolzano's Theorem]
    Let $f(x)$ be a continuous function on the interval $[a, b]$. If $f(a)$ and $f(b)$ have opposite signs, i.e. $f(a) f(b) < 0$, then there exists at least one real root in $[a, b]$.
\end{theorem}

Additionally, if $f(x)$ is strictly monotonic on $[a, b]$, then there is exactly one real root in $[a, b]$.

\section{Numerical Methods for Finding Roots}

A numerical method for finding roots typically consists of two stages:
\renewcommand{\theenumi}{\arabic{enumi}.}%
\begin{enumerate}
    \item \textbf{Estimate the location of the root}
    
    Obtain an initial approximate value of this root.
    \item \textbf{Improve on the estimate (via an iterative process)}

    An iterative process is a repetitive procedure designed to produce a sequence of approximations $\bc{x_n}$ so that the sequence converges to a root. The process is continued until the required accuracy is reached.
\end{enumerate}
\renewcommand{\theenumi}{(\alph{enumi})}

In this chapter, we will look at three numerical methods for finding roots, namely linear interpolation, fixed point iteration and the Newton-Raphson method.

\section{Linear Interpolation}

Linear interpolation is a numerical method based on approximating the curve $y = f(x)$ to a straight line in the vicinity of the root. The approximate root of the equation $f(x) = 0$ is the intersection of this straight line with the $x$-axis.

\subsection{Derivation}

\begin{figure}[H]\tikzsetnextfilename{113}
    \centering
    \begin{tikzpicture}
        \begin{axis}[
                domain = -4.8:-0.5,
                samples = 101,
                axis y line=none,
                axis x line=middle,
                xtick = {-3.68, -2.90},
                xticklabels = {$\a$, $c$},
                ytick = \empty,
                xlabel = {$x$},
                ylabel = {$y$},
            ]
        \addplot[color=red!50] {0.04 * x^3 + 2};
        
        \coordinate[label=left:$A$] (A) at (-4.5, -1.645);
        \coordinate[label=above:$B$] (B) at (-1, 1.96);
        \draw[dashed] (A) -- (B);
        \fill (A) circle[radius=2.5pt];
        \fill (B) circle[radius=2.5pt];
        \end{axis}
    \end{tikzpicture}
    \caption{}
\end{figure}

Suppose $f(x) = 0$ has exactly one root $\a$ in the interval $[a, b]$, where $f(a)$ and $f(b)$ have opposite signs. By the point-slope formula, the line connecting the points $(a, f(a))$ and $(b, f(b))$ is given by \[y - f(a) = \frac{f(b) - f(a)}{b - a} (x - a).\] At the point $(c, 0)$, \[0 - f(a) = \frac{f(b)-f(a)}{b-a} (c -a) \implies c = \frac{af(b) - bf(a)}{f(b)-f(a)}.\]

Linear interpolation can be repeatedly applied by replacing either the lower or upper bound of the interval with the previously found approximation.

\subsection{Convergence}

Convergence of the approximations is guaranteed for linear interpolation. However, how good the estimation is depends on how "straight" the graph of $y = f(x)$ is in $[a, b]$, i.e. the rate at which $f'(x)$ is changing in $[a, b]$. This rate also affects the rate of convergence: if $f'(x)$ changes considerably, the rate of convergence is slow; if $f'(x)$ does not change much, the rate of convergence is fast.

\section{Fixed Point Iteration}

Fixed point iteration is used to find a root of an equation $f(x) = 0$ which can be written in the form $x = F(x)$. The roots of the equation are the abscissae of the points of intersection of the line $y = x$ and $y = F(x)$.

\subsection{Derivation}

Let $\a$ be a root to $f(x) = 0$. Since $f(x) = 0$ can be written in the form $x = F(x)$, we clearly have $\a = F(\a)$. Now observe that we can replace the argument $\a$ with $F(\a)$: \[\a = F(\a) = F \circ F(\a) = F \circ F \circ F(\a) = \dots.\] Assuming certain conditions on $F$ which we will see below, the repeated composition of $F(x)$ converges to the root $\a$: \[\a = F \circ F \circ F \circ \dots \circ F(x).\]

\subsection{Geometrical Interpretation}

Geometrically, fixed-point iteration can be seen as repeatedly "reflecting" the initial approximation point $(x_1, F(x_1))$ about the line $y = x$, while keeping the resultant point on the curve $y = F(x)$.

\begin{figure}[H]\tikzsetnextfilename{331}
    \centering
    \begin{tikzpicture}
        \begin{axis}[
            domain = 0:2,
            samples = 101,
            axis y line=middle,
            axis x line=middle,
            xtick = {1.7, 0.721, 1.112},
            xticklabels = {$x_1$, $x_2$, $x_3$},
            ytick = \empty,
            xlabel = {$x$},
            ylabel = {$y$},
            ymax=1.4,
            xmax=2,
            after end axis/.code={
                \path (axis cs:0,0)
                    node [anchor=north east] {$O$};
                }
            ]
            \addplot[red!50] {cos(\x r) + 0.5 * x};
            \addplot[blue!50] {x};
            \draw[dotted, thick] (1.7, 0) -- (1.7, 0.721);
            \draw[dotted, thick] (0.721, 0.721) -- (0.721, 0);
            \draw[dotted, thick] (1.112, 0.999) -- (1.112, 0);
            \begin{scope}[decoration={
                markings,
                mark=at position 0.5 with {\arrow{>}}}
                ]
                \draw[postaction={decorate}, thick] (1.7, 0.721) -- (0.721, 0.721);
                \draw[postaction={decorate}, thick] (0.721, 0.721) -- (0.721, 1.112);
                \draw[postaction={decorate}, thick] (0.721, 1.112) -- (1.112, 1.112);
                \draw[postaction={decorate}, thick] (1.112, 1.112) -- (1.112, 0.999);
            \end{scope}
        \end{axis}
    \end{tikzpicture}
    \caption{}
\end{figure}

\subsection{Convergence}

Convergence is not guaranteed. The rate at which the approximations converge to $\a$ depends on the value of $\abs{F'(x)}$ near $\a$. The smaller $\abs{F'(x)}$ is, the faster the convergence. It should be noted that fixed-point iteration fails if $\abs{F'(x)} > 1$ near $\a$.\footnote{More rigorously, by the Banach fixed-point theorem, fixed-point iteration converges to $\a$ if and only if $F$ acts as a contraction mapping around $\a$, i.e. $\abs{F'(x)} < 1$ near $x = \a$.}

\begin{figure}[H]\tikzsetnextfilename{332}
    \centering
    \begin{tikzpicture}
        \begin{axis}[
            domain = 0:pi/2,
            samples = 101,
            axis y line=middle,
            axis x line=middle,
            xtick = {1.2, 0.725, 1.497},
            ytick = \empty,
            xticklabels = {$x_1$, $x_2$, $x_3$},
            xlabel = {$x$},
            ylabel = {$y$},
            ymax=2.2,
            xmax=1.7,
            after end axis/.code={
                \path (axis cs:0,0)
                    node [anchor=north east] {$O$};
                }
            ]
    
            \addplot[red!50] {2 *cos(\x r)};
            \addplot[blue!50] {x};
            \draw[dotted, thick] (1.2, 0) -- (1.2, 0.725);
            \draw[dotted, thick] (0.725, 0.725) -- (0.725, 0);
            \draw[dotted, thick] (1.497, 0.147) -- (1.497, 0);
            \begin{scope}[decoration={
                markings,
                mark=at position 0.5 with {\arrow{>}}}
                ]
                \draw[postaction={decorate}, thick] (1.2, 0.725) -- (0.725, 0.725);
                \draw[postaction={decorate}, thick] (0.725, 0.725) -- (0.725, 1.497);
                \draw[postaction={decorate}, thick] (0.725, 1.497) -- (1.497, 1.497);
                \draw[postaction={decorate}, thick] (1.497, 1.497) -- (1.497, 0.147);
            \end{scope}
        \end{axis}
    \end{tikzpicture}
    \caption{Divergence occurs when $\abs{F'(x)} > 1$ near $\a$.}
\end{figure}

\section{Newton-Raphson Method}

The Newton-Raphson method is a numerical method that improves on linear interpolation by considering the tangent line at the initial approximation to the root.

\subsection{Derivation}

\begin{figure}[H]\tikzsetnextfilename{329}
    \centering
    \begin{tikzpicture}[scale=0.8]
        \begin{axis}[
            domain = 1.5:2.5,
            samples = 50,
            axis y line=middle,
            axis x line=middle,
            xtick = {1.8393, 1.994, 2.4},
            xticklabels = {$\alpha$, $x_2$, $x_1$},
            ytick = \empty,
            xlabel = {$x$},
            ylabel = {$y$},
            ymin = -2,
            legend cell align={left},
            legend pos=outer north east,
            after end axis/.code={
                \path (axis cs:1.5,0) 
                    node [anchor=east] {$O$};
                }
            ]
    
            \addplot[red!50] {x^3-x^2-x-1};
    
            \draw[dotted, thick] (2.4, 0) -- (2.4, 4.644);
    
            \addplot[blue!50] {11.48 * (x-2.4) + 4.644};
        \end{axis}
    \end{tikzpicture}
    \caption{}
\end{figure}

Let $\a$ be a root to $f(x) = 0$. Consider the tangent to $y = f(x)$ at the point where $x = x_1$. In most circumstances, the point $(x_2, 0)$ where this tangent cuts the $x$-axis will be nearer to the point $(\a, 0)$ than $(x_1, 0)$ was. By the point-slope formula, the equation of the tangent to the curve at $x = x_1$ is \[y - f(x_1) = f'(x_1)(x-x_1).\] Since $(x_2, 0)$ lies on the tangent line, we have \[x_2 = x_1 - \frac{f(x_1)}{f'(x_1)}.\] By repeating the Newton-Raphson process, we are able to get better approximations to $\a$. In general, \[x_{n+1} = x_n - \frac{f(x_n)}{f'(x_n)}.\]

\subsection{Convergence}

The rate of convergence when using the Newton-Raphson method depends on the first approximation used and the shape of the curve in the neighbourhood of the root. In extreme cases, these factors may lead to failure (divergence). The three main cases are:
\begin{itemize}
    \item $\abs{f'(x_1)}$ is too small (extreme case when $f'(x_1) = 0$),
    \item $f'(x)$ increases/decreases too rapidly ($\abs{f''(x)}$ is too large),
    \item $x_1$ is too far away from $\a$.
\end{itemize}

\begin{figure}[H]\tikzsetnextfilename{330}
    \centering
    \begin{tikzpicture}[scale=0.8]
        \begin{axis}[
            domain = -1.3:1.3,
            samples = 201,
            axis y line=middle,
            axis x line=middle,
            xlabel = {$x$},
            ylabel = {$y$},
            xtick={0.864, 0.5, -1.022},
            xticklabels={$\alpha$, $x_1$, $x_2$},
            ytick=\empty,
            ymin=-0.5,
            ymax=1,
            legend cell align={left},
            legend pos=outer north east,
            after end axis/.code={
                \path (axis cs:0,0) 
                    node [anchor=north east] {$O$};
                }
            ]
            
            \addplot[red!50] {sec(\x r)^2 - e^x};
    
            \draw[dotted] (0.5, 0) -- (0.5, -0.35);
    
            \addplot[blue!50] {-0.23 * (x - 0.5) - 0.35};
        \end{axis} 
    \end{tikzpicture}
    \caption{Divergence occurs when $x_1$ is too far away from $\a$.}
\end{figure}