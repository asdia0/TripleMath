\chapter{Equations and Inequalities}\label{chap:Equations-and-Inequalities}

\section{Quadratic Equations}

In this section, we will look at the properties of quadratic equations as well as their roots.

\begin{proposition}[Quadratic Formula]
    The roots $\a$ and $\b$ of a quadratic equation $ax^2 + bx + c = 0$, where $a \neq 0$ can be found using the quadratic formula: \[\a, \, \b = \frac{-b \pm \sqrt{b^2 - 4ac}}{2a}.\]
\end{proposition}
\begin{proof}
    Complete the square and solve for $x$.
\end{proof}

\begin{definition}
    The expression under the radical, $b^2 - 4ac$, is known as the \vocab{discriminant} and is denoted $\D$.
\end{definition}

\begin{proposition}[Nature of Roots]
    \phantom{.}
    \begin{itemize}
        \item If $\D > 0$, the roots are real and distinct.
        \item If $\D = 0$, the roots are equal.
        \item If $\D < 0$, the roots are complex.
    \end{itemize}
\end{proposition}
\begin{proof}
    Let the roots to the quadratic equation $ax^2 + bx + c = 0$ be $\a$ and $\b$. By the quadratic formula, \[\a, \, \b = \frac{-b}{2a} \pm \frac{\sqrt{\D}}{2a}.\] Clearly, if $\D > 0$, then $\sqrt{D} > 0$, whence the two roots are different. If $\D = 0$, then $\sqrt{D} = 0$, whence $\a = \b = -b/2a$. If $\D < 0$, then $\sqrt{D}$ is not real, whence $\a$ and $\b$ are complex.
\end{proof}

\begin{remark}
    Not only are $\a$ and $\b$ complex, but they are also \emph{complex conjugates}. We will cover this later in \SS\ref{chap:Introduction-to-Complex-Numbers}.
\end{remark}

\begin{proposition}[Vieta's Formulas for Quadratics]
    Let $\a$ and $\b$ be the roots of the quadratic $ax^2 + bx + c = 0$, where $a \neq 0$. Then \[\a + \b = -\frac{b}{a} \quad \tand \quad \a \b = \frac{c}{a}.\]
\end{proposition}
\begin{proof}
    Since $\a$ and $\b$ are roots, we can rewrite the quadratic as \[ax^2 + bx + c = a(x - \a)(x - \b) = a\bs{x^2 - (\a + \b) x + \a\b}.\] Comparing coefficients yields Vieta's formulas.
\end{proof}

\section{System of Linear Equations}

\begin{definition}
    A set of two or more equations to be solved simultaneously is called a \vocab{system of equations}. If the system has only equations that contain unknowns of the \emph{first degree}, it is a \vocab{system of linear equations}.
\end{definition}

\begin{definition}
    A system of equations is said to be \vocab{consistent} if it admits solutions. Conversely, if there are no solutions to the system, it is said to be \vocab{inconsistent}.
\end{definition}

\begin{example}
    The system \[\systeme{3x + 6y = 3,3x+8y=9}\] is consistent, since $x = -5$, $y = 3$ is a solution. On the other hand, the system \[\systeme{3x + 6y = 3, 6x + 12y = 7}\] is inconsistent, as it does not admit any solutions.
\end{example}

\begin{proposition}
    If a system of linear equations is consistent, it either has a unique solution or infinitely many solutions.
\end{proposition}
For simplicity, we prove this claim for a system of linear equations in two variables, say $x$ and $y$. A more general proof requiring concepts from linear algebra can be found in the proof of Proposition~\ref{prop:lin-alg-solution-space}.
\begin{proof}
    Suppose we have two distinct solutions to our system of linear equations, say $(x_1, y_1)$ and $(x_2, y_2)$. Each linear equation in our system is of the form $ax + by = c$, for constants $a$, $b$ and $c$. Both $(x_1, y_1)$ and $(x_2, y_2)$ must satisfy this equation, i.e. $ax_1 + by_1 = c$ and $ax_2 + by_2 = c$. Subtracting these two equations, we see that \[a\bp{x_1 - x_2} + b\bp{y_1 - y_2} = 0.\] Now choose any real number $t$, and define the point \[(x_t, y_t) = \bp{x_1 + t\bp{x_1 - x_2}, \, y_1 + t\bp{y_1 - y_2}}.\] Observe then that this new point satisfies the equation $ax + by = c$:
    \begin{align*}
        ax_t + by_t &= a\bs{x_1 + t\bp{x_1 - x_2}} + b\bs{y_1 + t\bp{y_1 - y_2}}\\
        &= \bp{ax_1 + by_1} + t\bs{a\bp{x_1 - x_2} + b\bp{y_1 - y_2}}\\
        &= c + t(0)\\
        &= c.
    \end{align*}
    Since this works for every equation in the system, it follows that $(x_t, y_t)$ is also a solution to the system for all values of $t$. Thus, the solution has infinitely many solutions.
\end{proof}

Geometrically, each linear equation $ax + by = c$ represents a line. If a collection of lines passes through more than one common point, then the lines must be the same, so each of the infinitely many points on the line is a solution to the system.

\section{Inequalities}

\begin{fact}[Properties of Inequalities]
    Let $a, b, c \in \RR$.
    \begin{itemize}
        \item (transitivity) If $a > b$ and $b > c$, then $a > c$.
        \item (addition) If $a > b$, then $a + c > b + c$.
        \item (multiplication) Suppose $a > b$. If $c > 0$, then $ac > bc$; if $c < 0$, then $ac < bc$.
    \end{itemize}
\end{fact}

There are two main methods of solving inequalities.

\begin{recipe}[Graphical Method]
    Plot the function and observe which $x$-values satisfy the inequality.
\end{recipe}
\begin{recipe}[Test-Value Method]
    \phantom{.}
    \renewcommand{\theenumi}{\arabic{enumi}.}
    \begin{enumerate}
        \item Rearrange the inequality so that all terms are on one side, resulting in a form like $f(x) > 0$.
        \item Mark the root(s) of $f(x) = 0$ on a number line.
        \item Divide the number line into intervals based on these roots and select a test value from each interval.
        \item Determine the sign of $f(x)$ in each interval by substituting the corresponding test value.
        \item Construct the solution set.
        \begin{itemize}
            \item Include intervals where $f(x)$ has the desired sign.
            \item If the inequality is not strict (e.g. $f(x) \geq 0$), include also the roots of $f(x) = 0$.
        \end{itemize}
    \end{enumerate}
    \renewcommand{\theenumi}{(\alph{enumi})}
\end{recipe}

\begin{sample}[Test-Value Method]
    Solve the inequality $2x - x^2 \geq -3$.
\end{sample}
\begin{sampans}
    Rearranging the inequality, we get $x^2 - 2x - 3 \leq 0$. Let $f(x) = x^2 - 2x - 3$. The roots of $f(x) = 0$ are $x = -1$ and $x = 3$. We thus divide the number line into the intervals $(-\infty, -1)$, $(-1, 3)$ and $(3, \infty)$. Within each interval, we pick a test-value, say $x = -2$, $x = 0$ and $x = 4$. Evaluating $f(x)$ at these test-values, we see that $f(x)$ is negative only on the interval $(-1, 3)$. Since the inequality is not strict, we include the roots $x = -1$ and $x = 3$, to get the solution set $[-1, 3]$.
\end{sampans}

Note that the test-value method implicitly assumes that $f(x)$ is continuous; if $f(x)$ is discontinuous, the sign of $f(x)$ between two roots may change. Hence, for most discontinuous functions, we can only use the graphical method.

However, in the case where $f(x) = g(x)/h(x)$ is a quotient of two continuous functions, we can multiply the inequality throughout by the square of the denominator, $\bs{h(x)}^2$, and proceed with the test-value method. Note that the square ensures that the direction of the inequality is preserved.

\section{Modulus Function}

\begin{definition}
    The modulus function $\abs{x}$, where $x \in \RR$, is defined as \[\abs{x} = \begin{cases}
        \phantom{-}x & \text{if $x \geq 0$,}\\
        -x & \text{if $x < 0$.}
    \end{cases}\]
\end{definition}

The modulus function can be thought of as the ``distance'' between a number and the origin (the number 0) on the real number line.

\begin{fact}[Properties of Modulus Function]
    For any $x \in \RR$ and $k > 0$,
    \begin{itemize}
        \item $\abs{x} \geq 0$.
        \item $\abs{x^2} = \abs{x}^2 = x^2$ and $\sqrt{x^2} = \abs{x}$.
        \item $\abs{x} < k \iff -k < x < k$.
        \item $\abs{x} = k \iff x = -k \tor x = k$.
        \item $\abs{x} > k \iff x < -k \tor x > k$.
    \end{itemize}
\end{fact}