\chapter{Vectors}

\section{Basic Definitions and Notations}

\begin{definition}
    A \vocab{vector} is an object that has both magnitude and direction.
\end{definition}

To visualize a vector, we often represent it graphically by an arrow. An arrow from a point $P$ to a point $Q$ represents a vector, which we write as $\oa{PQ}$. The \vocab{magnitude} of the vector, denoted $\abs{\oa{PQ}}$, is the length of the arrow.

When working abstractly, vectors are usually denoted by bold print (e.g. $\vec a$).

\begin{definition}
    Two vectors $\vec a$ and $\vec b$ are said to be \vocab{equal vectors} if they both have the same magnitude and direction, and we write $\vec a = \vec b$. If instead they have the same magnitude but opposite directions, they are said to be \vocab{negative vectors}, and we write $\vec a = -\vec b$.
\end{definition}

\begin{definition}[Multiplication of a Vector by a Scalar]
    Let $\l$ be a scalar. If $\l > 0$, then $\l \vec a$ is a vector of magnitude $\l \abs{\vec a}$ and has the same direction as $\vec a$. If $\l < 0$, then $\l \vec a$ is a vector of magnitude $-\l \abs{\vec a}$ and is in the opposite direction of $\vec a$.
\end{definition}

 \begin{figure}[H]\tikzsetnextfilename{545}
    \centering
    \begin{tikzpicture}
        \draw[->] (0, 0) -- (2, 1) node[anchor=south west] {$\vec a$};
        \draw[->] (2.5, 0) -- (6.5, 2) node[anchor=south west] {$2\vec a$};
        \draw[->] (9, 1) -- (7, 0) node[anchor=north east] {$-\vec a$};
    \end{tikzpicture}
    \caption{}
\end{figure}

\begin{definition}
    The \vocab{zero vector} is the vector with a magnitude of 0 and is denoted $\vec 0$.
\end{definition}

\begin{definition}
    Let $\vec a$ and $\vec b$ be non-zero vectors. Then $\vec a$ and $\vec b$ are said to be \vocab{parallel} if $\vec b$ can be expressed as a non-zero scalar multiple of $\vec a$, That is, $\vec b = \l \vec a$ for some $\l \neq 0$.
\end{definition}

\begin{definition}
    A \vocab{unit vector} is a vector with a magnitude of 1.
\end{definition}

Unit vectors are typically denoted with a hat (e.g. $\hat{\vec a}$).

For any non-zero vector $\vec a$, the unit vector parallel to $\vec a$ is given by \[\hat{\vec a} = \frac{\vec a}{\abs{\vec a}}.\]

\begin{definition}
    The \vocab{Triangle Law of Vector Addition} states that $\oa{AB} + \oa{BC} = \oa{AC}$. Geometrically, we add two vectors $\vec a$ and $\vec b$ by placing them head to tail, taking the resultant vector as their sum.

    \begin{figure}[H]\tikzsetnextfilename{335}
        \centering
        \begin{tikzpicture}
            \coordinate[label=left:$A$] (A) at (0, 0);
            \coordinate[label=above:$B$] (B) at (2, 1);
            \coordinate[label=right:$C$] (C) at (5, 1);

            \draw[->-=0.5] (A) -- (B);
            \draw[->-=0.5] (B) -- (C);
            \draw[->-=0.5] (A) -- (C);

            \node[anchor=south east] at (1, 0.5) {$\vec a$};
            \node[anchor=south] at (3.5, 1) {$\vec b$};
            \node[anchor=north west, rotate=11.3] at (2, 0.2) {$\vec a + \vec b$};
        \end{tikzpicture}
        \caption{}
    \end{figure}
\end{definition}

To subtract vectors, we add $\vec a$ and $-\vec b$ using the same principle.

\begin{definition}
    The \vocab{angle between two vectors} refers to the angle between their directions when the arrows representing them both converge or both diverge.
\end{definition}

\begin{definition}
    The \vocab{position vector} of some point $A$ relative to the origin $O$ is unique and is denoted $\oa{OA}$.
\end{definition}

\begin{definition}
    A set of vectors are said to be \vocab{coplanar} if their directions are all parallel to the same plane.
\end{definition}

\begin{fact}
    Any vector $\vec c$ that is coplanar with $\vec a$ and $\vec b$ can be expressed as a \bd{unique linear combination} of $\vec a$ and $\vec b$. That is, there exist unique scalars $\l$ and $\m$ such that $\vec c = \l \vec a + \m \vec b$.
\end{fact}

\begin{theorem}[Ratio Theorem]
    If $P$ divides $AB$ in the ratio $\l : \m$, then \[\oa{OP} = \frac{\m \vec a + \l \vec b}{\l + \m}.\]
\end{theorem}
\begin{proof}
    Since $P$ divides $AB$ in the ratio $\l : \m$, we have \[\oa{AP} = \frac{\l}{\l + \m} \oa{AB} = \frac{\l}{\l + \m} (\vec b - \vec a).\] Thus, \[\oa{OP} = \oa{OA} + \oa{AP} = \vec a + \frac{\l}{\l + \m} (\vec b - \vec a) = \frac{\m \vec a + \l \vec b}{\l + \m}.\]
\end{proof}

\begin{corollary}[Mid-Point Theorem]
    If $P$ is the mid-point of $AB$, then \[\oa{OP} = \frac{\vec a + \vec b}{2}.\]
\end{corollary}

\section{Vector Representation using Cartesian Unit Vectors}

\subsection{2-D Cartesian Unit Vectors}

\begin{definition}[2-D Cartesian Unit Vectors]
    In the 2-D Cartesian plane, $\vec i = \cveciix10$ is defined to be the unit vector in the positive direction of the $x$-axis, while $\vec j = \cveciix01$ is defined to be the unit vector in the positive direction of the $y$-axis.
\end{definition}

Thus, if $P$ is the point with coordinates $(a, b)$, then we can express $\oa{OP}$ in terms of the unit vectors $\vec i$ and $\vec j$. In particular, $\oa{OP} = a\vec i + b \vec j$.

\begin{proposition}[Magnitude in 2-D]
    \[\abs{\cvecii{a}{b}} = \sqrt{a^2 + b^2}.\]
\end{proposition}
\begin{proof}
    Follows immediately from Pythagoras' theorem.
\end{proof}

\subsection{3-D Cartesian Unit Vectors}

\begin{definition}[3-D Cartesian Unit Vectors]
    In the 3-D Cartesian plane, $\vec i = \cveciiix100$, $\vec j = \cveciiix010$ and $\vec k = \cveciiix001$ denote the unit vectors in the positive direction of the $x$, $y$ and $z$-axes respectively.
\end{definition}

\begin{proposition}[Magnitude in 3-D]
    \[\abs{\cveciii{a}{b}{c}} = \sqrt{a^2 + b^2 + c^2}.\]
\end{proposition}
\begin{proof}
    Use Pythagoras' theorem twice.
\end{proof}

\begin{fact}[Operations on Cartesian Vectors]
    To add vectors given in Cartesian unit vector form, the coefficients of $\vec i$, $\vec j$ and $\vec k$ are added separately. \[\cveciii{x_1}{y_1}{z_1} + \cveciii{x_2}{y_2}{z_2} = \cveciii{x_1+x_2}{y_1+y_2}{z_1+z_2}.\] Subtraction and scalar multiplication follows immediately.
\end{fact}

\section{Scalar Product}

\begin{definition}
    The \vocab{scalar product} (or dot product) of two vectors $\vec a$ and $\vec b$ is defined as $\vec a \dotp \vec b = \abs{\vec a} \abs{\vec b} \cos{\t}$, where $\t$ is the angle between the two vectors (note that $0 \leq \t \leq \pi$).
\end{definition}
\begin{remark}
    $\vec a \dotp \vec b$ is called the scalar product as the result is a real number (a scalar). It is also called the dot product because of the notation.
\end{remark}

\begin{fact}[Algebraic Properties of Scalar Product]
    Let $\vec a$, $\vec b$ and $\vec c$ be vectors and let $\l$ be a scalar. Then
    \begin{itemize}
        \item (commutativity) $\vec a \dotp \vec b = \vec b \dotp \vec a$.
        \item (distributivity over addition) $\vec a \dotp (\vec b + \vec c) = \vec a \dotp \vec b + \vec a \dotp \vec c$.
        \item $\vec a \dotp \vec a = \abs{\vec a}^2$.
        \item $(\l \vec a) \dotp \vec b = \vec a \dotp (\l \vec b) = \l \bp{\vec a \dotp \vec b}$.
    \end{itemize}
\end{fact}

\begin{proposition}[Geometric Properties of Scalar Product]
    Let $\vec a$ and $\vec b$ be non-zero vectors, and let $\t$ be the angle between them.
    \begin{itemize}
        \item $\vec a \dotp \vec b = 0$ if and only if $\t = \pi/2$, i.e. $\vec a$ is perpendicular to $\vec b$.
        \item $\vec a \dotp \vec b > 0$ if and only if $\t$ is acute.
        \item $\vec a \dotp \vec b < 0$ if and only if $\t$ is obtuse.
    \end{itemize}
\end{proposition}
\begin{proof}
    The sign of $\vec a \dotp \vec b$ is determined solely by $\cos \t$.
\end{proof}

\begin{proposition}[Scalar Product in Cartesian Unit Vector Form]
    \[\cveciii{x_1}{y_1}{z_1} \dotp \cveciii{x_2}{y_2}{z_2} = x_1 x_2 + y_1 y_2 + z_1 z_2.\]
\end{proposition}
\begin{proof}
    Since $\vec i$, $\vec j$ and $\vec k$ are pairwise perpendicular, their pairwise scalar products are 0. That is, $\vec i \dotp \vec j = \vec j \dotp \vec k = \vec k \dotp \vec i = 0$. Also, because $\vec i$, $\vec j$ and $\vec k$ are all unit vectors, $\vec i \dotp \vec i = \vec j \dotp \vec j = \vec k \dotp \vec k = 1$. Thus, by the distributive property of the scalar product,
    \begin{align*}
        (x_1 \vec i + y_1 \vec j + z_1 \vec k) \dotp (x_2 \vec i + y_2 \vec j + z_2 \vec k) &= x_1 x_2 \vec i \dotp \vec i + y_1 y_2 \vec j \dotp \vec j + z_1 z_2 \vec k \dotp \vec k\\
        &= x_1 x_2 + y_1 y_2 + z_1 z_2.
    \end{align*}
\end{proof}

\subsection{Applications of Scalar Product}

\begin{proposition}[Angle between Two Vectors]
    Let $\t$ be the angle between two non-zero vectors $\vec a$ and $\vec b$. Then \[\cos \t = \frac{\vec a \dotp \vec b}{\abs{\vec a} \abs{\vec b}}.\]
\end{proposition}
\begin{proof}
    Follows immediately from the definition of the scalar product.
\end{proof}

\begin{definition}
    Let $\vec a$ and $\vec b$ denote the position vectors of $A$ and $B$ respectively, relative to the origin $O$. Let $\t$ be the angle between $\vec a$ and $\vec b$, and let $N$ be the foot of the perpendicular from the point $A$ to the line passing through $O$ and $B$.

    Then, the length $ON$ is defined to be the \vocab{length of projection} of the vector $\vec a$ onto the vector $\vec b$. Also, $\oa{ON}$ is the \vocab{vector projection} of $\vec a$ onto $\vec b$.
\end{definition}

\begin{proposition}[Length of Projection]
    The length of projection of $\vec a$ onto $\vec b$ is $\abs{\vec a \dotp \hat{\vec b}}$.
\end{proposition}
\begin{proof}
    Consider the case where $\t$ is acute.

    \begin{figure}[H]\tikzsetnextfilename{336}
        \centering
        \begin{tikzpicture}
            \coordinate[label=right:$A$] (A) at (2, 2);
            \coordinate[label=right:$B$] (B) at (3, 0);
            \coordinate[label=below:$N$] (N) at (2, 0);
            \coordinate[label=below left:$O$] (O) at (0, 0);
    
            \draw[->] (O) -- (A);
            \draw[->] (O) -- (B);
            \draw[dotted] (A) -- (N);

            \node[anchor=south east] at (1, 1) {$\vec a$};
            \node[anchor=north] at (0.5, 0) {$\vec b$};
    
            \draw pic [draw, angle radius=3mm] {right angle = B--N--A};
            \draw pic [draw, angle radius=10mm, "$\t$"] {angle = B--O--A};
        \end{tikzpicture}
        \caption{\label{fig:336}}
    \end{figure}

    From the diagram, \[ON = OA \cos \t = \abs{\vec a} \frac{\vec a \dotp \vec b}{\abs{\vec a} \abs{\vec b}} = \frac{\vec a \dotp \vec b}{\abs{\vec b}} = \vec a \dotp \hat{\vec b}.\] A similar argument shows that when $\t$ is obtuse, $ON = -\vec a \dotp \hat{\vec b}$. Hence, in any case, $ON = \abs{\vec a \dotp \hat{\vec b}}$.
\end{proof}

\begin{corollary}[Vector Projection]
    The vector projection of $\vec a$ onto $\vec b$ is $\bp{\vec a \dotp \hat{\vec b}} \hat{\vec b}$.
\end{corollary}

\section{Vector Product}

\begin{definition}
    The \vocab{vector product} (or cross product) of two vectors $\vec a$ and $\vec b$ is denoted by $\vec a \crossp \vec b$ and is defined as $\vec a \crossp \vec b = \abs{\vec a}\abs{\vec b} \sin{\t} \hat{\vec n}$, where $\t$ is the angle between $\vec a$ and $\vec b$ and $\hat{\vec n}$ is the unit vector perpendicular to both $\vec a$ and $\vec b$, in the direction determined by the right-hand grip rule.
\end{definition}
\begin{remark}
    $\vec a \crossp \vec b$ is called the vector product as the result is a vector. It is also called the cross product due to its notation.
\end{remark}

\begin{fact}[Algebraic Properties of Vector Product]
    Let $\vec a$, $\vec b$ and $\vec c$ be three vectors, and $\t$ be the angle between $\vec a$ and $\vec b$.
    \begin{itemize}
        \item (anti-commutativity) $\vec a \crossp \vec b = -\vec b \crossp \vec a$.
        \item (distributivity over addition) $\vec a \crossp (\vec b + \vec c) = (\vec a \crossp \vec b) + (\vec a \crossp \vec c)$.
        \item $(\l \vec a) \crossp \vec b = \vec a \crossp (\l \vec b) = \l \bp{\vec a \crossp \vec b}$, where $\l$ is a scalar.
    \end{itemize}
\end{fact}

\begin{proposition}[Geometric Properties of Vector Product]
    Let $\vec a$ and $\vec b$ be non-zero vectors and $\t$ be the angle between them.
    \begin{itemize}
        \item $\abs{\vec a \crossp \vec b} = 0$ if and only if $\vec a$ is parallel to $\vec b$.
        \item $\abs{\vec a \crossp \vec b} = \abs{\vec a} \abs{\vec b}$ if and only if $\vec a$ is perpendicular to $\vec b$.
    \end{itemize}
\end{proposition}
\begin{proof}
    Since $\hat{\vec n}$ is a unit vector, $\abs{\vec a \crossp \vec b} = \abs{\vec a} \abs{\vec b} \sin \t$. The claim follows upon considering $\t = 0$, $\pi/2$, and $\pi$.
\end{proof}

\begin{proposition}[Vector Product in Cartesian Unit Vector Form]
    \[\cveciii{x_1}{y_1}{z_1} \dotp \cveciii{x_2}{y_2}{z_2} = \cveciii{y_1 z_2 - z_1 y_2}{z_1 x_2 - x_1 z_2}{x_1 y_2 - y_1 x_2}.\]
\end{proposition}
\begin{proof}
    From the geometric properties of the vector product, we have $\vec i \crossp \vec i = \vec j \crossp \vec j = \vec k \crossp \vec k = \vec 0$. Furthermore, since $\vec i$, $\vec j$ and $\vec k$ are pairwise perpendicular, by the right-hand grip rule, one has $\vec i \crossp \vec j = \vec k$, $\vec j \crossp \vec k = \vec i$, and $\vec k \crossp \vec i = \vec j$. Hence, by the distributive property of the vector product,
    \begin{align*}
        &(x_1 \vec i + y_1 \vec j + z_1 \vec k) \crossp (x_2 \vec i + y_2 \vec j + z_2 \vec k) \\
        &\hspace{2em}= x_1 y_2 \vec k + x_1 z_2 (-\vec j) + y_1 x_2 (-\vec k) + y_1 z_2 \vec i + z_1 x_2 \vec j + z_1 y_2 (-\vec i) \\
        &\hspace{2em}=(y_1 z_2 - z_1 y_2) \vec i + (z_1 x_2 - x_1 z_2) \vec j + (x_1 y_2 - y_1 x_2) \vec k.
    \end{align*}
\end{proof}

\subsection{Applications of Vector Product}

\begin{proposition}[Length of Side of Right-Angled Triangle]
    Let $\vec a$ and $\vec b$ denote the position vectors of $A$ and $B$ respectively, relative to the origin $O$. Let $\t$ be the angle between $\vec a$ and $\vec b$, and let $N$ be the foot of the perpendicular from $A$ to $OB$. Then $AN = \abs{\vec a \crossp \hat{\vec b}}$.
\end{proposition}
\begin{proof}
    With reference to Fig.~\ref{fig:336}, we have \[AN = OA \sin \t = \abs{\vec a} \frac{\abs{\vec a \crossp \vec b}}{\abs{\vec a}\abs{\vec b}} = \frac{\abs{\vec a \crossp \vec b}}{\vec b} = \abs{\vec a \crossp \hat{\vec b}}.\]
\end{proof}

\begin{proposition}[Area of Triangles and Parallelogram]
    Let $ABCD$ be a parallelogram, let $\vec a = \oa{AB}$ and $\vec b = \oa{AC}$, and let $\t$ be the angle between $\vec a$ and $\vec b$. Then \[[\triangle ABC] = \frac12 \abs{\vec a \crossp \vec b} \quad \tand \quad [ABCD] = \abs{\vec a \crossp \vec b}.\]
\end{proposition}
\begin{proof}
    Recall that the formula for the area of a triangle is \[[\triangle ABC] = \frac12 (AB)(AC) \sin \t = \frac12 \abs{\vec a} \abs{\vec b} \sin \t = \frac12 \abs{\vec a \crossp \vec b}.\] Since the area of parallelogram $ABCD$ is twice that of $\triangle ABC$, we immediately have $[ABCD] = \abs{\vec a \crossp \vec b}$.
\end{proof}