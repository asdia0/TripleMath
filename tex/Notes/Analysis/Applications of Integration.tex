\chapter{Applications of Integration}

\section{Area}\label{sec:Applications-of-Integration:Area}

\subsection{The Riemann Sum and Integral}

Suppose we wish to find exact area bounded by the graph of $y = f(x)$, the $x$-axis and the lines $x = a$ and $x = b$, where $a \leq b$ and $f(x) \geq 0$ for $a \leq x \leq b$. 

We can approximate this area by drawing $n$ rectangles of equal width, as shown in the diagram below:

\begin{figure}[H]\tikzsetnextfilename{364}
    \centering
    \begin{tikzpicture}[trim axis left, trim axis right]
        \begin{axis}[
            domain = 0:1.2,
            samples = 101,
            axis y line=middle,
            axis x line=middle,
            xtick = {0.1, 1},
            xticklabels = {$a$, $b$},
            ytick = \empty,
            xlabel = {$x$},
            ylabel = {$y$},
            legend cell align={left},
            legend pos=outer north east,
            after end axis/.code={
                \path (axis cs:0,0) 
                    node [anchor=north east] {$O$};
                }
            ]
            \addplot[plotRed] {x + sin(pi * \x r)};

            \addlegendentry{$y = f(x)$};

            \draw[plotBlue, fill=plotBlue, fill opacity=0.3] (0.1, 0) rectangle (0.2, 0.788);
            \draw[plotBlue, fill=plotBlue, fill opacity=0.3] (0.2, 0) rectangle (0.3, 1.11);
            \draw[plotBlue, fill=plotBlue, fill opacity=0.3] (0.3, 0) rectangle (0.4, 1.35);
            \draw[plotBlue, fill=plotBlue, fill opacity=0.3] (0.4, 0) rectangle (0.5, 1.5);

            \draw[plotBlue, fill=plotBlue, fill opacity=0.3] (0.8, 0) rectangle (0.9, 1.21);
            \draw[plotBlue, fill=plotBlue, fill opacity=0.3] (0.9, 0) rectangle (1, 1);

            \node[plotBlue] at (0.65, 0.5) {$\ldots$};
        \end{axis}
    \end{tikzpicture}
    \caption{}
\end{figure}

Observe that the $k$th rectangle has width $\D x = (b-a)/n$ and height $f(a + k \D x)$. The total area of the rectangles is hence \[\sum_{k = 1}^n f(a + k \D x) \D x.\] This is known as the \vocab{Riemann sum} of $f$ over $[a, b]$.

As the number of rectangles approaches $\infty$, the width $\D x$ of the rectangles approaches 0, and the total area of rectangles approaches the actual area under the curve. In other words, \[\area = \lim_{\D x \to 0} \sum_{k = 1}^n f(a + k \D x) \D x.\] In the limit, the Riemann sum becomes the \vocab{Riemann integral}, which is conventionally written as the definite integral \[\int_a^b f(x) \d x.\]

Note that this is where the integral and differential sign comes from: in the limit, $\sum \to \int$ and $\D x \to \d x$.

\subsection{Definite Integral as the Area under a Curve}

\begin{proposition}[Area between a Curve and the $x$-axis]
    Let $A$ denote the area bounded by the curve of $y = f(x)$, the $x$-axis and the lines $x = a$ and $x = b$. Then \[\area A = \int_a^b \abs{y} \d x = \int_a^b \abs{f(x)} \d x.\]
\end{proposition}

\begin{proposition}[Area between Two Curves]
    The area $A$ between two curves $y = f(x)$ and $y = g(x)$ is given by \[\area A = \int_a^b \abs{f(x) - g(x)} \d x.\]
\end{proposition}

Similar results hold when integrating with respect to the $y$-axis instead.

\begin{proposition}[Area between a Parametric Curve and the $x$-axis]
    Let $C$ be the curve with parametric equations $x = f(t)$ and $y = g(t)$. Then the area $A$ bounded between $C$ and the $x$-axis is \[\area A = \int_a^b \abs{y} \d x = \int_{t_1}^{t_2} \abs{g(t)} \der{x}{t} \d t,\] where $t_1$ and $t_2$ are the values of $t$ when $x = a$ and $b$ respectively.
\end{proposition}

The formula can be applied similarly when we wish to find the area bounded between $C$ and the $y$-axis.

\begin{proposition}[Area Enclosed by Polar Curve]
    Let $r = f(\t)$ be a polar curve, and let $A$ be the area of the region bounded by a segment of the curve and two half-lines $\t = \a$ and $\t = \b$. Then \[\area A = \frac12 \int_{\a}^{\b} r^2 \d \t.\]
\end{proposition}
\begin{proof}
    Divide the enclosed region $A$ into $n$ sectors with the same interior angle $\D \t$. Consider that a typical sector of $A$ can be approximated by a sector of a circle. Thus, the area of that sector is approximately \[\D A \approx \frac12 r^2 \D \t.\] Summing up these approximations, we see that \[A \approx \sum_{\t = \a}^{\t = \b} \frac12 r^2 \D \t.\] This approximation will improve as the number of sectors increases, i.e. $\D \t \to 0$. Hence, \[\area A = \lim_{\D \t \to 0} \sum_{\t = \a}^{\t = \b} \frac12 r^2 \D \t = \frac12 \int_{\a}^{\b} r^2 \d \t.\]
\end{proof}

\section{Volume}

\begin{definition}
    If an enclosed region is rotated about a straight line, the three-dimensional object formed is called a \vocab{solid of revolution}, and its volume is a \vocab{volume of revolution}.
\end{definition}

The line about which rotation takes place is always an axis of symmetry for the solid of revolution, and any cross-section of the solid which is perpendicular to the axis of rotation is circular.

\subsection{Disc Method}

Consider the solid of revolution formed when the region bounded between $y = f(x)$, the $x$-axis and the lines $x = a$ and $x = b$ is rotated about the $x$-axis.

\begin{figure}[H]\tikzsetnextfilename{365}
    \centering
    \begin{tikzpicture}
        \begin{axis}[thick,ticks=none,
            scale only axis,
            axis lines=middle,
            ymin=-5,
            ymax=5,
            xmin=-1,
            xmax=3,
            xlabel = {$x$},
            ylabel = {$y$},
            ]
            \addplot[name path=f,thick,black,samples=1000,domain=0.2:2]{x^2+1};
            \addplot[name path=g,thick,black,samples=1000,domain=0.2:2]{-x^2-1};
            \draw[thick,black] (2, -5) arc(-90:90:0.5 and 5);
            \draw[thick,black] (1, -2) arc(-90:90:0.1 and 2);
            \draw[thick,dashed] (1, 2) arc(90:270:0.1 and 2);
            \draw[thick,black] (1.2, -2.44) arc(-90:90:0.17 and 2.44);
            \draw[thick,black] (0.2, 0) ellipse [x radius=0.05,y radius=1.04];

            \fill (1, 2) circle[radius=2.5pt] node[anchor=south east] {$P(x, y)$};

            \draw[->] (0.8, -1.22) -- (1.04, -1.22);
            \draw[->] (1.64, -1.22) -- (1.4, -1.22);
            \node[anchor=west] at (1.63, -1.22) {$\D x$};
        \end{axis}
    \end{tikzpicture}
    \caption{}
\end{figure}

To calculate the volume of this solid, we can cut it into thin slices (or discs) of thickness $\D x$. Each disc is approximately a cylinder and the approximate volume of the solid can be found by summing the volumes of these cylinders. The smaller $\D x$ is, the better the approximation.

Consider a typical disc formed by a one cut through the point $P(x, y)$ and the other cut distant $\D x$ from the first. The volume of this disc is approximately \[\D V \approx \pi y^2 \D x.\] Summing over all discs, \[V \approx \sum_{x = a}^{b} \pi y^2 \D x.\] As more cuts are made, $\D x \to 0$, whence \[V = \lim_{\D x \to 0} \sum_{x = a}^{b} \pi y^2 \D x = \pi \int_a^b y^2 \d x.\]

\begin{proposition}[Disc Method]
    When the region bound by the curve $y = f(x)$, the $x$-axis and the lines $x = a$ and $x = b$ is rotated $2\pi$ radians about the $x$-axis, the volume of the solid of revolution generated is given by \[V = \pi \int_a^b y^2 \d x = \pi \int_a^b \bs{f(x)}^2 \d x.\]
\end{proposition}

\begin{proposition}[Disc Method: Volume Enclosed by Two Curves]
    When the region enclosed by two curves $y = f(x)$ and $y = g(x)$ is rotated $2\pi$ radians about the $x$-axis, the volume of the solid of revolution generated is given by \[V = \pi \int_a^b \bs{f(x)}^2 \d x - \pi \int_a^b \bs{g(x)}^2 \d x = \pi \int_a^b \bp{\bs{f(x)}^2 - \bs{g(x)}^2} \d x.\]
\end{proposition}

Similar results hold when the axis of rotation is the $y$-axis.

\subsection{Shell Method}

Suppose a region $R$ is rotated about the $y$-axis. Consider a typical vertical strip in the region $R$ with height $y$ and thickness $\D x$. It will form a cylindrical shell with inner radius $x$, outer radius $x + \D x$ and height $y$ when rotated about the $y$-axis. Hence, it has volume \[\D V = \pi (x + \D x)^2 y - \pi x^2 y = 2 \pi xy \D x + \pi \D x^2 y \approx 2 \pi xy \D x.\] Hence, the volume of revolution is approximately \[V \approx \sum_{x = a}^b 2 \pi xy \D x.\] As more strips are considered, $\D x \to 0$, whence \[V = \lim_{\D x \to 0} 2\pi \int_a^b xy \d x.\]

\begin{proposition}[Shell Method]
    When the region bound by the curve $y = f(x)$, the $x$-axis and the lines $x = a$ and $x = b$ is rotated $2\pi$ radians about the $y$-axis, the volume of the solid of revolution is given by \[V = 2\pi \int_a^b xy \d x.\]
\end{proposition}

A similar result holds when the axis of rotation is the $x$-axis.

\section{Arc Length}

\subsection{Parametric Form}

\begin{proposition}[Arc Length of Parametric Curve]
    Let $A(t_1)$ and $B(t_2)$ be points the parametric curve with equations $x = f(t)$, $y = g(t)$, $t \in [t_1, t_2]$. Then \[\overbow{AB} = \int_{t_1}^{t_2} \sqrt{\bs{f'(t)}^2 + \bs{g'(t)}^2} \d t = \int_{t_1}^{t_2} \sqrt{\bp{\der{x}{t}}^2 + \bp{\der{y}{t}}^2} \d t.\]
\end{proposition}
\begin{proof}
    Let $s = \overbow{AB}$ be the arc length of $AB$. Let $P$ and $Q$ be points on $AB$ with parameters $t$ and $t + \D t$ respectively. By the Pythagorean theorem, the straight line $PQ$ is given by \[PQ^2 = \bs{f(t + \D t) - f(t)}^2 + \bs{g(t + \D t) - g(t)}.\] Dividing both sides by $(\D t)^2$, \[\bp{\frac{PQ}{\D t}}^2 = \bs{\frac{f(t + \D t) - f(t)}{\D t}}^2 + \bs{\frac{g(t + \D t) - g(t)}{\D t}}^2.\] As $\D t \to 0$, we can write the RHS in terms of $f'(t) $ and $g'(t)$: \[\lim_{\D t \to 0} \bp{\frac{PQ}{\D t}}^2 = \bs{f'(t)}^2 + \bs{g'(t)}^2.\] Rearranging, \[\lim_{\D t \to 0} PQ = \sqrt{\bs{f'(t)}^2 + \bs{g'(t)}^2} \D t.\] However, observe that as $\D t \to 0$, the straight line $PQ$ approximates the arc length $PQ$ (i.e. $\D s$) better and better. Hence, \[\D s = \overbow{PQ} = \sqrt{\bs{f'(t)}^2 + \bs{g'(t)}^2} \D t.\] Integrating from $A$ to $B$, we thus obtain \[s = \overbow{AB} = \int_{t_1}^{t_2} \sqrt{\bs{f'(t)}^2 + \bs{g'(t)}^2} \d t = \int_{t_1}^{t_2} \sqrt{\bp{\der{x}{t}}^2 + \bp{\der{y}{t}}^2} \d t.\]
\end{proof}

\subsection{Cartesian Form}

Taking $t = x$ or $t = y$, we get the following formulas involving $\derx{y}{x}$ and $\derx{x}{y}$, which is suitable for Cartesian curves.

\begin{proposition}[Arc Length of Cartesian Curve]
    Let $A(x_1, y_1)$ and $B(x_2, y_2)$ be points on the curve $y = f(x)$. The arc length $AB$ is given by \[\overbow{AB} = \int_{x_1}^{x_2} \sqrt{1 + \bp{\der{y}{x}}^2} \d x = \int_{y_1}^{y_2} \sqrt{\bp{\der{x}{y}}^2 + 1} \d y.\]
\end{proposition}

\subsection{Polar Form}

\begin{proposition}
    Let $A(r_1, \t_1)$ and $B(r_2, \t_2)$ be points on the polar curve $r = f(\t)$. Then the arc length $AB$ is given by \[\overbow{AB} = \int_{\t_1}^{\t_2} \sqrt{r^2 + \bp{\der{r}{\t}}^2} \d \t.\]
\end{proposition}
\begin{proof}
    Recall that $x = r\cos \t$ and $y = r\sin \t$. Hence, \[\der{x}{\t} = \cos \t \der{r}{\t} - r \sin \t, \qquad \der{y}{\t} = \sin \t \der{r}{t} + r \cos \t.\] It follows that \[\bp{\der{(r\cos \t)}{\t}}^2 + \bp{\der{(r\sin \t)}{\t}}^2 = \bp{\cos^2 \t + \sin^2 \t} \bs{r^2 + \bp{\der{r}{\t}}^2} = r^2 + \bp{\der{r}{\t}}^2.\] Taking $t = \t$, \[\overbow{AB} = \int_{\t_1}^{\t_2} \sqrt{\bp{\der{x}{\t}}^2 + \bp{\der{y}{\t}}^2} \d \t = \int_{\t_1}^{\t_2} \sqrt{r^2 + \bp{\der{r}{\t}}^2} \d \t.\]
\end{proof}

\section{Surface Area of Revolution}

\begin{definition}
    The surface area of a solid of revolution is called the \vocab{surface area of revolution}.
\end{definition}

\begin{proposition}[Surface Area of Revolution of Parametric Curve]
    Let $A(t_1)$ and $B(t_2)$ be points the parametric curve with equations $x = f(t)$, $y = g(t)$, $t \in [t_1, t_2]$. Then the surface area of revolution about the $x$-axis of arc $AB$ is given by \[A = 2\pi \int_{t_1}^{t_2} y \sqrt{\bp{\der{x}{t}}^2 + \bp{\der{y}{t}}^2} \d t.\] Similarly, the surface area of revolution about the $y$-axis is given by \[A = 2\pi \int_{t_1}^{t_2} x \sqrt{\bp{\der{x}{t}}^2 + \bp{\der{y}{t}}^2} \d t.\]
\end{proposition}
\begin{proof}
    Let $s = \overbow{AB}$ be the arc length of $AB$. Let $P$ and $Q$ be points on $AB$ with parameters $t$ and $t + \D t$ respectively. Recall that \[\D s = \overbow{PQ} = \sqrt{\bp{\der{x}{t}}^2 + \bp{\der{y}{t}}^2} \D t.\] Now consider the surface area of revolution about the $x$-axis of arc $PQ$. For small $\D s$, the solid of revolution is approximately a disc wish radius $y$ and width $\D s$. The surface area of this disc can be calculated as \[\D A = 2 \pi y \D s = 2 \pi y \sqrt{\bp{\der{x}{t}}^2 + \bp{\der{y}{t}}^2} \D t.\] Integrating from $A$ to $B$, we see that \[A = 2\pi \int_{t_1}^{t_2} y \sqrt{\bp{\der{x}{t}}^2 + \bp{\der{y}{t}}^2} \d t.\] A similar argument is used when the axis of rotation is the $y$-axis.
\end{proof}

\section{Approximating Definite Integrals}

In \SS\ref{sec:Applications-of-Integration:Area}, we saw how Riemann sums could approximate definite integrals using rectangles. This is a blunt tool which utilizes very little information from the curve and thus will often not give a good estimate of the definite integral for a fixed number of rectangles.

In this chapter, we will be exploring two other methods: the trapezium rule and Simpson's rule, for finding the approximate value of an area under a curve. These methods often give better approximations to the actual area as compared to using Riemann sums. Similar to Riemann sums, these methods can be extended to estimate the value of a definite integral.

\subsection{Trapezium Rule}

Consider the curve $y = f(x)$ which is non-negative over the interval $[a, b]$.

\begin{figure}[H]\tikzsetnextfilename{366}
    \centering
    \begin{tikzpicture}[trim axis left, trim axis right]
        \begin{axis}[
            domain = 0:1.2,
            samples = 101,
            axis y line=middle,
            axis x line=middle,
            xtick = {0.1, 1},
            xticklabels = {$a$, $b$},
            ytick = \empty,
            xlabel = {$x$},
            ylabel = {$y$},
            legend cell align={left},
            legend pos=outer north east,
            after end axis/.code={
                \path (axis cs:0,0) 
                    node [anchor=north east] {$O$};
                }
            ]
            \addplot[plotRed] {x + sin(pi * \x r)};

            \addlegendentry{$y = f(x)$};

            \draw[dashed, thick] (0.1, 0) -- (0.1, 0.409) -- (0.3, 1.11) -- (0.3, 0);
            \draw[dashed, thick] (0.3, 1.11) -- (0.5, 1.5) -- (0.5, 0);
            \draw[dashed, thick] (1, 0) -- (1, 1) -- (0.8, 1.39) -- (0.8, 0);

            \node at (0.65, 0.5) {$\ldots$};

            \node[anchor=west] at (0.1, 0.2) {$y_0$};
            \node[anchor=west] at (0.3, 0.2) {$y_1$};
            \node[anchor=west] at (0.5, 0.2) {$y_2$};

            \node[anchor=west] at (0.8, 0.2) {$y_{n-1}$};
            \node[anchor=west] at (1, 0.2) {$y_n$};
        \end{axis}
    \end{tikzpicture}
    \caption{}
\end{figure}

Divide the interval $[a, b]$ into $n$ equal intervals (strips) with each having width $h = (b-a)/n$. Then the area of the $n$ trapeziums is given by \[\area = \sum_{k = 0}^n \frac{h}{2} \bp{y_k + y_{k+1}} = \frac{h}{2} \bs{y_0 + 2(y_1 + y_2 + \dots + y_{n-1}) + y_n}.\]

\begin{recipe}[Trapezium Rule]
    The trapezium rule with ($n+1$) ordinates (or $n$ intervals) gives the approximation \[\int_a^b f(x) \d x \approx \sum_{k = 0}^n \frac{h}{2} \bp{y_k + y_{k+1}} = \frac{h}{2} \bs{y_0 + 2(y_1 + y_2 + \dots + y_{n-1}) + y_n},\] where $h = (b-a)/n$.
\end{recipe}

\begin{sample}\label{samp:Trapezium-Rule}
    Use the trapezium rule with 4 strips to find an approximation for \[\int_0^2 \ln{x + 2} \d x.\] Find the percentage error of the approximation.
\end{sample}
\begin{sampans}
    Let $f(x) = \ln{x+2}$. By the trapezium rule,
    \begin{align*}
        \int_0^2 \ln{x + 2} \d x &\approx \frac{1}{2} \cdot \frac{2 - 0}{4} \Big(f(0) + 2\bs{f(0.5) + f(1) + f(1.5)} + f(2)\Big) \\
        &= 2.15369 \todp{5}.
    \end{align*}
    One can easily verify that the integral evaluates to $2.15888 \todp{5}$. Hence, the percentage error is \[\abs{\frac{2.15888 - 2.15369}{2.15888}} = 0.240 \%.\]
\end{sampans}

\subsubsection{Error in Trapezium Rule Approximation}

If the curve is concave upward, the secant lines lie above the curve. Hence, the trapezium rule will give an overestimate.

\begin{figure}[H]\tikzsetnextfilename{367}
    \centering
    \begin{tikzpicture}[trim axis left, trim axis right]
        \begin{axis}[
            domain = 0:1.2,
            samples = 101,
            axis y line=middle,
            axis x line=middle,
            xtick = \empty,
            ytick = \empty,
            xlabel = {$x$},
            ylabel = {$y$},
            legend cell align={left},
            legend pos=outer north east,
            after end axis/.code={
                \path (axis cs:0,0) 
                    node [anchor=north east] {$O$};
                }
            ]
            \addplot[plotRed] {x^2};

            \addlegendentry{$y = f(x)$};

            \draw[dashed, thick] (0.2, 0) -- (0.2, 0.04) -- (1, 1) -- (1, 0);
        \end{axis}
    \end{tikzpicture}
    \caption{}
\end{figure}

If the curve is concave downward, the secant lines lie below the curve. Hence, the trapezium rule will give an underestimate.

\begin{figure}[H]\tikzsetnextfilename{368}
    \centering
    \begin{tikzpicture}[trim axis left, trim axis right]
        \begin{axis}[
            domain = 0:1.2,
            samples = 101,
            axis y line=middle,
            axis x line=middle,
            xtick = \empty,
            ytick = \empty,
            xlabel = {$x$},
            ylabel = {$y$},
            legend cell align={left},
            legend pos=outer north east,
            after end axis/.code={
                \path (axis cs:0,0) 
                    node [anchor=north east] {$O$};
                }
            ]
            \addplot[plotRed] {-x^2+1.44};

            \addlegendentry{$y = f(x)$};

            \draw[dashed, thick] (0.2, 0) -- (0.2, 1.40) -- (1, 0.44) -- (1, 0);
        \end{axis}
    \end{tikzpicture}
    \caption{}
\end{figure}

\subsection{Simpson's Rule}

Previously, we explored how Riemann sums approximate definite integrals using horizontal lines (i.e. degree 0 polynomials). We also saw how the trapezium rule improves this approximation by using sloped lines (i.e. degree 1 polynomials). Now, we introduce Simpson's rule, which takes this a step further by using quadratics (i.e. degree 2 polynomials) to achieve even greater accuracy in approximating definite integrals.

Consider the curve $y = f(x)$, which is non-negative over the interval $[a, b]$. Suppose the area represented by $\int_a^b f(x) \d x$ is divided by the ordinates $y_0$, $y_1$, $y_2$ into two strips each of width $h$ as shown below. A particular parabola can be found passing through the three points on the curve with ordinates $y_0$, $y_1$, $y_2$. Simpson's rule uses the area under the parabola to approximate the area represented by $\int_a^b f(x) \d x$.

To deduce the area under the parabola, we consider the case where $y = f(x)$ is translated $x_1$ units to the left, i.e. the line $x = x_1$ is now the $y$-axis. 

\begin{figure}[H]\tikzsetnextfilename{369}
    \centering
    \begin{tikzpicture}[trim axis left, trim axis right]
        \begin{axis}[
            domain = -0.6:0.6,
            samples = 101,
            axis y line=middle,
            axis x line=middle,
            xtick = {-0.4, 0.4},
            xticklabels = {$-h$, $h$},
            ytick = \empty,
            xlabel = {$x$},
            ylabel = {$y$},
            legend cell align={left},
            legend pos=outer north east,
            after end axis/.code={
                \path (axis cs:0,0) 
                    node [anchor=north east] {$O$};
                }
            ]

            \addplot[plotRed, very thick] {x + 0.8 + sin(pi * (\x + 0.4) r)};

            \addlegendentry{$y = f(x + x_1)$};

            \addplot[dashed, thick] {(-4.107) * x^2 + 1.735 * x + 1.751};

            \addlegendentry{$y = Ax^2 + Bx + C$};

            \draw[dashed, thick] (-0.4, 0) -- (-0.4, 0.4);
            \draw[dashed, thick] (0, 1.751) -- (0, 0);
            \draw[dashed, thick] (0.4, 0) -- (0.4, 1.788);

            \fill (-0.4, 0.4) circle[radius=2.5pt];
            \fill (0., 1.751) circle[radius=2.5pt];
            \fill (0.4, 1.788) circle[radius=2.5pt];

            \node[anchor=west] at (-0.4, 0.2) {$y_0$};
            \node[anchor=west] at (0, 0.2) {$y_1$};
            \node[anchor=west] at (0.4, 0.2) {$y_2$};
        \end{axis}
    \end{tikzpicture}
    \caption{}
\end{figure}

Under this translation, \[\int_a^b f(x) \d x = \int_{-h}^h f(x + x_1) \d x.\] This area will now be approximated by a parabola $y = g(x) = Ax^2 + Bx + C$, where $A$, $B$ and $C$ are constants. The area under the parabola is given by \[\int_{-h}^h \bp{Ax^2 + Bx + C} \d x = \evalint{\frac{A}3 x^3 + \frac{B}{2} x^2 + Cx}{-h}{h} = \frac{h}{3} \bp{2Ah^2 + 6C}.\]

Now, observe that the parabola $y = g(x)$ intersects the curve at $(-h, y_1)$, $(0, y_2)$ and $(h, y_3)$. Hence, \[g(-h) = Ah^2 - Bh + C = y_0, \quad g(0) = C = y_1, \quad g(h) = Ah^2 + Bh + C = y_2.\] Thus, \[\frac{h}{3} \bp{2Ah^2 + 6C} = \frac{h}{3} \bs{\bp{Ah^2 - Bh + C} + 4C + \bp{Ah^2 + Bh + C}} = \frac{h}{3} (y_0 + 4y_1 + y_2).\] We hence arrive at Simpson's rule with 2 strips: \[\int_a^b f(x) \d x \approx \frac{h}{3} (y_0 + 4y_1 + y_2).\] We can extend Simpson's rule to cover any even number of strips. In general,

\begin{recipe}[Simpson's Rule]
    Simpson's rule with $2n$ strips (or $2n+1$ ordinates) gives the approximation
    \begin{align*}
        \int_a^b f(x) \d x &\approx \sum_{k = 0}^n \frac{h}{3} (y_{2k} + 4y_{2k+1} + y_{2k+2})\\
        &= \frac{h}{3} \bs{y_0 + 4y_1 + 2y_2 + 4y_3 + 2y_4 + \dots + 2y_{2n-2} + 4y_{2n-1} + y_{2n}}.
    \end{align*}
\end{recipe}

\begin{sample}
    Use Simpson's rule with 4 strips to find an approximation for \[\int_0^2 \ln{x + 2} \d x.\] Find the percentage error of the approximation.
\end{sample}
\begin{sampans}
    Let $f(x) = \ln{x+2}$. By the trapezium rule,
    \begin{align*}
        \int_0^2 \ln{x + 2} \d x &\approx \frac{1}{3} \cdot \frac{2 - 0}{4} \Big[f(0) + 4f(0.5) + 2f(1) + 4f(1.5) + f(2)\Big] \\
        &= 2.15881 \todp{5}.
    \end{align*}
    As previously mentioned in Sample Problem~\ref{samp:Trapezium-Rule} the actual value of the integral is $2.15888 \todp{5}$. Hence, the percentage error is \[\abs{\frac{2.15888 - 2.15881}{2.15888}} = 0.00324\%.\]
\end{sampans}

In the previous example, the trapezium rule gave an estimate of $2.15369 \todp{5}$, which has an error of 0.240\%. In the case of Simpson's rule, the error is 0.00324\%, vastly better than that of the trapezium rule's.

In general, Simpson's rule gives a better approximation than the trapezium rule as the quadratics used account for the concavity of the curve.