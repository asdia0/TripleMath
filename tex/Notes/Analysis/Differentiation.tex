\chapter{Differentiation}

\section{First Principles}

\begin{definition}
    A function $f$ is said to be \vocab{differentiable} at some point $a$, if its domain contains an open interval containing $a$, and the limit \[L = \lim_{h \to 0} \frac{f(a+h) - f(a)}{h}\] exists. If $f$ is differentiable, then $L$ is the \vocab{derivative} of $f$ at $a$, denoted \[\der{}{x} f(a).\] If the derivative exists for all points in the domain, we may define a \vocab{derivative function} that maps $x$ to the value of the derivative at $x$, denoted \[\der{}{x} f(x) \quad \tor \quad f'(x).\]
\end{definition}

If $y = f(x)$, we write the derivative as \[\der{y}{x} \quad \tor \quad y'.\]

Geometrically, the derivative of $f$ at $a$ can be understood as the instantaneous rate of change of $f$ at $a$. Consider a curve $y = f(x)$. Let $A(a, f(a))$ and $B(a + h, f(a + h))$ be two points on the curve.

Observe that the gradient of the tangent to the curve at $A$ can be approximated by the gradient of the chord $AB$, denoted $m_{AB}$. The closer $B$ is to $A$, the better the approximation. Therefore, the gradient of the curve at point $A$ is $\lim_{B \to A} m_{AB}$. Now observe that \[m_{AB} = \frac{f(a + h) - f(a)}{(a + h) - a} = \frac{f(a + h) - f(a)}{h}.\] Additionally, as $B \to A$, $h \to 0$. Hence, \[\lim_{B \to A} m_{AB} = \lim_{h \to 0} \frac{f(a + h) - f(a)}{h},\] which is exactly the definition of the derivative of $f$ at $a$.

\begin{definition}
    The \vocab{$n$th derivative} of $y$ with respect to $x$ is \[\der[n]{y}{x} = f^{(n)}(x) = \der{}{x} \bp{\der[n-1]{y}{x}},\] where $n \in \ZZ^+$.
\end{definition}

\section{Differentiation Rules}

\begin{proposition}[Differentiation Rules]
    Let $k \in \RR$ and suppose $u$ and $v$ are functions of $x$. Then
    \begin{itemize}
        \item (Sum/Difference Rule) If $y = u \pm v$ then $y' = u' \pm v'$.
        \item (Product Rule) If $y = uv$, then $y' = u' v + u v'$.
        \item (Quotient Rule) If $y = u/v$, then \[y' = \frac{u' v - u v'}{v^2}.\]
        \item (Chain Rule) If $y = f(x)$ and $x = g(t)$, then \[\der{y}{t} = \der{y}{x} \der{x}{t}.\]
    \end{itemize}
\end{proposition}
The sum, product and quotient rules are easy to prove from first principles. We hence only prove the chain rule. To do so, we introduce an equivalent definition of differentiability at a point.
\begin{definition}\label{def:Differentiability}
    A function $f(x)$ is \vocab{differentiable} at $a$ if there exists some function $q(x)$ continuous at $a$ such that \[q(x) = \frac{f(x) - f(a)}{x - a}.\] Note that there is at most one such $q(x)$, and if it exists, then $q(x) = f'(x)$.
\end{definition}
We now prove the chain rule.
\begin{proof}[Proof of Chain Rule]    
    Suppose $y = f(x)$ and $x = g(t)$. Suppose also that $f(x)$ is differentiable at $x = g(a)$, and that $g(t)$ is differentiable at $a$.

    Since $f(x)$ is differentiable at $x = g(a)$, by the above definition, there exists a function $q(x)$ such that \[q(x) = \frac{f(x) - f(g(a))}{x - g(a)}.\] Replacing $x$ with $g(t)$, and rearranging terms, we get \[g(t) - g(a) = \frac{f(g(t)) - f(g(a))}{q(g(t))}. \tag{1}\]

    Similarly, since $g(t)$ is differentiable at $a$, by the above definition, there must exist a function $r(t)$ continuous at $a$ such that \[r(t) = \frac{g(t) - g(a)}{t - a}.\] This implies that \[g(t) - g(a) = r(t) (t - a). \tag{2}\]

    Equating (1) and (2), we have \[\frac{f(g(t)) - f(g(a))}{q(g(t))} = r(t) (t - a).\] Rearranging, \[q(g(t)) r(t) = \frac{f(g(t)) - f(g(a))}{t - a} = \frac{(f \circ g)(t) - (f \circ g)(a)}{t - a}.\] By our assumptions, $q(g(t)) r(t)$ is continuous at $t = a$. Hence, by the above definition, $q(g(t)) r(t)$ is the derivative of $(f \circ g)'(t)$. Since $q(x) = f'(x)$ and $r(t) = g'(t)$, we arrive at \[(f \circ g)'(t) = f'(g(t)) g'(t).\] In Liebniz notation, this reads as \[\der{}{t} f(g(t)) = \bs{\der{}{x} f(g(t))} \bs{\der{}{t} g(t)}.\] Since $x = g(t)$ and $y = f(x) = f(g(t))$, this can be written more compactly as \[\der{y}{t} = \der{y}{x} \der{x}{t}.\]
\end{proof}

From the chain rule, we can derive the following property:

\begin{proposition}
    Suppose $\derx{x}{y} \not\equiv 0$. Then \[\der{y}{x} = \frac1{\derx{x}{y}}.\]
\end{proposition}
\begin{proof}
    By the chain rule, \[1 = \der{y}{y} = \der{y}{x} \der{x}{y}.\] Since $\derx{x}{y} \not\equiv 0$, we may divide by $\derx{x}{y}$ to get \[\der{y}{x} = \frac1{\derx{x}{y}}.\]
\end{proof}

Note that this property does not generalize to higher derivatives. For instance, \[\der[2]{y}{x} \neq \frac1{\derx[2]{x}{y}}.\]

\section{Derivatives of Standard Functions}

Let $a \in \RR$.

{\renewcommand{\arraystretch}{1.2}
\begin{table}[H]
    \centering
    \begin{tabular}{|c|c|l|c|c|l|c|c|}
    \cline{1-2} \cline{4-5} \cline{7-8}
    $y$ & $y'$ &  & $y$ & $y'$ &  & $y$ & $y'$ \\ \cline{1-2} \cline{4-5} \cline{7-8} 
    $x^n$ & $n x^{n-1}$ &  & $\sin x$ & $\cos x$ &  & $\cos x$ & $-\sin x$ \\ \cline{1-2} \cline{4-5} \cline{7-8} 
    $a^x$ & $a^x \ln a$ &  & $\sec x$ & $\sec x \tan x$ &  & $\csc x$ & $-\csc x \cot x$ \\ \cline{1-2} \cline{4-5} \cline{7-8} 
    $\log_a x$ & $1/(x\ln a)$ &  & $\tan x$ & $\sec^2 x$ &  & $\cot x$ & $-\csc^2 x$ \\ \cline{1-2} \cline{4-5} \cline{7-8} 
    \end{tabular}
\end{table}}

{\renewcommand{\arraystretch}{1.2}
\begin{table}[H]
    \centering
    \begin{tabular}{|c|c|}
    \hline
    $y$ & $y'$ \\ \hline\hline
    $\arcsin x$ & $1/\sqrt{1 - x^2}$, $\abs{x} < 1$ \\ \hline
    $\arccos x$ & $-1/\sqrt{1 - x^2}$, $\abs{x} < 1$ \\ \hline
    $\arctan x$ & $1/(1 + x^2)$ \\ \hline
    \end{tabular}
\end{table}}

\section{Implicit Differentiation}

\begin{definition}
    An \vocab{explicit function} is one of the form $y = f(x)$, i.e. the dependent variable $y$ is expressed explicitly in terms of the independent variable $x$, e.g. $y = 2x \sin x + 3$. An \vocab{implicit function} is one where the dependent variable $y$ is expressed implicitly in terms of the independent variable $x$, e.g. $xy + \sin y = 2$.
\end{definition}

\begin{recipe}[Implicit Differentiation]
    $y'$ is found by differentiating every term in the equation with respect to $x$ and with subsequent arrangement, making $y'$ the subject.
\end{recipe}

Implicit differentiation requires the use of the chain rule: \[\der{}{x} g(y) = \der{}{y} g(y) \cdot \der{y}{x}.\]

\begin{example}[Implicit Differentiation]
    Consider the implicit function $3y^3 + x^2 y = 2$. Implicitly differentiating each term with respect to $x$, we obtain \[9y^2 y' + \bp{x^2 y' + 2x y} = 0.\] Rearranging, we get \[y' = \frac{-2xy}{9y^2 + x^2}.\]
\end{example}

\begin{proposition}[Derivative of Inverse Functions]
    \[\der{}{x} \inv f(x) = \frac1{{f'\bp{\inv f(x)}}}.\]
\end{proposition}
\begin{proof}
    Let $y = \inv f(x)$. Then $f(y) = x$. Implicitly differentiating, we have $f'(y) \, y' = 1$, so \[y' = \frac1{f'(y)} = \frac1{{f'\bp{\inv f(x)}}}.\]
\end{proof}

We can use the above result to derive the derivatives of the inverse trigonometric functions and the logarithm.

\begin{example}[Derivative of $\arcsin x$]
    Let $f(x) = \sin x$. Then $f'(x) = \cos x$. Using the above result, \[\der{}{x} \arcsin x = \frac{1}{\cos{\arcsin x}} = \frac1{\sqrt{1 - x^2}}.\]
\end{example}

\begin{example}[Derivative of $\log_a x$]
    Let $f(x) = a^x$. Then $f'(x) = a^x \ln a$. Using the above result, \[\der{}{x} \log_a x = \frac1{a^{\log_a x} \ln a} = \frac1{x \ln a}.\]
\end{example}

\section{Parametric Differentiation}

Sometimes it is difficult to obtain the Cartesian form of a parametric equation, so we are unable to express $\derx{y}{x}$ in terms of $x$. However, we are still able to obtain $\derx{y}{x}$ in terms of the parameter $t$ using the chain rule. If $x = f(t)$ and $g(t)$, then \[\der{y}{x} = \der{y}{t} \der{t}{x}.\]

\begin{example}[Parametric Differentiation]
    Suppose $x = \sin 2\t$, $y = \cos 4\t$. Differentiating $x$ and $y$ with respect to $\t$, we see that \[\der{x}{\t} = 2\cos 2\t, \qquad \der{y}{\t} = -4\sin 4\t.\] Hence, by the chain rule, \[\der{y}{x} = \der{y}{\t} \der{\t}{x} = \frac{-2 \sin 4\t}{\cos 2\t}.\]
\end{example}