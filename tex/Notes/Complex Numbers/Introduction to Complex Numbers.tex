\chapter{Introduction to Complex Numbers}\label{chap:Introduction-to-Complex-Numbers}

\begin{definition}
    The \vocab{imaginary unit} $\i$ is a root to the equation $x^2 + 1 = 0$.
\end{definition}

\section{Cartesian Form}

\begin{definition}
    A \vocab{complex number} $z$ has \vocab{Cartesian form} $x + \i y$, where $x$ and $y$ are real numbers. We call $x$ the \vocab{real part} of $z$, denoted $\Re z$. Likewise, we call $y$ the \vocab{imaginary part} of $z$, denoted $\Im z$.
\end{definition}

\begin{definition}
    The set of complex numbers is denoted $\CC$ and is defined as $\CC = \bc{z = x + \i y \mid x, y \in \RR}$.
\end{definition}
\begin{remark}
    The set of real numbers, $\RR$, is a proper subset of the set of complex numbers, $\CC$. That is, $\RR \subset \CC$.
\end{remark}

\begin{fact}[Algebraic Operations on Complex Numbers]
    Let $z_1, z_2, z_3 \in \CC$.
    \begin{itemize}
        \item Two complex numbers are equal if and only if their corresponding real and imaginary parts are equal.
        \item Addition of complex numbers is
        \begin{itemize}
            \item commutative, i.e. $z_1 + z_2 = z_2 + z_1$, and
            \item associative, i.e. $(z_1 + z_2) + z_3 = z_1 + (z_2 + z_3)$.
        \end{itemize}
        \item Multiplication of complex numbers is
        \begin{itemize}
            \item commutative, i.e. $z_1z_2 = z_2z_1$,
            \item associative, i.e. $z_1(z_2z_3) = (z_1z_2)z_3$, and
            \item distributive, i.e. $z_1(z_2 + z_3) = z_1z_2 + z_1z_3$.
        \end{itemize}
    \end{itemize}
\end{fact}

\begin{proposition}
    Complex numbers cannot be ordered.
\end{proposition}
\begin{proof}
    Seeking a contradiction, suppose complex numbers can be ordered.
    
    \case{1} Suppose $\i > 0$. Multiplying both sides by $\i$, we have $\i^2 = -1 > 0$, a contradiction.

    \case{2} Suppose $\i < 0$. Multiplying both sides by $\i$ and changing signs (since $\i < 0$), we have $\i^2 = -1 > 0$, a contradiction.
    
    Thus, the complex numbers cannot be ordered.
\end{proof}

\section{Argand Diagram}

We can represent complex numbers in the complex plane using an Argand diagram.

\begin{definition}
    The \vocab{Argand diagram} is a modified Cartesian plane where the $x$-axis represents real numbers and the $y$-axis represents imaginary numbers. The two axes are called the \vocab{real axis} and \vocab{imaginary axis} correspondingly.
\end{definition}

On the Argand diagram, the complex number $z = x + \i y$ can be represented by
\begin{itemize}
    \item the point $Z(x, y)$ or $Z(z)$; or
    \item the vector $\oa{OZ}$.
\end{itemize}

\begin{figure}[H]\tikzsetnextfilename{339}
    \centering
    \begin{tikzpicture}
        \begin{axis}[
            domain = 0:10,
            samples = 101,
            axis y line=middle,
            axis x line=middle,
            xtick = \empty,
            ytick = \empty,
            xmax=8,
            xmin=-2,
            ymin=-2,
            ymax=8,
            xlabel = {$\Re$},
            ylabel = {$\Im$},
            after end axis/.code={
                \path (axis cs:0,0) 
                    node [anchor=north east] {$O$};
                }
            ]

            \coordinate[label=above:$Z\bp{x, y} \tor Z\bp{z}$] (Z) at (5, 6);
            \coordinate (O) at (0, 0);

            \draw[->-=0.5, very thick] (O) -- (Z);
            \node[anchor=north west] at (2.5, 3) {$\oa{OZ}$};

            \fill (Z) circle[radius=2.5pt];
        \end{axis}
    \end{tikzpicture}
    \caption{}
\end{figure}

In an Argand diagram, let the points $Z$ and $W$ represent the complex numbers $z$ and $w$ respectively. Then $\oa{OZ}$ and $\oa{OW}$ are the corresponding vectors representing $z$ and $w$.

\subsection{Modulus}

Recall in \SS\ref{chap:Equations-and-Inequalities}, we defined the modulus of a real number $x$ as the ``distance'' between $x$ and the origin on the real number line. Generalizing this notion to complex numbers, it makes sense to define the modulus of a real number $z$ as the ``distance'' between $z$ and the origin on the complex plane. This uses Pythagoras' theorem.

\begin{definition}
    The \vocab{modulus} of a complex number $z = x + \i y$ is denoted $\abs{z}$ and is defined as $\abs{z} = \sqrt{x^2 + y^2}$.
\end{definition}

\subsection{Complex Conjugate}

\begin{definition}
    The \vocab{conjugate} of the complex number $z = x + \i y$ is denoted $z\conj$ with definition $z\conj = x - \i y$. We refer to $z$ and $z\conj$ as a \vocab{conjugate pair} of complex numbers.
\end{definition}

On an Argand diagram, the conjugate $z\conj$ is the reflection of $z$ about the real axis.

\begin{fact}[Properties of Complex Conjugates]
    \phantom{.}
    \begin{itemize}
        \item Conjugation is distributive over addition and multiplication, i.e. $(z + w)\conj = z\conj + w\conj$ and $(zw)\conj = z\conj w\conj$.
        \item Conjugation is an involution, i.e. $\bp{z\conj}\conj = z$.
        \item $z + z\conj = 2\Re{z}$ and $z - z\conj = 2\Im{z} \i$.
        \item $z z\conj = \Re{z}^2 + \Im{z}^2 = \abs{z}^2$.
    \end{itemize}
\end{fact}

Because conjugation is distributive over addition and multiplication, we also have the identities $(kz)\conj = k z\conj$ and $\bp{z^n}\conj = \bp{z\conj}^n$, where $k \in \RR$ and $n \in \ZZ$.

Using the conjugate of a complex number $z$, the reciprocal of $z$ can be computed as \[\frac1{z} = \frac{z\conj}{z z\conj} = \frac{z \conj}{\abs{z}^2}.\]

\subsection{Argument}

\begin{definition}
    The \vocab{argument} of a complex number $z$ is the directed angle $\t$ that $Z(z)$ makes with the positive real axis, and is denoted by $\arg{z}$. By convention, $\arg{z} > 0$ when measured in an anticlockwise direction from the positive real axis, and $\arg{z} < 0$ when measured in a clockwise direction from the positive real axis.
\end{definition}

Note that $\arg{z}$ is not unique; the position of $Z(z)$ is not affected by adding an integer multiple of $2\pi$ to $\t$. Hence, when talking about the argument of a complex number, we typically refer to the principal argument.

\begin{definition}
    The value of $\arg{z}$ in the interval $(-\pi, \pi]$ is known as the \vocab{principal argument} of $z$.
\end{definition}

The modulus $r = \abs{z}$, complex conjugate $z\conj$ and argument $\t = \arg{z}$ of a complex number $z$ can be identified on an Argand diagram as shown in the figure below.

\begin{figure}[H]\tikzsetnextfilename{342}
    \centering
    \begin{tikzpicture}[trim axis left, trim axis right]
        \begin{axis}[
            domain = 0:10,
            samples = 101,
            axis y line=middle,
            axis x line=middle,
            xtick = \empty,
            ytick = \empty,
            xmin=-3,
            xmax=9,
            ymin=-6,
            ymax=6,
            xlabel = {$\Re$},
            ylabel = {$\Im$},
            legend cell align={left},
            legend pos=outer north east,
            after end axis/.code={
                \path (axis cs:0,0) 
                    node [anchor=north east] {$O$};
                }
            ]

            \coordinate (A) at (4, 0);
            \coordinate (B) at (0, 0);
            \coordinate (C) at (3, 4);
            \coordinate (D) at (3, -4);

            \fill (3, 4) circle[radius=2.5pt];
            \fill (3, -4) circle[radius=2.5pt];
            \draw (0, 0) -- (3, 4);
            \draw[dashed] (0, 0) -- (D);
            \node[anchor=south east] at (1.5, 2) {$r$};
            \node[anchor=north east] at (1.5, -2) {$r$};
            \node[anchor=south] at (3, 4) {$Z_1(z)$};
            \node[anchor=north] at (D) {$Z_2(z\conj)$};

            \draw pic [draw, angle radius=10mm, "$\t$"] {angle = A--B--C};
            \draw pic [draw, angle radius=12mm, "$-\t$"] {angle = D--B--A};
        \end{axis}
    \end{tikzpicture}
    \caption{}
\end{figure}

\section{Polar Form}

Instead of using Cartesian coordinates on an Argand diagram, we can use polar coordinates, leading to the polar form of a complex number. This polar form can be expressed in two ways: trigonometric form and exponential form.

\begin{definition}
    The \vocab{trigonometric form} of the complex number $z$ is $z = r\bp{\cos \t + \i \sin \t}$, where $r = \abs{z}$ and $\t = \arg{z}$.
\end{definition}

\begin{theorem}[Euler's Identity]
    $\e^{\i \t} = \cos \t + \i \sin \t$.
\end{theorem}
\begin{proof}[Proof 1 (Series Expansion)]
    By the standard series expansion of $\e^x$, we have \[\e^{\i \t} = 1 + \i \t + \frac{(\i \t)^2}{2!} + \frac{(\i \t)^3}{3!} + \frac{(\i \t)^4}{4!} + \frac{(\i \t)^5}{5!} + \dots.\] Simplifying and grouping real and imaginary parts together, \[\e^{\i \t} = \bp{1 - \frac{\t^2}{2!} + \frac{\t^4}{4!} + \dots} + \i \bp{\t - \frac{\t^3}{3!} + \frac{\t^5}{5!} + \dots},\] which we recognize to be the standard series expansions of $\cos \t$ and $\sin \t$ respectively. Hence, $\e^{\i \t} = \cos \t + \i \sin \t$.
\end{proof}
\begin{proof}[Proof 2 (Differentiation)]
    Let $f(\t) = \e^{-\i \t} \bp{\cos \t + \i \sin \t}$. Differentiating with respect to $\t$, we see that \[f'(\t) = \e^{-\i \t} \bp{-\sin \t + \i \cos \t} - \i \e^{-\i \t} \bp{\cos \t + \i \sin \t} = 0,\] whence $f(\t)$ is constant. Since $f(0) = 1$, it follows that $\e^{-\i \t} \bp{\cos \t + \i \sin \t} = 1$, so $\e^{\i \t} = \cos \t + \i \sin \t$.
\end{proof}

\begin{definition}
    The \vocab{exponential form} of the complex number $z$ is $z = r \e^{\i \t}$, where $r = \abs{z}$ and $\t = \arg{z}$.
\end{definition}

\begin{fact}[Conjugation in Polar Form]
    Let $z = r \e^{\i \t}$. Then
    \begin{itemize}
        \item $z\conj = r\e^{-\i \t}$,
        \item $z + z\conj = r\e^{\i \t} + r\e^{-\i \t} = 2r\cos \t$, and
        \item $z - z\conj = r\e^{\i \t} - r\e^{-\i \t} = \bp{2r \sin \t} \i$.
    \end{itemize}
\end{fact}

Lastly, we observe the effect of multiplication and division on the modulus and argument of complex numbers.

\begin{proposition}[Multiplication in Polar Form]
    Let $z_1 = r_1 \e^{\i \t_1}$ and $z_2 = r_2 \e^{\i \t_2}$. Then $\abs{z_1 z_2} = r_1 r_2 = \abs{z_1} \abs{z_2}$ and $\arg{z_1 z_2} = \t_1 + \t_2 = \arg{z_1} + \arg{z_2}$.
\end{proposition}
\begin{proof}
    We have \[z_1 z_2 = \bp{r_1 \e^{\i \t_1}}\bp{r_2 \e^{\i \t_2}} = (r_1 r_2) \e^{\i (\t_1 + \t_2)}.\] The results follow immediately.
\end{proof}

\begin{corollary}[Division in Polar Form]
    Let $z_1 = r_1 \e^{\i \t_1}$ and $z_2 = r_2 \e^{\i \t_2}$. Then \[\abs{\frac{z_1}{z_2}} = \frac{r_1}{r_2} = \frac{\abs{z_1}}{\abs{z_2}} \quad \tand \quad \arg{\frac{z_1}{z_2}} = \t_1 - \t_2 = \arg{z_1} - \arg{z_2}.\]
\end{corollary}

\begin{corollary}[Exponentiation in Polar Form]
    For integers $n$, we have $\abs{z^n} = r^n = \abs{z}^n$ and $\arg{z^n} = n \t = n \arg{z}$.
\end{corollary}

\section{De Moivre's Theorem}

\begin{theorem}[De Moivre's Theorem]
    For rational numbers $n$, if $z = r\bp{\cos \t + \i \sin \t} = r \e^{\i \t}$, then $z^n = r^n \e^{\i n \t} = r^n \bp{\cos n\t + \i \sin n\t}$.
\end{theorem}
\begin{proof}
    Write $z^n$ in exponential form before converting it into trigonometric form.
\end{proof}

One application of de Moivre's theorem is to find the $n$th roots of a complex number.

\begin{recipe}[Finding $n$th Roots]
    Let $z$ be an $n$th root of $r\e^{\i\t}$.
    \phantom{.}
    \renewcommand{\theenumi}{\arabic{enumi}.}
    \begin{enumerate}
        \item Write $z^n = r \e^{\i \bp{\t + 2k\pi}}$, where $k$ is an integer.
        \item Take $n$th roots on both sides, so $z = r^{1/n} \e^{\i \bp{\t + 2k\pi} /n}$.
        \item Pick values of $k$ so that $\arg z = (\t + 2k\pi)/n$ lies in the principal interval $(-\pi, \pi]$.
    \end{enumerate}
    \renewcommand{\theenumi}{(\alph{enumi})}
\end{recipe}

\begin{definition}
    Let $n \in \ZZ$. The \vocab{$n$th roots of unity} are the $n$ solutions to the equation $z^n - 1 = 0$.
\end{definition}

\begin{proposition}[Roots of Unity in Polar Form]
    The $n$th roots of unity are given by \[z = \cos{\frac{2k \pi}{n}} + \i \sin{\frac{2k \pi}{n}} = \e^{\i \bp{2k \pi / n}},\] where $k \in \ZZ$.
\end{proposition}
\begin{proof}
    Use de Moivre's theorem.
\end{proof}

\begin{fact}[Geometric Properties of Roots of Unity]
    On an Argand diagram, the $n$th roots of unity
    \begin{itemize}
        \item all lie on a circle of radius 1.
        \item are equally spaced apart.
        \item form a regular $n$-gon.
    \end{itemize}
\end{fact}

De Moivre's theorem can also be used to derive trigonometric identities. The trigonometric identities one will be required to prove typically involve reducing ``powers'' to ``multiple angles'' (e.g. expressing $\sin^3 \t$ in terms of $\sin \t$ and $\sin 3\t$), or vice versa.

\begin{proposition}[Power to Multiple Angles]
    Let $z = \cos \t + \i \sin \t = \e^{\i \t}$. Then $z^n + z^{-n} = 2\cos n\t$ and $z^n - z^{-n} = 2\i \sin n\t$.
\end{proposition}
\begin{proof}
    By de Moivre's theorem,
    \begin{align*}
        z^n &= \cos{n\t} + \i \sin{n\t},\\
        z^{-n} &= \cos{-n\t} + \i \sin{-n\t}\\
        &= \cos{n\t} - \i \sin{n \t}.
    \end{align*}
    Adding and subtracting the above equations yields the desired claim.
\end{proof}

\begin{recipe}[Multiple Angles to Powers]
    Suppose we want to express $\cos n\t$ and $\sin n\t$ in terms of powers of $\sin \t$ and $\cos \t$. Invoking de Moivre's theorem and the binomial theorem, \[\cos n\t + \i \sin n\t = \bp{\cos \t + \i \sin \t}^n = \sum_{k = 0}^n \binom{n}{k} \cos^k \t \sin^{n-k} \t.\] We then take the real and imaginary parts of both sides to get \[\cos n\t = \Re \sum_{k = 0}^n \binom{n}{k} \cos^k \t \sin^{n-k} \t \quad \tand \quad \sin n\t = \Im \sum_{k = 0}^n \binom{n}{k} \cos^k \t \sin^{n-k} \t.\]
\end{recipe}

\begin{sample}
    Prove that $\sin 2\t = 2 \sin \t \cos \t$.
\end{sample}
\begin{sampans}
    Using de Moivre's theorem, \[\cos 2\t + \i \sin 2\t = \bp{\cos \t + \i \sin \t}^2 = \cos^2 \t + 2\i \cos \t \sin \t - \sin^2 \t.\] Comparing imaginary parts, we obtain $\sin 2\t = 2\cos\t\sin\t$ as desired.
\end{sampans}

Another way to derive new trigonometric identities is to differentiate known identities.

\begin{sample}
    Prove that \[\sin \t \cos^5 \t = \frac1{32} \bp{\sin 6\t + 4\sin 4\t + 5\sin 2\t}.\]
\end{sample}
\begin{sampans}
    Using the ``power to multiple angle'' formula above, one can show that \[\cos^6 \t = \frac1{32} \bp{\cos 6\t + 6\cos4\t + 15\cos2\t + 10}.\] Differentiating yields the desired identity.
\end{sampans}

\section{Solving Polynomial Equations over \texorpdfstring{$\CC$}{Complex Numbers}}

\begin{theorem}[Fundamental Theorem of Algebra]
    A non-zero, single-variable, degree $n$ polynomial with complex coefficients has $n$ roots in $\CC$, counted with multiplicity.
\end{theorem}

\begin{theorem}[Conjugate Root Theorem]
    For a polynomial equation with all real coefficients, non-real roots must occur in conjugate pairs.
\end{theorem}
\begin{proof}
    Suppose $z$ is a non-real root to the polynomial $P(z) = a_n z^n + a_{n-1} z^{n-1} + \dots + a_1 z + a_0$, where $a_n, a_{n-1}, \dots, a_1, a_0 \in \RR$. Consider $P(z\conj)$. \[P(z\conj) = a_n \bp{z\conj}^n + a_{n-1} \bp{z\conj}^{n-1} + \dots + a_1 \bp{z\conj} + a_0.\] By conjugation properties, this simplifies to \[P(z\conj) = \bp{a_n z^n + a_{n-1} z^{n-1} + \dots + a_1 z + a_0}\conj,\] which clearly evaluates to 0, whence $z\conj$ is also a root of $P(z)$.
\end{proof}