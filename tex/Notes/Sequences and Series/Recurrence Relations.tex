\chapter{Recurrence Relations}

\begin{definition}
    A \vocab{recurrence relation} is an equation that defines a sequence based on a rule that gives the next term as a function of the previous term(s).
\end{definition}

\section{First Order Linear Recurrence Relation with Constant Coefficients}

\begin{definition}
    A \vocab{first order linear recurrence relation with constant coefficients} is a recurrence relation of the form $u_n = a u_{n-1} + b$, where $a$ and $b$ are constants. If $b = 0$, the recurrence relation is said to be \vocab{homogeneous}.
\end{definition}

\begin{proposition}
    The recurrence relation $u_n = a u_{n-1} + b$ has solution \[u_n = \bp{u_1 + \frac{b}{a-1}}a^{n-1} - \frac{b}{a-1}.\]
\end{proposition}
\begin{proof}
    Let $k = b/(a-1)$ be the constant so that $u_n + k = a\bp{u_{n-1} + k}$. Define a new sequence $v_n = u_n + k$. This turns our recurrence relation into $v_n = av_{n-1}$, whence $v_n$ is in geometric progression. Thus, $v_n = v_1 a^{n-1}$. Writing this back in terms of $u_n$, we get $u_n + k = \bp{u_1 + k} a^{n-1}$, so \[u_n = \bp{u_1 + \frac{b}{a-1}}a^{n-1} - \frac{b}{a-1}.\]
\end{proof}

The proof above uses a substitution to reduce the recurrence to a geometric progression. This is one of the two main approaches for solving first-order recurrence relations. The second method, often called \vocab{solving by procedure}, is typically more direct. We now illustrate this alternative approach.

\begin{definition}
    Given a first order linear recurrence relation with constant coefficients $u_n = au_{n-1} + b$,
    \begin{itemize}
        \item $u_n = au_{n-1}$ is the \vocab{associated homogeneous recurrence relation}.
        \item $u_n^{(c)} = Ca^n$ is the general solution of the associated homogeneous recurrence relation and is called the \vocab{complementary solution}.
        \item $u_n^{(p)} = k$ is the \vocab{particular solution} to the recurrence relation.
    \end{itemize}
\end{definition}

\begin{fact}[Solving by Procedure]
    The general solution is given by $u_n = u_n^{(c)} + u_n^{(p)} = Ca^n + k$.
\end{fact}

\begin{example}[Solving by Procedure]
    Consider the recurrence relation $u_n = 3u_{n-1} + 10$ with first term $u_1 = 0$. The associated homogeneous recurrence relation is $u_n = 3u_{n-1}$, so the complementary solution is $u_n^{(c)} = C \cdot 3^n$ for some arbitrary constant $C$. Let the particular solution be $u_n^{(p)} = k$. Then $k = 3 k + 10$, so $k = -5$. Hence, the general solution is $u_n = u_n^{(c)} + u_n^{(p)} = C\cdot 3^n - 5$. Using the initial condition $u_1 = 0$, we have $C = 5/3$, so $u_n = 5 \bp{3^{n-1} - 1}$.
\end{example}

\section{Second Order Linear Homogeneous Recurrence Relation with Constant Coefficients}

\begin{definition}
    A \vocab{second order linear homogeneous recurrence relation with constant coefficients} is a recurrence relation of the form $u_n = a u_{n-1} + b u_{n-2}$, where $a$ and $b \neq 0$ are constants.
\end{definition}

Note that we exclude the case $b = 0$, as that would lead to the first order recurrence relation $u_n = a u_{n-1}$.

\begin{recipe}[Solving by Procedure]
    To solve the recurrence relation $u_n = a u_{n-1} + b u_{n-2}$,

    \renewcommand{\theenumi}{\arabic{enumi}.}
    \begin{enumerate}
        \item Form the \vocab{characteristic equation} $x^2 - ax - b = 0$.
        \item Find the roots $\a$ and $\b$ of this characteristic equation.
        \item Then $u_n$ has the \vocab{general solution} 
        \begin{itemize}
            \item $u_n = A\a^n + B\b^n$, if $\a \neq \b$ (distinct roots, may be real or non-real).
            \item $u_n = \bp{A + Bn} \a^n$, if $\a = \b$ (real and equal roots).
            \item $u_n = r^n\bp{A \cos n\t + B \sin n\t}$, if $\a = r\e^{\i \t}$ and $\b = r\e^{-\i \t}$ (non-real roots).
        \end{itemize}
    \end{enumerate}
    \renewcommand{\theenumi}{(\alph{enumi})}
\end{recipe}
\begin{proof}
    For $u_{n+1} = a u_n + b u_{n-1}$ with given initial conditions $u_1$ and $u_2$, let the constant $k$ be such that \[u_{n+1} - k u_n = (a-k) (u_n - ku_{n-1}). \tag{1}\] Note that this is a GP. Comparing coefficients of $u_{n-1}$, we have $(a-k)k = -b$, which expands to give the characteristic equation $k^2 - ak - b = 0$. Let the roots to the characteristic equation be $k = \a$ and $k = \b$. By Vieta's formula, $\a + \b = a$. Now, using the fact that (1) is in GP, we get \[u_{n+1} - ku_n = (a-k)^{n-1} (u_2 - ku_1). \tag{2}\] Substituting $k = \a$ into (2), we obtain \[u_{n+1} - \a u_n = \b^{n-1} \bp{u_2 - \a u_1}. \tag{3a}\] Substituting $k = \b$ into (2), we obtain \[u_{n+1} - \b u_n = \a^{n-1} \bp{u_2 - \b u_1}. \tag{3b}\] We now analyse the case where $\a = \b$ and $\a \neq \b$ separately.

    \case{1}[$\a = \b$] Since the two roots are equal, (3a) and (3b) are equivalent. Taking either, \[u_{n+1} - \a u_n = \a^{n-1} (u_2 - \a u_1).\] Noting that 0 cannot be a root of the characteristic equation (since $b \neq 0$), we may divide by $\a^{n-1}$ to get \[\frac{u_{n+1}}{\a^{n-1}} - \frac{u_n}{\a^{n-2}} = u_2 - \a u_1.\] The sequence $\bc{u_n/\a^{n-2}}$ is hence in AP with common difference $u_2 - \a u_1$. Invoking the closed form for AP, we obtain \[\frac{u_n}{\a^{n-2}} = \frac{u_1}{\a^{-1}} + (n-1)(u_2 - \a u_1).\] Simplifying, \[u_n = \bs{\bp{\frac{2u_1}{\a} - \frac{u_2}{\a^2}} + \bp{\frac{u_2}{\a^2} - \frac{u_1}{\a}}n} \a^n = \bp{A + Bn} \a^n.\]

    \case{2}[$\a \neq \b$] Observe that $\bp{\a - \b}^{-1} \bs{\text{(3b)} - \text{(3a)}}$ yields \[u_n = \frac{\a^{n-1} (u_2 - \b u_1) - \b^{n-1} (u_2 - \a u_1)}{\a - \b}.\] Simplifying, we have \[u_n = \bs{\frac{u_2 - \b u_1}{\a (\a - \b)}} \a^n + \bs{\frac{u_2 - \a u_1}{\b (\b - \a)}} \b^n = A\a^n + B\b^n.\]

    We now consider the case where $\a$ and $\b$ are non-real. By the conjugate root theorem, we can write $\a = r\e^{\i \t}$ and $\b = r \e^{-\i \t}$. Substituting this into the above result, we have \[u_n = A\bp{r\e^{\i \t}}^n + B\bp{r\e^{-\i \t}}^n = r^n \bp{A \e^{\i n \t} + B \e^{-\i n \t}}.\] By Euler's identity, \[u_n = r^n \bs{(A+B) \cos n\t + \i (A-B) \sin n\t} = r^n \bp{C \cos n\t + D \sin n\t}.\]
\end{proof}