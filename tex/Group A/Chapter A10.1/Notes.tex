\section{Notes}

\begin{definition}
    The \vocab{imaginary unit} $\i$ is a root to the equation \[x^2 + 1 = 0.\]
\end{definition}

\begin{definition}
    A \vocab{complex number} $z$ has \vocab{Cartesian form} $x + \i y$, where $x$ and $y$ are real numbers. We call $x$ the \vocab{real part} of $z$, denoted $\Re z$. Likewise, we call $y$ the \vocab{imaginary part} of $z$, denoted $\Im z$.
\end{definition}

\begin{definition}
    The set of complex numbers is denoted $\CC$ and is defined as \[\CC = \bc{z : z = x + \i y , \quad x, y \in \RR}.\]
\end{definition}
\begin{remark}
    The set of real numbers, $\RR$, is a proper subset of the set of complex numbers, $\CC$. That is, $\RR \subset \CC$.
\end{remark}

\begin{proposition}
    Complex numbers cannot be ordered.
\end{proposition}
\begin{proof}
    Seeking a contradiction, suppose $\i > 0$. Multiplying both sides by $\i$, we have $\i^2 = -1 > 0$, a contradiction. Hence, we must have $\i < 0$. However, multiplying both sides by $\i$ and changing signs (since $\i < 0$), we have $\i^2 = -1 > 0$, another contradiction. Thus, $\CC$ cannot be ordered.
\end{proof}

\begin{definition}
    Let $z = x + \i y$. $z$ is said to be \vocab{purely real} if and only if $y = 0$. Likewise, $z$ is said to be \vocab{purely imaginary} if and only if $x = 0$.
\end{definition}

\subsection{Complex Numbers in Cartesian Form}

\begin{definition}
    The \vocab{modulus} of a complex number $z$ is denoted $\abs{z}$ and is defined as \[\abs{z} = \sqrt{\Re{z}^2 + \Im{z}^2}.\]
\end{definition}

\begin{fact}[Algebraic Operations on Complex Numbers]
    Let $z_1, z_2, z_3 \in \CC$.
    \begin{itemize}
        \item Two complex numbers are equal if and only if their corresponding real and imaginary parts are equal. \[z_1 = z_2 \iff \Re z_1 = \Re z_2 \land \Im z_1 = \Im z_2.\]
        \item Addition of complex numbers is commutative, i.e. \[z_1 + z_2 = z_2 + z_1\] and associative, i.e. \[(z_1 + z_2) + z_3 = z_1 + (z_2 + z_3).\]
        \item Multiplication of complex numbers is commutative, i.e. \[z_1z_2 = z_2z_1,\] associative, i.e. \[z_1(z_2z_3) = (z_1z_2)z_3\] and distributive, i.e. \[z_1(z_2 + z_3) = z_1z_2 + z_1z_3.\]
    \end{itemize}
\end{fact}

\begin{definition}
    The \vocab{conjugate} of the complex number $z = x + \i y$ is denoted $z\conj$ with definition \[z\conj = x - \i y.\] We refer to $z$ and $z\conj$ as a \vocab{conjugate pair} of complex numbers.
\end{definition}

\begin{fact}[Properties of Complex Conjugates]
    \phantom{.}
    \begin{itemize}
        \item (distributive over addition) $(z + w)\conj = z\conj + w\conj$.
        \item (distributive over multiplication) $(zw)\conj = z\conj w\conj$.
        \item (involution) $\bp{z\conj}\conj = z$.
        \item $z + z\conj = 2\Re{z}$.
        \item $z - z\conj = 2\Im{z} \i$.
        \item $z z\conj = \Re{z}^2 + \Im{z}^2 = \abs{z}^2$.
    \end{itemize}
\end{fact}
\begin{remark}
    Because conjugation is distributive over addition and multiplication, we also have the following identities: \[(kz)\conj = k z\conj, \qquad \bp{z^n}\conj = \bp{z\conj}^n,\] where $k \in \RR$ and $n \in \ZZ$.
\end{remark}

\begin{proposition}[Division of Complex Numbers]
    For all non-zero complex numbers, \[z^{-1} = \frac{z\conj}{\abs{z}^2}.\]
\end{proposition}
\begin{proof}
    Multiplying by $\frac{z\conj}{z\conj}$, we have \[\frac{1}{z} = \frac{z\conj}{z z\conj} = \frac{z\conj}{\abs{z}^2}.\]
\end{proof}

\subsection{Roots of Polynomial Equations}

\begin{theorem}[Fundamental Theorem of Algebra]
    A non-zero, single-variable, degree $n$ polynomial with complex coefficients has $n$ roots in $\CC$, counted with multiplicity.
\end{theorem}

\begin{theorem}[Conjugate Root Theorem]
    For a polynomial equation with all real coefficients, non-real roots must occur in conjugate pairs.
\end{theorem}
\begin{proof}
    Suppose $z$ is a non-real root to the polynomial $P(z) = a_n z^n + a_{n-1} z^{n-1} + \dots + a_1 z + a_0$, where $a_n, a_{n-1}, \dots, a_1, a_0 \in \RR$. Consider $P(z\conj)$. \[P(z\conj) = a_n \bp{z\conj}^n + a_{n-1} \bp{z\conj}^{n-1} + \dots + a_1 \bp{z\conj} + a_0.\] By conjugation properties, this simplifies to \[P(z\conj) = \bp{a_n z^n + a_{n-1} z^{n-1} + \dots + a_1 z + a_0}\conj,\] which clearly evaluates to 0, whence $z\conj$ is also a root of $P(z)$.
\end{proof}