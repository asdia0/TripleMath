\section{Tutorial A5}

\begin{problem}
    Solve these recurrence relations together with the initial conditions.

    \begin{enumerate}
        \item $u_n = 2u_{n-1}$, for $n \geq 1$, $u_0 = 3$
        \item $u_n = 3u_{n-1} + 7$, for $n \geq 1$, $u_0 = 5$
    \end{enumerate}
\end{problem}
\begin{solution}
    \begin{ppart}
        $u_n = 2^n \cdot u_0 = 3 \cdot 2^n$.
    \end{ppart}
    \begin{ppart}
        Let $k$ be a constant such that $u_n + k = 3(u_{n-1} + k)$. Then $k = \frac72$. Hence, \[u_n + \frac72 = 3\bp{u_{n-1} + \frac72} \implies u_n + \frac72 = 3^n \bp{u_0 + \frac72} \implies u_n = \frac{17}2 \cdot3^n  - \frac72.\]
    \end{ppart}
\end{solution}

\begin{problem}
    Solve these recurrence relations together with the initial conditions.
        
    \begin{enumerate}
        \item $u_n = 5u_{n-1} - 6u_{n-2}$, for $n \geq 2$, $u_0 = 1$, $u_1 = 0$
        \item $u_n = 4u_{n-2}$, for $n \geq 2$, $u_0 = 0$, $u_1 = 4$
        \item $u_n = 4u_{n-1} - 4u_{n-2}$, for $n \geq 2$, $u_0 = 6$, $u_1 = 8$
        \item $u_n = -6u_{n-1} - 9u_{n-2}$, for $n \geq 2$, $u_0 = 3$, $u_1 = -3$
        \item $u_n = 2u_{n-1} - 2u_{n-2}$, for $n \geq 2$, $u_0 = 2$, $u_1 = 6$
    \end{enumerate}
\end{problem}
\begin{solution}
    \begin{ppart}
        Note that the characteristic equation of $u_n$, $x^2 - 5x + 6 = 0$, has roots 2 and 3. Thus, \[u_n = A\cdot2^n + B\cdot3^n.\] From $u_0 = 1$ and $u_1 = 0$, we have the equations $A + B = 1$ and $2A + 3B = 0$. Solving, we see that $A = 3$ and $B = 2$, whence \[u_n = 3 \cdot 2^n + 2 \cdot 3^n.\]
    \end{ppart}
    \begin{ppart}
        Note that the characteristic equation of $u_n$, $x^2 - 4 = 0$, has roots $-2$ and 2. Thus, \[u_n = A(-2)^n + B\cdot2^n.\] From $u_0 = 0$ and $u_1 = 4$, we get $A + B = 0$ and $-2A + 2B = 4$. Solving, we see that $A = -1$ and $B = 1$, whence \[u_n = -(-2)^n + 2^n.\]
    \end{ppart}
    \begin{ppart}
        Note that the characteristic equation of $u_n$, $x^2 - 4x + 4 = 0$, has only one root, 2. Thus, \[u_n = (A + Bn)2^n.\] From $u_0 = 6$ and $u_1 = 8$, we obtain $A = 6$ and $A + B = 4$, whence $B = -2$. Thus, \[u_n = (6 - 2n)2^n.\]
    \end{ppart}
    \begin{ppart}
        Note that the characteristic equation of $u_n$, $x^2 + 6x + 9 = 0$, has only one root, $-3$. Thus, \[u_n = (A + Bn)(-3)^n.\] From $u_0 = 3$ and $u_1 = -3$, we get $A = 3$ and $A + B = 1$, whence $B = -2$. Thus, \[u_n = (3 - 2n)2^n.\]
    \end{ppart}
    \begin{ppart}
        Consider the characteristic equation of $u_n$, $x^2 - 2x + 2 = 0$. By the quadratic formula, this has roots $x = 1 \pm i = \sqrt{2} \exp{\pm \frac{i\pi}4}$. Hence, \[u_n = A \cdot 2^{\frac12 n} \cos{\frac{n\pi}4 n } + B \cdot 2^{\frac12 n} \sin{\frac{n\pi}4}.\] From $u_0 = 2$, we obtain $A = 2$. From $u_0 = 6$, we obtain $A + B = 6$, whence $B = 4$. Thus, \[u_n = 2^{\frac12 n + 1} \cos{\frac{n\pi}4} + 2^{\frac12 n + 2} \sin{\frac{n\pi}4}.\]
    \end{ppart}
\end{solution}

\begin{problem}
    \begin{enumerate}
        \item A sequence is defined by the formula $b_n = \frac{n!n!}{(2n)!}\cdot2^n$, where $n \in \ZZ^+$. Show that the sequence satisfies the recurrence relation $b_{n+1} = \frac{n+1}{2n+1} b_n$.
        \item A sequence is defined recursively by the formula
        \[
            u_{n+1} = 2u_n + 3, \qquad n \in \ZZ_0^+, \, u_0 = a
        \]
        Show that $u_n = 2^na+3\bp{2^n - 1}$.
    \end{enumerate}
\end{problem}
\begin{solution}
    \begin{ppart}
        \[b_{n+1} = \frac{(n+1)! (n+1)!}{(2n+2)!}\cdot2^{n+1} = \frac{2(n+1)^2}{(2n+1)(2n+2)} \cdot \bs{\frac{n!n!}{(2n)!} \cdot 2^{n}} = \frac{n+1}{2n+1}b_n.\]
    \end{ppart}
    \begin{ppart}
        Let $k$ be a constant such that $u_{n+1} + k = 2(u_n + k)$. Then $k = 3$. Hence, \[u_{n+1} + 3 = 2(u_n + 3) \implies u_n +3 = 2^n(u_0 + 3) \implies u_n = 2^n (a + 3) - 3 = 2^n a + 3\bp{2^n - 1}.\]
    \end{ppart}
\end{solution}

\clearpage
\begin{problem}
    The volume of water, in litres, in a storage tank decreases by 10\% by the end of each day. However, 90 litres of water is also pumped into the tank at the end of each day. The volume of water in the tank at the end of $n$ days is denoted by $x_n$ and $x_0$ is the initial volume of water in the tank.

    \begin{enumerate}
        \item Write down a recurrence relation to represent the above situation.
        \item Show that $x_n = 0.9^n (x_0 - 900) + 900$.
        \item Deduce the amount of water in the tank when $n$ becomes very large.
    \end{enumerate}
\end{problem}
\begin{solution}
    \begin{ppart}
        $x_{n+1} = 0.9x_n + 90$, $n \in \NN$
    \end{ppart}
    \begin{ppart}
        Let $k$ be a constant such that $x_{n+1} + k = 0.9(x_n + k)$. Then $k = -900$. Hence, \[x_{n+1} - 900 = 0.9(x_n - 900) \implies x_n - 900 = 0.9^n (x_0 - 900) \implies x_n = 0.9^n (x_0 - 900) + 900.\]
    \end{ppart}
    \begin{ppart}
        As $n \to \infty$, $0.9^n \to 0$. Hence, the amount of water in the tank will converge to 900 litres.
    \end{ppart}
\end{solution}

\begin{problem}
    A deposit of \$100,000 is made to an investment fund at the beginning of a year. On the last day of each year, two dividends are awarded and reinvested into the fund. The first dividend is 20\% of the amount in the account during that year. The second dividend is 45\% of the amount in the account in the previous year.

    \begin{enumerate}
        \item Find a recurrence relation $\bc{P_n}$ where $P_n$ is the amount at the start of the $n$th year if no money is ever withdrawn.
        \item How much is in the account after $n$ years if no money is ever withdrawn?
    \end{enumerate}
\end{problem}
\begin{solution}
    \begin{ppart}
        \[P_{n+2} = P_{n+1} + 0.2 P_{n+1} + 0.45P_n = 1.2 P_{n+1} + 0.45 P_n.\]
    \end{ppart}
    \begin{ppart}
        Note that the characteristic equation of $P_n$, $x^2 - 1.2x - 0.45 = 0$, has roots $-\frac3{10}$ and $\frac32$. Thus, \[P_n = A  \bp{-\frac3{10}}^n + B \bp{\frac32}^n.\] From $P_0 = 0$ and $P_1 = 100000$, we have $A+B = 0$ and $-\frac3{10}A + \frac32B = 100000$. Solving, we have $A = -\frac{500000}{9}$ and $B = \frac{500000}{9}$. Thus, \[P_n = \frac{500000}9 \bs{\bp{\frac32}^n - \bp{-\frac3{10}}^n}.\] Hence, there will be \$$\bc{\frac{500000}9 \bs{\bp{\frac32}^n - \bp{-\frac3{10}}^n}}$ in the account after $n$ years if no money is ever withdrawn
    \end{ppart}
\end{solution}

\clearpage
\begin{problem}
    A pair of rabbits does not breed until they are two months old. After they are two months old, each pair of rabbit produces another pair each month.

    \begin{enumerate}
        \item Find a recurrence relation $\bc{f_n}$ where $f_n$ is the total number of pairs of rabbits, assuming that no rabbits ever die.
        \item What is the number of pairs of rabbits at the end of the $n$th month, assuming that no rabbits ever die?
    \end{enumerate}
\end{problem}
\begin{solution}
    \begin{ppart}
        $f_{n+2} = f_{n+1} + f_n$, $n \geq 2$, $f_0 = 0, f_1 = 1$
    \end{ppart}
    \begin{ppart}
        Consider the characteristic equation of $f_n$, $x^2 - x - 1 = 0$. By the quadratic formula, the roots of the characteristic equation are $\frac{1 + \sqrt5}{2}$ and $\frac{1 - \sqrt5}{2}$. Hence \[f_n = A \bp{\frac{1 + \sqrt5}{2}}^n + B \bp{\frac{1 - \sqrt5}{2}}^n.\] From $f_0 = 0$, we get $A + B = 0$. From $f_1 = 1$, we get $A \bp{\frac{1 + \sqrt5}2} + B \bp{\frac{1 - \sqrt5}2} = 1$. Solving, we get $A = \frac1{\sqrt5}$ and $B = -\frac1{\sqrt5}$. Hence, \[f_n = \frac1{\sqrt5} \bp{\frac{1 + \sqrt5}{2}}^n - \frac1{\sqrt5} \bp{\frac{1 - \sqrt5}{2}}^n.\]
    \end{ppart}
\end{solution}

\begin{problem}
    For $n \in \bc{2^j \colon j \in \ZZ, j \geq 1}$, it is given that $T_n = 3T_{n/2} + 17$, where $T_1 = 4$. By considering the substitution $n = 2^i$ and another suitable substitution, show that the recurrence relation can be expressed in the form
    \[
        t_i = 3t_{i-1} + 17, \qquad i \in \ZZ^+
    \]
    
    Hence, find an expression for $T_n$ in terms of $n$.
\end{problem}
\begin{solution}
        Let $n = 2^i \iff i = \log_2{n}$. The given recurrence relation transforms to \[T_{2^i} = 3T_{2^{i-1}} + 17, T_{2^0} = 4.\] Let $t_i = T_{2i}$. Then \[t_i = 3t_{i-1} + 17, t_0 = 4.\] Let $k$ be a constant such that $t_i + k = 3(t_{i-1} + k)$. Then $k = \frac{17}2$. We thus obtain a formula for $t_i$: \[t_i + \frac{17}2 = 3\bp{t_{i-1} + \frac{17}2} \implies t_i + \frac{17}{2} = 3^i \bp{t_0 + \frac{17}{2}} \implies t_i = \frac{25}{2} \cdot 3^i - \frac{17}{2}.\] Thus, \[T_{2i} = \frac{25}{2} \cdot 3^i - \frac{17}{2} \implies T_n = \frac{25}{2} \cdot 3^{\log_2 n} - \frac{17}2.\]
\end{solution}

\clearpage
\begin{problem}
    Consider the sequence $\bc{a_n}$ given by the recurrence relation \[a_{n+1} = 2a_n + 5^n, \qquad n \geq 1.\]

    \begin{enumerate}
        \item Given that $a_n = k\bp{5^n}$ satisfies the recurrent relation, find the value of the constant $k$.
        \item Hence, by considering the sequence $\bc{b_n}$ where $b_n = a_n - k(5^n)$, find the particular solution to the recurrence relation for which $a_1 = 2$.
    \end{enumerate}
\end{problem}
\begin{solution}
    \begin{ppart}
        \[a_{n+1} = 2a_n + 5^n \implies k\bp{5^{n+1}} = 2\cdot k\bp{5^n} + 5^n \implies 5k = 2k + 1 \implies k = \frac13.\]
    \end{ppart}
    \begin{ppart}
        \[b_n = a_n - \frac{5^n}3 = \bp{2a_{n-1} - 5^{n-1}} - \frac{5^n}3 = 2a_{n-1} - \frac23 \cdot 5^{n-1} = 2\bp{a_{n-1} - \frac{5^{n-1}}3} = 2b_{n-1}.\] Hence, $b_n = b_1 \cdot 2^{n-1}$. Note that $b_1 = a_1 - \frac53 = \frac13$. Thus, $b_n = \frac{2^{n-1}}3$, which gives 
        \[b_n = a_n - \frac{5^n}{3} = \frac{2^{n-1}}3 \implies a_n = \frac{2^n + 2 \cdot 5^n}6.\]
    \end{ppart}
\end{solution}

\begin{problem}
    The sequence $\bc{X_n}$ is given by \[\sqrt{X_{n+2}} = \frac{X_{n+1}}{X_n^2}, \qquad n \geq 1.\] By applying the natural logarithm to the recurrence relation, use a suitable substitution to find the general solution of the sequence, expressing your answer in trigonometric form.
\end{problem}
\begin{solution}
    Taking the natural logarithm of the recurrence relation and simplifying, we get \[\ln X_{n+2} = 2\ln X_{n+1} - 4\ln X_n.\] Let $L_n = \ln X_n \iff X_n = \exp{L_n}$. Then, \[L_{n+2} = 2L_{n+1} - 4L_n.\] Consider the characteristic equation of $L_n$, $x^2 - 2x + 4 = 0$. By the quadratic formula, this has roots $1 \pm \sqrt3 i = 2\exp{\pm \frac{i\pi}3}$. Thus, we can express $L_n$ as \[L_n = A \cdot 2^n \cos \frac{n\pi}3 + B \cdot 2^n \sin\frac{n\pi}3 = 2^n \bp{A \cos \frac{n\pi}3 + B \sin \frac{n\pi}3}.\] Thus, $X_n$ has the general solution \[X_n = \exp{2^n \bp{A \cos\frac{n\pi}3 + B \sin \frac{n\pi}3}}.\]
\end{solution}

\clearpage
\begin{problem}
    The sequence $\bc{X_n}$ is given by $X_1 = 2$, $X_2 = 15$ and \[X_{n+2} = 5\bp{1 + \frac1{n+2}}X_{n+1} - 6\bp{1 + \frac2{n+1}}X_n, \qquad n \geq 1.\] By dividing the recurrence relation throughout by $n+3$, use a suitable substitution to determine $X_n$ as a function of $n$.
\end{problem}
\begin{solution}
    Dividing the recurrence relation by $n+3$, we obtain \[\frac{X_{n+2}}{n+3} = 5\bp{\frac1{n+3} + \frac1{(n+2)(n+3)}}X_{n+1} - 6\bp{\frac1{n+3} + \frac2{(n+1)(n+3)}}X_n.\] Note that $\frac1{(n+2)(n+3)} = \frac1{n+2} - \frac1{n+3}$ and $\frac{2}{(n+1)(n+3)} = \frac1{n+1} - \frac1{n+3}$. Thus, \[\frac{X_{n+2}}{n+3} = 5\bp{\frac{X_{n+1}}{n+2}} - 6\bp{\frac{X_n}{n+1}}.\] Let $Y_n = \frac{n + 1}{X_n} \iff X_n = (n+1)Y_n$. Then, \[Y_{n+2} = 5 Y_{n+1} - 6Y_n.\] Note that the characteristic equation of $Y_n$, $x^2 - 5x + 6 = 0$, has roots 2 and 3. Hence, \[Y_n = A \cdot 2^n + B \cdot 3^n \implies X_n = (n+1) \bp{A \cdot 2^n + B \cdot 3^n}.\] From $X_1 = 2$ and $X_2 = 15$, we have $2A + 3B = 1$ and $4A + 9B = 5$. Solving, we obtain $A = -1$ and $B = 1$. Thus, \[X_n = (n+1)\bp{3^n - 2^n}.\]
\end{solution}

\begin{problem}
    A logistics company set up an online platform providing delivery services to users on a monthly paid subscription basis. The company's sales manager models the number of subscribers that the company has at the end of each month. She notes that approximately 10\% of the existing subscribers leave each month, and that there will be a constant number $k$ of new subscribers in each subsequent month after the first.

    Let $T_n$, $n \geq 1$, denote the number of subscribers the company has at the end of the $n$th month after the online platform was set up.

    \begin{enumerate}
        \item Express $T_{n+1}$ in terms of $T_n$.
    \end{enumerate}

    The company has 250 subscribers at the end of the first month.

    \begin{enumerate}
        \setcounter{enumi}{1}
        \item Find $T_n$ in terms of $n$ and $k$.
        \item Find the least number of subscribers the company needs to attract in each subsequent month after the first if it aims to have at least 350 subscribers by the end of the 12th month.
    \end{enumerate}

    Let $k = 50$ for the rest of the question.

    The monthly running cost of the company is assumed to be fixed at \$4,000. The monthly subscription fee is \$10 per user which is charged at the end of each month.

    \begin{enumerate}
        \setcounter{enumi}{3}
        \item Given that the $m$th month is the first month in which the company's revenue up to and including that month is able to cover its cost up to and including that month, find the value of $m$.
        \item Using your answer to part (b), determine the long-term behaviour of the number of subscribers that the company has. Hence, explain whether this behaviour is appropriate in terms of long-term prospects for the company's success.    
    \end{enumerate}
\end{problem}
\begin{solution}
    \begin{ppart}
        $T_{n+1} = 0.9T_n + k$
    \end{ppart}
    \begin{ppart}
        Let $m$ be a constant such that $T_{n+1} + m = 0.9\bp{T_n + m}$. Then $m = -10k$. Hence, \[T_{n+1} - 10k = 0.9\bp{T_n - 10k} \implies T_{n} - 10k = 0.9^{n-1} \bp{T_0 - 10k}.\] Since $T_0 = 250$, we get \[T_n = 0.9^{n-1} \bp{250 - 10k} + 10k.\]
    \end{ppart}
    \begin{ppart}
        Consider $T_{12} \geq 350$. \[T_{12} \geq 350 \implies 0.9^{12-1} \bp{250 - 10k} + 10k \geq 350.\] Using G.C., $k \geq 39.6$. Hence, the company needs to attract at least 40 subscribers in each subsequent month.
    \end{ppart}
    \begin{ppart}
        Since $k = 50$, $T_n = -250 \cdot 0.9^{n-1} + 500$. Let \$$S_m$ be the total revenue for the first $m$ months.
        \begin{align*}
            S_m &= 10 \sum_{n=1}^m T_n = 10 \sum_{n=1}^m \bp{-250 \cdot 0.9^{n-1} + 500}\\
            &= 10\bs{-250 \bp{\frac{1 - 0.9^{m}}{1 - 0.9}} + 500m} = 25000 \bp{0.9^m - 1} + 5000m.
        \end{align*}
        Note that the total cost for the first $m$ months is \$$4000m$. Hence, the total profit for the first $m$ months is given by \$$(S_m - 4000m)$. Hence, we consider $S_m - 4000m \geq 0$: \[S_m - 4000m \geq 0 \implies 25000\bp{0.9^m - 1} + 1000m \geq 0.\] Using G.C., we obtain $m \geq 22.7$, whence the least value of $m$ is 23.
    \end{ppart}
    \begin{ppart}
        As $n \to \infty$, $0.9^{n-1} \to 0$. Hence, $T_n \to 500$. Hence, as $n$ becomes very large, the profit per month approaches $500 \cdot 10 - 4000 = 1000$ dollars. Thus, this behaviour is appropriate as the business will remain profitable in the long run.
    \end{ppart}
\end{solution}