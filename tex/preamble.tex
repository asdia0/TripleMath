\usepackage{asdia}

\usepackage[a4paper,left=3cm,right=3cm,top=2.5cm,bottom=2.5cm]{geometry}
\usepackage[headsepline]{scrlayer-scrpage}
\usepackage{tocloft}
\usepackage{scalerel}
\usepackage{import}

\usetikzlibrary{external}
\tikzexternalize[prefix=figures/]

\title{Triple Math}
\subtitle{\url{https://asdia.dev/projects/triplemath}}
\author{{\normalsize Notes taken by}\\Eytan Chong}
\date{2024--2025}

\let\endtitlepage\relax

\hypersetup{pdfauthor={Eytan Chong},pdfsubject={Dunman High Maths Notes},pdftitle={Triple Math},pdfkeywords={Dunman High Math Notes}}

\AfterStartingTOC{\clearpage}

\newcommand{\partx}[1]{\cleardoubleoddpage\part{#1}\cleardoubleoddpage}

\clearpairofpagestyles
\ohead*{\pagemark}
\chead{\headmark}

\setlength\cftpartnumwidth{2.2em}

\raggedbottom

% Problem and Solution Environments
\newcounter{problemnum}[section]
\newcounter{partnum}[problemnum]
\newcounter{subpartnum}[partnum]

\renewcommand{\theproblemnum}{\arabic{problemnum}}
\renewcommand{\thepartnum}{\alph{partnum}}
\renewcommand{\thesubpartnum}{\roman{subpartnum}}

\newbox\sepbox 
\newcommand\separator{%
  \setbox\sepbox=\vbox{\begin{center}$\ast$ $\ast$ $\ast$ $\ast$ $\ast$\end{center}}%
  \cleaders\copy\sepbox\vskip\ht\sepbox}

\NewDocumentCommand\chili{}{\scalerel*{\includegraphics{media/chili.png}}{X}}

\NewDocumentCommand{\problem}{o}{\ifnum\value{problemnum}>0\separator \fi\stepcounter{problemnum}\noindent{\large\sffamily\bfseries{Problem \theproblemnum\IfNoValueF{#1}{ (#1)}.}}}
\NewDocumentCommand{\solution}{o}{\medbreak\noindent{\large\sffamily\bfseries{\IfNoValueTF{#1}{Solution.}{Solution #1.}}}}
\newcommand*{\ppart}{\stepcounter{partnum}\smallbreak\noindent{\sffamily\bfseries{Part (\thepartnum).}}}
\newcommand*{\psubpart}{\stepcounter{subpartnum}\smallbreak\noindent{\sffamily\bfseries\small{Part (\thepartnum)(\thesubpartnum).}}}

\renewcommand{\theenumi}{(\alph{enumi})}
\renewcommand{\labelenumi}{\theenumi}
\renewcommand{\theenumii}{(\roman{enumii}}

% environment boxes (modified from evan.sty)

\mdfdefinestyle{mdbluebox}{%
roundcorner=10pt,
linewidth=2pt,
skipabove=12pt,
innerbottommargin=1pt,
innertopmargin=-2pt,
skipbelow=2pt,
linecolor=MidnightBlue,
nobreak=true,
%backgroundcolor=TealBlue!5,
rightline=false,
leftline=true,
topline=false,
bottomline=false,
}
\declaretheoremstyle[
headfont=\sffamily\bfseries\color{MidnightBlue},
mdframed={style=mdbluebox},
headpunct={.},
]{thmbluebox}

\declaretheorem[style=thmbluebox,name=Theorem,numberwithin=section]{theorem}
\declaretheorem[style=thmbluebox,name=Fact,sibling=theorem]{fact}
\declaretheorem[style=thmbluebox,name=Statement,sibling=theorem]{statement}
\declaretheorem[style=thmbluebox,name=Lemma,sibling=theorem]{lemma}
\declaretheorem[style=thmbluebox,name=Corollary,sibling=theorem]{corollary}
\declaretheorem[style=thmbluebox,name=Proposition,sibling=theorem]{proposition}

\mdfdefinestyle{mdpurplebox}{%
roundcorner=10pt,
linewidth=2pt,
skipabove=12pt,
innerbottommargin=1pt,
innertopmargin=-2pt,
skipbelow=2pt,
linecolor=Mulberry!80!black,
nobreak=true,
%backgroundcolor=Thistle!5,
rightline=false,
leftline=true,
topline=false,
bottomline=false,
}
\declaretheoremstyle[
headfont=\sffamily\bfseries\color{Mulberry!80!black},
mdframed={style=mdpurplebox},
headpunct={.},
]{thmpurplebox}

\declaretheorem[style=thmpurplebox,name=Recipe,sibling=theorem]{recipe}

\mdfdefinestyle{mdgreenbox}{%
roundcorner=10pt,
linewidth=2pt,
skipabove=12pt,
innerbottommargin=1pt,
innertopmargin=-2pt,
skipbelow=2pt,
linecolor=ForestGreen,
nobreak=true,
%backgroundcolor=ForestGreen!5,
rightline=false,
leftline=true,
topline=false,
bottomline=false,
}
\declaretheoremstyle[
headfont=\sffamily\bfseries\color{ForestGreen!70!black},
mdframed={style=mdgreenbox},
headpunct={.},
]{thmgreenbox}

\let\definition\relax
\declaretheorem[style=thmgreenbox,name=Definition,sibling=theorem]{definition}
\declaretheorem[style=thmgreenbox,name=Axiom,sibling=theorem]{axiom}
\declaretheorem[style=thmgreenbox,name=Condition,sibling=theorem]{condition}


\mdfdefinestyle{mdredbox}{%
skipabove=8pt,
skipbelow=0pt,
linewidth=2pt,
rightline=false,
leftline=true,
topline=false,
bottomline=false,
linecolor=RawSienna,
%backgroundcolor=Salmon!5,
rightline=false,
leftline=true,
topline=false,
bottomline=false,
innerbottommargin=1pt,
innertopmargin=-2pt,
}
\declaretheoremstyle[
headfont=\bfseries\sffamily\color{RawSienna},
bodyfont=\normalfont,
spaceabove=2pt,
spacebelow=1pt,
mdframed={style=mdredbox},
headpunct={.},
]{thmredbox}

\declaretheorem[style=thmredbox,name=Example,sibling=theorem]{example}
\declaretheorem[style=thmredbox,name=Sample Problem,sibling=theorem]{sample}

\newenvironment{sampans}{\noindent\textit{Solution.}}{\hfill$\square$}
\newenvironment{sketch}{\noindent\textit{Sketch.}}{\hfill$\square$}

\theoremstyle{remark}
\newtheorem*{remark}{Remark}

% SUPERPART
% Define the superpart command
\newcommand{\superpart}[1]{%
  \cleardoublepage
  \thispagestyle{empty}% No header/footer
  \vbox to \textheight{%
    \vfill
    \begin{center}
      \Huge\bfseries\sffamily
      \MakeUppercase{#1} % Title
    \end{center}
    \vfill
  }
  %\addcontentsline{toc}{part}{\numberline{}\MakeUppercase{#1}}%
  \addcontentsline{toc}{part}{\MakeUppercase{#1}}%
}

% CHAPTER NUMBERING
\newcommand{\stopchapternumbers}{%
\setcounter{secnumdepth}{-1}
}

% DIFFICULTY

\makeatletter
\newcount\my@repeat@count
\newcommand{\myrepeat}[2]{%
  \begingroup
  \my@repeat@count=\z@
  \@whilenum\my@repeat@count<#1\do{#2\advance\my@repeat@count\@ne}%
  \endgroup
}
\makeatother

\newcommand{\difficulty}[1]{\myrepeat{#1}{$\blacktriangleright$}}

\newcommand{\overbow}[1]{
  \tikzexternaldisable
  \tikz [baseline = (N.base), every node/.style={}] {
    \node [inner sep = 0pt] (N) {$#1$};
    \draw [line width = 0.4pt] plot [smooth, tension=1.3] coordinates {
        ($(N.north west) + (0.1ex,0)$)
        ($(N.north)      + (0,0.5ex)$)
        ($(N.north east) + (0,0)$)
    };
  }
  \tikzexternalenable
}

\def\svgwidth{\linewidth}

\makeatletter
\renewcommand*\env@matrix[1][*\c@MaxMatrixCols c]{%
\hskip -\arraycolsep
\let\@ifnextchar\new@ifnextchar
\array{#1}}
\makeatother