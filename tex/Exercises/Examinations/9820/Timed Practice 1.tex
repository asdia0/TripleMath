\section{9820 Timed Practice 1}

\begin{problem}
    At a school charity bazaar, students sell chicken nuggets in boxes of 5 or 11. Customers can buy any combination of boxes of 5 or 11 chicken nuggets.

    \begin{enumerate}
        \item Show that it is possible to buy exactly $k$ chicken nuggets, for all integer values of $k$ between 40 and 44 inclusive.
        \item Use mathematical induction to show that it is possible to buy exactly $k$ chicken nuggets for all integer values of $k$ greater than or equal to 40.
        \item Determine if it is possible to buy exactly 39 chicken nuggets.
        \item In the general case where they sell chicken nuggets in boxes of $p_1$ or $p_2$, where $p_1$ and $p_2$ are distinct primes, show that it is not possible to buy exactly $p_1p_2 - p_1 - p_2$ chicken nuggets.
    \end{enumerate}
\end{problem}
\begin{solution}
    \begin{ppart}
        \begin{table}[H]
            \centering
            \begin{tabular}{|c|c|c|}
            \hline
            $k$ & Boxes of 5 & Boxes of 11 \\ \hline\hline
            40 & 8 & 0 \\ \hline
            41 & 6 & 1 \\ \hline
            42 & 4 & 2 \\ \hline
            43 & 2 & 3 \\ \hline
            44 & 0 & 4 \\ \hline
            \end{tabular}
        \end{table}
    \end{ppart}
    \begin{ppart}
        Let $P(5n + r)$ be the statement ``it is possible to buy exactly $5n+r$ chicken nuggets'', where $n \in \ZZ$ and $r \in \bc{0, 1, 2, 3, 4}$. From part (a), the statement holds for $n = 8$ and $r \in \bc{0, 1, 2, 3, 4}$. We now induct on $n$. Suppose $P(5k+r)$ is true for some $k \in \ZZ$. Then there exist positive integers $x$ and $y$ such that \[5x + 11y = 5k+r.\] It follows that \[5\bp{k+1} + r = 5\bp{x + 1} + 11y.\] Hence, taking $(x+1)$ boxes of 5 chicken nuggets and $y$ boxes of chicken nuggets, we obtain exactly $5(k+1) + r$ chicken nuggets. Hence, $P(5k+r) \implies P(5(k+1) + r)$. This closes the induction. We thus conclude that we can get exactly $m$ chicken nuggets for all $m \geq 8 \cdot 5 = 40$.
    \end{ppart}
    \begin{ppart}
        By way of contradiction, suppose we can buy exactly 39 chicken nuggets. Then there exist positive integers $x$ and $y$ such that \[5x + 11y = 39.\] Reducing modulo 5, we see that \[y \equiv 4 \pmod{5},\] so $\min y = 4$. Thus, \[\min \bp{5x + 11y} \geq 11 \min y = 44 > 39.\] Thus, such a $y$ cannot exist. Therefore, we cannot buy exactly 39 nuggets.
    \end{ppart}
    \begin{ppart}
        By way of contradiction, suppose we can buy exactly $p_1p_2 - p_1 - p_2$ nuggets. Then there exist positive integers $x$ and $y$ such that \[p_1 x + p_2 y = p_1 p_2 - p_1 - p_2.\] Reducing modulo $p_1$, we see that \[p_2 y \equiv -p_2 \pmod{p_1} \implies p_2 \bp{y + 1} \equiv 0 \pmod{p_1}.\] Because $p_1$ and $p_2$ are distinct, they must be coprime, so \[y + 1 \equiv 0 \pmod{p_1}.\] It follows that $\min y = p_1 - 1$. Thus, \[\min \bp{p_1 x + p_2 y} \geq p_2 \min y = p_2 \bp{p_1 - 1} = p_1 p_2 - p_2 > p_1 p_2 - p_1 - p_2.\] Thus, such a $y$ cannot exist. Therefore, we cannot buy exactly $p_1 p_2 - p_1 - p_2$ nuggets.
    \end{ppart}
\end{solution}

\begin{problem}
    Let $g(x) = a_0 + a_1 x + a_2 x^2 + \dots + a_n x^n$ be a polynomial with integer coefficients, that is, $a_0, a_1, \dots, a_n \in \ZZ$, with $a_0, a_n \neq 0$.

    \begin{enumerate}
        \item \begin{enumerate}
            \item By writing \[g(x) = x^n \bp{\frac{a_0}{x^n} + \frac{a_1}{x^{n-1}} + \frac{a_2}{x^{n-1}} + \dots + \frac{a_{n-1}}{x} + a_n},\] explain why as $x \to \infty$, either $g(x) \to \infty$ or $g(x) \to -\infty$.
            \item Let \[P = \bc{p : \text{$p$ is a prime and $\exists \, k \in \ZZ$ such that $p \mid g(k)$}}.\] Show that $P$ contains infinitely many elements.
        \end{enumerate}
        \item Show that there exists an integer $m > 2025$ such that $\abs{g(m)}$ is not prime.
    \end{enumerate}
\end{problem}
\begin{solution}
    \begin{ppart}
        \begin{psubpart}
            Note that for all $k \in \RR$ and $r \in \NN$, we have \[\lim_{x \to \infty} \frac{k}{x^r} = 0.\] Thus, \[\lim_{x \to \infty} g(x) = \lim_{x \to \infty} x^n \bp{\frac{a_0}{x^n} + \frac{a_1}{x^{n-1}} + \frac{a_2}{x^{n-1}} + \dots + \frac{a_{n-1}}{x} + a_n} = \lim_{x \to \infty} a_n x^n,\] which approaches $\infty$ if $a_n > 0$, or $-\infty$ if $a_n < 0$.
        \end{psubpart}
        \begin{psubpart}
            By way of contradiction, suppose $P$ is finite, say $P = \bc{p_1, \dots, p_k}$. Then for all $m \in \ZZ$, either $g(m) = 0$ or, by the Fundamental Theorem of Arithmetic, \[g(m) = \pm p_1^{a_1} p_2^{a_2} \dots p_k^{a_k},\] where $a_i \in \ZZ_0^+$. By part (a)(i), there exists some $M \in \NN$ such that $g(m) \geq 0$ for all $m \geq M$. Define \[S = \sum_{m = M}^\infty \frac1{g(m)^{1/n}}.\] Since $g$ is a polynomial of degree $n$, it can take on a given value at most $n$ times. Thus, for a given set of integers $a_1, a_2, \dots, a_k$, \[g(m)^{1/n} = p_1^{a_1/n} p_2^{a_2/n} \dots p_k^{a_k/n}\] for at most $n$ values of $m$. We thus obtain the following estimate: \[S \leq \sum_{a_1, \dots, a_k = 0}^\infty \frac{n}{p_1^{a_1/n} \dots p_k^{a_k/n}} = n \prod_{r = 1}^k \sum_{a_r = 0}^\infty p_r^{-a_r/n} = n \prod_{r = 1}^k \frac{1}{1 - p_r^{-1/n}},\] which is finite. But this is a clear contradiction, since \[\frac1{g(m)^{1/n}} \sim \frac1{a_n^{1/n} x},\] so $S$ diverges like the harmonic series. Thus, $P$ must be infinite.
        \end{psubpart}
    \end{ppart}
    \begin{ppart}
        By way of contradiction, suppose $\abs{g(m)}$ for all integers $m > 2025$. Let $p = \abs{g(m)}$, which is prime, and $t \in \NN$. Observe that \[g(m + tp) = \sum_{i = 0}^n a_i \bp{m + tp}^i \equiv \sum_{i = 0}^n a_i m^i = g(m) \equiv 0 \pmod{p}.\] Since $m + tp > 2025$, by our assumption, $g(m+tp)$ must also be prime. Thus, \[\pm p = g(m) = g(m + p) = g(m + 2p) = g(m + 3p) = \dots,\] so there are infinitely many solutions to the equation $g(x) = \pm p$. However, because $g$ is a polynomial of degree $n$, there are at most $2n$ solutions to $g(x) = \pm p$, a contradiction. Thus, there must exist some $m > 2025$ such that $\abs{g(m)}$ is not prime.
    \end{ppart}
\end{solution}