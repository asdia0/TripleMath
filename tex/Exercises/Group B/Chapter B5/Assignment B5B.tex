\section{Assignment B5B}

\begin{problem}
    Sketch the curve with parametric equations \[x = 3t, \, y = \frac3t.\] The point $P$ on the curve has parameter $t = 2$. The normal at $P$ meets the curve again at the point $Q$.

    \begin{enumerate}
        \item Show that the normal at $P$ has equation $2y = 8x - 45$.
        \item Find the value of $t$ at $Q$.
    \end{enumerate}
\end{problem}
\begin{solution}
    \begin{center}\tikzsetnextfilename{275}
        \begin{tikzpicture}[trim axis left, trim axis right]
            \begin{axis}[
                domain = -10:10,
                samples = 101,
                axis y line=middle,
                axis x line=middle,
                xtick = \empty,
                ytick = \empty,
                xlabel = {$x$},
                ylabel = {$y$},
                legend cell align={left},
                legend pos=outer north east,
                after end axis/.code={
                    \path (axis cs:0,0) 
                        node [anchor=north east] {$O$};
                    }
                ]
                \addplot[plotRed, unbounded coords = jump] ({3*x}, {3/x});

                \addlegendentry{$x = 3t$, $y = \frac3t$}
            \end{axis}
        \end{tikzpicture}
    \end{center}

    \begin{ppart}
        Consider $\derx{y}{x}$. \[\der{y}{x} = \frac{\derx{y}{t}}{\derx{x}{t}} = \frac{-3/t^2}{3} = -\frac1{t^2}.\] Hence, the tangent to the curve has gradient $-1/t^2$, whence the normal to the curve has gradient $\frac{-1}{-1/t^2} = t^2$. Thus, the normal to the curve at $P$ has gradient $2^2 = 4$. Note that $P$ has coordinates $(6, 3/2)$. Using the point-slope formula, the normal at $P$ has equation \[y - \frac32 = 4(x - 6) \implies 2y = 8(x-6) + 3 = 8x - 45.\]
    \end{ppart}
    \begin{ppart}
        Observe that \[x = 3t \implies t = \frac{x}3 \implies y = \frac3{x/3} = \frac9x.\] Substituting $y = 9/x$ into the equation of the normal at $P$, \[2\bp{\frac9x} = 8x - 45 \implies 8x^2 - 45x - 18 = (x-6)(8x+3) = 0.\] Hence, the $x$-coordinate of $Q$ is $-3/8$ (note that we reject $x = 6$ since that corresponds to $P$). Thus, $t = -1/8$ at $Q$.
    \end{ppart}
\end{solution}

\clearpage
\begin{problem}
    \begin{center}\tikzsetnextfilename{276}
        \begin{tikzpicture}
            \draw (0, 0) -- (3, 0);

            \draw (0, 5) -- (3, 5);

            \draw[dotted, thick] (0, 0) -- (0, 5);

            \draw[dotted, thick] (3, 0) -- (3, 5);

            \draw (3, 0) arc (-90:90:2.5);

            \draw (0, 0) arc (90:-90:-2.5);

            \node[anchor=west] at (0, 2.5) {$y$};

            \node[anchor=north] at (1.5, 5) {$x$};
        \end{tikzpicture}
    \end{center}

    A pond with a constant depth of 1 m is being designed for a park. The pond comprises a rectangle $x$ m by $y$ m and two semicircles of diameter $y$ m, as shown in the diagram. The cost to build a boundary around the pond is \$30 per metre for straight parts and \$60 per metre for the semicircular parts. Given that the budget for the boundary is fixed at \$6000, using differentiation or otherwise, find in terms of $\pi$, the exact values of $x$ and $y$ which give the pond a maximum volume.
\end{problem}
\begin{solution}
    Observe that the total length of the straight parts is $2x$ m and the total length of the semicircular parts is $\pi y$ m. Hence, \[30(2x) + 60(\pi y) = 6000 \implies x + \pi y = 100 \implies x = 100 - \pi y.\] Let $V(y)$ m$^3$ be the volume of the pond. \[V(y) = \pi \bp{\frac{y}{2}}^2 + xy = \frac{\pi}{4} y^2 + \bp{100 - \pi y}y = -\frac{3\pi}4 y^2 + 100 y.\] Consider the stationary points of $V(y)$. For stationary points, $V'(y) = 0$. \[V'(y) = -\frac{3\pi}2 y + 100 = 0 \implies y = \frac{200}{3\pi}.\]

    \begin{table}[h]
        \centering
        \begin{tabular}{|c|c|c|c|}
        \hline
        $y$ & $\bp{\frac{200}{3\pi}}^-$ & $\frac{200}{3\pi}$ & $\bp{\frac{200}{3\pi}}^+$ \\\hline
        $V'(y)$ & +ve   & 0 & $-$ve   \\\hline
        \end{tabular}
    \end{table}

    By the first derivative test, the maximum volume of the pond is achieved when $y = 200/3\pi$. Thus, $x = 100 - \pi y = 100/3$.
\end{solution}

\begin{problem}
    A circular cylinder is expanding in such a way that, at time $t$ seconds, the length of the cylinder is $20x$ cm and the area of the cross-section is $x$ cm$^2$. Given that, when $x=5$, the area of the cross-section is increasing at a rate of $0.025$ cm$^2$s$^{-1}$, find the rate of increase at this instant of

    \begin{enumerate}
        \item the length of the cylinder,
        \item the volume of the cylinder,
        \item the radius of the cylinder.
    \end{enumerate}
\end{problem}
\begin{solution}
    Let $A = x$ cm$^2$ be the cross-sectional area of the cylinder. Then \[\der{A}{t} = \der{A}{x} \cdot\der{x}{t} = \der{x}{t}\] and \[\evalder{\der{A}{t}}{x=5} = 0.025.\]

    \begin{ppart}
        Let $L = 20x$ cm be the length of the cylinder. Then \[\der{L}{t} = 20\cdot \der{x}{t} \implies \evalder{\der{L}{t}}{x=5} =  20\bp{0.025} = 0.5.\] Thus, the length of the cylinder is increasing at a rate of 0.5 cm/s.
    \end{ppart}
    \begin{ppart}
        Let $V = AL = 20x^2$ cm$^3$ be the volume of the cylinder. Then \[\der{V}{t} = 40x \cdot \der{x}{t} \implies \evalder{\der{V}{t}}{x=5} = 40(5)(0.025) = 5.\] Thus, the volume of the cylinder is increasing at a rate of 5 cm$^3$/s.
    \end{ppart}
    \begin{ppart}
        Let $R$ cm be the radius of the cylinder. Observe that \[\pi R^2 = A = x \implies R = \sqrt{\frac{x}\pi}= \frac{\sqrt{x}}{\sqrt\pi}.\] Hence, \[\der{R}{t} = \frac{1}{\sqrt\pi} \cdot \frac1{2\sqrt{x}} \cdot \der{x}{t} \implies \evalder{\der{R}{t}}{x=5} = \frac{1}{\sqrt\pi} \bp{\frac1{2\sqrt{5}}} \bp{0.025} = 0.00315 \tosf{3}.\] Thus, the radius of the cylinder is increasing at a rate of 0.00315 cm/s.
    \end{ppart}
\end{solution}

\begin{problem}
    The curve $C$ has equation $2^{-y} = x$. The point $A$ on $C$ has $x$-coordinate $a$ where $a > 0$. Show that $\der{y}{x} = -\frac1{a\ln2}$ at $A$ and find the equation of the tangent to $C$ at $A$. Hence, find the equation of the tangent to $C$ which passes through the origin.
        
    The straight line $y = mx$ intersects $C$ at 2 distinct points. Write down the range of values of $m$.
\end{problem}
\begin{solution}
    Observe that \[2^{-y} = x \implies y = -\log_2 x = -\frac{\ln x}{\ln 2} \implies \der{y}{x} = -\frac{1}{x \ln 2}.\] At $A$, $x = a$. Hence, \[\der{y}{x} = -\frac{1}{a \ln 2}.\] Also, we clearly have $A(a, -\ln a/\ln 2)$. Using the point-slope formula, the tangent to $C$ at $A$ has equation \[y-\bp{-\frac{\ln a}{\ln 2}} = -\frac1{a\ln2}(x-a) \implies y = -\frac{x}{a \ln 2} + \frac{1 - \ln a}{\ln 2}.\] Consider the straight line $y = mx$ that is tangent to $C$ and passes through the origin. \[0 = -\frac{0}{a\ln2} + \frac{1 - \ln a}{\ln2} \implies 1-\ln a = 0 \implies a = \e.\] Hence, the equation of the tangent to $C$ that passes through the origin is \[y = -\frac{x}{\e\ln 2}.\]
    
    Consider the graph of $2^{-y} = x$.

    \begin{center}\tikzsetnextfilename{277}
        \begin{tikzpicture}[trim axis left, trim axis right]
            \begin{axis}[
                domain = 0:10,
                samples = 101,
                axis y line=middle,
                axis x line=middle,
                xtick = \empty,
                ytick = \empty,
                xlabel = {$x$},
                ylabel = {$y$},
                legend cell align={left},
                legend pos=outer north east,
                after end axis/.code={
                    \path (axis cs:0,0) 
                        node [anchor=east] {$O$};
                    }
                ]
                \addplot[plotRed] {-ln(x)/ln(2)};
    
                \addlegendentry{$2^{-y} = x$};

                \addplot[plotBlue] {-x/(e*ln(2))};

                \addlegendentry{$y = -\frac{x}{\e\ln2}$}
            \end{axis}
        \end{tikzpicture}
    \end{center}

    Hence, $m \in (-1/\e \ln 2, 0)$.
\end{solution}