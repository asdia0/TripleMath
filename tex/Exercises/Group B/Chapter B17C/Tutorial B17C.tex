\section{Tutorial B17C}

\begin{problem}
    For each of the following matrices $\mat A$, determine the eigenvalue(s) and corresponding eigenvector(s). Where $\mat A$ is diagonalizable, write down the matrix $\mat Q$ and $\mat D$ where $\mat A= \mat Q \mat D \mat Q^{-1}$.

    \begin{tasks}(3)
        \task $\begin{pmatrix}0 & -1 \\ -2 & 0\end{pmatrix}$
        \task $\begin{pmatrix}2 & 0 \\ -3 & 2\end{pmatrix}$
        \task $\begin{pmatrix}-3 & 0 \\ 0 & -3\end{pmatrix}$
        \task $\begin{pmatrix}19 & -9 & -6 \\ 25 & -11 & -9 \\ 17 & -9 & -4\end{pmatrix}$
        \task $\begin{pmatrix}5 & 0 & 0 \\ 1 & 5 & 0 \\ 0 & 1 & 5\end{pmatrix}$
        \task $\begin{pmatrix}0 & 0 & 0 \\ 0 & 0 & 0 \\ 3 & 0 & 1\end{pmatrix}$
    \end{tasks}
\end{problem}
\begin{solution}
    \begin{ppart}
        Consider $\det{\mat A - \l \mat I} = 0$: \[0 = \det{\mat A - \l \mat I} = \begin{vmatrix}-\l & -1 \\ -2 & -\l\end{vmatrix} = \l^2 - 2 \implies \l = \pm \sqrt2.\]

        Let $\vec x = \cveciix{x}{y} \neq \vec 0$ be an eigenvector.

        \case{1}[$\l = \sqrt2$] Consider $\bp{\mat A - \l \mat I} \vec x = \vec 0$: \[\bp{\mat A - \l \mat I} \vec x = \begin{pmatrix}-\sqrt2 & -1 \\ -2 & -\sqrt2\end{pmatrix} \cvecii{x}{y} = \cvecii00.\] Solving, we get $x + \frac1{\sqrt2} y = 0$. Taking $y = -\sqrt2$, we have $\vec x = \cveciix{1}{-\sqrt2}$.

        \case{2}[$\l = -\sqrt2$] Consider $\bp{\mat A - \l \mat I} \vec x = \vec 0$: \[\bp{\mat A - \l \mat I} \vec x = \begin{pmatrix}\sqrt2 & -1 \\ -2 & \sqrt2\end{pmatrix} \cvecii{x}{y} = \cvecii00.\] Solving, we get $x - \frac1{\sqrt2} y = 0$. Taking $y = \sqrt2$, we have $\vec x = \cveciix1{\sqrt2}$.

        Thus, \[\mat Q = \begin{pmatrix}1 & 1 \\ -\sqrt2 & \sqrt2 \end{pmatrix} \quad \tand \quad \mat D = \begin{pmatrix}\sqrt2 & 0 \\ 0 & -\sqrt2 \end{pmatrix}.\]
    \end{ppart}
    \begin{ppart}
        Consider $\det{\mat A - \l \mat I} = 0$: \[0 = \det{\mat A - \l \mat I} = \begin{vmatrix}2-\l & 0 \\ -3 & 2-\l\end{vmatrix} = (2-\l)^2 \implies \l = 2.\] Let $\vec x = \cveciix{x}{y} \neq \vec 0$ be an eigenvector. Consider $\bp{\mat A - \l \mat I} \vec x = \vec 0$: \[\bp{\mat A - \l \mat I} \vec x = \begin{pmatrix}0 & 0 \\ -3 & 0 \end{pmatrix} \cvecii{x}{y} = \cvecii00.\] Solving, we get $x = 0$ and $y \in \RR$. Taking $y = 1$, we have $\cveciix01$.

        Since there are fewer eigenvectors (1) than dimensions (2), $\mat A$ is not diagonalizable.
    \end{ppart}
    \begin{ppart}
        \[\mat A = \begin{pmatrix}-3 & 0 \\ 0 & -3\end{pmatrix} = \begin{pmatrix}1 & 0 \\ 0 & 1\end{pmatrix} \begin{pmatrix}-3 & 0 \\ 0 & -3\end{pmatrix} \begin{pmatrix}1 & 0 \\ 0 & 1\end{pmatrix}^{-1},\] so $\l = -3$ with eigenvectors $\cveciix10$ and $\cveciix01$.
    \end{ppart}
    \begin{ppart}
        Note that
        \begin{align*}
            \l_1 + \l_2 + \l_3 &= \abs{19} + \abs{-11} + \abs{-4} = 4,\\
            \l_1 \l_2 + \l_2 \l_3 + \l_3 \l_1 &= \begin{vmatrix}19 & -9 \\ 25 & -11\end{vmatrix} + \begin{vmatrix}-11 & -9 \\ -9 & -4\end{vmatrix} + \begin{vmatrix}19 & -6 \\ 17 & -4\end{vmatrix} = 5,\\
            \l_1 \l_2 \l_3 &= \begin{vmatrix}19 & -9 & -6 \\ 25 & -11 & -9 \\ 17 & -9 & -4\end{vmatrix} = 2.
        \end{align*}
        Thus, the characteristic polynomial of $\mat A$ is $-\l^3 + 4\l^2 - 5\l + 2$. Solving, we get $\l = 1, 2$.

        Let $\vec x = \cveciiix{x}{y}{z} \neq \vec 0$ be an eigenvector.

        \case{1}[$\l = 1$] Consider $\bp{\mat A - \l \mat I} \vec x = \vec 0$: \[\bp{\mat A - \l \mat I} \vec x = \begin{pmatrix}18 & -9 & -6 \\ 25 & -12 & -9 \\ 17 & -9 & -5\end{pmatrix} \cveciii{x}{y}{z} = \cveciii000.\] Reducing to RREF, we have \[\begin{pmatrix}1 & 0 & -1 \\ 0 & 1 & -4/3 \\ 0 & 0 & 0\end{pmatrix} \cveciii{x}{y}{z} = \cveciii000.\] Letting $z = t \in \RR$, we have \[\vec x = \cveciii{x}{y}{z} = t\cveciii1{4/3}{1}.\] Taking $t = 3$, our eigenvector is $\cveciiix343$.

        \case{2}[$\l = 2$] Consider $\bp{\mat A - \l \mat I} \vec x = \vec 0$: \[\bp{\mat A - \l \mat I} \vec x = \begin{pmatrix}17 & -9 & -6 \\ 25 & -13 & -9 \\ 17 & -9 & -6\end{pmatrix} \cveciii{x}{y}{z} = \cveciii000.\] Reducing to RREF, we have \[\begin{pmatrix}1 & 0 & -3/4 \\ 0 & 1 & -3/4 \\ 0 & 0 & 0\end{pmatrix} \cveciii{x}{y}{z} = \cveciii000.\] Letting $z = t \in \RR$, we have \[\vec x = \cveciii{x}{y}{z} = t\cveciii{3/4}{3/4}{1}.\] Taking $t = 4$, our eigenvector is $\cveciiix334$.

        Since there are fewer eigenvectors (2) than dimensions (3), $\mat A$ is not diagonalizable.
    \end{ppart}
    \begin{ppart}
        Note that the characteristic polynomial of $\mat A$ is simply $(5-\l)^3$. Hence, the only eigenvalue of $\mat A$ is $\l = 5$. Let $\vec x = \cveciiix{x}{y}{z} \neq \vec 0$ be an eigenvector. Then \[\bp{\mat A - \l \mat I} \vec x = \begin{pmatrix}0 & 0 & 0 \\ 1 & 0 & 0 \\ 0 & 1 & 0\end{pmatrix} \cveciii{x}{y}{z} = \cveciii000.\] Hence, $x = y = 0$ and $z = t \in \RR$. Thus, the only eigenvector is $\cveciiix001$. Since there are fewer eigenvectors (1) than dimensions (3), $\mat A$ is not diagonalizable.
    \end{ppart}
    \begin{ppart}
        Consider $\det{\mat A - \l \mat I} = 0$: \[0 = \det{\mat A - \l \mat I} = \begin{vmatrix}-\l & 0 & 0 \\ 0 & -\l & 0 \\ 3 & 0 & 1 - \l\end{vmatrix} = \l^2 (1-\l) \implies \l = 0, 1.\]

        Let $\vec x = \cveciiix{x}{y}{z} \neq \vec 0$ be an eigenvector.

        \case{1}[$\l = 0$] Consider $\bp{\mat A - \l \mat I} \vec x = \vec 0$: \[\bp{\mat A - \l \mat I} \vec x = \begin{pmatrix}0 & 0 & 0 \\ 0 & 0 & 0 \\ 3 & 0 & 1\end{pmatrix}\cveciii{x}{y}{z} = \cveciii000.\] Let $y = s \in \RR$ and $z = t \in \RR$. Then $x = -1/3 t$, so \[\vec x = \cveciii{-t/3}{s}{t} = s \cveciii010 + \frac13 t \cveciii{-1}03.\] Thus, the corresponding eigenvectors are $\cveciiix010$ and $\cveciiix{-1}03$.

        \case{2}[$\l = 1$] Consider $\bp{\mat A - \l \mat I} \vec x = \vec 0$: \[\bp{\mat A - \l \mat I} \vec x = \begin{pmatrix}-1 & 0 & 0 \\ 0 & -1 & 0 \\ 3 & 0 & 0\end{pmatrix}\cveciii{x}{y}{z} = \cveciii000.\] Thus, $x = y = 0$ while $z = t \in \RR$. Hence, the corresponding eigenvector is $\cveciiix001$.

        Thus, \[\mat Q = \begin{pmatrix}0 & -1 & 0 \\ 1 & 0 & 0 \\ 0 & 3 & 1\end{pmatrix} \quad \tand \quad \mat D = \begin{pmatrix}0 & 0 & 0 \\ 0 & 0 & 0 \\ 0 & 0 & 1\end{pmatrix}.\]
    \end{ppart}
\end{solution}

\begin{problem}
    Find the eigenvalues and corresponding eigenvectors of the matrix $\mat A$, where \[\mat A = \begin{pmatrix}-3 & 5 & 5 \\ -4 & 6 & 5 \\ 4 & -4 & -3\end{pmatrix}.\] Hence, find a matrix $\mat P$ and a diagonal matrix $\mat D$ such that $\mat A = \mat P^{-1} \mat D \mat P$. Find also a diagonal matrix $\mat E$ such that $\mat A^3 = \mat P^{-1} \mat E \mat P$.
\end{problem}
\begin{solution}
    Note that
    \begin{align*}
        \l_1 + \l_2 + \l_3 &= \abs{-3} + \abs{6} + \abs{-3} = 0\\
        \l_1 \l_2 + \l_2 \l_3 + \l_3 \l_1 &= \begin{vmatrix}-3 & 5 \\ -4 & 6\end{vmatrix} + \begin{vmatrix}6 & 5 \\ -4 & -3\end{vmatrix} + \begin{vmatrix}-3 & 5 \\ 4 & -3\end{vmatrix} = -7 \\
        \l_1 \l_2 \l_3 &= \begin{vmatrix}-3 & 5 & 5 \\ -4 & 6 & 5 \\ 4 & -4 & -3\end{vmatrix} = -6.
    \end{align*}
    Hence, the characteristic polynomial of $\mat A$ is $-\l^3 + 7\l -6$, whence the roots are $\l = -3, 1, 2$.
    
    Note that \[\mat A - \l \mat I = \begin{pmatrix}-3-\l & 5 & 5 \\ -4 & 6-\l & 5 \\ 4 & -4 & -3-\l\end{pmatrix}.\] The eigenvectors are thus
    \begin{align*}
        \vec e_1 &= \frac1{20} \cveciii{-3-(-3)}55 \crossp \cveciii4{-4}{-3-(-3)} = \cveciii11{-1}\\
        \vec e_2 &= \frac1{4} \cveciii{-3-1}55 \crossp \cveciii4{-4}{-3-1} = \cveciii01{-1}\\
        \vec e_3 &= -\frac1{5} \cveciii{-3-2}55 \crossp \cveciii4{-4}{-3-2} = \cveciii110.
    \end{align*}
    Thus, \[\mat P = \begin{pmatrix}1 & 0 & 1 \\ 1 & 1 & 1 \\ -1 & -1 & 0\end{pmatrix}^{-1} = \begin{pmatrix}1 & -1 & -1 \\ -1 & 1 & 0 \\ 0 & 1 & 1\end{pmatrix} \quad \tand \quad \mat D = \begin{pmatrix}-3 & 0 & 0 \\ 0 & 1 & 0 \\ 0 & 0 & 2\end{pmatrix}.\]

    Note that $\mat A^3 = \bp{\mat P^{-1} \mat D \mat P}^3 = \mat P^{-1} \mat D^3 \mat P$. Hence, \[\mat E = \mat D^3 = \begin{pmatrix}-27 & 0 & 0 \\ 0 & 1 & 0 \\ 0 & 0 & 8\end{pmatrix}.\]
\end{solution}

\begin{problem}
    Find the eigenvalues and corresponding eigenvectors of the matrix $\mat A$, where \[\mat A = \begin{pmatrix}2 & -3 & 0 \\ 1 & - 1 & 1 \\ - 1 & 3 & 1\end{pmatrix}.\] Hence, or otherwise,
    \begin{enumerate}
        \item find the eigenvalues and corresponding eigenvectors of the matrix $\mat A + 10 \mat I$, where $\mat I$ is the unit matrix of order 3,
        \item find a matrix $\mat P$ such that $\mat P \mat A \mat P^{-1}$ is a diagonal matrix.
    \end{enumerate}
\end{problem}
\begin{solution}
    Note that
    \begin{align*}
        \l_1 + \l_2 + \l_3 &= \abs{2} + \abs{-1} + \abs{1} = 2\\
        \l_1 \l_2 + \l_2 \l_3 + \l_3 \l_1 &= \begin{vmatrix}2 & -3 \\ 1 & -2\end{vmatrix} + \begin{vmatrix}-1 & 1 \\ 3 & 1\end{vmatrix} + \begin{vmatrix}2 & 0 \\ -1 & 1\end{vmatrix} = -1 \\
        \l_1 \l_2 \l_3 &= \begin{vmatrix}2 & -3 & 0 \\ 1 & - 1 & 1 \\ - 1 & 3 & 1\end{vmatrix} = -2.
    \end{align*}
    Hence, the characteristic equation of $\mat A$ is $-\l^3 + 2 \l^2 + \l - 2$, whence its roots are $\l = -1, 1, 2$.

    Note that \[\mat A -\l \mat I = \begin{pmatrix}2 - \l & -3 & 0 \\ 1 & -1 -\l & 1 \\ -1 & 3 & 1 -\l\end{pmatrix}.\] The eigenvectors are thus
    \begin{align*}
        \vec e_1 &= -\frac13 \cveciii{2-(-1)}{-3}{0} \crossp \cveciii1{-1-(-1)}{1} = \cveciii11{-1},\\
        \vec e_2 &= -\cveciii{2-1}{-3}{0} \crossp \cveciii1{-1-1}{1} = \cveciii31{-1},\\
        \vec e_3 &= \frac13 \cveciii{2-2}{-3}{0} \crossp \cveciii1{-1-2}{1} = \cveciii{-1}01.
    \end{align*}

    \begin{ppart}
        Note that \[\bp{\mat A + 10\mat I} \vec e = \mat A \vec e + 10 \vec e = \l \vec e + 10 \vec e = (\l + 10) \vec e.\] Hence, the eigenvalues of $\mat A + 10\mat I$ are 9, 11 and 12. Their corresponding eigenvectors are $\cveciiix11{-1}$, $\cveciiix31{-1}$ and $\cveciiix{-1}01$.
    \end{ppart}
    \begin{ppart}
        Note that $\mat A = \mat Q \mat D \mat Q^{-1}$, where \[\mat Q = \begin{pmatrix}1 & 3 & -1 \\ 1 & 1 & 0 \\ -1 & -1 & 1\end{pmatrix} \quad \tand \quad \mat D = \begin{pmatrix}-1 & 0 & 0 \\ 0 & 1 & 0 \\ 0 & 0 & 2\end{pmatrix}.\] Rearranging, we have $\mat Q^{-1} \mat A \mat Q = \mat D$. Hence, \[\mat P = \mat Q^{-1} = \frac12 \begin{pmatrix}-1 & 2 & -1 \\ 1 & 0 & 1 \\ 0 & 2 & 2\end{pmatrix}.\]
    \end{ppart}
\end{solution}

\begin{problem}
    \begin{enumerate}
        \item Given two square matrices $\mat A$ and $\mat B$ of the same order, show that if there exists a non-singular matrix $\mat P$ such that $\mat A = \mat P \mat B \mat P^{-1}$, then, $\mat A$ and $\mat B$ have the same eigenvalues.
        \item Hence, by considering the product \[\begin{pmatrix}1 & 0 & -1 \\ 0 & 1 & 0 \\ 0 & 0 & 1\end{pmatrix} \begin{pmatrix}4 & 0 & 0 \\ 1 & 5 & 0 \\ 2 & 3 & 7\end{pmatrix} \begin{pmatrix}1 & 0 & 1 \\ 0 & 1 & 0 \\ 0 & 0 & 1\end{pmatrix}\] or otherwise, find the eigenvalues and corresponding eigenvectors of the matrix $\mat M$, where \[\mat M = \begin{pmatrix}2 & -3 & -5 \\ 1 & 5 & 1 \\ 2 & 3 & 9\end{pmatrix}.\]
        \item Find a matrix $\mat Q$ and a diagonal matrix $\mat D$ such that $\mat M = \mat Q \mat D \mat Q^{-1}$, and deduce that if $n$ is a positive integer, then \[\mat M^n = \mat Q \begin{pmatrix}4^n & 0 & 0 \\ 0 & 5^n & 0 \\ 0 & 0 & 7^n\end{pmatrix} \mat Q^{-1}.\]
    \end{enumerate}
\end{problem}
\begin{solution}
    \begin{ppart}
        Let $\mat B = \mat Q \mat D \mat Q^{-1}$, where $\mat D$ is a diagonal matrix. Then the principal diagonal of $\mat D$ contains the eigenvalues of $\mat B$. Further, \[\mat A = \mat P \mat B \mat P^{-1} = \mat P \mat Q \mat D \mat Q^{-1} \mat P^{-1} = \bp{\mat P \mat Q} \mat D \bp{\mat P \mat Q},\] so the principal diagonal of $\mat D$ also contains the eigenvalues of $\mat A$. Thus, $\mat A$ and $\mat B$ must have the same eigenvalues.
    \end{ppart}
    \begin{ppart}
        Since \[\begin{pmatrix}4 & 0 & 0 \\ 1 & 5 & 0 \\ 2 & 3 & 7\end{pmatrix}\] is triangular, its eigenvalues are simply the entries of its principal diagonal, i.e. $\l = 4, 5, 7$. Since \[\mat M = \begin{pmatrix}1 & 0 & -1 \\ 0 & 1 & 0 \\ 0 & 0 & 1\end{pmatrix} \begin{pmatrix}4 & 0 & 0 \\ 1 & 5 & 0 \\ 2 & 3 & 7\end{pmatrix} \begin{pmatrix}1 & 0 & 1 \\ 0 & 1 & 0 \\ 0 & 0 & 1\end{pmatrix} = \begin{pmatrix}1 & 0 & -1 \\ 0 & 1 & 0 \\ 0 & 0 & 1\end{pmatrix} \begin{pmatrix}4 & 0 & 0 \\ 1 & 5 & 0 \\ 2 & 3 & 7\end{pmatrix} \begin{pmatrix}1 & 0 & -1 \\ 0 & 1 & 0 \\ 0 & 0 & 1\end{pmatrix}^{-1},\] it follows from (a) that $\mat M$ also has eigenvalues $\l = 4, 5, 7$.
    \end{ppart}
    \begin{ppart}
        Note that \[\begin{pmatrix}4 & 0 & 0 \\ 1 & 5 & 0 \\ 2 & 3 & 7\end{pmatrix} - \l \mat I = \begin{pmatrix}4 - \l & 0 & 0 \\ 1 & 5-\l & 0 \\ 2 & 3 & 7-\l\end{pmatrix}.\] The eigenvectors are hence
        \begin{align*}
            \vec e_1 &= \cveciii23{7-4} \crossp \cveciii1{5-4}0 = \cveciii{-3}3{-1},\\
            \vec e_2 &= \cveciii{4-5}00 \crossp \cveciii23{7-5} = \cveciii02{-3},\\
            \vec e_3 &= \cveciii{4-7}00 \crossp \cveciii1{5-7}0 = \cveciii006.
        \end{align*}
        Thus, \[\mat Q = \begin{pmatrix}1 & 0 & -1 \\ 0 & 1 & 0 \\ 0 & 0 & 1\end{pmatrix} \begin{pmatrix}-3 & 0 & 0 \\ 3 & 2 & 0 \\ -1 & -3 & 6\end{pmatrix} = \begin{pmatrix}-2 & 3 & -6 \\ 3 & 2 & 0 \\ -1 & -3 & -6\end{pmatrix} \quad \tand \quad \mat D = \begin{pmatrix}4 & 0 & 0 \\ 0 & 5 & 0 \\ 0 & 0 & 7\end{pmatrix}.\] Note also that \[\mat M = \underbrace{\bp{\mat Q \mat D \mat Q^{-1}} \bp{\mat Q \mat D \mat Q^{-1}} \dots \bp{\mat Q \mat D \mat Q^{-1}}}_{\text{$n$ times}} = \mat Q \mat D^n \mat Q^{-1} = \mat Q \begin{pmatrix}4^n & 0 & 0 \\ 0 & 5^n & 0 \\ 0 & 0 & 7^n\end{pmatrix} \mat Q^{-1}.\]
    \end{ppart}
\end{solution}

\begin{problem}
    The matrix $\mat A$ is given by \[\mat A = \begin{pmatrix}-1 & 5 & 0 \\ 2 & 3 & b \\ 0 & a & -1\end{pmatrix}.\]

    \begin{enumerate}
        \item Given that 5 is an eigenvalue of $\mat A$ with eigenvector $\cveciiix562$, find the values of $a$ and $b$.
        \item Find the other eigenvalues of $\mat A$ and their corresponding eigenvectors.
        \item Hence, state the matrices $P$ and $D$ such that $\mat A = \mat P \mat D \mat P^{-1}$
    \end{enumerate}

    The matrix $\mat B$ is such that $\mat B = \mat A^2 - 2 \mat A + 3 \mat I$.

    \begin{enumerate}
        \setcounter{enumi}{3}
        \item Find a matrix $\mat Q$ and a diagonal matrix $\mat E$ such that $\mat E = \mat Q \mat B \mat Q^{-1}$.
    \end{enumerate}
\end{problem}
\begin{solution}
    \begin{ppart}
        We have \[\begin{pmatrix}-1 & 5 & 0 \\ 2 & 3 & b \\ 0 & a & -1\end{pmatrix} \cveciii562 = 5\cveciii562 \implies \cveciii{25}{28+2b}{6a-2} = \cveciii{25}{30}{10}.\] Thus, $a = 2$ and $b = 1$.
    \end{ppart}
    \begin{ppart}
        Note that \[5 + \l_2 + \l_3 = \abs{-1} + \abs{3} + \abs{-1} = 1 \quad \tand \quad 5\l_2\l_3 = \begin{vmatrix}-1 & 5 & 0 \\ 2 & 3 & 1 \\ 0 & 2 & -1\end{vmatrix} = 15.\] By inspection, we have $\l_2 = -3$ and $\l_3 = -1$.

        Note that \[\mat A - \l \mat I = \begin{pmatrix}-1 - \l & 5 & 0 \\ 2 & 3-\l & 1 \\ 0 & 2 & -1-\l\end{pmatrix}.\] The eigenvectors are thus
        \begin{align*}
            \vec e_2 &= \cveciii{-1-(-3)}50 \crossp \cveciii2{3-(-3)}1 = \cveciii5{-2}2\\
            \vec e_3 &= \frac15 \cveciii{-1-(-1)}50 \crossp \cveciii2{3-(-1)}1 = \cveciii10{-2}.
        \end{align*}
    \end{ppart}
    \begin{ppart}
        We have \[\mat P = \begin{pmatrix}5 & 5 & 1 \\ 6 & -2 & 0 \\ 2 & 2 & -2\end{pmatrix} \quad \tand \quad \mat D = \begin{pmatrix}5 & 0 & 0 \\ 0 & -3 & 0 \\ 0 & 0 & -1\end{pmatrix}.\]
    \end{ppart}
    \begin{ppart}
        Note that the eigenvalues of $\mat B$ are $\l^2 - 2\l + 3 = 18, 18, 6$. Hence, \[\mat B = \mat A^2 - 2 \mat A + 3 \mat I = \mat P \bp{\mat D^2 - 2\mat D + 3\mat I} \mat P^{-1} = \mat P \begin{pmatrix}18 & 0 & 0 \\ 0 & 18 & 0 \\ 0 & 0 & 6\end{pmatrix} \mat P^{-1}.\] Thus, \[\mat Q = \mat P^{-1} = \frac1{48} \begin{pmatrix}2 & 6 & 1 \\ 6 & -6 & 3 \\ 8 & 0 & -20\end{pmatrix} \quad \tand \quad \mat E = \begin{pmatrix}18 & 0 & 0 \\ 0 & 18 & 0 \\ 0 & 0 & 6\end{pmatrix}.\]
    \end{ppart}
\end{solution}

\begin{problem}
    \begin{enumerate}
        \item Determine the eigenvalues of a square matrix, $\mat A$, if
        \begin{enumerate}
            \item $\mat A^n = \mat 0$ for some positive integer $n$,
            \item $\mat A^3 = \mat A$.
        \end{enumerate}
        \item The matrices $\mat A$ and $\mat B$ have the same eigenvectors $\vec e_1$, $\vec e_2$, $\vec e_3$. The corresponding eigenvalues of $\mat A$ are $\l_1$, $\l_2$ and $\l_3$ while the corresponding eigenvalues of $\mat B$ are $\m_1$, $\m_2$ and $\m_3$.
        \begin{enumerate}
            \item Show that the matrix $\mat A + \mat B$ has eigenvalues $\l_1 + \m_1$, $\l_2 + \m_2$ and $\l_3 + \m_3$ with corresponding common eigenvectors $\vec e_1$, $\vec e_2$, $\vec e_3$.
        \end{enumerate}

        It is given that \[\mat A = \begin{pmatrix}0 & -1 & 0 \\ -4 & -9 & -6 \\ 5 & 11 & 7\end{pmatrix}, \quad \mat B = \begin{pmatrix}-4 & -16 & -11 \\ -9 & -27 & -19 \\ 14 & 44 & 31\end{pmatrix}\] and $\m_1 = -3$, $\m_2 = 2$, $\m_3 = 1$.

        \begin{enumerate}
            \setcounter{enumii}{1}
            \item Find $\l_1$, $\l_2$, $\l_3$, where $\l_1 < \l_2 < \l_3$, and the corresponding $\vec e_1$, $\vec e_2$, $\vec e_3$.
            \item Find matrices $\mat R$ and $\mat S$ and a diagonal matrix $\mat D$ such that $(\mat A + \mat B)^5 = \mat R \mat D \mat S$.
        \end{enumerate}
    \end{enumerate}
\end{problem}
\begin{solution}
    \begin{ppart}
        Let $\l$ be an eigenvalue of $\mat A$.

        \begin{psubpart}
            Since $\mat A^5 = \mat 0$, we have $\l^5 = 0$. Hence, the eigenvalues of $\mat A$ are all 0.
        \end{psubpart}
        \begin{psubpart}
            Since $\mat A^3 = \mat A$, we have $\l^3 = \l$. Hence, $\l = -1, 0, 1$.
        \end{psubpart}
    \end{ppart}
    \begin{ppart}
        \begin{psubpart}
            Since \[\bp{\mat A + \mat B} \vec e_i = \mat A \vec e_i + \mat B \vec e_i = \l_i \vec e_i + \m_i \vec e_i = (\l_i + \m_i) \vec e_i,\] it follows that $\mat A + \mat B$ has eigenvalues $\l_i + \m_i$ with corresponding eigenvectors $\vec e_i$ for $i = 1, 2, 3$.
        \end{psubpart}
        \begin{psubpart}
            We have the system of equations
            \begin{align*}
                \l_1 + \l_2 + \l_3 &= \begin{vmatrix}0\end{vmatrix} + \begin{vmatrix}-9\end{vmatrix} + \begin{vmatrix}7\end{vmatrix} = -2,\\
                \l_1 \l_2 + \l_2 \l_3 + \l_3 \l_1 &= \begin{vmatrix}0 & -1 \\ -4 & -9\end{vmatrix} + \begin{vmatrix}-9 & -6 \\ 11 & 7\end{vmatrix} + \begin{vmatrix}0 & 0 \\ 5 & 7\end{vmatrix} = -1 \\
                \l_1 \l_2 \l_3 &= \begin{vmatrix}0 & -1 & 0 \\ -4 & -9 & -6 \\ 5 & 11 & 7\end{vmatrix} = 2.
            \end{align*}
            
            Hence, the characteristic polynomial of $\mat A$ is $-\l^3 - 2 \l^2 + \l + 2$. Solving, we have $\l_1 = -2$, $\l_2 = -1$ and $\l_3 = 1$.

            Note that \[\mat A - \l_i \mat I = \begin{pmatrix}-\l_i & -1 & 0 \\ -4 & -9 - \l_i & -6 \\ 5 & 11 & 7 - \l_i\end{pmatrix}.\] Hence, we take
            \begin{gather*}
                \vec e_1 = \frac16 \cveciii{- (-2)}{-1}{0} \crossp \cveciii{-4}{-9 - (-2)}{-6} = \cveciii12{-3},\\
                \vec e_2 = \frac16 \cveciii{- (-1)}{-1}{0} \crossp \cveciii{-4}{-9 - (-1)}{-6} = \cveciii11{-2}\\
                \vec e_3 = \frac16 \cveciii{-1}{-1}0 \crossp \cveciii{-4}{-9 - 1}{-6} = \cveciii1{-1}1.
            \end{gather*}
        \end{psubpart}
        \begin{psubpart}
            Note that $\mat A + \mat B = \mat P \mat E \mat P^{-1}$, where \[\mat P = \begin{pmatrix}\vec e_1 & \vec e_2 & \vec e_3 \end{pmatrix} = \begin{pmatrix}1 & 1 & 1 \\ 2 & 1 & -1 \\ -3 & -2 & 1\end{pmatrix}\] and \[\mat E = \begin{pmatrix}\l_1 + \m_1 & 0 & 0 \\ 0 & \l_2 + \m_2 & 0 \\ 0 & 0 & \l_3 + \m_3\end{pmatrix} = \begin{pmatrix}-5 & 0 & 0 \\ 0 & 1 & 0 \\ 0 & 0 & 2\end{pmatrix}.\] Hence, $\bp{\mat A + \mat B}^5 = \mat P \mat E^5 \mat P^{-1} = \mat R \mat D \mat S$. We thus take
            \begin{gather*}
                \mat R = \mat P = \begin{pmatrix}1 & 1 & 1 \\ 2 & 1 & -1 \\ -3 & -2 & 1\end{pmatrix},\\
                \mat S = \mat P^{-1} = \begin{pmatrix}1 & 3 & 2 \\ -1 & -4 & -3 \\ 1 & 1 & 1\end{pmatrix},\\
                \mat D = \mat E^5 = \begin{pmatrix}-3125 & 0 & 0 \\ 0 & 1 & 0 \\ 0 & 0 & 32\end{pmatrix}.
            \end{gather*}
        \end{psubpart}
    \end{ppart}
\end{solution}

\clearpage
\begin{problem}
    A square matrix $\mat A$ has eigenvector $\vec x$ with corresponding eigenvalue $\l$.

    \begin{enumerate}
        \item Show that $\vec x$ is also an eigenvector of $\mat A + \mat A^2 + \mat A^3 + \mat A^4$ with corresponding eigenvalue $\l + \l^2 + \l^3 + \l^4$.
    \end{enumerate}

    Let $\mat B = \mat A + \mat A^2 + \mat A^3 + \mat A^4$, where \[\mat A = \begin{pmatrix}1 & 3 & 4 \\ 0 & 2 & 8 \\ 0 & 0 & -3\end{pmatrix}.\]

    \begin{enumerate}
        \setcounter{enumi}{1}
        \item Using the result from part (a), find the eigenvectors and corresponding eigenvalues of $\mat B$.
        \item Hence, write down a non-singular matrix $\mat Q$ and a diagonal matrix $\mat D$ such that $\mat B = \mat Q \mat D \mat Q^{-1}$.
    \end{enumerate}
\end{problem}
\begin{solution}
    \begin{ppart}
        Observe that
        \begin{gather*}
            \bp{\mat A + \mat A^2 + \mat A^3 + \mat A^4} \vec x = \mat A \vec x + \mat A^2 \vec x + \mat A^3 \vec x + \mat A^4 \vec x\\
            = \l \vec x + \l^2 \vec x + \l^3 \vec x + \l^4 \vec x = \bp{\l + \l^2 + \l^3 + \l^4} \vec x.
        \end{gather*}
        Hence, $\vec x$ is also an eigenvector of $\mat A + \mat A^2 + \mat A^3 + \mat A^4$ with corresponding eigenvalue $\l + \l^2 + \l^3 + \l^4$.
    \end{ppart}
    \begin{ppart}
        Since $\mat A$ is triangular, the entries on its principal diagonal are precisely its eigenvectors. Hence, $\l = 1, 2, -3$. Thus, $\mat B$ has eigenvalues $\l + \l^2 + \l^3 + \l^4 = 4, 30, 60$.

        Note that \[\mat A - \l \mat I = \begin{pmatrix}1 - \l & 3 & 4 \\ 0 & 2-\l & 8 \\ 0 & 0 & -3-\l\end{pmatrix}.\] The eigenvectors of $\mat A$ and $\mat B$ are hence 
        \begin{align*}
            \vec e_1 &= -\frac1{12} \cveciii{1-1}34 \crossp \cveciii00{-3-1} = \cveciii100,\\
            \vec e_2 &= -\frac15 \cveciii{1-2}34 \crossp \cveciii00{-3-2} = \cveciii310,\\
            \vec e_3 &= \frac14 \cveciii{1-(-3)}34 \crossp \cveciii0{2-(-3)}8 = \cveciii1{-8}5.
        \end{align*}
    \end{ppart}
    \begin{ppart}
        We have \[\mat Q = \begin{pmatrix}1 & 3 & 1 \\ 0 & 1 & -8 \\ 0 & 0 & 5\end{pmatrix} \quad \tand \quad \mat D = \begin{pmatrix}4 & 0 & 0 \\ 0 & 30 & 0 \\ 0 & 0 & 60\end{pmatrix}.\]
    \end{ppart}
\end{solution}

\clearpage
\begin{problem}
    Two $n \times n$ square matrices $\mat A$ and $\mat B$ are said to commute if $\mat A \mat B = \mat B \mat A$.

    \begin{enumerate}
        \item Show that any two diagonal $n \times n$ square matrices commute.
        \item Let $\mat A$ and $\mat B$ be two $n \times n$ square matrices with the same eigenvectors. By diagonalizing the square matrices, show that $\mat A$ and $\mat B$ commute.
        \item The matrix $\mat A$ is given by \[\mat A = \begin{pmatrix}-5 & -5 & -2 \\ 8 & 4 & 4 \\ 16 & 10 & 7\end{pmatrix}.\] Find the eigenvalues of $\mat A$ and corresponding eigenvectors with integer entries.
        \item Write down a matrix $\mat P$ and a diagonal matrix $\mat D$ such that $\mat P^{-1} \mat A \mat P = \mat D$.
        \item The matrix $\mat B$ is given by \[\mat B = \begin{pmatrix}-11 & 3 & -6 \\ 8 & -2  & 4 \\ 16 & -6 & 9\end{pmatrix}.\] Determine whether $\mat B$ is diagonalizable with your matrix $\mat P$ from (d), and hence deduce whether $\mat A$ and $\mat B$ commutes.
    \end{enumerate}
\end{problem}
\begin{solution}
    \begin{ppart}
        Let $\vec x_i$ be the $i$th standard basis vector, where $1 \leq i \leq n$. Note that $\vec x_i$ is an eigenvector of a diagonal matrix $\mat D$ with corresponding eigenvalue $d_{ii}$. Hence, \[\mat A \mat B \vec x_i = \mat A \bp{b_{ii} \vec x_i} = a_{ii} b_{ii} \vec x_i \quad \tand \quad \mat B \mat A \vec x_i = \mat B \bp{a_{ii} \vec x_i} = b_{ii} a_{ii} \vec x_i.\] Hence, $\vec x_i$ has the same image under $\mat A \mat B$ and $\mat B \mat A$, whence $\mat A \mat B = \mat B \mat A$.
    \end{ppart}
    \begin{ppart}
        Since $\mat A$ and $\mat B$ have the same eigenvectors, we can write $\mat A = \mat P \mat D \mat P^{-1}$ and $\mat B = \mat P \mat E \mat P^{-1}$, where $\mat D$ and $\mat E$ are diagonal matrices. Then \[\mat A \mat B = \bp{\mat P \mat D \mat P^{-1}}\bp{\mat P \mat E \mat P^{-1}} = \mat P \mat D \mat E \mat P^{-1}\] and \[\mat B \mat A = \bp{\mat P \mat E \mat P^{-1}}\bp{\mat P \mat D \mat P^{-1}} = \mat P \mat E \mat D \mat P^{-1}.\] Since $\mat D$ and $\mat E$ are diagonal, by part (a), they commute, so $\mat D \mat E = \mat E \mat D$. Consequently, $\mat A \mat B = \mat B \mat A$, so $\mat A$ and $\mat B$ also commute.
    \end{ppart}
    \begin{ppart}
        We have the system of equations
        \begin{align*}
            \l_1 + \l_2 + \l_3 &= \begin{vmatrix}-5\end{vmatrix} + \begin{vmatrix}4\end{vmatrix} + \begin{vmatrix}7\end{vmatrix} = 6,\\
            \l_1 \l_2 + \l_2 \l_3 + \l_3 \l_1 &= \begin{vmatrix}-5 & -5 \\ 8 & 4\end{vmatrix} + \begin{vmatrix}4 & 4 \\ 10 & 7\end{vmatrix} + \begin{vmatrix}-5 & -2 \\ 16 & 7\end{vmatrix} = 5 \\
            \l_1 \l_2 \l_3 &= \begin{vmatrix}-5 & -5 & -2 \\ 8 & 4 & 4 \\ 16 & 10 & 7\end{vmatrix} = -12.
        \end{align*}
        Hence, the characteristic equation of $\mat A$ is $-\l^3 + 6\l^2 - 5 \l - 12$. Solving, we get $\l = -1, 3, 4$.

        Note that \[\mat A - \l \mat I = \begin{pmatrix}-5 -\l & -5 & -2 \\ 8 & 4-\l & 4 \\ 16 & 10 & 7-\l\end{pmatrix}.\] The corresponding eigenvectors are thus
        \begin{align*}
            \vec e_1 &= \frac1{10} \cveciii{-5-(-1)}{-5}{-2} \crossp \cveciii8{4-(-1)}4 = \cveciii{-1}02,\\
            \vec e_2 &= \frac{1}{2} \cveciii{-5-3}{-5}{-2} \crossp \cveciii8{4-3}4 = \cveciii{-9}8{16},\\
            \vec e_3 &= \frac1{20} \cveciii{-5-4}{-5}{-2} \crossp \cveciii8{4-4}{4} = \cveciii{-1}12.
        \end{align*}
    \end{ppart}
    \begin{ppart}
        Rearranging, we have $\mat A = \mat P \mat D \mat P^{-1}$. Thus, \[\mat P = \begin{pmatrix}-1 & -9 & -1 \\ 0 & 8 & 1 \\ 2 & 16 & 2\end{pmatrix} \quad \tand \quad \mat D = \begin{pmatrix}-1 & 0 & 0 \\ 0 & 3 & 0 \\ 0 & 0 & 4\end{pmatrix}.\]
    \end{ppart}
    \begin{ppart}
        Observe that \[\mat B \vec e_1 = \cveciii{-1}02 = \vec e_1, \quad \mat B \vec e_2 = \cveciii{27}{-24}{-48} = -3\vec e_1, \quad \mat B \vec e_3 = \cveciii2{-2}{-4} = -2\cveciii{-1}12 = -2\vec e_3.\] Hence, $\mat A$ and $\mat B$ share the share eigenvectors. Thus, by part (b), $\mat A$ and $\mat B$ commute.
    \end{ppart}
\end{solution}

\begin{problem}
    The linear transformation $T : \RR^2 \to \RR^2$ is represented by the matrix $\mat A$, where \[\mat A = \begin{pmatrix}2 & -4 \\ -1 & -1\end{pmatrix}.\]

    \begin{enumerate}
        \item Show that $T$ transforms any point on the line $y = x$ to a point on the same line.
        \item Explain what happens to the points on the line $4y + x = 0$ when they are transformed by $T$.
        \item State the two eigenvalues of $\mat A$ and state two eigenvectors corresponding to the two eigenvalues.
    \end{enumerate}
\end{problem}
\begin{solution}
    \begin{ppart}
        Let $\vec x$ be a point on the line $y = x$. Then $\vec x$ has the form $t \cveciix11$, where $t \in \RR$. Now observe that \[T \vec x = \begin{pmatrix}2 & -4 \\ -1 & -1\end{pmatrix} \bs{t\cvecii11} = t \cvecii{-2}{-2} = -2\vec x.\] Hence, the image of $\vec x$ under $T$ is a point on the same line $y = x$.
    \end{ppart}
    \begin{ppart}
        Let $\vec x$ be a point on the line $4y + x = 0$. Then $\vec x$ has the form $t \cveciix{-4}1$, where $t \in \RR$. Now observe that \[T \vec x = \begin{pmatrix}2 & -4 \\ -1 & -1\end{pmatrix} \bs{t\cvecii{-4}1} = t \cvecii{-12}{3} = 3\vec x.\] Hence, the image of $\vec x$ under $T$ is a point on the same line $4y + x = 0$.
    \end{ppart}
    \begin{ppart}
        The eigenvalues of $\mat A$ are $\l = -2, 3$, and their corresponding eigenvectors are $\cveciix11$ and $\cveciix{-4}1$.
    \end{ppart}
\end{solution}

\begin{problem}
    \begin{enumerate}
        \item \begin{enumerate}
            \item Suppose that an invertible $3 \times 3$ matrix $\mat A$ has non-zero eigenvalues $\l_1$, $\l_2$, $\l_3$ with corresponding eigenvectors $\vec e_1$, $\vec e_2$, $\vec e_3$. Show that the matrix $\mat A^{-1}$ has the same eigenvectors and find the corresponding eigenvalues of $\mat A^{-1}$.
            \item Another $3 \times 3$ matrix $\mat B$ has eigenvalues $\m_1$, $\m_2$, $\m_3$ with corresponding eigenvectors $\vec e_1$, $\vec e_2$, $\vec e_3$. Determine the eigenvalues and the corresponding eigenvectors of the matrix $\mat A^{-1} \mat B$. Hence, find the eigenvalue of the matrix $\mat C$ corresponding to the eigenvector $\vec e_1$, where \[\mat C = \mat I + \mat A^{-1} \mat B + \bp{\mat A^{-1} \mat B}^2 + \dots + \bp{\mat A^{-1} \mat B}^n,\] where $n \in \ZZ^+$ and $\m_1 \neq \l_1$.
        \end{enumerate}
        \item The matrix \[\mat E = \begin{pmatrix}a & b \\ -1 & 0\end{pmatrix},\] where $a, b \in \RR$, has real eigenvalues $\b_1$, $\b_2$.
        \begin{enumerate}
            \item If $\b_1 \neq \b_2$, show that $a^2 > 4b$.
            \item Assume that $a \neq 0$. State the value of $b$ when $\mat E$ is singular. Find a matrix $\mat P$ and a diagonal matrix $\mat D$ such that $\mat E = \mat P \mat E \mat P^{-1}$ when $\mat E$ is singular.
        \end{enumerate}
    \end{enumerate}
\end{problem}
\begin{solution}
    \begin{ppart}
        \begin{psubpart}
            We have $\mat A \vec e_i = \l_i \vec e_i$. Pre-multiplying by $\frac1{\l_i} \mat A^{-1}$ yields $\frac1{\l_i} \vec e_i = \mat A^{-1} \vec e_i$. Hence, $\mat A^{-1}$ has the same eigenvectors $\vec e_i$ with corresponding eigenvalues $1/\l_i$.
        \end{psubpart}
        \begin{psubpart}
            Note that \[\mat A^{-1} \mat B \vec e_i = \mat A^{-1} \m_i \vec e_i = \frac{\m_i}{\l_i} \vec e_i.\] Thus,
            \begin{gather*}
                \mat C \vec e_1 = \bs{\mat I + \mat A^{-1} \mat B + \bp{\mat A^{-1} \mat B}^2 + \dots + \bp{\mat A^{-1} \mat B}^n} \vec e_1\\
                = \bs{1 + \frac{\m_1}{\l_1} + \bp{\frac{\m_1}{\l_1}}^2 + \dots + \bp{\frac{\m_1}{\l_1}}^n} \vec e = \frac{(\m_1/\l_1)^{n+1} - 1}{\m_1/\l_1 - 1} \vec e_1.
            \end{gather*}
            Thus, the eigenvalue corresponding to $\vec e_1$ is $\frac{(\m_1/\l_1)^{n+1} - 1}{\m_1/\l_1 - 1}$.
        \end{psubpart}
    \end{ppart}
    \begin{ppart}
        \begin{psubpart}
            The characteristic polynomial of $\mat E$ is $\l^2 - a \l + b$. Since there are two real and distinct eigenvalues, the discriminant of the characteristic polynomial must be strictly greater than 0. Hence, $a^2 - 4b > 0 \implies a^2 > 4b$.
        \end{psubpart}
        \begin{psubpart}
            For $\mat E$ to be singular, we require $b = 0$. In this case, the characteristic polynomial of $\mat E$ is $\l^2 - a\l$. Its roots are $\l = 0, a$. Let $\vec x = \cveciix{x}{y} \neq 0$ be an eigenvector.

            \case{1}[$\l = 0$] Consider $\bp{\mat E - \l \mat I} \vec x = \vec 0$: \[\begin{pmatrix}a & 0 \\ -1 & 0 \end{pmatrix}\cvecii{x}{y} = \cvecii00.\] Hence, $x = 0$, while $y$ is free. Hence, $\vec x = \cvecii01$.

            \case{2}[$\l = a$] Consider $\bp{\mat E - \l \mat I} \vec x = \vec 0$: \[\begin{pmatrix}0 & 0 \\ -1 & -a\end{pmatrix} \cvecii{x}{y} = \cvecii00.\] We get the equation $x + ay = 0$. Taking $y = 1$, we have $\vec x = \cveciix{-a}{1}$.

            Thus, \[\mat P = \begin{pmatrix}0 & -a \\ 1 & 1\end{pmatrix} \quad \tand \quad \mat D = \begin{pmatrix}a & 0 \\ 0 & 0\end{pmatrix}.\]
        \end{psubpart}
    \end{ppart}
\end{solution}

\begin{problem}
    A microbiologist wants to investigate the growth of three different species of microorganisms in a controlled habitat. She sets up the following model \[\cveciii{a_n}{b_n}{c_n} = \mat Q\cveciii{a_{n-1}}{b_{n-1}}{c_{n-1}}, \quad \mat Q = \begin{pmatrix}0.42 & 0.076 & 1.16 \\ 0.6 & 0.68 & -1.2 \\ -0.1 & 0.02 & 1.2\end{pmatrix},\] where $a_n$, $b_n$ and $c_n$ represents the number of microorganisms $A$, $B$ and $C$ (in billions) respectively, $n$ hours after the start of the experiment.

    \begin{enumerate}
        \item Give an observation that the microbiologist may expect to see pertaining to the population of each of the microorganisms $A$, $B$ and $C$, justifying your answer by drawing references to the entries in the matrix $\mat Q$.
        \item Explain numerically why the model will fail in predicting the growth of the microorganisms for the initial population of $a_0 = 10$, $b_0 = 20$ and $c_0 = 30$.
    \end{enumerate}

    The microbiologist wants to predict the long-term growth of the microorganisms for the case where the scenario in (b) does not occur. She turns to a mathematician for help. The mathematician advises her to diagonalize $\mat Q$ into the form $\mat P \mat D \mat P^{-1}$, such that \[\mat P = \begin{pmatrix}5 & 1 & 2 \\ -10 & 5 & 0 \\ 1 & 0 & 1\end{pmatrix}.\]

    \begin{enumerate}
        \setcounter{enumi}{2}
        \item Find the matrices $\mat D$ and $\mat P^{-1}$ corresponding to the given matrix $\mat P$.
        \item Show how this diagonalization process can be used to help the microbiologist predict that the population of the microorganisms stabilize in the long run. Determine the equilibrium population of the microorganisms in the long run for the case where $a_0 = 40$, $b_0 = 20$ and $c_0 = 6$.
        \item Comment on one possible drawback of the model.
    \end{enumerate}
\end{problem}
\begin{solution}
    \begin{ppart}
        Microorganism $C$ will likely have a large population. Further, $q_{13} = 1.16$ means that a high population of $C$ will result in a large population of $A$, while $q_{23} = -1.2$ means that a large population of $C$ will result in a small population of $C$.
    \end{ppart}
    \begin{ppart}
        Note that \[\cveciii{a_1}{b_1}{c_1} = \mat Q\cveciii{a_{0}}{b_{0}}{c_{0}} = \begin{pmatrix}0.42 & 0.076 & 1.16 \\ 0.6 & 0.68 & -1.2 \\ -0.1 & 0.02 & 1.2\end{pmatrix} \cveciii{10}{20}{30} = \cveciii{40.52}{-16.4}{35.4}.\] This is absurd since populations cannot be negative. Hence, the model fails when $a_0 = 10$, $b_0 = 20$ and $c_0 = 30$.
    \end{ppart}
    \begin{ppart}
        Clearly, \[\mat P^{-1} = \frac1{25} \begin{pmatrix}5 & -1 & -10 \\ 10 & 3 & -20 \\ -5 & 1 & 35\end{pmatrix}.\] Rearranging the given equation, we obtain \[\mat D = \mat P^{-1} \mat Q \mat P = \begin{pmatrix}0.5 & 0 & 0 \\ 0 & 0.8 & 0 \\ 0 & 0 & 1\end{pmatrix}.\]
    \end{ppart}
    \begin{ppart}
        Note that \[\lim_{n \to \infty} \mat D^n = \lim_{n \to \infty} \begin{pmatrix}0.5^n & 0 & 0 \\ 0 & 0.8^n & 0 \\ 0 & 0 & 1^n\end{pmatrix} = \begin{pmatrix}0 & 0 & 0 \\ 0 & 0 & 0 \\ 0 & 0 & 1\end{pmatrix}.\] Hence, \[\lim_{n \to \infty} \cveciii{a_n}{b_n}{c_n} = \lim_{n \to \infty} \mat Q^n \cveciii{a_0}{b_0}{c_0} = \mat P \begin{pmatrix}0 & 0 & 0 \\ 0 & 0 & 0 \\ 0 & 0 & 1\end{pmatrix} \mat P^{-1} \cveciii{40}{20}{6} = \cveciii{2.4}0{1.2}.\] Thus, in the long run, microorganisms $A$ and $C$ will have a population of $2.4$ and $1.2$ billion respectively, while microorganism $B$ will die out.
    \end{ppart}
    \begin{ppart}
        The model does not take depletion of resources (e.g. space and food) into account.
    \end{ppart}
\end{solution}