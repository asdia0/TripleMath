\section{Assignment B17C}

\begin{problem}
    \begin{enumerate}
        \item The matrix $\mat A$ is given by \[\mat A = \begin{pmatrix}1 & c & 3 \\ 4 & 1 & 0 \\ 3 & 0 & 1\end{pmatrix}.\] It is given that $\mat A$ has an eigenvalue of 6. Find the value of $c$ and the remaining eigenvalues.
        \item Hence, find matrices $\mat P$ and $\mat D$ such that $\mat A = \mat P \mat D \mat P^{-1}$, where $\mat D$ is a diagonal matrix.
        \item It is given that three functions $y_1, y_2, y_3$ are the solutions of the following system of differential equations:
        \begin{align*}
            \der{y_1}{x} &= y_1 + c y_2 + 3 y_3,\\
            \der{y_2}{x} &= 4y_1 + y_2,\\
            \der{y_3}{x} &= 3y_1 + y_3,
        \end{align*}
        where $c$ is the value found in (a).

        By considering $\vec U = \mat P^{-1} \vec Y$, where \[\vec U = \cveciii{u_1}{u_2}{u_3} \quad \tand \quad \vec Y = \cveciii{y_1}{y_2}{y_3},\] show that the above system can be rewritten as $\vec U' = \mat D \vec U$, where \[\vec U' = \cveciii{\derx{u_1}{x}}{\derx{u_2}{x}}{\derx{u_3}{x}}.\] You may assume that $\vec U' = \mat P^{-1} \vec Y'$, where \[\vec Y' = \cveciii{\derx{y_1}{x}}{\derx{y_2}{x}}{\derx{y_3}{x}}.\]
        \item Hence, or otherwise, find the general solution of the functions $y_1, y_2, y_3$ in terms of $x$.
    \end{enumerate}
\end{problem}
\begin{solution}
    \begin{ppart}
        Let the characteristic polynomial of $\mat A$ be $\c(\l) = -\l^3 + E_1 \l^2 - E_2 \l^1 + E_3$. We have
        \begin{align*}
            E_1 &= \abs{1} + \abs{1} + \abs{1} = 3,\\
            E_2 &= \begin{vmatrix}1 & c \\ 4 & 1 \end{vmatrix} + \begin{vmatrix}1 & 0 \\ 0 & 1\end{vmatrix} + \begin{vmatrix}1 & 3 \\ 3 & 1\end{vmatrix} = -6 - 4c,\\
            E_3 &= \begin{vmatrix}1 & c & 3 \\ 4 & 1 & 0 \\ 3 & 0 & 1\end{vmatrix} = -8-4c.
        \end{align*}
        Thus, $\c(\l) = -\l^3 + 3\l^2 + \bp{6 + 4c} \l - \bp{8 + 4c}$. Since $\c(6) = 0$, we obtain $c = 4$. Hence, $\c(\l) = -\l^3 + 3\l^2 + 22\l - 24 = -\bp{\l-6}\bp{\l-1}\bp{\l+4}$, and the other eigenvalues are 1 and $-4$.
    \end{ppart}
    \begin{ppart}
        Note that \[\mat A - \l \mat I = \begin{pmatrix}1 - \l & 4 & 3 \\ 4 & 1 - \l & 0 \\ 3 & 0 & 1 - \l\end{pmatrix}.\]
        
        \case{1}[$\l = 6$] Note that \[\cveciii4{1-6}0 \crossp \cveciii30{1-6} = 5\cveciii543.\] We thus take $\cveciiix543$ to be our eigenvector corresponding to $\l = 6$.

        \case{2}[$\l = 1$] Note that \[\cveciii{1-1}43 \crossp \cveciii30{1-1} = 3\cveciii03{-4}.\] We thus take $\cveciiix03{-4}$ to be our eigenvector corresponding to $\l = 1$.

        \case{1}[$\l = -4$] Note that \[\cveciii4{1-(-4)}0 \crossp \cveciii30{1-(-4)} = -5\cveciii{-5}43.\] We thus take $\cveciiix{-5}43$ to be our eigenvector corresponding to $\l = -4$.

        Thus, \[\mat P = \begin{pmatrix}5 & 0 & -5 \\ 4 & 3 & 4 \\ 3 & -4 & 3\end{pmatrix} \quad \tand \quad \mat D = \begin{pmatrix}6 & 0 & 0 \\ 0 & 1 & 0 \\ 0 & 0 & -4\end{pmatrix}.\]
    \end{ppart}
    \begin{ppart}
        Since $\mat Y' = \mat A \mat Y = \mat P \mat D \mat P^{-1} \mat Y$, we get $\mat P^{-1} \mat Y' = \mat D \mat P^{-1} \mat Y$. But $\mat U = \mat P^{-1} \mat Y$ and $\mat U' = \mat P^{-1} \mat Y'$, so we obtain $\mat U' = \mat D \mat U$ as desired.
    \end{ppart}
    \begin{ppart}
        $\mat U' = \mat D \mat U$ expands as \[\vec U' = \cveciii{\derx{u_1}{x}}{\derx{u_2}{x}}{\derx{u_3}{x}} = \begin{pmatrix}6 & 0 & 0 \\ 0 & 1 & 0 \\ 0 & 0 & -4\end{pmatrix} \cveciii{u_1}{u_2}{u_3}.\] This gives us the differential equations \[\der{u_1}{x} = 6 u_1, \quad \der{u_2}{x} = u_2, \quad \der{u_3}{x} = -4 u_3,\] which have respective solutions $u_1 = c_1 \e^{6x}$, $u_2 = c_2 \e^{x}$, and $u_3 = c_3 \e^{-4x}$. Since $\mat U = \mat P^{-1} \mat Y$, we have $\mat Y = \mat P \mat U$, so \[\cveciii{y_1}{y_2}{y_3} = \begin{pmatrix}5 & 0 & -5 \\ 4 & 3 & 4 \\ 3 & -4 & 3\end{pmatrix} \cveciii{c_1 \e^{6x}}{c_2 \e^{x}}{c_3 \e^{-4x}},\] whence
        \begin{align*}
            y_1 &= 5c_1 \e^{6x} - 5c_3 \e^{-4x},\\
            y_2 &= 4c_1 \e^{5x} + 3c_2 \e^{x} + 4c_3 \e^{-4x},\\
            y_3 &= 3c_1 \e^{6x} - 4c_2 \e^{x} + 3c_3 \e^{-4x}.
        \end{align*}
    \end{ppart}
\end{solution}

\clearpage
\begin{problem}
    The $3 \times 3$ non-singular matrix $\mat A$ has eigenvectors $\vec e_1, \vec e_2, \vec e_3$ with corresponding eigenvalues $\a$, $\b$ and $\g$ respectively. The three eigenvectors are linearly independent, and the eigenvalues are all non-zero real numbers. The eigenvectors of the $3 \times 3$ matrix $\mat B$ are also $\vec e_1, \vec e_2, \vec e_3$ and the corresponding eigenvalues are $\a - \b \g$, $\b - \g \a$ and $\g - \a \b$ respectively.

    \begin{enumerate}
        \item The characteristic equation of $\mat A$ is $x^3 - x^2 + k x + 4 = 0$, where $k$ is a real constant. Find an expression for the matrix $\mat B$ but in terms of the matrix $\mat A$.
    \end{enumerate}

    The transformation $T : \RR^3 \to \RR^3$ is represented by the matrix $\mat B$.

    \begin{enumerate}
        \setcounter{enumi}{1}
        \item Show that $\bc{\vec e_1, \vec e_2, \vec e_3}$ forms a basis for the range of $T$.
    \end{enumerate}
\end{problem}
\begin{solution}
    \begin{ppart}
        Note that $x^3 - x^2 + k x + 4 = (x - \a)(x - \b)(x - \g)$. By Vieta's formula, we see that $\a\b\g = -4$. Thus, $\mat B$ has eigenvalues $\a + 4/\a$, $\b + 4/\b$ and $\g + 4/\g$, so $\mat B = \mat A + 4 \mat A^{-1}$.
    \end{ppart}
    \begin{ppart}
        Let $\l$ be an eigenvalue of $\mat A$. The corresponding eigenvalue of $\mat B$ is $\l + 4/\l$, which is non-zero, since \[\l + \frac4\l = \frac1\l \bp{\l^2 + 4}\] has no real roots. Hence, $\mat B$ has full rank, so $\Range{T} = \RR^3$. But because the eigenvectors $\vec e_1, \vec e_2, \vec e_3$ are linearly independent, they span $\RR^3$, so they form a basis for $\Range{T}$.
    \end{ppart}
\end{solution}

\begin{problem}
    The matrix $\mat A$ is given by \[\mat A = \begin{pmatrix}p & 5 & 0 \\ -1 & - 1 & 1 - p \\ -1 & 2 & 1\end{pmatrix}.\]

    \begin{enumerate}
        \item Find the possible number(s) of real eigenvalues of $\mat A$, and the corresponding range of values of $p$.
    \end{enumerate}

    Joe makes the following assertion: ``If $p = -1$, then $\mat A$ is not diagonalizable.''

    \begin{enumerate}
        \setcounter{enumi}{1}
        \item Explain, with working, whether Joe's assertion is true.
    \end{enumerate}
\end{problem}
\begin{solution}
    \begin{ppart}
        Let $\c(\l) = -\l^3 + E_1 \l^2 - E_2 \l + E_3$. Note that
        \begin{align*}
            E_1 &= \abs{p} + \abs{-1} + \abs{1} = p,\\
            E_2 &= \begin{vmatrix}p & 5 \\ -1 & - 1\end{vmatrix} + \begin{vmatrix}-1 & 1-p \\ 2 & 1\end{vmatrix} + \begin{vmatrix}p & 0 \\ -1 & 1\end{vmatrix} = 2p+2,\\
            E_3 &= \begin{vmatrix}p & 5 & 0 \\ -1 & - 1 & 1 - p \\ -1 & 2 & 1\end{vmatrix} = p\bp{2p+2}.
        \end{align*}
        Thus, $\c(\l) = -\l^3 + p\l^2 - \bp{2p+2}\l + p \bp{2p+2} = \bp{-\l + p} \bp{\l^2 + 2p + 2}$.

        \case{1} If $2p + 2 < 0$, i.e. $p < -1$, then $\mat A$ has three real and distinct eigenvalues, namely $p$, $\sqrt{-\bp{2p- 2}}$ and $\sqrt{-\bp{2p-2}}$.

        \case{2} If $2p + 2 = 0$, i.e. $p = -1$, then $\mat A$ has two real and distinct eigenvalues, namely $p = -1$ and $0$.

        \case{3} If $2p + 2 > 0$, i.e. $p > -1$, then $\mat A$ has one real eigenvalue, namely $p$.
    \end{ppart}
    \begin{ppart}
        When $p = -1$, $\mat A$ has two eigenvalues, namely $-1$ and $0$. Note that \[\mat A - \l \mat I = \begin{pmatrix}-1-\l & 5 & 0 \\ -1 & - 1-\l & 2 \\ -1 & 2 & 1-\l\end{pmatrix}.\]

        \case{1}[$\l = -1$] Note that \[\cveciii050 \crossp \cveciii{-1}02 = 5\cveciii201.\] The eigenvector corresponding to $\l = -1$ is thus $\cveciiix201$.

        \case{2}[$\l = 0$] Note that \[\cveciii{-1}50 \crossp \cveciii{-1}{-1}{2} = 2\cveciii513.\] The eigenvector corresponding to $\l = 0$ is thus $\cveciiix513$.

        Thus, $\mat A$ has two independent eigenvectors when $p = -1$. Since there are less than 3 independent eigenvectors, $\mat A$ is not diagonalizable, so Joe is correct.
    \end{ppart}
\end{solution}