\section{Assignment B19}

\begin{problem}
    Explain why it is better to use a Wilcoxon matched-pair signed rank test, rather than a sign test, to test for a difference between two populations.

    The task completion times, in minutes, for a random sample of 12 operatives using two different methods are given in the table below.

    \begin{table}[H]
        \centering
        \begin{tabular}{|c|c|c|c|c|c|c|c|c|c|c|c|c|}
            \hline
            Operative & 1 & 2 & 3 & 4 & 5 & 6 & 7 & 8 & 9 & 10 & 11 & 12 \\ \hline
            Method A & 9.1 & 8.6 & 8.2 & 9.0 & 8.7 & 9.1 & 9.5 & 8.9 & 10.0 & 9.6 & 9.5 & 8.3 \\ \hline
            Method B & 8.4 & 8.8 & 7.6 & 9.4 & 9.2 & 8.2 & 9.4 & 7.9 & 8.7 & 8.4 & 8.7 & 8.0 \\ \hline
        \end{tabular}
    \end{table}

    \begin{enumerate}
        \item Use both tests mentioned above to test, at the 5\% significance level, whether Method B results in a smaller median completion time than Method A. Comment on the results.
        \item Test, at the 5\% significance level, whether the median time for Method A is 8.3 minutes.
    \end{enumerate}
\end{problem}
\begin{solution}
    A Wilcoxon matched-pair signed rank test accounts for the sign and magnitude of the differences between samples of two populations, while a sign test only accounts for the sign. Hence, a Wilcoxon matched-pair signed rank test is more powerful than a sign test.

    \begin{ppart}
        Consider the differences ``Method A'' $-$ ``Method B''. Let $m$ be the median of these differences. Our hypotheses are \nullhyp: $m = 0$, \althyp: $m > 0$. We take a 5\% significance level.

        \begin{table}[H]
            \centering
            \begin{tabular}{|c|c|c|c|c|c|c|c|c|c|c|c|c|}
            \hline
            A $-$ B & 0.7 & $-0.2$ & 0.6 & $-0.4$ & $-0.5$ & 0.9 & 0.1 & 1 & 1.3 & 1.2 & 0.8 & 0.3 \\ \hline
            Rank & 7 & 2 & 6 & 4 & 5 & 9 & 1 & 10 & 12 & 11 & 8 & 3 \\ \hline
            \end{tabular}
        \end{table}

        \case{1}[Sign Test] Let $K_+$ be the number of positive differences. From the above table, $k_+ = 9$. Under \nullhyp, $K_+ \sim \Binom{12}{1/2}$. Hence, the $p$-value is $\P{K_+ \geq 9} = 0.0730$, which is greater than our 5\% significance level. Thus, we do not reject \nullhyp{} and conclude there is insufficient evidence to claim at a 5\% significance level that Method B results in a smaller median completion time than Method A.

        \case{2}[Wilcoxon Matched-Pair Signed Rank Test] Let $P$ and $Q$ be the sum of ranks corresponding to the positive and negative differences respectively. Let $T$ be the smaller of the two. From the above table, $p = 67$ and $q = 11$, so $t = 11$. From the formula list, we reject \nullhyp{} if $t \leq 17$. Since $t = 11 \leq 17$, we reject \nullhyp{} and conclude there is sufficient evidence to claim at a 5\% significance level that Method B results in a smaller median completion time than Method A.

        Since the Wilcoxon test is more powerful than the sign test, we should reject \nullhyp.
    \end{ppart}
    \begin{ppart}
        Let $m'$ be the median of completion times for Method A. We perform a sign test at a 5\% significance level. Our hypothesis are \nullhyp: $m' = 8.3$, \althyp: $m' \neq 8.3$. Let $K_+$ be the number of completion times for Method A that takes longer than 8.3 minutes.

        From the data, the signs are \[+, \, +, \, -, \, +, \, +, \, +, \, +, \, +, \, +, \, +, \, +, \, 0,\] so $k_+ = 10$. We discard the 0 and reduce our sample size to $n = 11$. Under \nullhyp, $K_+ \sim \Binom{11}{1/2}$, so the $p$-value is $2\P{K_+ \geq 10} = 0.0117$, which is less than our 5\% significance level. Thus, we reject \nullhyp{} and conclude there is sufficient evidence to claim at a 5\% significance level that the median time for Method A differs from 8.3 minutes.
    \end{ppart}
\end{solution}

\begin{problem}
    A group of 14 students from the 19/20 DHS FM course participated in an experiment on writing speeds using dominant and non-dominant hands. Each student was to write with each of their 2 hands in 30 seconds the alphabets A to Z, and repeating if time permits. Out of the 14, 7 were randomly assigned to use the dominant hand, followed by the non-dominant hand whereas the rest were to use the non-dominant followed by the dominant.

    The number of alphabets each student wrote were recorded in the table below.

    \begin{table}[H]
        \centering
        \begin{tabular}{|c|c|c|c|c|c|c|c|c|c|c|c|c|c|c|}
            \hline
            Student & 1 & 2 & 3 & 4 & 5 & 6 & 7 & 8 & 9 & 10 & 11 & 12 & 13 & 14 \\ \hline
            Dominant & 57 & 52 & 60 & 46 & 26 & 63 & 54 & 65 & 67 & 48 & 78 & 70 & 45 & 40 \\ \hline
            Non-dominant & 35 & 21 & 20 & 16 & 23 & 31 & 17 & 28 & 36 & 41 & 33 & 22 & 13 & 25 \\ \hline
        \end{tabular}
    \end{table}

    Taking the 14 students as a random sample of the DHS 19/20 cohort, test at the 1\% significance level, the hypothesis that DHS students are able to write, on average, at least 3 times as fast using their dominant hand as compared to their non-dominant hand.
\end{problem}
\begin{solution}
    Consider the difference ``Dominant'' $-$ 3$\times$``Non-dominant'', and let $m$ be its median. Our hypotheses are \nullhyp: $m = 0$ and \althyp: $m < 0$. We perform a sign test at a 1\% significance level. Let $K_+$ be the number of positive differences.

    From the data, the signs of the differences are \[-, \, -, \, 0, \, -, \, -, \, -, \, +, \, -, \, -, \, -, \, -, \, +, \, +, \, -,\] so $k_+ = 3$. We also discard the 0 and reduce our sample size to $n = 13$. Under \nullhyp, $K_+ \sim \Binom{13}{1/2}$, so the $p$-value is $\P{K_+ \leq 3} = 0.0461$, which is greater than our 1\% significance level. Thus, we do not reject \nullhyp{} and conclude there is insufficient evidence to claim at the 1\% significance level that the students are able to write, on vaerage, at least 3 times as fast using their dominant hand as compared to their non-dominant hand.
\end{solution}