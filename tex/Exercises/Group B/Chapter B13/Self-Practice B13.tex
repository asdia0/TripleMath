\section{Self-Practice B13}

\begin{problem}
    Food energy taken in by a man goes partly to maintain the healthy functioning of his body and partly to increase body mass. The total food energy intake of the man per day is assumed to be a constant denoted by $I$ (in joules). The food energy required to maintain the healthy functioning of his body is proportional to his body mass $M$ (in kg). The increase of $M$ with respect to time $t$ (in days) is proportional to the energy not used by his body. If the man does not eat for one day, his body mass will be reduced by 1\%. 
    
    \begin{enumerate}
        \item Show that $I$, $M$ and $t$ are related by the following differential equation: \[\der{M}{t} = \frac{I - aM}{100a},\] where $a$ is a constant. State an assumption for this model to be valid.
        \item Find the total food energy intake per day, $I$, of the man in terms of $a$ and $M$ if he wants to maintain a constant body mass.
    \end{enumerate}

    It is given that the man's initial mass is 100 kg.
    
    \begin{enumerate}
        \setcounter{enumi}{2}
        \item Solve the differential equation in part (a), giving $M$ in terms of $I$, $a$ and $t$.
        \item Sketch the graph of $M$ against $t$ for the case where $I > 100a$. Interpret the shape of the graph with regard to the man's food energy intake.
        \item If the man's total food energy intake per day is $50a$, find the time taken in days for the man to reduce his body mass from 100 kg to 90 kg.
    \end{enumerate}
\end{problem}

\begin{problem}
    Find the general solution of the differential equation \[x \der{y}{x} - 3y = x^5 \e^{2x}.\] Sketch the family of solution curves, showing clearly all the essential features sufficiently.
\end{problem}

\begin{problem}
    Let the variables $x$ and $y$ be related by the differential equation \[\der{y}{x} + \frac{y}{x} = xy^n,\] where $n$ is a real number. Find the general solution for $y$ in terms of $x$ for the following cases:
    \begin{enumerate}
        \item $n = 0$;
        \item $n = 1$;
        \item $n \geq 2$, using the substitution $u = y^{1-n}$.
    \end{enumerate}
\end{problem}

\begin{problem}
    \bd{Orthogonal trajectories} are a family of curves that intersect another family of curves perpendicularly.

    The electrostatic field created by a single positive charge is a collection of straight lines that radiate away from the charge. Equipotential lines are where the electric potentials are equal on a 2-dimensional surface (\bd{these lines can be curves}). It is given that the equipotential lines are orthogonal trajectories of the electric field lines. 

    \begin{enumerate}
        \item By forming a differential equation satisfied by equipotential lines and solving it, show that the equipotential line of a point charge forms a family of circles with centre at the origin, taking the point charge to be at the origin.
    \end{enumerate}

    When a point charge is placed at $(0, h_1)$, there is an equipotential line tangential to the $x$-axis. The collection of these equipotential lines for all $h_1 \in \RR$, $h_1 \neq 0$ forms a family of circles denoted by $C$.

    \begin{enumerate}
        \setcounter{enumi}{1}
        \item By first writing the Cartesian equation of a circle tangential to the $x$-axis and with centre $(0, h_1)$, show that the orthogonal trajectories of the family of circles $C$ satisfy \[\der{y}{x} = \frac{y^2 - x^2}{2xy}.\]
        
        Hence, by using the substitution $Y = y^2$ show that the orthogonal trajectories of the family of circles C$,$ form a family of circles that are tangential to the $y$-axis at the origin.
    \end{enumerate}
\end{problem}