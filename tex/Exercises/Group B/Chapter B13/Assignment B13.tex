\section{Assignment B13}

\begin{problem}
    Two biological cultures, $X$ and $Y$, react with each other, and their volumes at time $t$ are $x$ and $y$ respectively, in appropriate units. Their rates of growth are modelled by the simultaneous equations \[\der{x}{t} = (2-x)y \quad \tand \quad \der{y}{t} = \frac{y^2}{x}.\] When $t = 0$, $x = y = 1$.
    \begin{enumerate}
        \item Show that \[x = \frac{2y^2}{1+y^2}.\]
        \item Find and simplify expressions for $y$ and $x$ in terms of $t$.
        \item Sketch the graph of $y$ against $x$ for $0 < t < \pi/2$.
    \end{enumerate}
\end{problem}
\begin{solution}
    \begin{ppart}
        Note that $x, y > 0$ since they represent volume. Also, for $x \in (0, 2)$, we have $\derx{x}{t} = (2-x)y > 0$. When $x = 2$, we have $\derx{x}{t} = 0$. Hence, $0 < x \leq 2$. Now observe that \[\der{y}{x} = \frac{\derx{y}{t}}{\derx{x}{t}} = \frac{y^2/x}{(2-x)y} = \frac{y}{x(2-x)}.\] Rearranging, \[\frac1y \der{y}{x} = \frac1{x(2-x)}.\] Integrating both sides with respect to $x$, we get \[\int \frac1y \d y = \int \frac1{x(2-x)} \d x = \frac12 \int \bp{\frac1x + \frac1{2-x}} \d x.\] Thus, \[\ln y = \frac{\ln x - \ln{2-x}}2 + C_1,\] so \[y = C_2 \sqrt{\frac{x}{2-x}}.\]

        We are given that $x = y = 1$ when $t = 0$, so $C_2 = 1$. Thus, \[y = \sqrt{\frac{x}{2-x}}.\] Solving for $x$, we get \[x = \frac{2y^2}{1 + y^2}.\]
    \end{ppart}
    \begin{ppart}
        Observe that \[\der{y}{t} = \frac{y^2}{x} = \frac{y^2}{2y^2/(1+y^2)} = \frac12 (1 + y^2).\] Rearranging, \[\frac1{1+y^2} \der{y}{t} = \frac12.\] Integrating both sides with respect to $t$, we get \[\int \frac1{1+y^2} \d y = \int \frac12 \d t.\] Thus, $\arctan y = t/2 + C$, whence $y = \tan{t/2 + C}$. Since $y = 1$ when $t = 0$, we get $C = \pi/4$, whence \[y = \tan{\frac{t}2 + \frac\pi4} = \frac{1-\cos{t + \pi/2}}{\sin{t+\pi/2}} = \frac{1 + \sin t}{\cos t} = \sec t + \tan t.\]

        Using the result in part (a), we have
        \begin{align*}
            x &= \frac{2y^2}{1 + y^2} = \frac{2(\sec t + \tan t)^2}{1 + (\sec t + \tan t)^2} = \frac{2\bp{1 + \sin t}^2}{\cos^2 t + \bp{1 + \sin t}^2} = \frac{2\bp{1 + \sin t}}{2 + 2 \sin t} = 1 + \sin t.
        \end{align*}
    \end{ppart}
    \begin{ppart}
        Note that $0 < t < \pi/2$, so $1 < x < 2$.
        \begin{center}\tikzsetnextfilename{216}
            \begin{tikzpicture}[trim axis left, trim axis right]
                \begin{axis}[
                    domain = 1:2,
                    samples = 101,
                    axis y line=middle,
                    axis x line=middle,
                    xtick = {2},
                    ytick = \empty,
                    xlabel = {$x$},
                    ylabel = {$y$},
                    xmin=0,
                    xmax=2.1,
                    ymax=10,
                    ymin=0,
                    legend cell align={left},
                    legend pos=outer north east,
                    after end axis/.code={
                        \path (axis cs:0,0) 
                            node [anchor=north east] {$O$};
                        }
                    ]
                    \addplot[plotRed] {sqrt(x/(2-x))};
        
                    \addlegendentry{$y = \sqrt{x/(2-x)}$};

                    \draw[dotted] (2, 0) -- (2, 10);

                    \draw (1, 1) circle[radius=2.5pt] node[above] {$\bp{1, 1}$};
                \end{axis}
            \end{tikzpicture}
        \end{center}
    \end{ppart}
\end{solution}

\begin{problem}
    Find the general solution of the differential equation \[x\der{y}{x} + 4y - 10x = 0.\]

    Find the particular solution such that $y \to 0$ as $x \to 0$.

    Show, on a single diagram, a sketch of this particular solution and one typical member of the family, $F$ of solution curves for which $\derx{y}{x}$ is positive whenever $x$ is positive.

    Show that there is a straight line which passes through the maximum point of every member of $F$ and find its equation.
\end{problem}
\begin{solution}
    We have \[x^4\der{y}{x} + 4x^3y = \der{}{x} \bp{x^4 y} = 10x^4.\] Thus, \[x^4 y = \int 10 x^4 \d x = 2x^5 + C,\] and $y = 2x + Cx^{-4}$.

    As $x \to 0$, $x^{-4} \to \infty$. Hence, $C$ must be 0, whence the particular solution is $y = 2x$.

    Note that \[\der{y}{x} = 2 - 4Cx^{-5} > 0,\] so $C < x^5/2$. Since $x > 0$, we hence have the constraint $C \leq 0$ for members of $F$.

    \begin{center}\tikzsetnextfilename{217}
        \begin{tikzpicture}[trim axis left, trim axis right]
            \begin{axis}[
                domain = -5:3,
                restrict y to domain=-10:5,
                samples = 120,
                axis y line=middle,
                axis x line=middle,
                xtick = \empty,
                ytick = \empty,
                xlabel = {$x$},
                ylabel = {$y$},
                legend cell align={left},
                legend pos=outer north east,
                after end axis/.code={
                    \path (axis cs:0,0) 
                        node [anchor=north east] {$O$};
                    }
                ]
                \addplot[plotRed] {2*x};
    
                \addlegendentry{$C=0$};
    
                \addplot[plotBlue] {2*x - 1/x^4};
    
                \addlegendentry{$C=-1$};
            \end{axis}
        \end{tikzpicture}
    \end{center}

    Differentiating the original differential equation with respect to $x$, we obtain \[\bp{x \der[2]{y}{x} + \der{y}{x}} + 4\der{y}{x} - 10 = 0.\] At stationary points, $\derx{y}{x} = 0$, hence \[4y - 10 x = 0 \quad \tand \quad \der[2]{y}{x} = \frac{10}{x}.\] Thus, the stationary points lie on the line $y = 5x/2$. For members of $F$, stationary points can only occur when $x < 0$ since $\derx{y}{x} > 0$ for $x > 0$. Thus, $\derx[2]{y}{x} < 0$, so the stationary points are maximum points. Thus, $y = 5x/2$ passes through the maximum point of every member of $F$.
\end{solution}

\begin{problem}
    \begin{enumerate}
        \item The variables $x$ and $y$ are related by the differential equation \[x^2 \der{y}{x} - 2xy + y = 0.\]
        \begin{enumerate}
            \item Find the general solution of this differential equation, expressing $y$ in terms of $x$.
            \item Find the particular solution for which $y = -\e$ when $x = 1$. Obtain the coordinates of the turning point of the solution curve of this particular solution and sketch the curve for $x > 0$.
        \end{enumerate}
        \item Find the general solution of the differential equation \[\der{y}{x} + xy = \e^x x^2,\] expressing $y$ in terms of $x$.
    \end{enumerate}
\end{problem}
\begin{solution}
    \begin{ppart}
        \begin{psubpart}
            Rearranging the given differential equation, we get \[\frac1y \der{y}{x} = \frac2{x} - \frac1{x^2}.\] Integrating with respect to $x$ on both sides, we get \[\int \frac1y \d y = \int \bp{\frac2{x} - \frac1{x^2}} \d x.\] Thus, \[\ln \abs{y} = 2 \ln \abs{x} + \frac1x + C_1.\] Solving for $y$, we have $y = C_2 x^2 \e^{1/x}$.
        \end{psubpart}
        \begin{psubpart}
            Since $y = -\e$ when $x = 1$, we have $C_2 = -1$, so $y = -x^2 e^{1/x}$.

            For stationary points, $\derx{y}{x} = 0$. From the given differential equation, we see that $y(2x - 1) = 0$, whence $x = 1/2$. Note that we reject $y = 0$ since $\e^{1/x} \neq 0$ and $x \neq 0$ due to the presence of a $1/x$ term. Hence, $y$ has a stationary point at $(1/2, -\e^2/4)$.

            Differentiating the original differential equation with respect to $x$, we obtain \[x^2 \der[2]{y}{x} - 2y = 0.\] Hence, at $(1/2, -\e^2/4)$, we have \[\der[2]{y}{x} = \frac{2y}{x^2} = \frac{-e^2/2}{1/4} < 0,\] whence it is a turning point. More precisely, it is a maximum point.

            \begin{center}\tikzsetnextfilename{218}
                \begin{tikzpicture}[trim axis left, trim axis right]
                    \begin{axis}[
                        domain = 0:3,
                        restrict y to domain =-10:5,
                        samples = 101,
                        axis y line=middle,
                        axis x line=middle,
                        xtick = \empty,
                        ytick = \empty,
                        xlabel = {$x$},
                        ylabel = {$y$},
                        legend cell align={left},
                        legend pos=outer north east,
                        after end axis/.code={
                            \path (axis cs:0,0) 
                                node [anchor=north east] {$O$};
                            }
                        ]
                        \addplot[plotRed] {-x^2 * e^(1/x)};
            
                        \addlegendentry{$y = -x^2 \e^{1/x}$};

                        \fill (0.5, -e^2/4) circle[radius=2.5 pt] node[above] {$\bp{\frac12, -\frac{\e^2}{4}}$};
                    \end{axis}
                \end{tikzpicture}
            \end{center}
        \end{psubpart}
    \end{ppart}
    \begin{ppart}
        Multiplying through by the integrating factor of $\exp{\int x \d x} = \e^{x^2/2}$, we get \[e^{x^2/2}\der{y}{x} + x\e^{x^2/2}y = \der{}{x} \bp{\e^{x^2/2} y} = \e^{x + x^2/2} x^2.\] Thus, \[\e^{x^2/2} y = \int \e^{x + x^2/2} x^2 \d x.\]

        Completing the square, we see that \[x + \frac{x^2}{2} = \frac{(x+1)^2}{2} - \frac12.\] Since $x^2 = (x+1)^2 - 2(x+1) + 1$, we may rewrite the integral as \[\e^{x^2/2} y = \int \e^{x + x^2/2} x^2 \d x = \int \bp{(x+1)^2 - 2(x+1) + 1}\e^{(x+1)^2/2 - 1/2} \d x.\] Take $u = x + 1$. Then the integral becomes \[\e^{x^2/2} y = \e^{-1/2} \int \bp{u^2 - 2u + 1} \e^{u^2/2} \d u.\] Since \[\int u\e^{u^2/2} \d u = \e^{u^2/2} + C \quad \tand \quad \int u^2 \e^{u^2/2} \d u = u\e^{u^2/2} - \int \e^{u^2/2} \d u,\]  we have that \[\e^{x^2/2} y = \e^{-1/2} \bp{u\e^{u^2/2} - 2\e^{u^2/2}} + C = \bp{x-1} \e^{x + x^2/2} + C.\] Thus, \[y = \bp{x-1} \e^x + C\e^{-x^2/2}.\]
    \end{ppart}
\end{solution}