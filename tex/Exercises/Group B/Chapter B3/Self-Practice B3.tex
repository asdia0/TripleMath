\section{Self-Practice B3}

\begin{problem}
    Functions $f$ and $g$ are defined by
    \begin{align*}
        f &: x \mapsto \frac{3x-2}{x+1}, \quad x \in \RR, \quad x \neq -1,\\
        g &: x \mapsto 3x + 4, \quad x \in \RR.
    \end{align*}

    \begin{enumerate}
        \item Find $\inv f(x)$ and state the domain and range of $\inv f$.
        \item Express in similar form, the functions $g^2$ and $gf$.
        \item Find the value of $x$ for which $\inv{(gf)} (x) = 0$.
    \end{enumerate}
\end{problem}

\begin{problem}
    The function $f$ is defined by \[f : x \mapsto 2 - (x-1)^2, \quad x \leq 1.\]

    \begin{enumerate}
        \item Sketch the graphs of $y = f(x)$, $y = \inv f(x)$ and $y = \inv f f(x)$ on a single diagram.
        \item If $f(\b) = \inv f(\b)$, find the values of the constant $p$ and $q$ such that \[\b^2 - p\b + q = 0.\]
        \item Define $\inv f$ in a similar form.   
    \end{enumerate}
\end{problem}

\begin{problem}
    The function $f$ is defined by $f : x \mapsto \cos \frac{\pi x}{2}$, $x \in \RR$, $-2 < x \leq 0$.

    \begin{enumerate}
        \item Sketch the graph of $y=f(x)$ indicating clearly the coordinates of all axial intercepts and end points.
        \item Show that $\inv f$ exists, and find its rule and domain.
    \end{enumerate}

    The function $g$ is defined by $g : x \mapsto (2x+1)^{2/3}$, $x \in \RR$, $-2 < x \leq 2$.

    \begin{enumerate}
        \setcounter{enumi}{2}
        \item Find the set of values of $x$ such that $g(x) \geq f(x)$
        \item Explain clearly why $gf$ exists. Hence, find the range of $gf$.
    \end{enumerate}
\end{problem}

\begin{problem}
    Functions $f$ and$ g$ are defined by:
    \begin{align*}
        f &: x \mapsto x^2 + c, \quad x \leq 2,\\
        g &: x \mapsto 5 + \frac3x, \quad x \geq k,
    \end{align*}
    where $c$, $k$ are positive constants and $c>k$.
    \begin{enumerate}
        \item Show that $\inv g$ exists.
        \item Find $\inv g$ in similar form, expressing its domain in terms of $k$.
        \item Determine whether each of the two functions, $fg$ and $gf$, exists. Where it exists, express the composite function in similar form and state its range.
    \end{enumerate}
\end{problem}

\begin{problem}[\chili]
    Functions $f$ and $g$ are defined such that    
    \begin{align*}
        f &: x \mapsto \arccos{x^2}, \quad -1 \leq x \leq 1,\\
        g &: x \mapsto x^3 + 1, \quad x \in \RR.
    \end{align*}

    \begin{enumerate}
        \item Explain why the composite function $fg$ does not exist.
    \end{enumerate}
    
    The function $h$ is defined such that $h(x) = g(x)$ and the domain of $h$ is $a \leq x \leq 0$. It is given that $a = -5/4$.

    \begin{enumerate}
        \setcounter{enumi}{1}
        \item Find the range of $fh$ in exact form.
        \item Determine all the possible value(s) of $x$ that satisfies $\inv g(x^2) = 2$. Hence, explain why $\inv h (x^2) = 2$ has no solution.
    \end{enumerate}
\end{problem}