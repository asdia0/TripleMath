\section{Self-Practice B2}

\begin{problem}
    Show that the equation $y=\frac{2x+7}{x+2}$ can be written as $y=A+\frac{B}{x+2}$, where $A$ and $B$ are constants to be found. Hence, state a sequence of transformations which transform the graph of $y=\frac{1}{x}$ to the graph of $y=\frac{2x+7}{x+2}$.

    Sketch the graph of $y=\frac{2x+7}{x+2}$, giving the equations of any asymptotes and the coordinates of any points of intersection with the $x$- and $y$-axes.
\end{problem}

\begin{problem}
    The diagram shows the curve with equation $y=f(x)$. The curve passes through the origin, and has asymptotes $x=a$ and $y=b$, where $a$ and $b$ are positive constants.
    
    \begin{figure}[H]\tikzsetnextfilename{403}
        \centering
        \begin{tikzpicture}[trim axis left, trim axis right]
            \begin{axis}[
                domain = -3:5,
                restrict y to domain = -3:5,
                samples = 101,
                axis y line=middle,
                axis x line=middle,
                xtick = \empty,
                ytick = \empty,
                xlabel = {$x$},
                ylabel = {$y$},
                legend cell align={left},
                legend pos=outer north east,
                after end axis/.code={
                    \path (axis cs:0,0) 
                        node [anchor=north east] {$O$};
                    }
                ]
                \addplot[plotRed, unbounded coords = jump] {1/(x-1) + 1};
    
                \addlegendentry{$y = f(x)$};

                \draw[dashed] (1,-3) -- (1, 5);
                \addplot[dashed] {1};
                \node[anchor=south west] at (-2, 1) {$y = b$};
                \node[anchor=south west] at (1, -2) {$x = a$};
            \end{axis}
        \end{tikzpicture}
        \caption{}
    \end{figure}

    On separate diagrams, draw sketches of the graphs of
    \begin{enumerate}
        \item $y = f(x + a) - b$,
        \item $y = 1/f(x)$,
    \end{enumerate}
    showing clearly the axial intercepts and asymptotes (if any).
\end{problem}

\begin{problem}
    The curves $C_{1}$ and $C_{2}$ are given by the equations $x^{2}+y^{2}=1$ and $x^{2}-2x+9y^{2}=a$ respectively, where $a$ is a real constant. The curve $C_{2}$ cuts the $x$-axis at the origin $O$ and is symmetrical about the line $x=b$, as shown in the diagram below.

    \begin{figure}[H]\tikzsetnextfilename{404}
        \centering
        \begin{tikzpicture}[trim axis left, trim axis right]
            \begin{axis}[
                samples = 101,
                axis y line=middle,
                axis x line=middle,
                ymin=-5,
                ymax=5,
                xmin=0,
                xmax=10,
                xtick = \empty,
                ytick = \empty,
                xlabel = {$x$},
                ylabel = {$y$},
                legend cell align={left},
                legend pos=outer north east,
                after end axis/.code={
                    \path (axis cs:0,0) 
                        node [anchor=east] {$O$};
                    }
                ]
                \draw[plotRed] (5, 0) ellipse (5 and 3);
                \draw[dashed] (5, -5) -- (5, 5) node[anchor=north west] {$x = b$};

                \addlegendimage{plotRed};
                \addlegendentry{$C_2$};
            \end{axis}
        \end{tikzpicture}
        \caption{}
    \end{figure}

    \begin{enumerate}
        \item Determine the values of $a$ and $b$.
        \item Describe clearly a sequence of transformations that maps $C_{1}$ onto $C_{2}$.
    \end{enumerate}
\end{problem}

\begin{problem}
    It is given that the curve $y = f(x)$, where $f(x) = \frac{ax + b}{2x + c}$, where $a$, $b$, $c$ are constants, has an asymptote $x=\frac{1}{2}.$ The point $A$ with coordinates $(2, \frac53)$ lies on the curve. The tangent to the curve at $A$ has gradient $\frac29$.

    \begin{enumerate}
        \item Write down the value of $c$.
        \item Show that $a=4$ and $b=-3$.
        \item Sketch the graph of $y=f(x),$ showing clearly all the asymptotes and the exact coordinates of the intersection with the axes.
        \item Describe a sequence of three transformations which transforms the graph $y = 2 + \frac1x$ to $y = f(x)$.
    \end{enumerate}
\end{problem}

\begin{problem}
    The curve whose equation is $\frac{(x-3)^2}{2^2} + \frac{y^2}{3^2} = 1$ undergoes, in succession, the following transformations:
    \begin{align*}
        A &: \quad \text{A translation of magnitude 1 unit in the direction of the $x$-axis.}\\
        B &: \quad \text{A reflection in the $y$-axis.}\\
        C &: \quad \text{A scaling parallel to the $y$-axis by a scale factor of $k$.}
    \end{align*}

    \begin{enumerate}
        \item Find the equation of the resulting curve.
        \item State the value of $k$ for which the resulting curve takes on the shape of a circle.
    \end{enumerate}
\end{problem}

\begin{problem}
    The diagram shows the graph of $y=f(x)$.

    \begin{figure}[H]\tikzsetnextfilename{406}
        \centering
        \begin{tikzpicture}[trim axis left, trim axis right]
            \begin{axis}[
                domain = -6:8,
                restrict y to domain =-2:7.5,
                samples = 91,
                axis y line=middle,
                axis x line=middle,
                xtick = \empty,
                ytick = {5},
                xlabel = {$x$},
                ylabel = {$y$},
                legend cell align={left},
                legend pos=outer north east,
                after end axis/.code={
                    \path (axis cs:0,0) 
                        node [anchor=north east] {$O$};
                    }
                ]
                \addplot[plotRed, unbounded coords = jump, domain=-6:-2] {1/(x+2) + 2};
                \addplot[plotRed, domain=-2:2] {-1/((x-2)*(x+2)) + 4.75};
                \addplot[plotRed, domain=2:6, unbounded coords = jump] {-2/(x-2) + 3.5};
                \addplot[plotRed, domain=6:8] {(x-5)/(x^2 - 11*x + 31) + 2};
    
                \addlegendentry{$y = f(x)$};
    
                \draw[dashed] (-2, -2) -- (-2, 7.5);
                \node[anchor=north east] at (-2, 5) {$x = -2$};
                \draw[dashed] (2, -2) -- (2, 7.5);
                \node[anchor=north west] at (2, 5) {$x = 2$};
                \addplot[dashed] {2};

                \node[anchor=north] at (6, 2) {$y = 2$};

                \fill (-2.5, 0) circle[radius=2.5pt] node[anchor=north east] {$A(-3,0)$};
                \fill (2.57, 0) circle[radius=2.5pt] node[anchor=north west] {$B(3,0)$};
                \fill (0, 5) circle[radius=2.5pt] node[anchor=north west] {$C$};
                \fill (6, 3) circle[radius=2.5pt] node[anchor=south] {$D(6, 4)$};
                \end{axis}
        \end{tikzpicture}
    \end{figure}

    On separate diagrams, sketch the graphs of
    \begin{enumerate}
        \item $y = f(4x + 3)$,
        \item $y = 1/f(x)$.
    \end{enumerate}
    In each case, state the equations of any asymptotes and the coordinates of the points corresponding to $A$, $B$, $C$ and $D$ where appropriate.
\end{problem}

\begin{problem}
    A curve $C_{1}$ is defined parametrically by \[x = \frac2{t-1}, \quad y = \frac{4}{t+1}, \quad t \neq \pm 1.\] Sketch a clearly labelled diagram of $C_{1}$.

    Describe a sequence of geometrical transformations which maps $C_{1}$ to $C_{2}$ defined by \[x = \frac1{1-t}, \quad y = \frac4{t+1}, \quad t \neq \pm 1.\]

    Sketch $C_{3}$, which is the reciprocal function of $C_{1}$, stating the equations of any asymptotes and any points of intersection with the axes.
\end{problem}

\begin{problem}[\chili]
    The curve of $y = \frac{x^2}{4-x}$ undergoes two transformations. The resulting curve whose equation is $y=\frac{(x-a)^2}{10-bx}$ has stationary points $A(1, 0)$ and $B(9,-8)$, and asymptotes $x=5$ and $y=cx+d$, where $a$, $b$, $c$ and $d$ are constants.

    \begin{figure}[H]\tikzsetnextfilename{405}
        \centering
        \begin{tikzpicture}[trim axis left, trim axis right]
            \begin{axis}[
                domain = -10:20,
                restrict y to domain =-17:14,
                samples = 91,
                axis y line=middle,
                axis x line=middle,
                xtick = \empty,
                ytick = \empty,
                xlabel = {$x$},
                ylabel = {$y$},
                legend cell align={left},
                legend pos=outer north east,
                after end axis/.code={
                    \path (axis cs:0,0) 
                        node [anchor=north east] {$O$};
                    }
                ]
                \addplot[plotRed, unbounded coords = jump] {(x-1)^2 / (10 - 2*x)};
    
                \addlegendentry{$y = (x-a)^2/(10-bx)$};
    
                \addplot[dashed] {-0.5 * x - 1.5};
                \node[anchor=south west, rotate=-27] at (10, -6.5) {$y = cx + d$};
                \draw[dashed] (5, -17) -- (5, 14) node[anchor=north west] {$x = 5$};

                \fill (1, 0) circle[radius=2.5pt] node[anchor=south] {$A$};
                \fill (9, -8) circle[radius=2.5pt] node[anchor=north] {$B$};
                \end{axis}
        \end{tikzpicture}
    \end{figure}

    \begin{enumerate}
        \item Show that $a=1$, and find the values of $b$, $c$ and $d$.
        \item Describe the sequence of transformations undergone by the graph of $y= \frac{x^2}{4-x}$ to attain that of $y=\frac{(x-a)^2}{10-bx}$, where $a$ and $b$ are the values found in (a).
    \end{enumerate}
\end{problem}