\section{Tutorial B16}

\begin{problem}
    Find the general solution of \[3\der[2]{x}{t} + 4\der{x}{t} -7x = 0.\]
\end{problem}
\begin{solution}
    The characteristic equation of the DE is $3m^2 + 4m - 7 = 0$ with roots $m = -7/3$ and $m = 1$. Thus, $x = A \e^{-7t/3} + B \e^t$.
\end{solution}

\begin{problem}
    Solve the following homogeneous second-order linear differential equations.
    \begin{enumerate}
        \item $\displaystyle \der[2]{y}{x} + 4\der{y}{x} + 3y = 0$, given that $y = 0$ and $\displaystyle \der{y}{x} = -4$ when $x = 0$.
        \item $\displaystyle \der[2]{y}{x} + 6\der{y}{x} + 9y = 0$, given that $y = 1$ and $\displaystyle \der{y}{x} = 1$ when $x = 0$.
        \item $\displaystyle \der[2]{y}{x} + \sqrt3 \der{y}{x} + 3y = 0$, given that $y = 0$ and $\displaystyle \der{y}{x} = -4$ when $x = 0$.
    \end{enumerate}
\end{problem}
\begin{solution}
    \begin{ppart}
        The characteristic equation of the DE is $m^2 + 4m + 3 = 0$ with roots $m = -1$ and $m = -3$. Thus, $y = A\e^{-x} + B\e^{-3x}$. Differentiating, we get \[\der{y}{x} = -A\e^{-x} - 3 B\e^{-3x}.\] Using the given conditions, we obtain the system \[\systeme[AB]{A + B = 0, -A - 3B = -4}\] which has the unique solution $A = -2$ and $B = 2$. Thus, $y = -2\e^{-x} + 2\e^{-3x}$.
    \end{ppart}
    \begin{ppart}
        The characteristic equation of the DE is $m^2 + 6m + 9 = 0$, which has the repeated root $m = -3$. Thus, $y = \bp{A + Bx} \e^{-3x}$. Differentiating, we get \[\der{y}{x} = -3\bp{A + Bx}\e^{-3x} + B\e^{-3x}.\] Using the given conditions, we obtain the system \[\systeme[AB]{A = 1, -3 A + B = 1}\] which has the unique solution $A = 1$ and $B = 4$. Thus, $y = \bp{1 + 4x}\e^{-3x}$.
    \end{ppart}
    \begin{ppart}
        The characteristic equation of the DE is $m^2 + \sqrt3 m + 3 = 0$, which has imaginary roots \[m = \frac{-\sqrt3}{2} \pm \frac{3}{2} \i,\] whence \[y = \e^{-\sqrt{3}x / 2} \bs{A \cos{\frac32 x} + B \sin{\frac32 x}}.\] Since $y = 0$ when $x = 0$, we have $A = 0$, so \[y = B \e^{-\sqrt{3}x / 2} \sin{\frac32 x}.\] Differentiating, we get \[\der{y}{x} = B \e^{-\sqrt{3} x / 2} \bs{ \frac32\cos{\frac32 x} -\frac{\sqrt3}{2} \sin{\frac32x}}.\] Since $\derx{y}{x} = -4$ when $x = 0$, we get $B = -8/3$. Thus, \[y = -\frac83 \e^{-\sqrt{3}x/2} \sin{\frac32 x}.\]
    \end{ppart}
\end{solution}

\begin{problem}
    Find the general solution of
    \begin{enumerate}
        \item $\displaystyle 2\der[2]{y}{x} - 3\der{y}{x} - 5y = 10x^2 + 1$,
        \item $\displaystyle \der[2]{y}{x} - 2\der{y}{x} + 3y = 22\e^{4x}$,
        \item $\displaystyle \der[2]{s}{t} - 2\der{s}{t} + s = 4\e^t$,
        \item $\displaystyle \der[2]{x}{t} + 16x = 3\cos 4t$.
    \end{enumerate}
\end{problem}
\begin{solution}
    \begin{ppart}
        The characteristic equation of the DE is $2m^2 - 3m - 5 = 0$, which has roots $m = 5/2$ and $m = -1$, whence the complementary function is $y_c = A \e^{5x/2} + B \e^{-x}$. For the particular solution, we try $y_p = Cx^2 + Dx + E$ Differentiating, we see that $y_p' = 2Cx + D$ and $y_p'' = 2C$. Substituting this into the DE, \[2\bp{2C} + -3\bp{2Cx + D} - 5\bp{Cx^2 + Dx + E} = 10x^2 + 1.\] Comparing coefficients, we get the system \[\systeme[CDE]{-5C = 10,-6C-5D = 0,4C-3D-5E=1},\] which gives $C = -2$, $D = 12/5$ and $E = -81/25$. The general solution is thus \[y = y_c + y_p = A \e^{5x/2} + B \e^{-x} - 2x^2 + \frac{12}5 x - \frac{81}{25}.\]
    \end{ppart}
    \begin{ppart}
        The characteristic equation of the DE is $m^2 - 2m + 3 = 0$, which has imaginary roots $m = 1 \pm \sqrt{2} \i$. The complementary function is hence \[y_c = \e^x \bs{A\cos{\sqrt{2}x} + B\sin{\sqrt{2}x}}.\] For the particular solution, we try $y_p = C\e^{4x}$. Differentiating, we have $y_p' = 4C\e^{4x}$ and $y_p'' = 16C\e^{4x}$. Substituting this into the DE, we have \[16C\e^{4x} - 2\bp{4C\e^{4x}} + 3C\e^{4x} = 22\e^{4x},\] so $C = 2$. The general solution is thus \[y = y_c + y_p = \e^x \bs{A \cos{\sqrt{2}x} + B \sin{\sqrt{2}x}} + 2\e^{4x}.\]
    \end{ppart}
    \begin{ppart}
        The characteristic equation of the DE is $m^2 - 2m + 1 = 0$, which has the repeated root $m = 1$, whence the complementary function is $s_c = \bp{A + Bt} \e^{t}$. For the particular solution, we try $s_p = Ct^2 \e^t$. Differentiating, we have $s_p' = C\e^t\bp{t^2 + 2t}$ and $s_p'' = C\e^t \bp{t^2 + 4t + 2}$. Substituting this into the DE, we have \[C\e^t \bp{t^2 + 4t + 2} - 2C\e^t \bp{t^2 + 2t} + C\e^t \bp{t^2} = 4\e^t,\] so $C = 2$. The general solution is thus $s = s_c + s_p = \bp{A + Bt + 2t^2} \e^{t}$.
    \end{ppart}
    \begin{ppart}
        The characteristic equation of the DE is $m^2 + 16m = 0$, which has imaginary roots $m = \pm 4 \i$. The complementary function is hence $x_c = A\cos 4t + B \sin 4t$. For the particular solution, we try $x_p = t\bp{C\cos 4t + D \sin 4t}$. Differentiating, we have
        \begin{align*}
            x_p' &= 4t\bp{-C\sin 4t + D\cos 4t} + \bp{C \cos 4t + D \sin 4t},\\
            x_p'' &= 16t\bp{-C\cos 4t - D\sin 4t} + 8\bp{-C\sin 4t + D \cos 4t}.
        \end{align*}
        Substituting this into the DE, 
        \begin{align*}
            &16t\bp{-C\cos 4t - D\sin 4t} + 8\bp{-C\sin 4t + D \cos 4t} + 16t\bp{C\cos 4t + D \sin 4t} = 3\cos 4\t.
        \end{align*}
        Comparing coefficients of $\cos 4t$ and $\sin 4t$ terms, we have $C = 0$ and $D = 3/8$. Thus, the general solution is \[x = A\cos4t + B \sin 4t + \frac38 t\sin 4t.\]
    \end{ppart}
\end{solution}

\begin{problem}
    \begin{enumerate}
        \item Find the general solution of the differential equation \[\der[2]{y}{x} - 4y = 10 \e^{3x}.\]
        \item Hence, find the solution for which $y = -2$ and $\derx{y}{x} = -6$ when $x = 0$.
    \end{enumerate}
\end{problem}
\begin{solution}
    \begin{ppart}
        The characteristic equation of the DE is $m^2 -4 = 0$, so the roots are $m = \pm 2$. The complementary solution is thus $y_c = A\e^{2x} + B\e^{-2x}$. For the particular solution, we try $y_p = C\e^{3x}$. Differentiating, we have $y_p' = 3C\e^{3x}$ and $y_p'' = 9C\e^{3x}$. Substituting this into the DE, we get $9C\e^{3x} - 4C\e^{3x} = 10\e^{3x}$, so $C = 2$, whence the general solution is $y = y_c + y_p = A\e^{2x} + B\e^{-2x} + 2\e^{3x}$.
    \end{ppart}
    \begin{ppart}
        Note that \[\der{y}{x} = 2A\e^{2x} - 2B\e^{-2x} + 6\e^{3x}.\] The given conditions thus give the system \[\systeme[AB]{A + B + 2= -2, 2A - 2B + 6 = -6},\] whence $A = -5$ and $B = 1$. Hence, $y = -5\e^{2x} + \e^{-2x} + 2\e^{3x}$.
    \end{ppart}
\end{solution}

\begin{problem}
    \begin{enumerate}
        \item Find the general solution of the differential equation $\derx[2]{y}{x} = \sin x$. Find the particular solution that passes through the points $(0, \sqrt2)$ and $\bp{\pi/4, -\sqrt2}$.
        \item Find the general solution of the differential equation
        \begin{enumerate}
            \item $\derx[2]{y}{x} = 16 - 9x^2$,
            \item $\bp{9 - x^2}^2 \derx[2]{y}{x} - x = 0$,
        \end{enumerate}
        giving your answer in the form $y = f(x)$.
    \end{enumerate}
\end{problem}
\begin{solution}
    \begin{ppart}
        Integrating the DE with respect to $x$, \[\der{y}{x} = \int \sin x \d x = -\cos x + A.\] Integrating once more, \[y = \int \bp{-\cos x + A} \d x = -\sin x + Ax + B.\] At $(0, \sqrt2)$, we have $B = \sqrt{2}$. At $(\pi/4, -\sqrt2)$, we have \[-\frac{\sqrt2}{2} + A\bp{\frac\pi4} + B = -\sqrt2,\] so \[A = -\frac{6\sqrt2}{\pi}.\] Thus, the particular solution is \[y = -\sin x - \frac{6\sqrt2}{\pi} x + \sqrt2.\]
    \end{ppart}
    \begin{ppart}
        \begin{psubpart}
            Integrating with respect to $x$, \[\der{y}{x} = \int \bp{16 - 9x^2} \d x= 16 x - 3x^2 + A.\] Integrating once more, \[y = \int \bp{16x - 3x^2 + A} \d x = 8x^2 - \frac34 x^4 + Ax + B.\]
        \end{psubpart}
        \begin{psubpart}
            Rewriting, we get \[\der[2]{y}{x} = \frac{x}{\bp{9-x^2}^2}.\] Integrating with respect to $x$, \[\der{y}{x} = \int \frac{x}{\bp{9 - x^2}^2} \d x.\] Using the substitution $x = 3 \sin \t$, we have
            \begin{align*}
                \der{y}{x} &= \int \frac{3\sin\t}{81 \cos^4 \t} 3\cos \t \d \t\\
                &= \frac19 \int \tan \t \sec^2 \t \d \t\\
                &= \frac19\bp{\frac{\tan^2 \t}{2}} + C\\
                &= \frac1{18} \frac{\sin^2 \t}{1 - \sin^2 \t} + C\\
                &= \frac1{18}\bp{\frac{(x/3)^2}{1 - (x/3)^2}} + C\\
                &= \frac1{18} \bp{\frac{x^2}{9 - x^2}} + C.
            \end{align*}
            Integrating once more,
            \begin{align*}
                y &= \int \bs{\frac1{18} \bp{\frac{x^2}{9 - x^2}} + C} \d x\\
                &= \int \bs{\frac1{18} \bp{\frac{9}{9 - x^2} - 1} + C} \d x\\
                &= \frac1{18} \bs{\frac{3}{2} \ln \abs{\frac{3+x}{3-x}} - x} + Cx + D\\
                &= \frac{1}{12} \ln \abs{\frac{3+x}{3-x}} + Ex + D,
            \end{align*}
            where $E = -1/18 + C$.
        \end{psubpart}
    \end{ppart}
\end{solution}

\begin{problem}
    \begin{enumerate}
        \item Find the particular solution of $\derx[2]{x}{t} + 16x = 0$, given that $x = 3$ and $\derx{x}{t} = -8$ when $t = 0$.
        \item By writing the particular solution as $R\cos{4t + \a}$, find the first positive value of $t$ for which $x$ is maximum.
    \end{enumerate}
\end{problem}
\begin{solution}
    \begin{ppart}
        Note that the characteristic equation of the DE is $m^2 + 16 = 0$, whence the roots are $m = \pm 4 \i$. Hence, $x = A\cos 4t + B \sin 4\t$. Differentiating with respect to $t$, we obtain \[\der{x}{t} = -4A\sin4t + 4B\cos4t.\] When $x = 3$ and $t = 0$, we have $A = 3$. When $\derx{x}{t} = -8$ and $t =0$, we have $B = -2$. Thus, $x = 3\cos4t - 2\sin 4t$.
    \end{ppart}
    \begin{ppart}
        We have \[x = 3\cos4t - 2\sin 4t = \sqrt{3^2 + 2^2} \cos{4t - \arctan \frac{-2}{3}} = \sqrt{13} \cos{4t + 0.58800}.\] It follows that $x$ is at a maximum whenever $\cos{4t + 0.58800} = 1$, i.e. when $4t + 0.58800 = 2\pi n$, where $n$ is an integer. Solving for $t$, we get \[t = \frac{2\pi n - 0.58800}{4}.\] The first positive value of $t$ is hence \[t = \frac{2\pi - 0.58800}{4} = 1.42 \tosf{3},\] where $n = 1$.
    \end{ppart}
\end{solution}

\begin{problem}
    Using the substitution $x= \e^u$, find the general solution of
    \begin{enumerate}
        \item $\displaystyle x^2 \der[2]{y}{x} + 2x \der{y}{x} - 2y = 0$,
        \item $\displaystyle x^2 \der[2]{y}{x} - 5x \der{y}{x} - 6y = 0$.
    \end{enumerate}
\end{problem}
\begin{solution}
    Note that \[\der{y}{x} = \der{y}{u} \der{u}{x} \quad \tand \quad \der[2]{y}{x} = \der[2]{y}{u} \bp{\der{u}{x}}^2 + \der{y}{u} \der[2]{u}{x}.\] Since $u = \ln x$, we have $\derx{u}{x} = 1/x$ and $\derx[2]{u}{x} = -1/x^2$. Thus, \[\der{y}{x} = \frac1x \der{y}{u} \quad \tand \quad \der[2]{y}{x} = \frac1{x^2} \der[2]{y}{u} - \frac1{x^2} \der{y}{u}.\]

    \begin{ppart}
        Substituting the above expressions into the DE, we have \[x^2 \bp{\frac1{x^2} \der[2]{y}{u} - \frac1{x^2} \der{y}{u}} + 2x \bp{\frac1x \der{y}{u}} - 2y = 0.\] Simplifying, we get \[\der[2]{y}{u} + \der{y}{u} - 2y = 0.\] The characteristic equation $m^2 + m - 2 = 0$ has roots $m = -2$ and $m = 1$. Thus, $y = A\e^{-2u} + B\e^u = Ax^{-2} + Bx$.
    \end{ppart}
    \begin{ppart}
        Substituting the above expressions into the DE, we have \[x^2 \bp{\frac1{x^2} \der[2]{y}{u} - \frac1{x^2} \der{y}{u}} - 5x \bp{\frac1x \der{y}{u}} - 6y = 0.\] Simplifying, we get \[\der[2]{y}{u} - 6 \der{y}{u} - 6y = 0.\] The characteristic equation $m^2 - 6m - 6 = 0$ has roots $m = 3 \pm \sqrt{15}$. Thus, \[y = A\e^{(3 + \sqrt{15}) u} + B\e^{(3 - \sqrt{15}) u} = Ax^{3+\sqrt{15}} + Bx^{3 - \sqrt{15}}.\]
    \end{ppart}
\end{solution}
\clearpage

\begin{problem}
    Show, by means of the substitution $y = x^{-4} z$, that the differential equation \[x^2 \der[2]{y}{x} + \bp{4x^2 + 8x} \der{y}{x} + \bp{3x^2 + 16x + 12} y = 0\] can be reduced to the form \[\der[2]{z}{x} + a\der{z}{x} + bz = 0,\] where $a$ and $b$ are constants to be determined. Hence, find the general solution of the above differential equation.
\end{problem}
\begin{solution}
    Note that $z = x^4y$. Differentiating with respect to $x$ we get \[\der{z}{x} = x^4 \der{y}{x} + 4yx^3 \quad \tand \quad \der[2]{z}{x} = x^4 \der[2]{y}{x} + 8x^3 \der{y}{x} + 12yx^2.\]

    Consider the DE in question. Multiplying through by $x^2$, \[x^4 \der[2]{y}{x} + \bp{4x^4 + 8x^3} \der{y}{x} + \bp{3x^4 + 16x^3 + 12x^2} y = 0.\] Regrouping the LHS, we have \[\bp{x^4 \der[2]{y}{x} + 8x^3 \der{y}{x} + 12yx^2} + 4\bp{x^4 \der{y}{x} + 4yx^3} + 3x^4 y = 0,\] so \[\der[2]{z}{x} + 4\der{z}{x} + 3z = 0.\] Hence, $a = 4$ and $b = 3$.

    The characteristic equation of this new DE is $m^2 + 4m + 3 = 0$, which has roots $m =-3$ and $m =-1$, whence $z = A\e^{-3x} + B\e^{-x}$. Thus, $y = x^{-4}\bp{A\e^{-3x} + B\e^{-x}}$.
\end{solution}

\begin{problem}
    By letting $x = \sqrt t$, show that the differential equation \[\der[2]{y}{x} + \bp{2x - \frac1x} \der{y}{x} + 24 x^2 = 0\] where $x > 0$, may be transformed to \[\der[2]{y}{t} + a \der{y}{t} + b = 0,\] where $a$ and $b$ are constants to be determined. Hence, find the general solution of $y$ in terms of $x$.
\end{problem}
\begin{solution}
    Note that \[\der{y}{x} = \der{y}{t} \der{t}{x} \quad \tand \quad \der[2]{y}{x} = \der[2]{y}{t} \bp{\der{t}{x}}^2 + \der{y}{t} \der[2]{t}{x}.\] Since $t = x^2$, we have $\derx{t}{x} = 2x$ and $\derx[2]{t}{x} = 2$. Thus, \[\der{y}{x} = 2x \der{y}{t} \quad \tand \quad \der[2]{y}{x} = 4x^2 \der[2]{y}{t} + 2 \der{y}{u}.\] Substituting this into the given DE, \[\bp{4x^2 \der[2]{y}{t} + 2 \der{y}{u}} + \bp{2x - \frac1x} \bp{2x \der{y}{t}} + 24x^2 = 0.\] Simplifying, we get \[\der[2]{y}{t} + \der{y}{t} + 6 = 0,\] whence $a = 1$ and $b = 6$.

    Rewriting, \[\der[2]{y}{t} + \der{y}{t} = -6.\] Integrating with respect to $t$, we get \[\der{y}{t} + y = -6t + C.\] Multiplying through by the integrating factor $\e^t$ yields \[\der{}{t} \bp{\e^t y} = \e^t \der{y}{t} + \e^t y = \e^t \bp{-6t + C}.\] Integrating with respect to $t$, \[\e^t y = \int \e^t \bp{-6t + C} \d t = -6 \bp{t\e^t - \e^t} + C\e^t = \e^t \bp{-6t + A} + B.\] Thus, \[y = -6t + A + B\e^{-t} = -6x^2 + A + B\e^{-x^2}.\]

\end{solution}

\begin{problem}
    A damped vibrating spring system is described by the differential equation \[m \der[2]{y}{t} = -ky - \l \der{y}{t},\] where $m$, $k$ and $\l$ are positive constants. The variable $y$ represents the displacement of the object from equilibrium position in centimetres, and $t$ is time measured in seconds. Given that $m = 1$, $k = 25$ and $\l = 10$, and the object was initially released from rest at $y = 1$, find the equation of motion and sketch its graph. Briefly explain if this motion is suitable to be used to close a door.
\end{problem}
\begin{solution}
    We have \[\der[2]{y}{t} + 10 \der{y}{t} + 25y = 0.\] The characteristic equation $r^2 + 10r + 25 = 0$ has a single root $r = -5$. Thus, $y = \bp{A + Bt} \e^{-5 t}$. Since $y = 1$ when $t = 0$, we get $A = 1$. Differentiating, we get \[\der{y}{t} = B\e^{-5t} - 5\bp{1 + Bt}\e^{-5t}.\] Since the object is initially at rest, $\derx{y}{t} = 0$ when $t = 0$. This gives $B = 5$. Thus, $y = \bp{1 + 5t}\e^{-5t}$.
    
    \begin{figure}[H]\tikzsetnextfilename{42}
        \centering
        \begin{tikzpicture}[trim axis left, trim axis right]
            \begin{axis}[
                domain = 0:2,
                ymin=0,
                ymax=1.5,
                xmin=0,
                xmax=2,
                samples = 101,
                axis y line=middle,
                axis x line=middle,
                xtick = \empty,
                ytick = {1},
                xlabel = {$t$},
                ylabel = {$y$},
                legend cell align={left},
                legend pos=outer north east,
                after end axis/.code={
                    \path (axis cs:0,0) 
                    node [anchor=north east] {$O$};
                    }
                ]
                \addplot[plotRed] {(1 + 5*x) * e^(-5*x)};
                \addlegendentry{$y = (1 + 5t) \e^{-5t}$};
            \end{axis}
        \end{tikzpicture}
    \end{figure}

    Since the object does not oscillate ($y$ does not change sign) and $y$ approaches $y = 0$ quite quickly, the motion is suitable to be used to close a door.
\end{solution}

\begin{problem}
    The motion of the tip of a tuning fork can be modelled by the differential equation \[m \der[2]{x}{t} + k \der{x}{t} + m \o^2 x = 0,\] where $x$ is the displacement of the tip from its equilibrium position at time $t$ and $m$, $k$ and $\o$ are positive constants. It is known that $k$ is so small that $k^2$ can be ignored as $k$ models the slight damping due to the resistance of the air. It is given that the tip of the fork is initially in its equilibrium position and moving with speed $v$ in the positive $x$-direction.

    \begin{enumerate}
        \item Solve the differential equation.
    \end{enumerate}

    The amplitude of a vibration is the maximum displacement of the tip from its equilibrium position and one period of a vibration is the time interval between the occurrences of two consecutive amplitudes.

    \begin{enumerate}
        \setcounter{enumi}{1}
        \item Comment on the period of the vibrations over time and show that the amplitude of successive vibrations follows a geometric progression.
        \item Given that $k$ is no longer small and $k^2 > 4m^2 \o^2$, describe the behaviour of $x$ as time progresses and sketch a possible graph of $x$ against $t$. Justify your answer.
    \end{enumerate}
\end{problem}
\begin{solution}
    \begin{ppart}
        The characteristic equation of the DE is given by $m r^2 + kr + m \o^2 = 0$. Let the roots be $r_1$ and $r_2$. We have \[r_{1, 2} = \frac{-k \pm \sqrt{k^2 - 4m^2 \o^2}}{2m}.\] Since $k^2$ can be ignored, \[r_{1, 2} = \frac{-k \pm \sqrt{-4m^2 \o^2}}{2m} = \frac{-k \pm 2m \o \i}{2m} = -\frac{k}{2m} \pm \o \i.\] The general solution is thus given by $x = \e^{-kt/2m} \bp{A \cos \o t + B \sin \o t}$. Since the object is initially at equilibrium, we have $x = 0$ at $t = 0$. There is hence no contribution from the cosine term, i.e. $A = 0$. Thus, $x = B \e^{-kt/2m} \sin \o t$. Differentiating with respect to $t$, \[\der{x}{t} = B \e^{-kt/2m} \bp{\o \cos \o t - \frac{k}{2m} \sin \o t}.\] Since the object was initially released with speed $v > 0$, we have $\derx{x}{t} = v$ at $t = 0$. Hence, $B \o = v$. We hence obtain the solution \[x = \frac{v}{\o} \e^{-kt/2m} \sin \o t.\]
    \end{ppart}
    \begin{ppart}
        Let $x_n$ be the (signed) amplitude of the $n$th vibration, and let $t_n$ be the corresponding time, where $n \in \NN$.

        Since $\abs{\sin \o t_n} = 1$ for all $n \in \NN$, we have $\o t_n = \pi/2 + \pi n$, so $t_{n+1} - t_n = \pi/\o$. Thus, \[\frac{\abs{x_{n+1}}}{\abs{x_n}} = \frac{(v/\o) \e^{-k\pi t_{n+1}/2m} \abs{\sin \o t_{n+1}}}{(v/\o) \e^{-k\pi t_{n}/2m} \abs{\sin \o t_{n}}} = \e^{-k\pi (t_{n+1} - t_n) / 2m} = \e^{-k\pi/2m \o},\] whence the amplitudes $\bc{\abs{x_n}}$ are in geometric progression with common ratio $\exp{-k\pi/2m \o}$.
    \end{ppart}
    \begin{ppart}
        Recall that the roots of the characteristic equation are given by \[r_{1, 2} = \frac{-k \pm \sqrt{k^2 - 4m^2 \o^2}}{2m}.\] If $k^2 > 4m^2 \o^2$, then the roots are real and distinct, whence $x$ has general solution $x = A\e^{r_1 t} + B\e^{r_2 t}$. Since $x = 0$ at $t = 0$, we have $A + B = 0$, so $x = A\bp{\e^{r_1 t} - \e^{r_2 t}}$. Note that both roots are negative (since $\sqrt{k^2 - X^2} < \sqrt{k^2} = k$). Hence, as $t$ tends to infinity, $\e^{r_1 t} - \e^{r_2 t}$ (and by extension $x$) tends to 0.

        A possible graph of $x$ is
        \begin{figure}[H]\tikzsetnextfilename{41}
            \centering
            \begin{tikzpicture}[trim axis left, trim axis right]
                \begin{axis}[
                    domain = 0:2,
                    samples = 101,
                    ymin = 0,
                    ymax = 0.25,
                    xmin = 0,
                    xmax = 2,
                    axis y line=middle,
                    axis x line=middle,
                    xtick = \empty,
                    ytick = \empty,
                    xlabel = {$t$},
                    ylabel = {$x$},
                    legend cell align={left},
                    legend pos=outer north east,
                    after end axis/.code={
                        \path (axis cs:0,0) 
                        node [anchor=north east] {$O$};
                        }
                    ]
                    \addplot[black] {e^(-3*x) - e^(-5*x)};
                \end{axis}
            \end{tikzpicture}
        \end{figure}
    \end{ppart}
\end{solution}