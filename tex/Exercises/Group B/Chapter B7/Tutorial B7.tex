\section{Tutorial B7}

\begin{problem}
    Find
    \begin{enumerate}
        \item $\displaystyle\int \frac1{\sqrt{3-2x}} \d x$
        \item $\displaystyle\int \frac1{3-2x} \d x$
        \item $\displaystyle\int \frac1{3-2x^2} \d x$
        \item $\displaystyle\int \frac1{\sqrt{3-2x^2}} \d x$
        \item $\displaystyle\int \frac{x}{\sqrt{3-2x^2}} \d x$
        \item $\displaystyle\int \frac1{3+4x+2x^2} \d x$
    \end{enumerate}
\end{problem}
\begin{solution}
    \begin{ppart}
        Consider the substitution $u = 3-2x$.
        \begin{align*}
            \int \frac1{\sqrt{3-2x}} \d x = -\int \frac1{2\sqrt u} \d u  = -\sqrt{u} + C = -\sqrt{3-2x} + C.
        \end{align*}
    \end{ppart}
    \begin{ppart}
        Consider the substitution $u = 3-2x$. \[\int \frac1{3-2x} \d x = -\frac12 \int \frac1{u} \d u = -\frac12 \ln \abs{u} + C = -\frac12 \ln \abs{3 - 2x} + C.\]
    \end{ppart}
    \begin{ppart}
        \begin{align*}
            \int \frac1{3-2x^2} \d x &= \frac12 \int \frac1{3/2 - x^2} \d x\\
            &= \frac12 \bp{\frac1{2\sqrt{3/2}}} \ln{\frac{\sqrt{3/2} + x}{\sqrt{3/2} - x}} + C \\
            &= \frac1{2\sqrt6} \ln{\frac{\sqrt3 + \sqrt2 x}{\sqrt3 - \sqrt2 x}} + C.
        \end{align*}
    \end{ppart}
    \begin{ppart}
        \begin{align*}
            \int \frac1{\sqrt{3-2x^2}} \d x &= \frac1{\sqrt2} \int \frac1{\sqrt{3/2 - x^2}} \d x\\
            &= \frac1{\sqrt2} \arcsin{\frac{x}{\sqrt{3/2}}} + C \\
            &= \frac{\sqrt2}2 \arcsin{\frac{\sqrt6 x}{3}} + C.
        \end{align*}
    \end{ppart}
    \begin{ppart}
        Consider the substitution $u = 3-2x^2$.
        \begin{align*}
            \int \frac{x}{\sqrt{3-2x^2}} \d x = -\frac12 \int \frac1{2\sqrt {u}} \d u = -\frac{\sqrt{u}}2 + C = -\frac{\sqrt{3 - 2x^2}}2 + C.
        \end{align*}
    \end{ppart}
    \begin{ppart}
        \begin{align*}
            \int \frac1{3+4x+2x^2} \d x &= \frac12 \int \frac1{(x+1)^2 + 1/2} \d x\\
            &= \frac12 \bp{\frac1{\sqrt{1/2}}} \arctan{\frac{x+1}{1/\sqrt{1/2}}} + C \\
            &= \frac{\arctan{\sqrt2 (x + 1)}}{\sqrt2}  + C.
        \end{align*}
    \end{ppart}
\end{solution}

\begin{problem}
    Find
    \begin{enumerate}
        \item $\displaystyle\int \frac{\sec^2 3x}{\tan 3x} \d x$
        \item $\displaystyle\int \cos (3x + \a) \d x$, where $\a$ is a constant
        \item $\displaystyle\int \cos^2 3x \d x$
        \item $\displaystyle\int \e^{1-2x} \d x$
    \end{enumerate}
\end{problem}
\begin{solution}
    \begin{ppart}
        \[\int \frac{\sec^2 3x}{\tan 3x} \d x = \frac13 \int \frac{3\sec^2 3x}{\tan 3x} \d x = \frac{\ln \tan 3x}3 + C.\]
    \end{ppart}
    \begin{ppart}
        \begin{align*}
            \int \cos (3x + \a) \d x = \frac{\sin(3x + \a)}3 + C
        \end{align*}
    \end{ppart}
    \begin{ppart}
        Recall that $\cos^2\t = (1 + \cos2\t)/2$. Hence, \[\int \cos^2 3x \d x = \frac12 \int (1 + \cos6x) \d x = \frac12 \bp{x + \frac{\sin6x}6} + C = \frac{x}2 + \frac{\sin6x}{12} + C.\]
    \end{ppart}
    \begin{ppart}
        \[\int \e^{1-2x} \d x = -\frac12 \int 2\e^{1-2x} \d x = -\frac12 \e^{1-2x} + C.\]
    \end{ppart}
\end{solution}

\clearpage
\begin{problem}
    Find
    \begin{enumerate}
        \item $\displaystyle\int 2x\sqrt{3x^2 - 5} \d x$
        \item $\displaystyle\int \frac{x^2 - 1}{\sqrt{x^3 - 3x}} \d x$
        \item $\displaystyle\int \sin x \sqrt{\cos x} \d x$
        \item $\displaystyle\int \e^{2x} \bp{1 - \e^{2x}}^4 \d x$
    \end{enumerate}
\end{problem}
\begin{solution}
    \begin{ppart}
        Consider the substitution $u = 3x^2 - 5$. \[\int 2x\sqrt{3x^2 - 5} \d x = \frac13 \int \sqrt{u} \d u = \frac13 \bp{\frac23 u^{3/2}} + C = \frac29 \bp{3x^2 - 5}^{3/2} + C.\]
    \end{ppart}
    \begin{ppart}
        Consider the substitution $u = x^3 - 3x$: \[\int \frac{x^2 - 1}{\sqrt{x^3 - 3x}} \d x = \frac23 \int \frac{\d u}{2\sqrt{u}} = \frac23 \sqrt{u} + C = \frac23 \sqrt{x^3 - 3x} + C.\]
    \end{ppart}
    \begin{ppart}
        Consider the substitution $u = \cos x$. \[\int \sin x \sqrt{\cos x} \d x = - \int \sqrt{u} \d u = -\frac23 u^{3/2} + C = -\frac23 \cos^{3/2} x + C.\]
    \end{ppart}
    \begin{ppart}
        Consider the substitution $u = 1 - \e^{2x}$. \[\int \e^{2x}(1 - \e^{2x})^4 \d x = -\frac12 \int u^4 \d u = -\frac12 \bp{\frac{u^5}{5}} + C = -\frac{\bp{1 - \e^{2x}}^5}{10} + C.\]
    \end{ppart}
\end{solution}

\begin{problem}
    Find
    \begin{enumerate}
        \item $\displaystyle\int \frac1{\sqrt{x}(1- \sqrt{x})} \d x$
        \item $\displaystyle\int \frac{3x}{x+3} \d x$
        \item $\displaystyle\int \frac{\sin x + \cos x}{\sin x - \cos x} \d x$
    \end{enumerate}
\end{problem}
\begin{solution}
    \begin{ppart}
        Consider the substitution $u = 1 - \sqrt x$. \[\int \frac1{\sqrt{x}(1- \sqrt{x})} \d x = -2 \int \frac1{u} \d u = -2 \ln \abs{u} + C = -2 \ln \abs{1 - \sqrt{x}} + C.\]
    \end{ppart}
    \begin{ppart}
        \[\int \frac{3x}{x+3} \d x = \int \bp{3 - \frac9{x+3} }\d x = 3x - 9\ln\abs{x+3} + C.\]
    \end{ppart}
    \begin{ppart}
        Observe that the derivative of $\sin x - \cos x$ is $\sin x + \cos x$. Hence, \[\int \frac{\sin x + \cos x}{\sin x - \cos x}\d x = \ln \abs{\sin x - \cos x} + C.\]
    \end{ppart}
\end{solution}

\begin{problem}
    Find 
    \begin{enumerate}
        \item $\displaystyle \int \frac{\e^{-\sqrt{x}}}{\sqrt{x}} \d x$
        \item $\displaystyle \int (\sin x)(\cos x)(\e^{\cos 2x}) \d x$
    \end{enumerate}
\end{problem}
\begin{solution}
    \begin{ppart}
        Consider the substitution $u = -\sqrt x$. \[\int \frac{\e^{-\sqrt{x}}}{\sqrt{x}} \d x = -2\int \e^{u} \d u = -2\e^u + C = -2\e^{-\sqrt{x}} + C.\]
    \end{ppart}
    \begin{ppart}
        Consider the substitution $u = \cos 2x$.
        \begin{align*}
            \int (\sin x)(\cos x)(\e^{\cos 2x}) \d x &= \frac12 \int \e^{\cos 2x} \sin 2x \d x\\
            &= -\frac14 \int \e^u \d u \\
            &= -\frac{\e^u }4 + C\\
            &= -\frac{\e^{\cos 2x}}4 + C.
        \end{align*}
    \end{ppart}
\end{solution}

\begin{problem}
    Find
    \begin{enumerate}
        \item $\displaystyle\int \tan^2 2x \d x$
        \item $\displaystyle\int \frac1{1 + \cos 2t} \d t$
        \item $\displaystyle\int \sin{\frac52\t}\cos{\frac12\t}  d\t$
        \item $\displaystyle\int \tan^4 x \d x$
    \end{enumerate}
\end{problem}
\begin{solution}
    \begin{ppart}
        \[\int \tan^2 2x \d x = \int \bp{\sec^2 2x - 1} \d x = \frac{\tan 2x}2 - x + C.\]
    \end{ppart}
    \begin{ppart}
        Note that \[\frac1{1 + \cos 2t} = \frac1{1 + \bp{2 \cos^2 t - 1}} = \frac{\sec^2 t}2.\] Hence, \[\int \frac1{1 + \cos 2t} \d t = \frac12 \int \sec^2 t \d t = \frac{\tan t}2 + C.\]
    \end{ppart}
    \begin{ppart}
        By the product-to-sum identity, \[\sin{\frac{5\t}2}\cos{\frac{\t}2} = \frac{\sin 3\t + \sin 2\t}2.\] Hence,
        \begin{align*}
            \int \sin{\frac52\t}\cos{\frac12\t}  d\t = \frac12 \int \bp{\sin 3\t + \sin2\t}  d\t = -\frac{\cos 3\t}6 - \frac{\cos 2\t}4 + C
        \end{align*}
    \end{ppart}
    \begin{ppart}
        Note that \[\int \tan^2 x \d x = \int \bp{\sec^2 x - 1} \d x = \tan x - x + C\] and \[\int \tan[2]{x} \sec[2]{x} \d x = \frac{\tan^3 x}3 + C.\] Hence,
        \begin{align*}
            \int \tan^4 x \d x &= \int \tan^2 x \bp{\sec^2 x - 1} \d x\\
            &= \int \bp{\tan[2]{x} \sec[2]{x} - \tan^2 x} \d x\\
            &= \frac{\tan^3 x}3 - \tan x + x + C
        \end{align*}
    \end{ppart}
\end{solution}

\begin{problem}
    Find
    \begin{enumerate}
        \item $\displaystyle\int \frac1{4x^2 + 2x + 10} \d x$
        \item $\displaystyle\int \frac{x^2}{1 - x^2} \d x$
        \item $\displaystyle\int \frac1{\sqrt{3 + 2x - x^2}} \d x$
    \end{enumerate}
\end{problem}
\begin{solution}
    \begin{ppart}
        \begin{align*}
            \int \frac1{4x^2 + 2x + 10} \d x &= 4 \int \frac1{\bp{4x + 1}^2 + 39} \d x\\
            &= 4 \bp{\frac14} \bp{\frac1{\sqrt{39}}} \arctan \bp{\frac{4x+1}{\sqrt{39}}} + C\\
            &= \frac1{\sqrt{39}} \arctan \bp{\frac{4x+1}{\sqrt{39}}} + C.
        \end{align*}
    \end{ppart}
    \begin{ppart}
        \[\int \frac{x^2}{1 - x^2} \d x = \int \bp{\frac1{1 - x^2} - 1} \d x = \frac1{2}\ln\bp{\frac{1 + x}{1 - x}} - x + C.\]
    \end{ppart}
    \begin{ppart}
        \[\int \frac1{\sqrt{3 + 2x - x^2}} \d x = \int \frac1{\sqrt{2^2 - (x-1)^2}} \d x = \arcsin{\frac{x-1}{2}} + C.\]
    \end{ppart}
\end{solution}

\begin{problem}
    Evaluate the following without the use of graphic calculator:

    \begin{enumerate}
        \item $\displaystyle\int_{\pi/3}^{2\pi/3} 4\cot\frac{x}2 \csc^2\frac{x}2 \d x$
        \item $\displaystyle\int_0^4 \frac{x+2}{\sqrt{2x+1}} \d x$
        \item $\displaystyle\int_0^1 \frac2{(1+x)(1+x^2)} \d x$
        \item $\displaystyle\int_{-4}^{-2} \frac{x^3 + 2}{x^2 - 1} \d x$
    \end{enumerate}
\end{problem}
\begin{solution}
    \begin{ppart}
        \[\int_{\pi/3}^{2\pi/3} 4\cot\frac{x}2 \csc^2\frac{x}2 \d x = -4\int_{\pi/3}^{2\pi/3} \cot\frac{x}2 \bp{-\csc^2 \frac{x}2} \d x = -8\evalint{\frac{\tan[2]{x/2}}2}{\pi/2}{2\pi/3} = \frac{32}3.\]
    \end{ppart}
    \begin{ppart}
        Consider the substitution $u = 2x + 1$.
        \begin{align*}
            \int_0^4 \frac{x+2}{\sqrt{2x+1}} \d x &= \frac12 \int_0^4 \bp{\sqrt{2x+1} + \frac3{\sqrt{2x +1}}} \d x \\
            &= \frac14 \int_1^9 \bp{\sqrt{u} + \frac3{\sqrt{u}}} \d u\\
            &= \frac14 \evalint{\frac{u^{3/2}}{3/2} + \frac{3u^{1/2}}{1/2}}{1}{9}\\
            &= \frac{22}3.
        \end{align*}
    \end{ppart}
    \begin{ppart}
        Note that \[\int_0^1 \frac{x}{1 + x^2} \d x = \frac12 \int_0^1 \frac{2x}{1 + x^2} \d x = \frac12 \evalint{\ln \abs{1 + x^2}}01 = \frac{\ln 2}2.\] Thus,
        \begin{align*}
            \int_0^1 \frac2{(1+x)(1+x^2)} \d x &= \int_0^1 \bp{\frac1{1+x} + \frac1{1+x^2} - \frac{x}{1 + x^2}}\d x \\
            &= \evalint{\ln \abs{1 + x}}01 + \evalint{\arctan x}01 - \frac12 \ln 2\\
            &= \frac12 \ln 2 + \frac\pi4.
        \end{align*}
    \end{ppart}
    \begin{ppart}
        \begin{align*}
            \int_{-4}^{-2} \frac{x^3 + 2}{x^2 - 1} \d x &= \int_{-4}^{-2} \bp{x + \frac{3/2}{x-1} - \frac{1/2}{x+1}} \d x\\
            &= \evalint{\frac{x^2}2 + \frac32 \ln \abs{x-1} - \frac12 \ln \abs{x+1}}{-4}{-2} \\
            &= -6 + 2\ln3 - \frac32 \ln 5.
        \end{align*}
    \end{ppart}
\end{solution}

\begin{problem}
    Using the given substitution, find

    \begin{enumerate}
        \item $\displaystyle\int \frac{x}{(2x+3)^3} \d x$ \hfill $[u = 2x+3]$
        \item $\displaystyle\int \frac1{\e^x + 4\e^{-x}} \d x$ \hfill $[u = \e^x]$
        \item $\displaystyle\int_0^{\sqrt2} \sqrt{4 - y^2} \d y$ \hfill $[y = 2\sin\t]$
        \item $\displaystyle\int_0^{\pi/2} \frac1{1 + \sin\t} \d \t$ \hfill $\bs{t = \tan{\t/2}}$
    \end{enumerate}
\end{problem}
\begin{solution}
    \begin{ppart}
        Using the substitution $u = 2x + 3$,
        \begin{align*}
            \int \frac{x}{(2x+3)^3} \d x &= \frac14 \int \frac{u-3}{u^3} \d x\\
            &= \frac14 \int \bp{\frac1{u^2} - \frac3{u^3}} \d u\\
            &= \frac14 \bp{\frac{u^{-1}}{-1} - \frac{3u^{-2}}{-2} } + C \\
            &= \frac38 (2x+3)^{-2} - \frac14 (2x+3)^{-1} + C.
        \end{align*}
    \end{ppart}
    \begin{ppart}
        Using the substitution $u = \e^x$,
        \begin{align*}
            \int \frac1{\e^x + 4\e^{-x}} \d x &= \int \frac{e^x}{e^{2x} + 4} \d x\\
            &= \int \frac1{u^2 + 4} \d u\\
            &= \frac12 \arctan \bp{\frac{u}2} + C\\
            &= \frac12 \arctan \bp{\frac{\e^x}2}+ C.
        \end{align*}
    \end{ppart}
    \begin{ppart}
        Using the substitution $y = 2 \sin \t$,
        \begin{align*}
            \int_0^{\sqrt2} \sqrt{4 - y^2} \d y &= 2\int_0^{\pi/4} \cos\t\sqrt{4 - 4\sin^2\t}\d \t \\
            &= 4\int_0^{\pi/4} \cos\t\sqrt{1 - \sin^2\t}\d \t\\
            &= 4\int_0^{\pi/4} \cos^2\t \d \t\\
            &= 4\int_0^{\pi/4} \frac{1 + \cos 2\t}{2} \d \t\\
            &= 2 \evalint{\t + \frac{\sin2\t}2}{0}{\pi/4}\\
            &= 1 + \frac\pi2.
        \end{align*}
    \end{ppart}
    \begin{ppart}
        Consider the substitution $t = \tan{\t/2}$. Then $\t = 2\arctan t$, so \[\d \t = \frac{2}{1 + t^2} \d t\] and \[\sin \t = \frac{2 \sin{\t/2} \cos{\t/2}}{\sin[2]{\t/2} + \cos[2]{\t/2}} = \frac{2\tan{\t/2}}{\tan[2]{\t/2} + 1} = \frac{2t}{t^2 + 1}.\] Hence, \[\int_0^{\pi/2} \frac1{1 + \sin\t} \d \t = \int_0^1 \frac{2 / (1+t^2)}{1 + 2t/(1+t^2)} \d u = \int_0^1 \frac{2}{(t+1)^2} \d t = 2 \evalint{-\frac1{t+1}}{0}{1} = 1.\]
    \end{ppart}
\end{solution}

\begin{problem}
    Find        
    \begin{enumerate}
        \item $\displaystyle\int \ln{2x+1} \d x$
        \item $\displaystyle\int x \arctan(x^2) \d x$
        \item $\displaystyle\int \e^{-2x}\cos2x \d x$
        \item $\displaystyle\int_0^2 x^2e^{-x} \d x$
    \end{enumerate}
\end{problem}
\begin{solution}
    \begin{ppart}
        Consider the substitution $u = 2x + 1$. \[\int \ln{2x + 1} \d x = \frac12 \int \ln u \d u.\] Integrating by parts, \[\begin{array}{r c @{\hspace*{1.0cm}} c}\toprule
                & D & I \\\cmidrule{1-3}
                + & \ln u & 1 \\
                - & 1/u & u \\\bottomrule
            \end{array}\]
        Thus, 
        \begin{align*}
            \int \ln{2x + 1} \d x &= \frac12 \bp{u\ln u - \int u \bp{\frac{1}{u}} \d u}\\
            &= \frac{u\ln u - u}2 + C\\
            &= \frac{(2x + 1)\ln{2x+1} - (2x+1)}2 + C\\
            &= x\ln{2x+1} + \frac{\ln{2x+1}}2 - x + C.
        \end{align*}
    \end{ppart}
    \begin{ppart}
        Consider the substitution $u = x^2$. \[\int x \arctan(x^2) \d x = \frac12 \int \arctan u \d u.\] Integrating by parts, \[\begin{array}{r c @{\hspace*{1.0cm}} c}\toprule
                & D & I \\\cmidrule{1-3}
                + & \arctan u & 1 \\
                - & 1/(1 + u^2) & u \\\bottomrule
            \end{array}\]
        Thus,
        \begin{align*}
            \int x \arctan(x^2) \d x &= \frac12 \bp{u \arctan u - \int \frac{u}{1 + u^2} \d u}\\
            &= \frac12 \bs{u \arctan u - \frac{\ln{1 + u^2}}2} + C\\
            &= \frac{x^2 \arctan x^2}2 - \frac{\ln{1 + x^4}}4 + C.
        \end{align*}
    \end{ppart}
    \begin{ppart}
        Let \[I = \int \e^{-2x}\cos2x \d x.\] Integrating by parts, we have \[\begin{array}{r c @{\hspace*{1.0cm}} c}\toprule
                & D & I \\\cmidrule{1-3}
                + & \e^{-2x} & \cos 2x \\
                - & -2\e^{-2x} & \sin{2x}/2  \\
                + & 4\e^{-2x} & -\cos{2x}/4 \\\bottomrule
            \end{array}\]
        Thus, \[I = \frac{\e^{-2x} \sin 2x}2 - \frac{\e^{-2x} \cos 2x}2 - I.\] Solving for $I$, we get \[I = \frac{\e^{-2x} \bp{\sin 2x - \cos 2x}}4 + C.\]
    \end{ppart}
    \begin{ppart}
        Integrating by parts, we get \[\begin{array}{r c @{\hspace*{1.0cm}} c}\toprule
                & D & I \\\cmidrule{1-3}
                + & x^2 & \phantom{-}\e^{-x} \\
                - & 2x & -\e^{-x}  \\
                + & 2 & \phantom{-}\e^{-x} \\
                - & 0 & -\e^{-x}\\\bottomrule
            \end{array}\]
        Thus, \[\int_0^2 x^2 \e^{-x} \d x = \evalint{-x^2 \e^{-x} - 2x\e^{-x} - 2\e^{-x}}{0}{2} = 2 - 10\e^{-2}.\]
    \end{ppart}
\end{solution}

\begin{problem}
    \begin{enumerate}
        \item Show that \[\der{}{x} \ln{\sec x + \tan x} = \sec x.\]
        \item Find \[\int x \sin x \d x.\]
        \item Find the exact value of \[\int_0^{\pi/4} (x\sin x)\ln{\sec x + \tan x} \d x.\]
    \end{enumerate}
\end{problem}
\begin{solution}
    \begin{ppart}
        \[\der{}{x} \ln{\sec x + \tan x} = \frac{\sec x \tan x + \sec^2 x}{\sec x + \tan x} = \sec x \bp{\frac{\tan x + \sec x}{\sec x + \tan x}} = \sec x.\]
    \end{ppart}
    \begin{ppart}
        Integrating by parts, \[\begin{array}{r c @{\hspace*{1.0cm}} c}\toprule
                & D & I \\\cmidrule{1-3}
                + & x & \sin x \\
                - & 1 & -\cos x  \\
                + & 0 & -\sin x \\\bottomrule
            \end{array}\]
        Hence, \[\int x \sin x \d x = -x\cos x + \sin x + C.\]
    \end{ppart}
    \begin{ppart}
        Integrating by parts, \[\begin{array}{r c @{\hspace*{1.0cm}} c}\toprule
                & D & I \\\cmidrule{1-3}
                + & \ln{\sec x + \tan x} & x\sin x \\
                - & \sec x & -x\cos x + \sin x  \\\bottomrule
            \end{array}\]
        Thus,
        \begin{align*}
            & \int_0^{\pi/4} (x\sin x)\ln{\sec x + \tan x} \d x\\
            &\hspace{2em}= \evalint{\ln{\sec x + \tan x}\bp{-x\cos x + \sin x}}{0}{\pi/4} - \int_0^{\pi/4} \bp{-x + \tan x} \d x\\
            &\hspace{2em}= \evalint{\ln{\sec x + \tan x}\bp{-x\cos x + \sin x}-\frac{x^2}2 - \ln \abs{\cos x}}{0}{\pi/4}\\
            &\hspace{2em}= \frac{\sqrt2}2\bp{1 - \frac\pi4}\ln{\sqrt2 + 1} + \frac{\pi^2}{32} - \frac{\ln 2}2.
        \end{align*}
    \end{ppart}
\end{solution}

\clearpage
\begin{problem}
    \begin{enumerate}
        \item Use the fact that \[7\cos x - 4\sin x = \frac32 (\cos x + \sin x) + \frac{11}2 (\cos x - \sin x)\] to find the exact value of \[\int_0^{\pi/2} \frac{7\cos x - 4\sin x}{\cos x + \sin x} \d x.\]
        \item Use integration by parts to find the exact value of \[\int_1^\e (\ln x)^2 \d x.\]
    \end{enumerate}
\end{problem}
\begin{solution}
    \begin{ppart}
        Note that \[\frac{7\cos x - 4\sin x}{\cos x + \sin x} = \frac12 \bp{\frac{3(\cos x + \sin x) + 11(\cos x - \sin x)}{\cos x + \sin x}} = \frac32 + \frac{11}2 \bp{\frac{\cos x - \sin x}{\cos x + \sin x}}.\] Thus,
        \begin{align*}
            \int_0^{\pi/2} \frac{7\cos x - 4\sin x}{\cos x + \sin x} \d x &= \frac12 \int_0^{\pi/2} \bp{3 + 11 \cdot \frac{\cos x - \sin x}{\cos x + \sin x}} \d x\\
            &= \evalint{\frac{3x}2 + \frac{11}{2}\ln\abs{\cos x + \sin x}}{0}{\pi/2}\\
            &= \frac{3\pi}4.
        \end{align*}
    \end{ppart}
    \begin{ppart}
        Integrating by parts, \[\begin{array}{r c @{\hspace*{1.0cm}} c}\toprule
                & D & I \\\cmidrule{1-3}
                + & (\ln x)^2 & 1 \\
                - & 2\ln{x}/x & x \\\bottomrule
            \end{array}\]
        Thus,
        \begin{align*}
            \int_1^\e (\ln x)^2 \d x &= \evalint{x (\ln x)^2}1\e - 2\int_1^\e \ln x \d x\\
            &= \evalint{x (\ln x)^2 - 2\bp{x \ln x - x}}1\e\\
            &= \e - 2.
        \end{align*}
    \end{ppart}
\end{solution}

\begin{problem}
    \begin{enumerate}
        \item Solve the inequality $x^2 + 2x - 3 < 0$.
        \item Without using the graphing calculator, evaluate
        \begin{enumerate}
            \item $\displaystyle\int_{-4}^4 \abs{x^2 + 2x - 3} \d x$
            \item $\displaystyle\int_0^2 x\abs{x^2 + 2x - 3} \d x$
        \end{enumerate}
    \end{enumerate}
\end{problem}
\begin{solution}
    \begin{ppart}
        Completing the square, we have $x^2 + 2x - 3 = (x+1)^2 - 4 < 0$. Thus, $(x+1)^2 < 4$, so $-3 < x < 1$.
    \end{ppart}
    \begin{ppart}
        \begin{psubpart}
            Let \[F(x) = \int \bp{x^2 + 2x - 3}\d x = \frac13 x^3 + x^2 - 3x + C.\] Then,
            \begin{align*}
                &\int_{-4}^4 \abs{x^2 + 2x - 3} \d x\\
                &\hspace{2em}= \int_{-4}^{-3} \abs{x^2 + 2x - 3} \d x + \int_{-3}^1 \abs{x^2 + 2x - 3} \d x + \int_1^4 \abs{x^2 + 2x - 3} \d x\\
                &\hspace{2em}= \int_{-4}^{-3} x^2 + 2x - 3 \d x - \int_{-3}^1 x^2 + 2x - 3 \d x + \int_1^4 x^2 + 2x - 3 \d x\\
                &\hspace{2em}= \Big[F(-3) - F(-4)\Big] - \Big[F(1) - F(-3)\Big] + \Big[F(4) - F(1)\Big]\\
                &\hspace{2em}= 40.
            \end{align*}
        \end{psubpart}
        \begin{psubpart}
            Let \[F(x) = \int x \bp{x^2 + 2x - 3} \d x = \frac14 x^4 + \frac23 x^3 - \frac32 x^2 + C.\] Then,
            \begin{align*}
                &\int_0^2 x\abs{x^2 + 2x - 3} \d x \\
                &\hspace{2em}= \int_0^1 x\abs{x^2 + 2x - 3} \d x + \int_1^2 x\abs{x^2 + 2x - 3} \d x\\
                &\hspace{2em}= -\int_0^1 x(x^2 + 2x - 3) \d x + \int_1^2 x(x^2 + 2x - 3) \d x\\
                &\hspace{2em}= -\Big[F(1) - F(0)\Big] + \Big[F(2) - F(1)\Big]\\
                &\hspace{2em}= \frac92.
            \end{align*}
        \end{psubpart}
    \end{ppart}
\end{solution}

\begin{problem}
    The indefinite integral \[\int \frac{P(x)}{x^3 + 1} \d x,\] where $P(x)$ is a polynomial in $x$, is denoted by $I$.
        
    \begin{enumerate}
        \item Find $I$ when $P(x) = x^2$.
        \item By writing $x^3 + 1 = (x+1)\bp{x^2 + Ax + B}$, where $A$ and $B$ are constants, find $I$ when
        \begin{enumerate}
            \item $P(x) = x^2 - x + 1$
            \item $P(x) = x + 1$
        \end{enumerate}
        \item Using the results of parts (a) and (b), or otherwise, find $I$ when $P(x) = 1$.
    \end{enumerate}
\end{problem}
\begin{solution}
    \begin{ppart}
        \[\int \frac{x^2}{x^3 + 1} \d x = \frac13 \frac{3x^2}{x^3 + 1} \d x = \frac{\ln \abs{x^3 + 1}}3 + C.\]
    \end{ppart}
    \begin{ppart}
        We have the factorization $x^3 + 1 = (x+1)\bp{x^2 - x + 1}$.
        \begin{psubpart}
            \[\int \frac{x^2 - x + 1}{x^3 + 1} \d x = \int \frac{x^2 - x + 1}{(x+1)\bp{x^2 - x + 1}} \d x = \int \frac1{x+1} \d x = \ln \abs{x+1} + C.\]
        \end{psubpart}
        \begin{psubpart}
            \begin{align*}
                \int \frac{x+1}{x^3 + 1} \d x &= \int \frac{x+1}{(x+1)\bp{x^2 - x + 1}} \d x\\
                &= \int \frac1{x^2 - x + 1} \d x\\
                &= \int \frac1{(x-1/2)^2 + 3/4} \d x\\
                &= \frac1{\sqrt{3/4}} \arctan \bp{ \frac{x - 1/2}{\sqrt{3/4}}} + C\\
                &= \frac2{\sqrt3}\arctan \bp{\frac{2x-1}{\sqrt3}} + C.
            \end{align*}
        \end{psubpart}
    \end{ppart}
    \begin{ppart}
        Observe that \[1 = \frac{\bp{x^2 - x + 1} - x^2 + (x+1)}2.\] Hence,

        \begin{align*}
            \int \frac1{x^3 + 1} \d x &= \frac12\bp{\int\frac{x^2 - x + 1}{x^3 + 1} \d x - \int\frac{x^2}{x^3 + 1} \d x + \int\frac{x+1}{x^3 + 1} \d x}\\
            &= \frac12 \bs{\ln \abs{x+1} - \frac{\ln\abs{x^3 + 1}}3 + \frac2{\sqrt3}\arctan \bp{\frac{2x-1}{\sqrt3}}} + C\\
            &= \frac12\ln \abs{x+1} - \frac{\ln\abs{x^3 + 1}}6 + \frac1{\sqrt3}\arctan \bp{\frac{2x-1}{\sqrt3}} + C.
        \end{align*}
    \end{ppart}
\end{solution}