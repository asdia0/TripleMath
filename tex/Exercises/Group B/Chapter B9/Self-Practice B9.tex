\section{Self-Practice B9}

\begin{problem}
    The arc of the curve $y^{2}=4ax$, for which $y\geq0$ and $0\leq x\leq a$, is rotated through $2\pi$ radians about the $x$-axis. Prove that the area of the surface so generated is \[\frac83 (2\sqrt2 - 1) \pi a^2.\]
\end{problem}
\begin{solution}
    Note that $y = \sqrt{4ax}$, so $\derx{y}{x} = \sqrt{a/x}$, hence the surface area of the solid generated is given by
    \begin{align*}
        \area &= 2\pi \int_0^a y \sqrt{1 + \bp{\der{y}{x}}^2} \d x\\
        &= 2\pi \int_0^a \sqrt{4ax} \sqrt{1 + \frac{a}{x}} \d x\\
        &= 4\pi \sqrt{a} \int_0^a \sqrt{x + a} \d x\\
        &= 4\pi \sqrt{a} \evalint{\frac23 (x+a)^{3/2}}{0}{a}\\
        &= \frac{8}{3} \pi \sqrt{a} \bs{(2a)^{3/2} - a^{3/2}}\\
        &= \frac{8}{3} \bp{2\sqrt2 - 1}\pi a^2 \units[2].
    \end{align*}
\end{solution}

\begin{problem}
    The area bounded by the ellipse with parametric equations $x=3\cos\t$, $y=2\sqrt{2}\sin\t$ and the positive $x$- and $y$-axis is rotated completely about the $y$-axis. Find the curved surface area of the solid.
\end{problem}
\begin{solution}
    Note that $\derx{x}{\t} = -3\sin\t$ and $\derx{y}{\t} = 2\sqrt2 \cos \t$, so the surface area of the solid is \[2\pi \int_{0}^{\pi/2} x \sqrt{\bp{\der{x}{\t}}^2 + \bp{\der{y}{\t}}^2} \d \t = 2\pi \int_0^{\pi/2} \bp{3\cos \t} \sqrt{9\sin^2 \t + 8 \cos^2 \t} \d \t = 54.4 \units[2].\]
\end{solution}

\begin{problem}
    A curve is defined parametrically by \[x=2\sqrt{2}a\sin\t \quad \tand \quad y=\frac{1}{2}a\sin2\t.\]

    Show that \[\bp{\der{x}{\t}}^2 + \bp{\der{y}{\t}}^2 = a^2 \bp{2 + \cos 2\t}^2.\]

    The portion of the curve from $\t=0$ to $\t=\pi/3$ is rotated completely about the $x$-axis. Find the exact surface area generated.
\end{problem}
\begin{solution}
    Note that \[\der{x}{\t} = 2\sqrt{2} a \cos \t \quad \tand \quad \der{y}{\t} = a \cos 2\t,\] so \[\bp{\der{x}{\t}}^2 + \bp{\der{y}{\t}}^2 = 8a^2 \cos^2 \t + \a^2 \cos^2 2\t = a^2 \bp{4 + 4\cos 2\t + \cos^2 2\t} = a^2 \bp{\cos 2\t + 2}^2.\]

    The surface area generated is
    \begin{align*}
        \area &= 2\pi \int_0^{\pi/3} y \sqrt{\bp{\der{x}{\t}}^2 + \bp{\der{y}{\t}}^2} \d \t\\
        &= 2\pi \int_0^{\pi/3} \frac12 a^2 \sin 2\t \bp{\cos 2\t + 2} \d \t\\
        &= \pi a^2 \int_0^{\pi/3} \bp{\frac12 \sin 4\t + 2\sin 2\t} \d \t\\
        &= \pi a^2 \evalint{-\frac18 \cos 4\t - \cos 2\t}0{\pi/3}\\
        &= \frac{27}{16} \pi a^2 \units[2].
    \end{align*}
\end{solution}

\begin{problem}
    A curve is defined parametrically by $x = t^{2}-2\ln t$, $y = 4(t-1)$, where $t \in \RR$, $t \geq 1$.

    \begin{enumerate}
        \item The points $A$ and $B$ on the curve are given by $t=1$ and $t=2$ respectively. Show that the length of the arc $AB$ of the curve is $3+2\ln2$.
        \item The arc $AB$ is rotated through one revolution about the $x$-axis. Show that the area of the curved surface generated is \[\frac83 \pi \bp{11 - 6\ln2}.\]
    \end{enumerate}
\end{problem}
\begin{solution}
    \begin{ppart}
        Note that $\derx{x}{t} = 2t - 2/t$ and $\derx{y}{t} = 4$, so \[\bp{\der{x}{\t}}^2 + \bp{\der{y}{\t}}^2 = \bp{2t - \frac2t}^2 + 4^2 = \bp{2t + \frac2t}^2.\] Thus, the arc length $AB$ is given by \[\overbow{AB} = \int_1^2 \sqrt{\bp{\der{x}{\t}}^2 + \bp{\der{y}{\t}}^2} \d t = \int_1^2 \bp{2t + \frac2t} \d t = \evalint{t^2 + 2\ln t}12 = 3 + 2\ln 2 \units.\]
    \end{ppart}
    \begin{ppart}
        The surface area of the solid generated is given by
        \begin{align*}
            \area &= 2\pi \int_1^2 y \sqrt{\bp{\der{x}{\t}}^2 + \bp{\der{y}{\t}}^2} \d t\\
            &= 8\pi \int_1^2 \bp{t-1} \bp{2t + \frac2t} \d t\\
            &= 16\pi \int_1^2 bp{t^2 - t + 1 - \frac1t} \d t\\
            &= 16\pi \evalint{\frac13 t^3 - \frac12 t^2 + t - \ln t}12\\
            &= \frac{8\pi}{3} \bp{11 - 6 \ln 2} \units[2].
        \end{align*}
    \end{ppart}
\end{solution}

\begin{problem}
    The curve $\G$ has polar equation $r=ke^{\t}$, where $k$ is a positive constant and $0\leq\t\leq\pi$. The points $P$ and $Q$ on $\G$ correspond to $\t=\a$ and $\t=\b$ respectively ($\b > \a$). The area of the region bounded by the lines $\t=\a$, $\t=\b$ and the arc $PQ$ is denoted by $A$. The length of the arc $PQ$ is denoted by $s$.

    \begin{enumerate}
        \item Find expressions for $A$ and $s$ in terms of $\a,\b$ and $k$.
        \item Deduce that \[\frac{A}{s^2} = \frac18 \bp{\frac{\e^\b + \e^a}{\e^b - \e^a}}.\]
    \end{enumerate}
\end{problem}
\begin{solution}
    \begin{ppart}
        We have \[A = \frac12 \int_{\a}^{\b} r^2 \d \t = \frac12 \int_{\a}^{\b} k^2 \e^{2\t} \d \t = \frac12 k^2 \evalint{\frac12 \e^{2\t}}\a\b = \frac14 k^2 \bp{\e^{2\b} - \e^{2\a}} \units[2].\] Note that $\derx{r}{\t} = k\e^\t = r$, so
        \begin{align*}
            s &= \int_\a^\b \sqrt{r^2 + \bp{\der{r}{\t}}^2} \d \t\\
            &= \int_\a^\b \sqrt{r^2 + r^2} \d \t\\
            &= \sqrt 2 \int_\a^\b r \d \t\\
            &= \sqrt 2 \int_\a^\b k \e^\t \d t\\
            &= \sqrt 2 k \evalint{\e^\t}\a\b\\
            &= \sqrt2 k \bp{\e^\b - \e^\a} \units.
        \end{align*}
    \end{ppart}
    \begin{ppart}
        We have \[\frac{A}{s^2} = \frac{\frac14 k^2 \bp{\e^{2\b} - \e^{2\a}}}{2 k^2 \bp{\e^\b - \e^\a}^2} = \frac18 \frac{\bp{\e^\b - \e^\a} \bp{\e^\b + \e^\a}}{\bp{\e^\b - \e^\a}^2} = \frac18 \bp{\frac{\e^\b + \e^a}{\e^b - \e^a}}.\]
    \end{ppart}
\end{solution}