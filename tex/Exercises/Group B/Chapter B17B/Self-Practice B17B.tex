\section{Self-Practice B17B}

\begin{problem}
    Determine which of the following are subspaces of the space of all real-valued functions $f$ defined on the entire real line.
    \begin{enumerate}
        \item all $f$ such that $f(x) \leq 0$ for all $x$.
        \item all $f$ such that $f(0) = 0$.
        \item all $f$ such that $f(0) = 2$.
        \item all constant functions.
    \end{enumerate}
\end{problem}
\begin{solution}
    Let $V$ be the space of all real-valued functions defined on $\RR$. Note that the zero element of $V$ is the zero function $f(x) \equiv 0$.

    \begin{ppart}
        Let $S$ be the set of all functions $f \in V$ such that $f(x) \leq 0$ for all $x \in \RR$. Then $g \in S$, where $g(x) \equiv -1$. Taking $c = -1$, we see that $cg(x) \equiv 1 \not\leq 0$ for all $x$, so $S$ is not closed under scalar multiplication. Hence, $S$ is not a subspace of $V$.
    \end{ppart}
    \begin{ppart}
        Let $S$ be the set of all functions $f \in V$ such that $f(0) = 0$. Clearly, the zero function is in $S$. Let $g, h \in S$ and $c \in \RR$. Then \[g(0) + h(0) = 0 + 0 = 0,\] so $S$ is closed under addition, and \[cg(0) = c(0) = 0,\] so $S$ is closed under scalar multiplication. Thus, $S$ is a subspace of $V$.
    \end{ppart}
    \begin{ppart}
        Let $S$ be the set of all functions $f \in V$ such that $f(0) = 2$. The zero function is not in $S$, so $S$ is not a subspace of $V$.
    \end{ppart}
    \begin{ppart}
        Let $S$ be the set of all constant functions. Clearly, the zero function is in $S$. Let $g, h \in S$ such that $g(x) \equiv a$ and $h(x) \equiv b$, where $a$ and $b$ are constants. Let also $c \in \RR$. Then \[g(x) + h(x) \equiv a + b,\] which is a constant, so $S$ is closed under addition. Further, \[cg(x) \equiv ca,\] which is also a constant, so $S$ is closed under scalar multiplication. Thus, $S$ is a subspace of $V$.
    \end{ppart}
\end{solution}

\begin{problem}
    Find a basis for the vector space spanned by the vectors $\cveciiix12{-2}$, $\cveciiix213$, $\cveciiix1{-4}{12}$ and $\cveciiix3{-9}{29}$. What is the dimension of this space? For what value (or values) of $a$ does $\cveciiix43a$ belong to this space?
\end{problem}
\begin{ppart}
    Consider \[\mat A = \begin{pmatrix}1 & 2 & 1 & 3 \\ 2 & 1 & -4 & -9 \\ -2 & 3 & 12 & 29\end{pmatrix}.\] Its RREF is given by \[\begin{pmatrix}1 & 0 & -3 & -7 \\ 0 & 1 & 2 & 5 \\ 0 & 0 & 0 & 0\end{pmatrix},\] so the column space of $\mat A$, which is also the span of the given vectors, has basis \[\bc{\cveciii12{-2}, \cveciii213}\] and thus has dimension 2.

    Suppose $\cveciiix43a$ belongs to the column space of $\mat A$. Then there exist $x, y \in \RR$ such that \[x \cveciii12{-2} + y \cveciii213 = \cveciii43a.\] From the first two rows, we see that $x$ and $y$ satisfy \[\systeme{x + 2y = 4, 2x + y = 3},\] so $x = 2/3$ and $y = 5/3$. From the last row, we have \[a = -2x + 3y = \frac{11}3.\]
\end{ppart}

\begin{problem}
    Given that $\cveciiix{a}{b}{c}$ belongs to the row space of the matrix \[\begin{pmatrix}3 & 2 & 1 \\ -2 & -2 & 1 \\ 1 & -2 & 7\end{pmatrix},\] find a linear relation that must be satisfied by $a$, $b$ and $c$.
\end{problem}
\begin{solution}
    The given matrix has RREF \[\begin{pmatrix}1 & 0 & 2 \\ 0 & 1 & -5/2 \\ 0 & 0 & 0\end{pmatrix},\] so its row space has basis \[\bc{\cveciii102, \cveciii01{-5/2}}.\] Since $\cveciiix{a}{b}{c}$ is in the row space, there exist $x, y \in \RR$ such that \[\cveciii{a}{b}{c} = x \cveciii102 + y \cveciii01{-5/2}.\] From the first two rows, we see that $x = a$ and $y = b$. Thus, from the last row, \[c = 2a - \frac52 b.\]
\end{solution}

\begin{problem}
    Determine the rank of the matrix \[\mat A = \begin{pmatrix}1 & 1 & 2 & 3 \\ 3 & 2 & 3 & 13 \\ 4 & 4 & 9 & 7 \\ 11 & 9 & 17 & 36\end{pmatrix}.\] Deduce that if $\vec x$ is a solution of the equation \[\mat A \vec x = p \cveciv134{11} + q \cveciv1249 + r \cveciv239{17},\] where $p$, $q$ and $r$ are given real numbers, then \[\vec x = \cveciv{p - 2\l}{q-11\l}{r+5\l}{\l},\] where $\l \in \RR$.

    Hence, or otherwise, for solutions $\vec x = \cvecivx\a\b\g\de$ of the equation $\mat A \vec x = \cvecivx48{17}{37}$,
    \begin{enumerate}
        \item find $\vec x$ such that $\a = 0$,
        \item show that there is no $\vec x$ for which $\a^2 + \b^2 + \g^2 + \de^2 = 1$.
    \end{enumerate}
\end{problem}
\begin{solution}
    The RREF of $\mat A$ is \[\begin{pmatrix}1 & 0 & 0 & 2 \\ 0 & 1 & 0 & 11 \\ 0 & 0 & 1 & -5 \\ 0 & 0 & 0 & 0\end{pmatrix},\] so $\mat A$ has rank 3.

    Let $\vec x = \cvecivx{x}{y}{z}{w}$. Then the LHS of the equation can be expanded as \[x \cveciv134{11} + y \cveciv1249 + z \cveciv239{17} + w \cveciv3{13}{7}{36} = p \cveciv134{11} + q \cveciv1249 + r \cveciv239{17},\] which rearranges as \[\bp{x - p}\cveciv134{11} + \bp{y - q} \cveciv1249 + \bp{z - r} \cveciv239{17} + w \cveciv3{13}{7}{36} = \mat A \cveciv{x-p}{y-q}{z-r}{w} = \cveciv0000. \tag{$\ast$}\]
    
    We now find the null space of $\mat A$. Consider $\mat A \vec y = \vec 0$, where $\vec y = \cvecivx{a}{b}{c}{d}$. From the RREF of $\mat A$, we see that \[\begin{pmatrix}1 & 0 & 0 & 2 \\ 0 & 1 & 0 & 11 \\ 0 & 0 & 1 & -5 \\ 0 & 0 & 0 & 0\end{pmatrix} \cveciv{a}{b}{c}{d} = \cveciv0000 \implies \systeme{a + 2d = 0, b + 11d = 0, c - 5d = 0}.\] Let $d = \l \in \RR$. Then the general solution is \[\vec y = \cveciv{a}{b}{c}{d} = \l \cveciv{-2}{-11}{5}{1}.\]

    Going back to ($\ast$), we see that \[\cveciv{x-p}{y-q}{z-r}{w} = \l \cveciv{-2}{-11}{5}{1} \implies \vec x = \cveciv{p - 2\l}{q-11\l}{r+5\l}{\l}.\]

    \begin{ppart}
        Take $p = q = r = 1$. Then the solution to \[\mat A \vec x = \cveciv134{11} + \cveciv1249 + \cveciv239{17} = \cvecivx48{17}{37}\] has the form \[\vec x = \cveciv{1 - 2\l}{1-11\l}{1+5\l}{\l}.\] Since $\a = 0$, we take $\l = 1/2$ to obtain \[\vec x = \cveciv0{-9/2}{7/2}{1/2}.\]
    \end{ppart}
    \begin{ppart}
        Observe that \[\a^2 + \b^2 + \g^2 + \de^2 = \bp{1 - 2\l}^2 + \bp{1 - 11\l}^2 + \bp{1 + 5\l} + \l^2 = 151\l^2 - 16\l + 3.\] Suppose now that $\a^2 + \b^2 + \g^2 + \de^2 = 1$ has a solution. Then there exists $\l \in \RR$ such that $151\l^2 - 16\l + 2$. The discriminant of this quadratic is $(-16)^2 - 4(151)(2) < 0$, so $\l \notin \RR$, a contradiction. Thus, there does not exist $\vec x$ such that $\a^2 + \b^2 + \g^2 + \de^2 = 1$.
    \end{ppart}
\end{solution}

\begin{problem}
    The set $P$ of all quadratic polynomials in $x$ is a vector space over $\RR$. For each of the following subsets of $P$, determine whether it is a subspace, and if so, give a basis.

    \begin{enumerate}
        \item $S_1 = \bc{f \in P : f(0) = 0}$.
        \item $S_2 = \bc{f \in P : f(0) = 1}$.
        \item $S_3 = \bc{f \in P : f(1) = 0}$.
        \item $S_4 = \bc{f \in P : f(-x) = f(x) \, \forall x \in \RR}$.
    \end{enumerate}
\end{problem}
\begin{solution}
    Note that the zero element of $P$ is the zero polynomial $f(x) \equiv 0$.

    \begin{ppart}
        Clearly, the zero polynomial is in $S_1$. Let $g, h \in S_1$ and $c \in \RR$. Observe that \[g(0) + h(0) = 0 + 0 = 0,\] so $S_1$ is closed under addition. Also, \[cg(0) = c(0) = 0,\] so $S_1$ is closed under scalar multiplication. Thus, $S_1$ is a subspace of $P$.

        Let $f \in S_1$, where $f(x) = ax^2 + bx + c$. Since $f(0) = 0$, we must have $c = 0$, so $f(x) = ax^2 +bx$. Thus, a basis of $S_1$ is $\bc{x^2, x}$.
    \end{ppart}
    \begin{ppart}
        $S_2$ does not contain the zero polynomial, hence it is not a subspace of $P$.
    \end{ppart}
    \begin{ppart}
        Clearly, the zero polynomial is in $S_3$. Let $g, h \in S_3$ and $c \in \RR$. Observe that \[g(1) + h(1) = 0 + 0 = 0,\] so $S_3$ is closed under addition. Also, \[cg(1) = c(1) = 0,\] so $S_3$ is closed under scalar multiplication. Thus, $S_3$ is a subspace of $P$.

        Let $f \in S_3$, where $f(x) = ax^2 + bx + c$. Since $f(1) = 0$, we must have $a + b + c = 0$, so \[f(x) = ax^2 +bx - a - b = a\bp{x^2 - 1} + b\bp{x -1}.\] Thus, a basis of $S_3$ is $\bc{x^2 - 1, x - 1}$.
    \end{ppart}
    \begin{ppart}
        Clearly, the zero polynomial is in $S_4$. Let $g, h \in S_4$ and $c \in \RR$. Observe that \[g(x) + h(x) = g(-x) + h(-x),\] so $S_4$ is closed under addition. Also, \[cg(-x) = cg(x),\] so $S_4$ is closed under scalar multiplication. Thus, $S_4$ is a subspace of $P$.

        Let $f \in S_4$, where $f(x) = ax^2 + bx + c$. Since $f(-x) = f(x)$, we must have $b = 0$, so $f(x) = ax^2 + c$. Thus, a basis of $S_4$ is $\bc{x^2, 1}$.
    \end{ppart}
\end{solution}

\clearpage
\begin{problem}
    The linear transformation $T : \RR^4 \to \RR^4$ is represented by the matrix $\mat A$ where \[\mat A = \begin{pmatrix}1 & -2 & 0 & 4 \\ 3 & 1 & 1 & 0 \\ -1 & -5 & -1 & 8 \\ 3 & 8 & 2 & -12\end{pmatrix}.\]
    
    \begin{enumerate}
        \item Find the rank of $\mat A$ and a basis for the range space of $T$.
        \item Show that the vector $\vec b = \cvecivx453{-2}$ belongs to the range space of $T$.
        \item Find a basis for the null space of $T$. Hence, find the general solution of $\mat A \vec x = \vec b$.
    \end{enumerate}
\end{problem}
\begin{solution}
    \begin{ppart}
        The RREF of $\mat A$ is given by \[\begin{pmatrix}1 & 0 & 2/7 & 4/7 \\ 0 & 1 & 1/7 & -12/7 \\ 0 & 0 & 0 & 0 \\ 0 & 0 & 0 & 0\end{pmatrix}.\] Thus, $\mat A$ has rank 2, and a basis for the range space of $T$ is given by \[\bc{\cveciv13{-1}3, \cveciv{-2}1{-5}8}.\]
    \end{ppart}
    \begin{ppart}
        Consider \[x \cveciv12{-1}3 + y \cveciv{-2}{1}{-5}8 = \cveciv453{-2}.\] Since there exists a solution ($x = 2$, $y = -1$), it follows that $\cvecivx453{-2}$ belongs to the range space of $T$.
    \end{ppart}
    \begin{ppart}
        Consider $\mat A \vec x = \vec 0$, where $\vec x = \cveciv{x}{y}{z}{w}$. From the RREF of $\mat A$, we have \[\systeme[xyzw]{x + \frac27z + \frac47w = 0, y + \frac17z - \frac{12}{7} w = 0}.\] Let $z = 7\l$ and $w = 7\m$, where $\l, \m \in \RR$. Then \[\vec x = \cveciv{x}{y}{z}{w} = \l \cveciv{-2}{-1}70 + \m \cveciv{-4}{12}07,\] so a basis for the null space of $T$ is given by \[\bc{\cveciv{-2}170, \cveciv4{12}07}.\]
        
        From (b), we know a particular solution to $\mat A \vec x = \vec b$: \[\cveciv453{-2} = 2 \cveciv12{-1}3 + (-1) \cveciv{-2}{1}{-5}8 + (0)\cveciv01{-1}2 + (0)\cveciv408{-12} = \mat A \cveciv2{-1}00.\] Thus, the general solution of $\mat A \vec x = \vec b$ is thus \[\vec x = \cveciv2{-1}00 + \l \cveciv{-2}{-1}70 + \m \cveciv{-4}{12}07.\]
    \end{ppart}
\end{solution}

\begin{problem}
    The linear transformation $L : \RR^4 \to \RR^3$ is represented by the matrix $\mat A$, where \[\mat A = \begin{pmatrix}1 & 2 & 0 & 1 \\ 2 & -1 & 2 & -1 \\ 1 & -3 & 2 & -2\end{pmatrix}.\]

    \begin{enumerate}
        \item Find the rank of $\mat A$ and deduce that the dimension of the null space, $N$, of $L$ is 2.
        \item Show that there is a basis $\bc{\vec e_1, \vec e_2}$ for $N$ such that \[\vec e_1 = \cveciv{p}2q0 \quad \tand \quad \vec e_2 = \cveciv{r}0s2,\] where $p$, $q$, $r$ and $s$ are integers to be found.
        \item Given that $\vec e_0 = \cvecivx1111$, $\vec b = \cveciiix42{-2}$, show that the solution set, $W$, of the equation $\mat A \vec x = \vec b$ is \[W = \bc{\vec e_0 + \l \vec e_1 + \m \vec e_2 \mid \l, \m \in \RR}.\]
    \end{enumerate}

    The set of elements of $\RR^4$ which do not belong to $W$ is denoted by $V$.

    \begin{enumerate}
        \setcounter{enumi}{3}
        \item Show that $N$ is a subset of $V$.
        \item Write down the vectors $\vec e_3$ and $\vec e_4$ such that the set $\bc{\vec e_1, \vec e_2, \vec e_3, \vec e_4}$ forms a basis for $\RR^4$, justifying your answer.
    \end{enumerate}
\end{problem}
\begin{solution}
    \begin{ppart}
        The RREF of $\mat A$ is \[\begin{pmatrix}1 & 0 & 4/5 & -1/5 \\ 0 & 1 & -2/5 & 3/5 \\ 0 & 0 & 0 & 0\end{pmatrix}.\] Hence, the rank of $\mat A$ is 2, so by the rank-nullity theorem, the dimension of $N$ is $4 - 2 = 2$.
    \end{ppart}
    \begin{ppart}
        Consider $\mat A \vec x = \vec 0$, where $\vec x = \cvecivx{x}{y}{z}{w}$. From the RREF, we have \[\systeme[xyzw]{x + \frac45 z - \frac15 w = 0, y - \frac25 z + \frac35 w = 0}.\] Let $z = 5\l$ and $w = 5\m$, where $\l, \m \in \RR$. Then, \[\vec x = \l \cveciv{-4}250 + \m \cveciv1{-3}05.\] Clearly, $\vec e_1 = \cvecivx{-4}250$, so $p = -4$ and $q = 5$. Now consider \[\cveciv{r}052 = a \cveciv{-4}250 + b\cveciv1{-3}05.\] From the first and third rows, we have $5b = 2$ and $2a - 3b = 0$, so $a = 3/5$ and $b = 2/5$. Thus, \[\vec e_2 = \cveciv{r}052 = \frac35 \cveciv{-4}250 + \frac25 \cveciv1{-3}05 = \cveciv{-2}032,\] so $r = -2$ and $s = 3$. The basis for $N$ is \[\bc{\cveciv{-4}250, \cveciv{-2}032}.\]
    \end{ppart}
    \begin{ppart}
        Note that \[\mat A \vec e_0 = \begin{pmatrix}1 & 2 & 0 & 1 \\ 2 & -1 & 2 & -1 \\ 1 & -3 & 2 & -2\end{pmatrix} \cveciv1111 = \cveciii42{-2} = \vec b.\] Let $\vec x \in W$. Then \[\mat A \vec x = \vec b = \mat A \vec e_0 \implies \mat A \bp{\vec x - \vec e_0} = \vec 0,\] so $\vec x - \vec e_0 \in N$, whence $\vec x = \vec e_0 + \l \vec e_1 + \m \vec e_2$.
    \end{ppart}
    \begin{ppart}
        Suppose $\vec v \in N$. Then $\mat A \vec v = \vec 0 \neq \vec b$, so $\vec v \notin W$, whence $\vec v \in V$. Thus, $N \subseteq V$.
    \end{ppart}
    \begin{ppart}
        Note that the RREF of the matrix below is $\mat I$: \[\begin{pmatrix}-4 & -2 & 1 & 0 \\ 2 & 0 & 0 & 1 \\ 5 & 3 & 0 & 0 \\ 0 & 2 & 0 & 0\end{pmatrix} \rightarrow \begin{pmatrix}1 & 0 & 0 & 0 \\ 0 & 1 & 0 & 0 \\ 0 & 0 & 1 & 0 \\ 0 & 0 & 0 & 1\end{pmatrix}.\] Thus, a basis for $\RR^4$ is \[\bc{\cveciv{-4}250, \cveciv{-2}032, \cveciv1000, \cveciv0100},\] hence $\vec e_3 = \cvecivx1000$ and $\vec e_4 = \cvecivx0100$.
    \end{ppart}
\end{solution}

\begin{problem}
    The linear transformations $T_1 : \RR^4 \to \RR^4$ and $T_2 : \RR^4 \to \RR^4$ are represented by the matrices $\mat M_1$ and $\mat M_2$ respectively, where \[\mat M_1 = \begin{pmatrix}1 & 2 & 1 & 2 \\ 0 & \t & 3 & 3 \\ 3 & 1 & 10 & 4 \\ 0 & 1 & -19 & -10 \end{pmatrix} \quad \tand \quad \mat M_2 = \begin{pmatrix}2 & 1 & 3 & 2 \\ 1 & 0 & 2 & 3 \\ 1 & 3 & 14 & 16 \\ 1 & 0 & -1 & -2\end{pmatrix}.\] The null spaces of $T_1$ and $T_2$ are denoted by $K_1$ and $K_2$ respectively. The range spaces of $T_1$ and $T_2$ are denoted by $R_1$ and $R_2$ respectively.

    \begin{enumerate}
        \item Determine the value of $\t$ given that the dimension of $K_1$ is 1. Find a basis of $R_1$.
        \item Write down a basis of $K_2$ and a basis of $R_2$.
        \item Prove that $R_1 \cap R_2$ is a vector space. Show that the dimension of $R_1 \cap R_2$ is 2.
        \item Without evaluating $\mat M_1 \mat M_2$, find a vector in the null space the transformation $T_3 : \RR^4 \to \RR^4$ represented by $\mat M_1 \mat M_2$.
    \end{enumerate}
\end{problem}
\begin{solution}
    \begin{ppart}
        Using G.C., $\mat M_1$ can be reduced to \[\mat M_1 \rightarrow \begin{pmatrix}1 & 0 & 0 & -\frac{23}{22} \\[0.5em] 0 & 1 & 0 & \frac{27}{22} \\[0.5em] 0 & 0 & 1 & \frac{13}{22} \\[0.5em] 0 & \t & 3 & 3\end{pmatrix} \rightarrow \begin{pmatrix}1 & 0 & 0 & -\frac{23}{22} \\[0.5em] 0 & 1 & 0 & \frac{27}{22} \\[0.5em] 0 & 0 & 1 & \frac{13}{22} \\[0.5em] 0 & 0 & 0 & 3 - \t \bp{\frac{27}{22}} - 3 \bp{\frac{13}{22}}\end{pmatrix}.\] Since $\Dim K_1 = 1$, we must have exactly one row of zeroes, so \[3 - \t \bp{\frac{27}{22}} - 3 \bp{\frac{13}{22}} \implies \t = 1.\] A basis for $R_1$ is \[\bc{\cveciv1030, \cveciv2111, \cveciv13{10}{-19}}.\]
    \end{ppart}
    \begin{ppart}
        The RREF of $\mat M_2$ is given by \[\begin{pmatrix}1 & 0 & 0 & -\frac13 \\[0.5em] 0 & 1 & 0 & -\frac73 \\[0.5em] 0 & 0 & 1 & \frac53 \\[0.5em] 0 & 0 & 0 & 0\end{pmatrix}.\] A basis for $R_2$ is \[\bc{\cveciv2111, \cveciv1030, \cveciv32{14}{-1}}.\]

        Consider $\mat M_2 \vec x = \vec 0$, where $\vec x = \cvecivx{x}{y}{z}{w}$. Then \[\begin{pmatrix}1 & 0 & 0 & -\frac13 \\[0.5em] 0 & 1 & 0 & -\frac73 \\[0.5em] 0 & 0 & 1 & \frac53 \\[0.5em] 0 & 0 & 0 & 0\end{pmatrix} \cveciv{x}{y}{z}{w} = \cveciv0000 \implies \systeme[xyzw]{x - \frac13 w = 0, y - \frac73 w = 0, z + \frac53 w = 0}.\] Let $w = 3\l \in \RR$. Then \[\vec x = \cveciv{x}{y}{z}{w} = \l \cveciv17{-5}3,\] so a basis of $K_2$ is \[\bc{\cveciv17{-5}3}.\]
    \end{ppart}
    \begin{ppart}
        Let $v \in R_1 \cap R_2$. Then there exist constants $a, b, c, d, e, f \in \RR$ such that \[a \cveciv1030 + b \cveciv2111 + c\cveciv13{10}{-19} = d \cveciv1030 + e\cveciv2111 + f\cveciv32{14}{-1}.\] Rearranging, we have \[\begin{pmatrix}1 & 2 & 1 & 3 \\ 0 & 1 & 3 & 2 \\ 3 & 1 & 10 & 14 \\ 0 & 1 & -19 &-1\end{pmatrix} \cveciv{a-d}{b-e}{c}{-f} = \cveciv0000.\] Since the matrix on the LHS is invertible, we must have $a = d$, $b = e$ and $c = f = 0$. Hence, $\vec v$ is of the form \[\vec v = a \cveciv1030 + b \cveciv2111.\] Thus, \[R_1 \cap R_2 = \Span{\cveciv1030, \cveciv2111}\] and is thus a vector space with dimension 2.
    \end{ppart}
    \begin{ppart}
        Picking $\vec x = \cvecivx17{-5}3$, which is in $K_2$, we have \[\mat M_1 \mat M_2 \vec x = \mat M_1 \vec 0 = \vec 0\] as desired.
    \end{ppart}
\end{solution}

\clearpage
\begin{problem}
    \begin{enumerate}
        \item Let $L : V \to V$ be a linear transformation on the vector space $V$. A linear transformation is said to be \emph{nilpotent} if there exists $k \in \ZZ^+$ such that $L^k(x) = 0$ for all $x \in V$. A linear transformation $L$ is said to be \emph{invertible} if there exists a transformation $T$ such that $LT(x) = TL(x) = x$ for all $x \in V$. Show that if $L$ is nilpotent, then the transformation $M = I - L$ is invertible by finding an explicit formula for $(I - L)^{-1}$, where $I$ is the identity transformation.
        \item Let $V$ be the space of quadratic polynomials and $L$ the differential transformation, that is, \[L\of{ax^2 + bx + c} = 2ax + b.\] Show that $L$ is a linear transformation on $V$.
        \item Using (a) and (b), find a particular solution $y_p \notin V$ of $y' - y = 5x^2 - 3$.
        \item Hence, find the general solution of the differential equation in (c).
    \end{enumerate}
\end{problem}
\begin{solution}
    \begin{ppart}
        A simple series expansion gives \[(I - L)^{-1} = I + L + L^2 + \dots + L^{k-1} + L^k + \dots = I + L + L^2 + \dots + L^{k-1},\] so $M$ is invertible.
    \end{ppart}
    \begin{ppart}
        Let $f, g$ be quadratic polynomials, where $f(x) = a_1 x^2 + b_1x + c_1$ and $g(x) = a_2 x^2 + b_2 x + c_2$. Then
        \begin{gather*}
            L(f + g) = L\of{\bp{a_1 + a_2} x^2 + \bp{a_2 + b_2} x + \bp{c_1 + c_2}} = 2\bp{a_1 + a_2}x + \bp{b_1 + b_2}\\
            = \bp{2a_1x + b_1} + \bp{2a_2x + b_2} = L(f) + L(g).
        \end{gather*}
        Let $k \in \RR$. Then \[L(kf) = L\of{ka_1 x^2 + kb_1 x + kc_1} = 2ka_1 x + kb_1 = k\bp{2a_1 x + b_1} = k L(f).\] Thus, $L$ is a linear transformation on $V$.
    \end{ppart}
    \begin{ppart}
        Rewriting the given differential equation, we see that \[y'_p - y_p = (L-I) y_p = 5x^2 - 3 \implies (I-L) y_p = -5x^2 + 3.\] Since $L^3 = 0$, by (a), we have the particular solution
        \begin{gather*}
            y_p = (I-L)^{-1} \bp{-5x^2 + 3} = \bp{I + L + L^2} \bp{-5x^2 + 3}\\
            = \bp{-5x^2 + 3} + \bp{-10x} + \bp{-10} = -5x^2 -10x - 7.
        \end{gather*}
    \end{ppart}
    \begin{ppart}
        The associated homogeneous differential equation is $y' = y$, so the complementary solution is $y_c = Ae^x$. Thus, the general solution is given by \[y = y_c + y_p = Ae^x - 5x^2 - 10 x - 7.\] 
    \end{ppart}
\end{solution}