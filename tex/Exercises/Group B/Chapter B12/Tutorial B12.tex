\section{Tutorial B12}

\begin{problem}
    Given that $y = 1$ when $x = 1$, find the particular solution of the differential equation \[\der{y}{x} = \frac{y^2}{x}.\]
\end{problem}
\begin{solution}
    We have \[\frac1{y^2} \der{y}{x} = \frac1x.\] Integrating both sides with respect to $x$, \[\int \frac1{y^2} \d y = \int \frac1x \d x.\] Thus, \[-\frac1{y} = \ln \abs{x} + C_1.\] Solving for $y$, we get \[y = \frac1{C - \ln \abs{x}}.\] Evaluating at $x = 1$ and $y = 1$, we get $C = 1$, so \[y = \frac1{1 - \ln \abs{x}}.\]
\end{solution}

\begin{problem}
    Two variables $x$ and $t$ are connected by the differential equation \[\der{x}{t} = \frac{kx}{10-x},\] where $0 < x < 10$ and where $k$ is a constant. It is given that $x = 1$ when $t = 0$ and that $x = 2$ when $t = 1$. Find the value of $t$ when $x = 5$, given your answer to three significant figures.
\end{problem}
\begin{solution}
    We have \[\frac{10-x}{x} \der{x}{t} = k.\] Integrating with respect to $t$, we have \[\int \frac{10-x}{x} \d x = \int k \d t,\] so $10\ln x - x = kt + C$. Evaluating at $x = 1$ and $t = 0$, we get $C = -1$. Evaluating at $x = 2$ and $t = 1$, we get $k = 10 \ln 2 - 1$. Hence, evaluating at $x = 5$, we get $10\ln{5} - 5 = \bp{10 \ln 2 - 1}t - 1$, so $t = 2.04 \tosf{3}$.
\end{solution}

\begin{problem}
    Use the substitution $y = u - 2x$ to find the general solution of the differential equation \[\der{y}{x} = -\frac{8x + 4y + 1}{4x+2y+1}.\]
\end{problem}
\begin{solution}
    Note that \[\der{y}{x} = -\frac{8x + 4y + 1}{4x+2y+1} = -2 + \frac1{4x + 2y + 1}.\] Also note that under the substitution $y = u - 2x$, we have \[\der{y}{x} = \der{u}{x} - 2.\] Thus, \[\der{u}{x} = \frac1{4x + 2y + 1} = \frac1{4x + 2(u - 2x) + 1} = \frac1{2u + 1}.\] Rearranging, \[(2u + 1) \der{u}{x} = 1.\] Integrating with respect to $x$, \[\int \bp{2u + 1} \d u = \int 1 \d x.\] Thus, $u^2 + u = x + C$. Substituting $u = y + 2x$ back, we get the general solution $(y+2x)^2 + y + x = C$.
\end{solution}

\begin{problem}
    By using the substitution $z = y\e^{2x}$, find the general solution of the differential equation \[\der{y}{x} + 2y = x\e^{-2x}.\]

    Find the particular solution of the differential equation given that $\derx{y}{x} = 1$ when $x = 0$.
\end{problem}
\begin{solution}
    Differentiating $z = y\e^{2x}$, we get \[\der{z}{x} = \der{y}{x} \e^{2x} + 2y \e^{2x} = \der{y}{x} \e^{2x} + 2z.\] Multiplying through by $\e^{-2x}$, we see that \[\der{z}{x} \e^{-2x} = \der{y}{x} + 2y.\] Substituting this into the given differential equation and simplifying, we get \[\der{z}{x} = x.\] Integrating with respect to $x$, we see that \[y \e^{2x} = z = \frac{x^2}{2} + C,\] so \[y = \e^{-2x} \bp{\frac{x^2}{2} + C}.\] When $x = 0$, the given differential equation yields $y(0) = -1/2$, so $C = -1/2$. The particular solution is hence \[y = \frac{x^2 - 1}{2\e^{2x}}.\]
\end{solution}

\begin{problem}
    Find the general solution of the differential equation \[\der{y}{x} = 6xy^3.\]

    Find its particular solution given that $y = 0.5$ when $x = 0$.

    Determine the interval of validity for the particular solution.
\end{problem}
\begin{solution}
    We have \[\frac{1}{y^3} \der{y}{x} = 6x.\] Integrating both sides with respect to $x$, we get \[\int \frac{1}{y^3} \d y = \int 6x \d x,\] so \[-\frac{1}{2y^2} = 3x^2 + C_1\] and thus \[y^2 = \frac{1}{C - 6x^2}.\]

    Since $y(0) = 0.5$, we have $C = 4$. Thus, the particular solution is \[y^2 = \frac1{4 - 6x^2}.\]
 
    For the solution to be valid, we require $4 - 6x^2 > 0$, whence $-\sqrt{2/3} < x < \sqrt{2/3}$.
\end{solution}

\begin{problem}
    \begin{enumerate}
        \item Find the general solution of the differential equation \[\der{y}{x} = \frac{3x}{x^2 + 1}.\]
        \item What can you say about the gradient of every solution as $x \to \pm \infty$?
        \item Find the particular solution of the differential equation for which $y = 2$ when $x = 0$. Hence, sketch the graph of this solution.
    \end{enumerate}
\end{problem}
\begin{solution}
    \begin{ppart}
        We have \[\der{y}{x} = \frac{3x}{x^2 + 1} = \frac32 \bp{\frac{2x}{x^2 + 1}},\] so \[y = \frac32 \int \frac{2x}{x^2 + 1} \d x = \frac32 \ln{x^2 + 1} + C.\]
    \end{ppart}
    \begin{ppart}
        \[\lim_{x \to \pm \infty} \der{y}{x} = \lim_{x \to \pm \infty} \frac{3x}{x^2 + 1} = 0.\]
    \end{ppart}
    \begin{ppart}
        Evaluating the general solution at $x = 0$ and $y = 2$, we get $C = 2$. Thus, the particular solution is \[y = \frac32 \ln{x^2 + 1} + 2.\]

        \begin{center}\tikzsetnextfilename{211}
            \begin{tikzpicture}[trim axis left, trim axis right]
                \begin{axis}[
                    domain = -20:20,
                    samples = 101,
                    axis y line=middle,
                    axis x line=middle,
                    xtick = \empty,
                    ytick = {2},
                    xlabel = {$x$},
                    ylabel = {$y$},
                    ymin=0,
                    legend cell align={left},
                    legend pos=outer north east,
                    after end axis/.code={
                        \path (axis cs:0,0) 
                            node [anchor=north] {$O$};
                        }
                    ]
                    \addplot[plotRed] {1.5 * ln(x^2 + 1) + 2};
        
                    \addlegendentry{$y = \frac32 \ln{x^2 + 1} + 2$};
                \end{axis}
            \end{tikzpicture}
        \end{center}
    \end{ppart}
\end{solution}

\begin{problem}
    The variables $x$, $y$ and $z$ are connected by the following differential equations.
    \begin{align*}
        \der{z}{x} &= 3-2z \tag{$\ast$}\\
        \der{y}{x} &= z
    \end{align*}
    \begin{enumerate}
        \item Given that $z < 3/2$, solve equation ($\ast$) to find $z$ in terms of $x$.
        \item Hence, find $y$ in terms of $x$.
        \item Use the result in part (b) to show that \[\der[2]{y}{x} = a\der{y}{x} + b\] for constants $a$ and $b$ to be determined.
        \item The curve of the solution in part (b) passes through the points $(0, 1)$ and $\bp{2, 3+\e^{-4}}$. Sketch this curve, indicating its axial intercept and asymptote (if any).
    \end{enumerate}
\end{problem}
\begin{solution}
    \begin{ppart}
        We have \[\frac{1}{3-2z} \der{z}{x} = 1.\] Integrating both sides with respect to $x$, we get \[\int \frac{1}{3-2z} \d z = \int 1 \d x,\] so \[-\frac12 \ln{3-2z} = x + C_1.\] It follows that \[z = \frac32 - A \e^{-2x}.\] Thus, the general solution is \[z = \frac32 - A \e^{-2x}.\]
    \end{ppart}
    \begin{ppart}
        Since \[\der{y}{x} = \frac32 - A \e^{-2x},\] we have \[y = \int \bp{\frac32 - A \e^{-2x}} \d x = \frac32 x + \frac{A}2 \e^{-2x} + B.\]
    \end{ppart}
    \begin{ppart}
        We have \[\der{y}{x} = \frac32 - A\e^{-2x}.\] Differentiating this result with respect to $x$, we get \[\der[2]{y}{x} = 2A \e^{-2x} = 2\bp{\frac32 - \der{y}{x}} = -2\der{y}{x} + 3.\] Hence, $a = -2$ and $b = 3$.
    \end{ppart}
    \begin{ppart}
        Evaluating the general solution at $\bp{2, 3 + \e^{-4}}$, we obtain $A = 2$. Evaluating the general solution at $(0, 1)$, we obtain $B = 1 - A/2 = 0$.  The curve thus has equation \[y = \frac32 x + \e^{-2x}.\]

        \begin{center}\tikzsetnextfilename{212}
            \begin{tikzpicture}[trim axis left, trim axis right]
                \begin{axis}[
                    domain = -0.5:1.4,
                    samples = 101,
                    axis y line=middle,
                    axis x line=middle,
                    xtick = \empty,
                    ytick = {1},
                    xlabel = {$x$},
                    ylabel = {$y$},
                    ymin=0,
                    legend cell align={left},
                    legend pos=outer north east,
                    after end axis/.code={
                        \path (axis cs:0,0) 
                            node [anchor=north] {$O$};
                        }
                    ]
                    \addplot[plotRed, name path=f1, unbounded coords = jump] {1.5 * x + e^(-2*x)};
        
                    \addlegendentry{$y = \frac32 x + \e^{-2x}$};

                    \addplot[dotted] {1.5 * x};
                    \node[rotate=50, below] at (1, 1.5) {$y = \frac32 x$};
                \end{axis}
            \end{tikzpicture}
        \end{center}
    \end{ppart}
\end{solution}

\begin{problem}
    A bottle containing liquid is taken from a refrigerator and placed in a room where the temperature is a constant 20\,$\deg$C. As the liquid warms up, the rate of increase of its temperature $\t$\,$\deg$C after time $t$ minutes is proportional to the temperature difference $(20-\t)$\,$\deg$C. Initially the temperature of the liquid is $10$\,$\deg$C and the rate of increase of the temperature is 1\,$\deg$C per minute. By setting up and solving a differential equation, show that $\t = 20 - 10\e^{-t/10}$.

    Find the time it takes the liquid to reach a temperature of $15$\,$\deg$C, and state what happens to $\t$ for large values of $t$. Sketch a graph of $\t$ against $t$.
\end{problem}
\begin{solution}
    Since $\derx{\t}{t} \propto (20 - \t)$, we have $\derx{\t}{t} = k(20 - \t)$, where $k$ is a constant. We now solve for $\t$. Note that \[\frac1{20 - \t} \der{\t}{t} = k.\] Integrating both sides with respect to $t$, we get \[\int \frac1{20 - \t} \d \t = \int k \d t,\] so $-\ln{20 - \t} = kt + C_1$. Thus, $\t = 20 - C\e^{-kt}$.

    Evaluating at $\t = 0$ and $\t = 10$, we get $C = 10$. Additionally, since $\derx{\t}{t} = 1$ when $t = 0$, we have $k = 1/10$. Thus, $\t = 20 - 10\e^{-t/10}$.

    Using G.C., when $\t = 15$, we have $t = 6.93$. Thus, it takes 6.93 minutes for the liquid to reach a temperature of 15$\deg$C. As $t$ tends to infinity, $\t$ tends towards 20.

    \begin{center}\tikzsetnextfilename{213}
        \begin{tikzpicture}[trim axis left, trim axis right]
            \begin{axis}[
                domain = 0:50,
                samples = 101,
                axis y line=middle,
                axis x line=middle,
                xtick = \empty,
                ytick = {10, 20},
                xlabel = {$t$},
                ylabel = {$\t$},
                ymax=25,
                ymin=0,
                legend cell align={left},
                legend pos=outer north east,
                after end axis/.code={
                    \path (axis cs:0,0) 
                        node [anchor=north east] {$O$};
                    }
                ]
                \addplot[plotRed, name path=f1, unbounded coords = jump] {20 - 10*e^(-0.1 * x)};
    
                \addlegendentry{$\t = 20 - 10\e^{-t/10}$};

                \addplot[dotted]{20};
            \end{axis}
        \end{tikzpicture}
    \end{center}
\end{solution}

\begin{problem}
    \begin{enumerate}
        \item Find \[\int \frac1{100-v^2} \d x.\]
        \item A stone is dropped from a stationary balloon. It leaves the balloon with zero speed, and $t$ seconds later its speed $v$ metres per second satisfies the differential equation \[\der{v}{t} = 10 - 0.1v^2.\]
        \begin{enumerate}
            \item Find $t$ in terms of $v$. Hence, find the exact time the stone takes to reach a speed of 5 metres per second.
            \item Find the speed of the stone after 1 second.
            \item What happens to the speed of the stone for large values of $t$?
        \end{enumerate}
    \end{enumerate}
\end{problem}
\begin{solution}
    \begin{ppart}
        \[\int \frac1{100 - v^2} \d v = \frac{1}{2(10)} \ln{\frac{10 + v}{10 - v}} + C = \frac{1}{20} \ln{\frac{10 + v}{10 - v}} + C.\]
    \end{ppart}
    \clearpage
    \begin{ppart}
        \begin{psubpart}
            We have \[\frac1{100 - v^2} \der{v}{t} = \frac1{10}.\] Integrating with respect to $t$, we get \[\int \frac1{100 - v^2} \d v = \int \frac1{10} \d t.\] Thus, \[\frac1{20} \ln{\frac{10 + v}{10 - v}} + C_1 = \frac{t}{10},\] from which we find that \[t = \frac1{2} \ln{\frac{10 + v}{10 - v}} + C.\]

            Evaluating the solution at $t = 0$ and $v = 0$, we get $C = 0$. Thus, the general solution is \[t = \frac12 \ln{\frac{10 + v}{10 - v}}.\]

            Consider $v = 5$, we have \[t = \frac12 \ln{\frac{10+5}{10 - 5}} = \frac12 \ln 3.\] It thus takes $\ln{3}/2$ seconds for the stone to reach a speed of 5 m/s.
        \end{psubpart}
        \begin{psubpart}
            Consider $t = 1$. Using G.C., we get $v = 7.62$. Thus, after 1 second, the stone has a speed of $7.62$ m/s.
        \end{psubpart}
        \begin{psubpart}
            As $t \to \infty$, we have \[\ln{\frac{10 + v}{10 - v}} \to \infty \implies \frac{10 + v}{10 - v} \to \infty.\] Thus, $v \to 10$. Hence, for large values of $t$, the speed of the stone approaches 10 m/s.
        \end{psubpart}
    \end{ppart}
\end{solution}

\begin{problem}
    Two scientists are investigating the change of a certain population of an animal species of size $n$ thousand at time $t$ years. It is known that due to its inability to reproduce effectively, the species is unable to replace itself in the long run.
    \begin{enumerate}
        \item One scientist suggests that $n$ and $t$ are related by the differential equation \[\der[2]{n}{t} = 10 - 6t.\] Given that $n = 100$ when $t = 0$, show that the general solution of this differential equation is $n = 5t^2 - t^3 + Ct + 100$, where $C$ is a constant. Sketch the solution curve of the particular solution when $C = 0$, stating the axial intercepts clearly.
        \item The other scientist suggests that $n$ and $t$ are related by the differential equation \[\der{n}{t} = 3 - 0.02n.\] Find $n$ in terms of $t$, given again that $n = 100$ when $t = 0$. Explain in simple terms what will eventually happen to the population using this model.
    \end{enumerate}
    Which is a more appropriate model in modelling the population of the animal species?
\end{problem}
\begin{solution}
    \begin{ppart}
        Integrating the given differential equation, we have \[\der{n}{t} = \int (10 - 6t) \d \t = 10t - 3t^2 + C.\] Integrating once more, we get \[n = \int \bp{10t - 3t^2 + C} \d t = 5t^2 - t^3 + C t + D\]

        Evaluating the solution at $t = 0$ and $n = 100$, we obtain $D = 100$. Thus, $n = 5t^2 - t^3 + Ct + 100$. Hence, when $C = 0$, $n = 5t^2 - t^3 + 100$.

        \begin{center}\tikzsetnextfilename{214}
            \begin{tikzpicture}[trim axis left, trim axis right]
                \begin{axis}[
                    domain = 0:7.03,
                    samples = 101,
                    axis y line=middle,
                    axis x line=middle,
                    xtick = {7.03},
                    ytick = {100},
                    xlabel = {$n$},
                    ylabel = {$t$},
                    ymax = 150,
                    xmax=7.1,
                    legend cell align={left},
                    legend pos=outer north east,
                    after end axis/.code={
                        \path (axis cs:0,0) 
                            node [anchor=north east] {$O$};
                        }
                    ]
                    \addplot[plotRed, name path=f1, unbounded coords = jump] {5*x^2 - x^3 + 100};
        
                    \addlegendentry{$n = 5t^2 - t^3 + 100$};
                \end{axis}
            \end{tikzpicture}
        \end{center}
    \end{ppart}
    \begin{ppart}
        We have \[\frac{1}{150-n} \der{n}{t} = \frac1{50}.\] Integrating, we obtain \[\int \frac{1}{150-n} \d n = \int \frac1{50} \d t,\] so $-\ln{150 - n} = \frac1{50} t + C_1$, whence $n = 150 - C\e^{-t/50}$.

        When $t = 0$ and $n = 100$, we have $C = 50$. Hence, $n = 150 - 50\e^{-t/50}$. As $t \to \infty$, $n \to 150$. Hence, the population will decrease before plateauing at 150 thousand.
    \end{ppart}
    
    The first model is more appropriate, as it account for the fact that the species will eventually go extinct ($n = 0$) due to the fact that they cannot replace itself in the long run.
\end{solution}

\begin{problem}
    A rectangular tank has a horizontal base. Water is flowing into the tank at a constant rate, and flows out at a rate which is proportional to the depth of water in the tank. At time $t$ seconds, the depth of water in the tank is $x$ metres. If the depth is $0.5$ m, it remains at this constant value. Show that \[\der{x}{t} = -k(2x-1),\] where $k$ is a positive constant. When $t = 0$, the depth of water in the tank is $0.75$ m and is decreasing at a rate of $0.01$ m s$^{-1}$. Find the time at which the depth of water is $0.55$ m.
\end{problem}
\begin{solution}
    Let $V_i$ m$^3$/s be the rate at which water is flowing into the tank. Note that $V_i \geq 0$. Let the rate at which water is flowing out of the tank be $V_o x$ m$^3$/s. Let the base of the container be $A$ m$^2$. Then \[\der{x}{t} = \frac{V_i - V_o x}{A}.\] At $x = 0.5$, the volume of water in the tank is constant, i.e. $\derx{x}{t} = 0$. Thus, $V_i - 0.5 V_o = 0$, so \[\der{x}{t} = -\frac{V_i (2x - 1)}{A}.\] Letting $k = V_i/A$, we have \[\der{x}{t} = -k(2x - 1).\]

    We now solve for $t$. We have \[\frac{1}{2x-1} \der{x}{t} = -k.\] Integrating both sides with respect to $t$, we get \[\int \frac{1}{2x-1} \d x = \int -k \d t.\] Thus, \[\frac{\ln{2x-1}}2 + C_1 = -kt,\] whence \[t = -\frac{\ln{2x-1} + C}{2k}.\]

    Evaluating the solution at $t = 0$ and $x = 0.75$, we get $C_2 = \ln 2$. Additionally, since $\derx{x}{t} = -0.01$ when $t = 0$, we get $k = 0.02$. Thus, \[t = -\frac{\ln{2x-1} + \ln 2}{2(0.02)} = -25 \ln{4x-2}.\] 
    
    Consider $x = 0.55$. Then $t = -25\ln{4(0.55) - 2} = 25\ln5$.
\end{solution}

\begin{problem}
    In a model of mortgage repayment, the sum of money owned to the Building Society is denoted by $x$ and the time is denoted by $t$. Both $x$ and $t$ are taken to be continuous variables. The sum of money owned to the Building Society increases, due to interest, at a rate proportional to the sum of money owed. Money is also repaid at a constant rate $r$.

    When $x = a$, interest and repayment balance. Show that, for $x > 0$, \[\der{x}{t} = \frac{r}{a} (x - a).\]

    Given that, when $t = 0$, $x = A$, find $x$ in terms of $t$, $r$, $a$ and $A$.

    On a single, clearly labelled sketch, show the graph of $x$ against $t$ in the two cases:
    \begin{enumerate}
        \item $A > a$.
        \item $A < a$.
    \end{enumerate}

    State the circumstances under which the loan is repaid in a finite time $T$ and show that, in this case, \[T = \frac{a}{r} \ln \frac{a}{a - A}.\]
\end{problem}
\begin{solution}
    Let the rate at which money is owned to the Building Society be $kx$. Then \[\der{x}{t} = kx - r.\] At $x = a$, interest and repayment balance, i.e. $\derx{x}{t} = 0$. Thus, $ka - r = 0$, so $k = r/a$. Hence, \[\der{x}{t} = \frac{r}{a} x - r = \frac{r}{a} (x - a).\]

    We now solve for $x$. We have \[\frac{1}{x-a} \der{x}{t} = \frac{r}{a}.\] Integrating with respect to $t$, we get \[\int \frac{1}{x-a} \d x = \int \frac{r}{a} \d t,\] so \[\ln \abs{x-a} = \frac{r}{a} t + C_1,\] whence $x = C\e^{rt/a} + a$.

    When $t = 0$, we have $x = A$. Substituting this into the solution, we obtain $C = A - a$, so $x = (A-a)\e^{rt/a} + a$.

    \begin{center}\tikzsetnextfilename{215}
        \begin{tikzpicture}[trim axis left, trim axis right]
            \begin{axis}[
                domain = 0:1.386,
                samples = 101,
                axis y line=middle,
                axis x line=middle,
                xtick = {2 * ln 2},
                xticklabels = {$T$},
                ytick = {3, 1},
                yticklabels = {$A_1$, $A_2$},
                xlabel = {$t$},
                ylabel = {$x$},
                legend cell align={left},
                legend pos=outer north east,
                after end axis/.code={
                    \path (axis cs:0,0) 
                        node [anchor=east] {$O$};
                    }
                ]
                \addplot[plotRed] {(3-2)*e^(x/2)+2};
    
                \addlegendentry{$A>a$};

                \addplot[plotBlue] {(1-2)*e^(x/2)+2};
    
                \addlegendentry{$A<a$};
            \end{axis}
        \end{tikzpicture}
    \end{center}

    For the loan to be repaid in finite time, $A < a$. At time $T$, the loan has been repaid, i.e. $x = 0$. Thus, $(A-a) \e^{rt/a} + a = 0$. Solving for $t$, we get \[t = \frac{a}{r} \ln{\frac{a}{a-A}}.\]
\end{solution}