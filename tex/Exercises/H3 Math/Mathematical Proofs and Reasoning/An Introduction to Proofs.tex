\section{An Introduction to Proofs}

\begin{problem}
    Let $m$ and $N$ be positive integers. Prove that $\sqrt[m]{N}$ is either an integer or an irrational.
\end{problem}
\begin{proof}
    Let $x = \sqrt[m]{N}$. Let $A$ be the nearest integer to $x$.
    
    Consider $(x-A)^n$. By the binomial theorem, \[(x-A)^n = \sum_{k = 0}^n \binom{n}{k} x^k (-A)^{n-k}.\] Since $x^m = N \in \ZZ$, the above $n$-degree polynomial reduces to an $m-1$ degree polynomial with integer coefficients, i.e. \[(x-A)^n = \sum_{k = 0}^{m-1} c_k x^k, \tag{1}\] where $\bc{c_k}$ are integers.

    Now, suppose $x \in \QQ$. Then we can write $x = p/q$, where $p, q \in \ZZ$ and $q \neq 0$. Substituting this into (1), we get \[(x-A)^n = \sum_{k = 0}^{m-1} c_k \bp{\frac{p}{q}}^k = \sum_{k = 0}^{m-1} \frac{c_k p^k}{q^k}.\] By combining all terms into a single fraction, we can write \[(x-A)^n = \frac{l}{p^{m-1}},\] where $l$ is an integer. Thus, the only possible values that $(x-A)^n$ can take on are \[\dots, \frac{-2}{p^{m-1}}, \frac{-1}{p^{m-1}}, 0, \frac{1}{p^{m-1}}, \frac{2}{p^{m-1}}, \dots.\]

    Observe that $1/p^{m-1}$ is constant with respect to $n$, i.e. $p$ and $m$ do not depend on $n$. Since $\abs{x - A} < 1$, for arbitrarily large $n$, we can make $(x-A)^n$ as close to 0 as we wish. In other words, we can always find an $n$ large enough such that \[\abs{(x-A)^n} < \frac1{p^{m-1}}.\] Thus, $(x-A)^n$ must be 0, whence $x = A \in \ZZ$. Hence, if $x$ is rational, it must necessarily be an integer. This completes the proof.
\end{proof}

\begin{problem}
    Prove that $\pi$ is irrational.
\end{problem}
\begin{proof}
    Seeking a contradiction, suppose $\pi = p/q$, where $p, q \in \ZZ$ with $q \neq 0$. 
    
    Consider the function $\sin x$. It is well known that $\sin x$
    \begin{itemize}
        \item is non-negative for all $x \in [0, \pi]$, with equality only when $x = 0$ or $x = \pi$; and
        \item attains a maximum at $\pi/2$.
    \end{itemize}

    Now consider the $2n$th degree polynomial $f(x) = x^n (p - qx)^n$. Clearly, $f(x)$ is non-negative on $[0, \pi]$ and has roots only at $x = 0$ and $x = \pi$. Additionally, $f(x)$ attains a maximum of $(\pi/2)^{2n}$ at $x = \pi/2$. Thus, $f(x)$ also satisfies the above two properties.
    
    Consider now the integral \[I = \int_0^\pi f(x) \sin x \d x.\] Since both $f(x)$ and $\sin x$ are non-negative on $[0, \pi]$, it follows that $I$ must also be non-negative on $[0, \pi]$. Additionally, since $f(x) \sin x \not\equiv 0$ on $[0, \pi]$, we have the strict lower bound \[0 < I.\] We can also bound $I$ from above: \[I = \int_0^\pi f(x) \sin x \d x \leq \int_0^\pi \bp{\frac\pi2}^{2n} \d x = \frac{\pi^{2n+1}}{2^{2n}} \leq p^{2n+1}.\] Putting both inequalities together, \[0 < I \leq p^{2n+1}. \tag{1}\]

    We now evaluate $I$. Repeatedly integrating by parts, we get \[I = \sum_{k = 0}^{2n+1} \evalint{f^{(k)}(x) \sin^{(-k-1)} (x)}{0}{\pi} = \sum_{k = 0}^{2n+1} \bs{f^{(k)}(\pi) \sin^{(-k-1)}(\pi) - f^{(k)}(0) \sin^{(-k-1)}(0)}.\] Note that the sum ends at $k = 2n+1$ since $f^{(k)} = 0$ for $k \geq 2n+2$. Also observe that \[\sin^{(-k-1)}(x) = \begin{cases}
        -\cos x, & k \equiv 0 \pmod{4}\\
        -\sin x, & k \equiv 1 \pmod{4}\\
        \cos x, & k \equiv 2 \pmod{4}\\
        \sin x, & k \equiv 3 \pmod{4}
    \end{cases}.\]
    The odd $k$ terms hence vanish. We are thus left with \[I = \sum_{k = 0}^{n} (-1)^{k+1} \bs{f^{(2k)}(\pi) + f^{(2k)}(0)}.\] We now consider $f^{(2k)}(x)$. Firstly, notice that $f(x) = f(\pi - x)$. Hence, by differentiating this repeatedly, we get $f^{(2k)}(0) = f^{(2k)}(\pi)$, so \[I = 2 \sum_{k = 0}^{n} (-1)^{k+1} f^{(2k)}(0).\] Now, observe that when expanded, $f(x)$ is of the form \[f(x) = \sum_{i = n}^{2n} a_i x^{i},\] where $\bc{a_i}$ are integers. Repeatedly differentiating this yields \[f^{(k)}(x) = x^n bp{px - q}^n = \sum_{i = n}^{2n} a_i (i)(i-1)\dots(i-k+1) x^{i - k}.\] Thus, \[f^{(k)}(0) = \begin{cases}
        0, & 0 \leq k < n\\
        a_k k!, & n \leq k \leq 2n
    \end{cases}.\]
    Thus, $f^{(k)}(0)$ is divisible by $k!$ and by extension $n!$ too (since $n \leq k$). Hence, $I$ is divisible by $n!$, i.e. $I = Cn!$ for some integer $C$. From Inequality (1), we have \[0 < C n! < p^{2n+1}.\] However, $n!$ grows much faster than $p^{2n+1}$. Thus, for sufficiently large $n$, the inequality does not hold, a contradiction. Hence, $\pi$ must be irrational.
\end{proof}

\begin{problem}
    Prove that $\e$ is irrational.
\end{problem}
\begin{proof}
    By definition, \[\e = \sum_{n = 0}^\infty \frac1{n!}.\] Seeking a contradiction, suppose $\e$ is rational, i.e. $\e = a/b$, where $a, b \in \ZZ$ and $b \neq 0$. Define $x$ as \[x = b! \bp{\e - \sum_{n = 0}^b \frac1{n!}}. \tag{1}\]
    
    Replacing $\e$ with $a/b$, we get \[x = b! \bp{\frac{a}{b} - \sum_{n = 0}^b \frac1{n!}} = a(b-1)! - \sum_{n = 0}^b \frac{b!}{n!}.\] Since $b!/n!$ is an integer for $0 \leq n \leq b$, it follows that $x$ is also an integer.

    Using the definition of $\e$, we can rewrite (1) as \[x = b! \bp{\sum_{n = 0}^\infty \frac1{n!} - \sum_{n = 0}^b \frac1{n!}} = \sum_{n = b+1}^\infty \frac{b!}{n!}.\] It follows that $x > 0$. Now, observe that
    \begin{align*}
        x &= \sum_{n = b+1}^\infty \frac{b!}{n!}\\
        &= \frac1{b + 1} + \frac1{(b+1)(b+2)} + \frac1{(b+1)(b+2)(b+3)} + \dots\\
        &< \frac1{b+1} + \frac1{(b+1)^2} + \frac1{(b+1)^3} + \dots\\
        &= \frac1{b+1} \bp{\frac1{1 - \frac1{b+1}}} = \frac1b \leq 1.
    \end{align*}
    Hence, $0 < x < 1$ but $x \in \ZZ$, a contradiction. Thus, $\e$ must be irrational.
\end{proof}

\clearpage
\begin{problem}
    Let $a$ and $b$ be relatively prime. Prove that $\log_a b$ is irrational.
\end{problem}
\begin{proof}
    Seeking a contradiction, suppose $\log_a b$ is rational. Then $\log_a b = m/n$, where $m, n \in \ZZ$ and $n \neq 0$. Note that $m, n > 0$ since $a, b > 1$. Clearly, we have \[b = a^{m/n} \implies b^n = a^m.\] This implies that the integer $k = b^n = a^m$ has two distinct prime factorizations, which is a clear contradiction of the Fundamental Theorem of Algebra. Hence, $\log_a b$ must be irrational.
\end{proof}