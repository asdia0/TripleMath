\section{Tutorial A1.1 Set 1}

\begin{problem}
    The function $f$ is defined for all $x \in \RR$ by \[f(x) = \begin{cases}
        k, & \abs{x} \leq l,\\
        0, & \abs{x} > l,
    \end{cases}\] where $k$ and $l$ are positive constants.

    Sketch on three separate diagrams, using the same scales for each, the graph of the function $g$ defined by \[g(x) = \frac{f(x+a) + f(x-a)}2\] in the cases $a = l/4$, $a = 3l/4$ and $a = 3l/2$.
\end{problem}
\begin{solution}
    \begin{figure}[H]\tikzsetnextfilename{435}
    \centering
    \begin{tikzpicture}[trim axis left, trim axis right]
        \begin{axis}[
            domain = -2.75:2.75,
            samples = 101,
            axis y line=middle,
            axis x line=middle,
            xtick = {-1.25, -0.75, 0.75, 1.25},
            xticklabels = {$\frac{-5l}4$, $\frac{-3l}4$, $\frac{3l}4$, $\frac{5l}4$},
            ytick = {1, 2},
            yticklabels = {$\frac{k}2$, $k$},
            xlabel = {$x$},
            ylabel = {$y$},
            legend cell align={left},
            axis equal,
            axis on top,
            legend pos=outer north east,
            after end axis/.code={
                \path (axis cs:0,0) 
                node [anchor=south east] {$O$};
                }
            ]
            \addplot[plotRed, very thick, domain=-2.75:-1.25] {0};
            \addplot[plotRed, very thick, domain=-1.25:-0.75] {1};
            \addplot[plotRed, very thick, domain=-0.75:0.75] {2};
            \addplot[plotRed, very thick, domain=0.75:1.25] {1};
            \addplot[plotRed, very thick, domain=1.25:2.75] {0};
            \addlegendentry{$a = l/4$};
        \end{axis}
    \end{tikzpicture}
    \end{figure}

    \begin{figure}[H]\tikzsetnextfilename{436}
        \centering
        \begin{tikzpicture}[trim axis left, trim axis right]
            \begin{axis}[
                domain = -2.75:2.75,
                samples = 101,
                axis y line=middle,
                axis x line=middle,
                xtick = {-1.75, -0.25, 0.25, 1.75},
                xticklabels = {$\frac{-7l}4$, $\frac{-l}4$, $\frac{l}4$, $\frac{7l}4$},
                ytick = {1, 2},
                yticklabels = {$\frac{k}2$, $k$},
                xlabel = {$x$},
                ylabel = {$y$},
                legend cell align={left},
                axis equal,
                axis on top,
                legend pos=outer north east,
                after end axis/.code={
                    \path (axis cs:0,0) 
                    node [anchor=south east] {$O$};
                    }
                ]
                \addplot[plotRed, very thick, domain=-2.75:-1.75] {0};
                \addplot[plotRed, very thick, domain=-1.75:-0.25] {1};
                \addplot[plotRed, very thick, domain=-0.25:0.25] {2};
                \addplot[plotRed, very thick, domain=0.25:1.75] {1};
                \addplot[plotRed, very thick, domain=1.75:2.75] {0};
                \addlegendentry{$a = 3l/4$};
            \end{axis}
        \end{tikzpicture}
    \end{figure}
    \begin{figure}[H]\tikzsetnextfilename{437}
        \centering
        \begin{tikzpicture}[trim axis left, trim axis right]
            \begin{axis}[
                domain = -2.75:2.75,
                samples = 101,
                axis y line=middle,
                axis x line=middle,
                xtick = {-2.5, -0.5, 0.5, 2.5},
                xticklabels = {$\frac{-5l}2$, $\frac{-l}2$, $\frac{l}2$, $\frac{5l}2$},
                ytick = {1, 2},
                yticklabels = {$\frac{k}2$, $k$},
                xlabel = {$x$},
                ylabel = {$y$},
                legend cell align={left},
                axis equal,
                axis on top,
                legend pos=outer north east,
                after end axis/.code={
                    \path (axis cs:0,0) 
                    node [anchor=south east] {$O$};
                    }
                ]
                \addplot[plotRed, very thick, domain=-2.75:-2.5] {0};
                \addplot[plotRed, very thick, domain=-2.5:-0.5] {1};
                \addplot[plotRed, very thick, domain=-0.5:0.5] {0};
                \addplot[plotRed, very thick, domain=0.5:2.5] {1};
                \addplot[plotRed, very thick, domain=2.5:2.75] {0};
                \addlegendentry{$a = 3l/2$};
            \end{axis}
        \end{tikzpicture}
    \end{figure}
\end{solution}

\begin{problem}
    Prove that
    \begin{enumerate}
        \item $\floor{\sqrt{x}} = \floor{\sqrt{\floor{x}}}$,
        \item $\ceil{\sqrt{x}} = \ceil{\sqrt{\ceil{x}}}$.
    \end{enumerate}

    Is $\ceil{\sqrt{x}} = \ceil{\sqrt{\floor{x}}}$? If so, give a proof. If not, provide a counterexample.
\end{problem}
\begin{proof}[Proof of \emph{(a)}]
    Let $n = \floor{\sqrt{x}}$, where $n \in \ZZ$. By the definition of the floor function, \[n \leq \sqrt{x} < n+1.\] Squaring, we get \[n^2 \leq x < (n+1)^2.\] Taking the floor, we have \[n^2 \leq \floor{x} \leq x < (n+1)^2.\] Rooting, we have \[n \leq \sqrt{\floor{x}} < n+1.\] Once again, by the definition of the floor function, $n = \floor{\sqrt{\floor{x}}}$. Thus, \[\floor{\sqrt{x}} = n = \floor{\sqrt{\floor{x}}}.\]
\end{proof}
\begin{proof}[Proof of \emph{(b)}]
    Let $n = \ceil{\sqrt{x}}$, where $n \in \ZZ$. By the definition of the ceiling function, \[n - 1 < \sqrt{x} \leq n.\] Squaring, we get \[(n-1)^2 < x \leq n^2.\] Taking the ceiling, we have \[(n-1)^2 < x \leq \ceil{x} \leq n^2.\] Rooting, we have \[n-1 < \sqrt{\ceil{x}} \leq n.\] Once again, by the definition of the ceiling function, $n = \ceil{\sqrt{\ceil{x}}}$. Thus, \[\ceil{\sqrt{x}} = n = \ceil{\sqrt{\ceil{x}}}.\]
\end{proof}
\setcounter{partnum}{2}
\begin{ppart}
    It is not true that $\ceil{\sqrt{x}} = \ceil{\sqrt{\floor{x}}}$. Take $x = 1.21$. Then \[\ceil{\sqrt{x}} = \ceil{\sqrt{1.21}} = \ceil{1.1} = 2,\] but \[\ceil{\sqrt{\floor{x}}} = \ceil{\sqrt{\floor{1.1}}} = \ceil{\sqrt{1}} = \ceil{1} = 1.\]
\end{ppart}

\begin{problem}
    Given that $f : \RR \to \RR$ such that for all $x, y \in \RR \setminus \bc{0}$, $f(xy) = f(x) + f(y)$, find the values of $f(1)$ and $f(-1)$. Hence, show that $f$ is even. Give an example of a function that satisfies the above properties and sketch its graph.
\end{problem}
\begin{solution}
    Taking $x = 1$, we see that \[f(y) = f(1) + f(y) \implies f(1) = 0.\] Taking $x = y = -1$, \[f(1) = f(-1) + f(-1) \implies f(-1) = 0.\] Taking $y = -1$, \[f(-x) = f(x) + f(-1) = f(x).\] Thus, by definition, $f$ is even.

    An example of $f$ is $f(x) = \ln \abs{x}$.
    \begin{figure}[H]\tikzsetnextfilename{360}
    \centering
    \begin{tikzpicture}[trim axis left, trim axis right]
        \begin{axis}[
            domain = -5:5,
            restrict y to domain =-5:5,
            samples = 101,
            axis y line=middle,
            axis x line=middle,
            xtick = {-1, 1},
            ytick = \empty,
            xlabel = {$x$},
            ylabel = {$y$},
            legend cell align={left},
            legend pos=outer north east,
            after end axis/.code={
                \path (axis cs:0,0) 
                node [anchor=south east] {$O$};
                }
            ]
            \addplot[plotRed] {ln(abs(x))};
            \addlegendentry{$y = \ln \abs{x}$};
        \end{axis}
    \end{tikzpicture}
    \end{figure}
\end{solution}

\begin{problem}
    If $f$ and $g$ are convex functions, show that the function $h$ given by $h(x) = f(x) g(x)$ is not necessarily a convex function with a suitable example.
\end{problem}
\begin{solution}
    Let $f(x) = x^2$ and $g(x) = x^2 - 1$. Then $h(x) = x^2\bp{x^2 - 1}$ is clearly not convex.
\end{solution}

\begin{problem}
    For a triangle $ABC$ with corresponding angles $a$, $b$ and $c$, show that \[\sin a + \sin b + \sin c \leq \frac{3\sqrt3}{2}\] and determine when equality holds.
\end{problem}
\begin{proof}
    Since $y = \sin x$ is concave, by Jensen's inequality, \[\frac{\sqrt3}2 = \sin \frac\pi3 = \sin \frac{a + b + c}{3} \geq \frac{\sin a + \sin b + \sin c}{3}.\] The desired inequality follows immediately.
\end{proof}