\section{Assignment A1}

\begin{problem}
    A traveller just returned from Germany, France and Spain. The amount (in dollars) that he spent each day on housing, food and incidental expenses in each country are shown in the table below.
    \begin{table}[H]
        \centering
        \begin{tabular}{|c|c|c|c|}
        \hline
        \textbf{Country} & \textbf{Housing} & \textbf{Food} & \textbf{Incidental Expenses}  \\\hline
        Germany & 28      & 30   & 14                   \\\hline
        France  & 23      & 25   & 8                    \\\hline
        Spain   & 19      & 22   & 12 \\\hline                  
        \end{tabular}
    \end{table}
    The traveller's records of the trip indicate a total of \$391 spent for housing, \$430 for food and \$180 for incidental expenses. Calculate the number of days the traveller spent in each country.

    He did his account again and the amount spent on food is \$337. Is this record correct? Why?
\end{problem}
\begin{solution}
    Let $g$, $f$ and $s$ represent the number of days the traveller spent in Germany, France and Spain respectively. From the table, we obtain the following system of equations: \[\systeme{28g + 23f + 19s = 391,30g + 25f + 22s = 430,14g + 8f + 12s = 180}\] This gives the unique solution $g=4$, $f=8$ and $s=5$. The traveller thus spent 4 days in Germany, 8 days in France and 5 days in Spain. 

    Consider the scenario where the amount spent on food is \$337. \[\systeme{28g + 23f + 19s = 391,30g + 25f + 22s = 337,14g + 8f + 12s = 180}\] This gives the unique solution $g=66$, $f=-27$ and $s=-44$. The record is hence incorrect as $f$ and $s$ must be positive.
\end{solution}

\begin{problem}
    \begin{enumerate}
        \item Solve algebraically $x^2 - 9 \geq (x+3)\bp{x^2 - 3x + 1}$.
        \item Solve algebraically $\frac{7-2x}{3-x^2} \leq 1$.
    \end{enumerate}
\end{problem}
\begin{solution}
    \begin{ppart}
        \begin{alignat*}{2}
            &&x^2-9 &\geq (x+3)\bp{x^2 - 3x + 1} \\
            \implies &&(x+3)(x-3) &\geq (x+3)\bp{x^2 - 3x + 1} \\
            \implies &&(x+3)\bp{x^2 - 4x + 4} &\leq 0 \\
            \implies &&(x+3)(x-2)^2 &\leq 0
        \end{alignat*}
        \begin{center}\tikzsetnextfilename{25}
            \begin{tikzpicture}
                    \draw[-latex] (-4.0,0) -- (3,0) node[right]{$x$};
                    \foreach \x in {2,-3.0} \draw[shift={(\x,0)}] (0pt,3pt) -- (0pt,-3pt);
                    \foreach \x in {2,-3} \draw[shift={(\x,-3pt)}] node[below]  {$\x$};
                    \draw[very thick, -*, color=red] (-4.0, 0) -- (-2.88, 0);
                    \path [draw=red, fill=red] (2,0) circle (3pt);
                    \node[anchor=south, align=center] at (-3.5, 0) {$-$};
                    \node[anchor=south, align=center] at (-0.5, 0) {$+$};
                    \node[anchor=south, align=center] at (2.5, 0) {$+$};
            \end{tikzpicture}
        \end{center}
        Thus, $x \leq -3$ or $x = 2$.
    \end{ppart}
    \begin{ppart}
        Note that $3-x^2 \neq 0 \implies x \neq \pm \sqrt{3}$.
        \begin{alignat*}{2}
            &&\frac{7-2x}{3-x^2} &\leq 1\\
            \implies &&\frac{7-2x}{3-x^2} - \frac{3-x^2}{3-x^2} &\leq 0\\
            \implies&& \frac{x^2-2x+4}{3-x^2} &\leq 0
        \end{alignat*}
        Observe that $x^2 - 2x + 4 = (x-1)^2 + 3 > 0$. Dividing through by $x^2 - 2x + 4$, we obtain
        \begin{alignat*}{2}
            &&\frac1{3-x^2} &\leq 0\\
            \implies&&3-x^2 &\leq 0
        \end{alignat*}
        \begin{center}\tikzsetnextfilename{26}
            \begin{tikzpicture}
                    \draw[-latex] (-2.732,0) -- (2.732,0) node[right]{$x$};
                    \foreach \x in {1.732,-1.732} \draw[shift={(\x,0)}] (0pt,3pt) -- (0pt,-3pt);
                    \draw[shift={(1.732,-3pt)}] node[below]  {$\sqrt3$};
                    \draw[shift={(-1.732,-3pt)}] node[below]  {$-\sqrt3$};
                    \draw[very thick, -*, color=red] (-2.732, 0) -- (-1.612, 0);
                    \draw[very thick, *-, color=red] (1.612, 0) -- (2.632, 0);
                    \node[anchor=south, align=center] at (-2.232, 0) {$-$};
                    \node[anchor=south, align=center] at (0.0, 0) {$+$};
                    \node[anchor=south, align=center] at (2.232, 0) {$-$};
            \end{tikzpicture}
        \end{center}
        Thus, $x < -\sqrt3$ or $x > \sqrt3$.
    \end{ppart}
\end{solution}

\begin{problem}
    \begin{enumerate}
        \item Without using a calculator, solve the inequality $\frac{3x+4}{x^2+3x+2} \geq \frac{1}{x+2}$.
        \item Hence, deduce the set of values of $x$ that satisfies $\frac{3\abs{x}+4}{x^2+3\abs{x}+2} \geq \frac{1}{\abs{x}+2}$.
    \end{enumerate}
\end{problem}
\begin{solution}
    \begin{ppart}
        Note that $x^2 + 3x + 2 \neq 0$ and $x + 2 \neq 0$, whence $x \neq -1, -2$.
        \begin{alignat*}{2}
            &&\frac{3x+4}{x^2+3x+2} &\geq \frac{1}{x+2}\\
            \implies&& \frac{3x+4}{(x+2)(x+1)} &\geq \frac{1}{x+2}\\
            \implies&& (3x+4)(x+2)(x+1) &\geq (x+2)(x+1)^2 \\
            \implies&& (x+2)(x+1)(2x+3) &\geq 0
        \end{alignat*}
        \begin{center}\tikzsetnextfilename{27}
            \begin{tikzpicture}
                    \draw[-latex] (-6,0) -- (0,0) node[right]{$x$};
                    \foreach \x in {-1.5,-2,-1} \draw[shift={(2*\x,0)}] (0pt,3pt) -- (0pt,-3pt);
                    \foreach \x in {-1.5,-2,-1} \draw[shift={(2*\x,-3pt)}] node[below]  {$\x$};
                    \draw[very thick, o-*, color=red] (-4.12, 0) -- (-2.88, 0);
                    \draw[very thick, o-, color=red] (-2.12, 0) -- (-0.1, 0);
                    \node[anchor=south, align=center] at (-5, 0) {$-$};
                    \node[anchor=south, align=center] at (-3.5, 0) {$+$};
                    \node[anchor=south, align=center] at (-2.5, 0) {$-$};
                    \node[anchor=south, align=center] at (-1, 0) {$+$};
            \end{tikzpicture}
        \end{center}
        Thus, $-2 < x \leq -\frac32$ or $x > -1$.
    \end{ppart}
    \begin{ppart}
        Observe that $\abs{x}^2 = x^2$. Hence, with the map $x \mapsto \abs{x}$, we obtain \[-2 < \abs{x} \leq -\frac32 \tor \abs{x} > -1.\] Since $\abs{x} \geq 0$, we have that $\abs{x} > -1$ is satisfied for all real $x$. Hence, the solution set is $\RR$.
    \end{ppart}
\end{solution}    

\clearpage
\begin{problem}
    On the same diagram, sketch the graphs of $y = 4\abs{x}$ and $y = x^2 - 2x + 3$. Hence or otherwise, solve the inequality $4\abs{x} \geq x^2 - 2x + 3$.
\end{problem}
\begin{solution}
    \begin{center}\tikzsetnextfilename{28}
        \begin{tikzpicture}[trim axis left, trim axis right]
            \begin{axis}[
                domain = -3:7,
                samples = 101,
                axis y line=middle,
                axis x line=middle,
                xlabel = {$x$},
                ylabel = {$y$},
                xtick=\empty,
                ytick={3},
                yticklabels={3},
                legend cell align={left},
                legend pos=outer north east,
                after end axis/.code={
                    \path (axis cs:0,0) 
                        node [anchor=north] {$O$};
                    }
                ]
                
                \addplot[plotRed, name path=f1] {4*abs(x)};

                \addlegendentry{$y = 4\abs{x}$};

                \addplot[plotBlue, name path=f2] {x^2-2*x+3)};

                \addlegendentry{$y = x^2-2x+3$};

                \fill [name intersections={of=f1 and f2,by={E1, E2}}] (E1) circle[radius=2.5pt] node[anchor=south west, fill=white, opacity = 0.6, text opacity=1] {$(0.551, 2.20)$};

                \fill (E2) circle[radius=2.5pt] node[anchor=south east] {$(5.45, 21.8)$};

            \end{axis}
        \end{tikzpicture}
    \end{center}
    From the graph, we see that $0.551 \leq x \leq 5.45$.
\end{solution}