\section{Self-Practice A3}

\begin{problem}
    The sum of the first $n$ terms of a sequence $\bc{u_n}$ is given by the formula $S_n = 2n(n-3)$, where $n \in \ZZ^+$.

    \begin{enumerate}
        \item Express $u_n$ in terms of $n$, and show that the sequence $\bc{u_n}$ follows an arithmetic progression.
        \item Three terms $u_3$, $u_k$ and $u_{38}$ of this sequence are consecutive terms in a geometric sequence. Find the value of $k$.
        \item Explain why the infinite series $\e^{-u_1} + \e^{-u_2} + \e^{-u_3} + \dots$ exists, and determine the value of the infinite sum, leaving your answer in exact form.
    \end{enumerate}
\end{problem}

\begin{problem}
    At the end of December 2010, the amount of water in a large tank was 43 000 litres. The tank was filled with 7000 litres of water at the start of every month. It was observed that 25\% of the amount at the start of any month was lost by the end of that month.

    \begin{enumerate}
        \item Show that at the end of February 2011, the amount of water in the tank was 33 375 litres.
        \item Find the amount of water in the tank, measured in litres, at the end of the $n$th month after the end of December 2010, expressing your answer in the form $A\bp{\frac34}^n + B$, where $A$ and $B$ are positive integers to be determined.
    \end{enumerate}
\end{problem}

\begin{problem}
    \begin{enumerate}
        \item A runner wants to train for the marathon. He runs 8 km during the first day, and increases the distance he runs each subsequent day by 400 m. Find the minimum number of days, $n$, that heneeds to take to complete at least 2000 km.
        \item A sequence of real numbers $\bc{u_1, u_2, u_3, \dots}$, where $u_1 \neq 0$, is defined such that the $(n+1)$th term of the sequence is equal to the sum of the first $n$ terms, where $n \in \ZZ^+$. Prove that the sequence $\bc{u_2, u_3, u_4, \dots}$ follows a geometric progression. Hence, find $u_1 + u_2 + \dots + u_{N+1}$ in terms of $u_1$ and $N$.
    \end{enumerate}
\end{problem}

\begin{problem}
    \begin{enumerate}
        \item If the sum of the first $n$ terms of a series is $S_n$, where $S_n = n - 3n^2$, write down an expression for $S_{n-1}$. Hence, prove that the series is in an arithmetic series.
        \item Each time a ball falls vertically onto a horizontal surface, it rebounds to two-thirds of the height from which it fell. The ball is initially dropped from a point 12 m above the surface.
        
        Show that the distance the ball has travelled just before it touches the surface for the $n$th time is $60 - 72 \bp{\frac23}^n$.

        Hence, find the least number of times the ball has bounced to travel a total distance of more than 52 m.
    \end{enumerate}
\end{problem}

\begin{problem}
    The sequence $\bc{2^n, n = 0, 1, 2, \dots}$ is grouped into sets such that the $r$th bracket contains $r$ terms: $\bc{1}$, $\bc{2, 2^2}$, $\bc{2^3, 2^4, 2^5}$, $\bc{2^6, 2^7, 2^8, 2^9}$, $\dots$. Find the total number of terms in the first $n$ brackets. Hence, find the sum of numbers in the first $n$ brackets. Deduce (in any order), in terms of $n$, the first and the last number in the $n$th bracket.
\end{problem}