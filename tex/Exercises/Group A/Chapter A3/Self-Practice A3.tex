\section{Self-Practice A3}

\begin{problem}
    The sum of the first $n$ terms of a sequence $\bc{u_n}$ is given by the formula $S_n = 2n(n-3)$, where $n \in \ZZ^+$.

    \begin{enumerate}
        \item Express $u_n$ in terms of $n$, and show that the sequence $\bc{u_n}$ follows an arithmetic progression.
        \item Three terms $u_3$, $u_k$ and $u_{38}$ of this sequence are consecutive terms in a geometric sequence. Find the value of $k$.
        \item Explain why the infinite series $\e^{-u_1} + \e^{-u_2} + \e^{-u_3} + \dots$ exists, and determine the value of the infinite sum, leaving your answer in exact form.
    \end{enumerate}
\end{problem}
\begin{solution}
    \begin{ppart}
        Note that \[u_n = S_n - S_{n-1} = 2n(n-3) - 2(n-1)(n-1-3) = 4n-8.\] Thus, \[u_n - u_{n-1} = \bs{4n - 8} - \bs{4(n-1) - 8} = 4.\] Since $u_n - u_{n-1}$ is a constant, the sequence $\bc{u_n}$ follows an arithmetic progression with common difference 4.
    \end{ppart}
    \begin{ppart}
        Note that $u_3 = 4$ and $u_{38} = 144$. Let the common ratio be $r$. Then $u_{38} = r^2 u_3$, so \[r^2 = \frac{u_{38}}{u_3} = \frac{144}{4} = 36,\] whence $r = \pm 6$. Since $u_k > u_3 > 0$, the common ratio $r$ must be positive, so $r = 6$. Thus, \[4k - 8 = u_k = ru_3 = 6(4) = 24,\] whence $k = 8$.
    \end{ppart}
    \begin{ppart}
        Observe that \[\frac{\e^{-u_n}}{\e^{-u_{n-1}}} = \e^{u_{n-1} - u_n} = \e^{-4}.\] Hence, $\bc{\e^{-u_n}}$ is in geometric progression with common ratio $\e^{-4}$. Since $\abs{\e^{-4}} < 1$, the sum to infinity exists, and is given by \[\sum_{n = 1}^\infty \e^{-u_n} = \e^{-u_1} \bp{\frac{1}{1 - \e^{-4}}} = \frac{\e^4}{1 - \e^{-4}}.\]
    \end{ppart}
\end{solution}

\begin{problem}
    At the end of December 2010, the amount of water in a large tank was 43000 litres. The tank was filled with 7000 litres of water at the start of every month. It was observed that 25\% of the amount at the start of any month was lost by the end of that month.

    \begin{enumerate}
        \item Show that at the end of February 2011, the amount of water in the tank was 33375 litres.
        \item Find the amount of water in the tank, measured in litres, at the end of the $n$th month after the end of December 2010, expressing your answer in the form $A(3/4)^n + B$, where $A$ and $B$ are positive integers to be determined.
    \end{enumerate}
\end{problem}
\begin{solution}
    \begin{ppart}
        Let the amount of water in the tank, measured in litres, at the start of the $n$th month after the end of December 2010 be $u_n$. Clearly, $u_0 = 43000$ and \[u_n = \frac34 \bp{u_{n-1} + 7000}.\]

        Note that \[u_1 = \frac34 \bp{u_0 + 7000} = 37500 \quad \tand \quad u_2 = \frac34 \bp{u_1 + 7000} = 33375.\] Hence, at the end of February 2011, the amount of water in the tank was 33 375 litres.
    \end{ppart}
    \begin{ppart}
        Let $k$ be the constant such that \[u_n - k = \frac34 \bp{u_{n-1} - k}.\] It quickly follows that $k = 21 000$. Then \[u_n - 21 000 = \frac34 \bp{u_{n-1} - 21 000} = \bp{\frac34}^n \bp{u_0 - 21 000}.\] Thus, \[u_n = 22 000 \bp{\frac34}^n + 21 000,\] whence $A = 22000$ and $B = 21000$.
    \end{ppart}
\end{solution}

\begin{problem}
    \begin{enumerate}
        \item A runner wants to train for the marathon. He runs 8 km during the first day, and increases the distance he runs each subsequent day by 400 m. Find the minimum number of days, $n$, that he needs to take to complete at least 2000 km.
        \item A sequence of real numbers $\bc{u_1, u_2, u_3, \dots}$, where $u_1 \neq 0$, is defined such that the $(n+1)$th term of the sequence is equal to the sum of the first $n$ terms, where $n \in \ZZ^+$. Prove that the sequence $\bc{u_2, u_3, u_4, \dots}$ follows a geometric progression. Hence, find $u_1 + u_2 + \dots + u_{N+1}$ in terms of $u_1$ and $N$.
    \end{enumerate}
\end{problem}
\begin{solution}
    \begin{ppart}
        Let $u_n$ be the distance ran on the $n$th day, measured in km. Clearly, $\bc{u_n}$ is in arithmetic progression with common difference 0.4, and $u_1 = 8$. Thus, \[u_n = 0.4(n-1) + 8 = 0.4n + 7.6.\] Let $S_n$ be the total distance ran in $n$ days. We have \[S_n = \sum_{k = 1}^n u_k = \sum_{k=1}^n \bp{0.4k + 7.6} = 0.4 \bp{\frac{n(n+1)}{2}} + 7.6n.\] Consider \[S_n = 0.4 \bp{\frac{n(n+1)}{2}} + 7.6n \geq 2000.\] Using G.C., we have $n \geq 82.4$ or $n \leq -121.4$. Since $n$ is a positive integer, the least $n$ is 83. Thus, he needs at least 83 days to complete at least 2000 km.
    \end{ppart}
    \begin{ppart}
        Note that $u_2 = u_1$. Observe that $S_n - S_{n-1} = u_n = S_{n-1}$, so $S_n = 2 S_{n-1}$. Hence, \[\frac{u_{n+1}}{u_n} = \frac{S_n}{S_{n-1}} = 2,\] whence $\bc{u_2, u_3, u_4, \dots}$ is geometric progression with common ratio 2. Thus, \[u_1 + u_2 + \dots + u_{N+1} = u_1 + u_2 \bp{\frac{1 - 2^N}{1- 2}} = u_1 + u_1 \bp{2^N - 1} = u_1 2^N.\]
    \end{ppart}
\end{solution}

\begin{problem}
    \begin{enumerate}
        \item If the sum of the first $n$ terms of a series is $S_n$, where $S_n = n - 3n^2$, write down an expression for $S_{n-1}$. Hence, prove that the series is in an arithmetic series.
        \item Each time a ball falls vertically onto a horizontal surface, it rebounds to two-thirds of the height from which it fell. The ball is initially dropped from a point 12 m above the surface.
        
        Show that the distance the ball has travelled just before it touches the surface for the $n$th time is $60 - 72 \bp{2/3}^n$.

        Hence, find the least number of times the ball has bounced to travel a total distance of more than 52 m.
    \end{enumerate}
\end{problem}
\begin{solution}
    \begin{ppart}
        Clearly, \[S_{n-1} = (n-1) - 3(n-1)^2 = -3n^2 + 7n - 4.\] Hence, \[u_n = S_n - S_{n-1} = \bp{n - 3n^2} - \bp{-3n^2 + 7n - 4} = -6n + 4.\] Since \[u_n - u_{n-1} = \bs{-6n+4} - \bs{-6(n-1) - 4} = -6,\] the sequence $\bc{u_n}$ is in arithmetic progression with common ratio $-6$.
    \end{ppart}
    \begin{ppart}
        Let $u_n$ be the height of the $n$th ``drop'' of the ball. We have $u_1 = 12$, and the recurrence relation $u_{n+1} = 2u_n/3$. Clearly, \[u_n = \bp{\frac23}^{n-1} u_1 = 18 \bp{\frac23}^n.\]
        
        Let $D_n$ be the total distance travelled by the ball just before it touches the surface for the $n$th time. Observe that the after the initial 12 m, the ball travels up and down before touching the surface again. Hence, \[D_n = u_1 + 2 u_2 + 2u_3 + \dots + 2u_n = u_1 + \sum_{k = 2}^n 2u_n.\] This evaluates as \[D_n = u_1 + 2 \sum_{k=2}^n 18 \bp{\frac23}^n = 12 + 36 \cdot \bp{\frac23}^2 \bp{\frac{1 - (2/3)^{n-1}}{1 - 2/3}} = 60 - 72 \bp{\frac23}^n.\]

        Consider $D_n \geq 52$. Using G.C., we have $n \geq 5.4$. Thus, the ball must bounce at least 6 times.
    \end{ppart}
\end{solution}

\begin{problem}
    The sequence $\bc{2^n, n = 0, 1, 2, \dots}$ is grouped into sets such that the $r$th bracket contains $r$ terms: $\bc{1}$, $\bc{2, 2^2}$, $\bc{2^3, 2^4, 2^5}$, $\bc{2^6, 2^7, 2^8, 2^9}$, $\dots$. Find the total number of terms in the first $n$ brackets. Hence, find the sum of numbers in the first $n$ brackets. Deduce (in any order), in terms of $n$, the first and the last number in the $n$th bracket.
\end{problem}
\begin{solution}
    Clearly, the number of terms in the first $n$ brackets is \[1 + 2 + 3 + \dots + n = \frac{n(n-1)}{2}.\]

    Note that the $k$th number is given by $2^{k-1}$. The sum of number in the first $n$ brackets is hence given by \[\sum_{k = 0}^{n(n+1)/2 - 1} 2^k = \frac{1 - 2^{n(n+1)/2}}{1 - 2} = 2^{n(n+1)/2} - 1.\]

    The last number in the $n$th bracket is clearly $2^{n(n+1)/2 -1}$.

    Note that there are $n(n-1)/2$ terms in the first $(n-1)$ brackets. Thus, the first number in the $n$th bracket is $2^{n(n-1)/2}$.
\end{solution}