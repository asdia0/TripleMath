\section{Self-Practice A9}

\begin{problem}
    The position vectors of the vertices of $A$, $B$ and $C$ of a triangle are $\vec a$, $\vec b$ and $\vec c$ respectively.

    If $O$ is the origin, show that the area of triangle $OAB$ is $\frac12 \abs{\vec a \crossp \vec b}$ and deduce an expression for the area of the triangle $ABC$.

    Hence, or otherwise, show that the perpendicular distance from $B$ to $AC$ is \[\frac{\abs{\vec a \crossp \vec b + \vec b \crossp \vec c + \vec c \crossp \vec a}}{\abs{\vec c - \vec a}}.\]
\end{problem}

\begin{problem}
    Points $A$, $B$, $C$ and $D$ have position vectors, relative to the origin $O$, given by $\oa{OA} = \vec i + 2\vec j - \vec k$, $\oa{OB} = -\vec i + 2\vec j + c\vec k$, $\oa{OC} = 2\vec i + \vec j + 4 \vec k$ and $\oa{OD} = \vec i + \vec j + \vec k$, where $c$ is a constant. It is given that $OA$ and $OB$ are perpendicular.

    \begin{enumerate}
        \item Find the value of $c$.
        \item Show that $OA$ is normal to the plane $OBC$.
        \item Find an equation of the plane through $D$ and parallel to $OBC$.
    \end{enumerate}

    Also, find the position vector of the point of intersection of this plane and the line $AC$. Find the acute angle between the plane $OBC$ and the plane through $D$ normal to $OD$.
\end{problem}

\begin{problem}
    The equations of the line $l_1$ and the plane $\Pi_1$ are as follows:
    \begin{gather*}
        l_1 : \vec r = \cveciii5{-1}4 + \l \cveciii1{-1}0, \quad \l \in \RR, \\
        \Pi_1 : xa + z = 5a + 4, \quad a \in \RR^+.
    \end{gather*}

    \begin{enumerate}
        \item If the angle between $l_1$ and $\Pi_1$ is $\pi/6$, show that $a = 1$.
    \end{enumerate}

    Using the value of $a$ in (a),

    \begin{enumerate}
        \setcounter{enumi}{1}
        \item Verify that $l_1$ and $\Pi_1$ intersect at the point $A(5, -1, 4)$.
        \item Given that $C(7, -3, 4)$, find the length of projection of $\oa{AC}$ on $\Pi_1$.
        \item Find the position vector of $N$, the foot of perpendicular of $C$ to $\Pi_1$.
        \item Point $C'$ is obtained by reflecting $C$ about $\Pi_1$. Determine the vector equation of the line that passes through $A$ and $C'$.
    \end{enumerate}
\end{problem}

\begin{problem}
    The equation of the plane $\Pi_1$ is $x + y - 2z = 3$.

    \begin{enumerate}
        \item Find the vector equation of the line $l_1$, which lies in both the plane $\Pi_1$ and the $yz$ plane.
        \item Another plane $\Pi_2$ contains the line $l_2$ with equation $x = 1$, $\frac{y+1}{2} = z$ and is perpendicular to $\Pi_1$. Find the equation of the plane $\Pi_2$ in scalar product form. Determine whether $l_1$ lies on $\Pi_2$.
    \end{enumerate}
\end{problem}

\begin{problem}
    The lines $l_1$ and $l_2$ intersect at the point $P$ with position vector $\vec i + 5 \vec j + 12 \vec k$. The equations of $l_1$ and $l_2$ are $\vec r = (1 + 3\l)\vec i + (5 + 2\l)\vec j + (12 - 2\l)\vec k$ and $\vec r = (1 + 8\m)\vec i + (5 + 11\m)\vec j + (12 + 6\m) \vec k$ respectively, where $\l$ and $\m$ are real parameters.

    \begin{enumerate}
        \item Find an equation of the plane $\Pi_1$, which contains $l_1$ and $l_2$ in the form $\vec r \dotp \vec n = d$.
    \end{enumerate}

    $\Pi_2$ and $\Pi_3$ are two planes with equations $2x + az = b$ and $x - 3y - z = 7$ respectively, where $a$ and $b$ are constants.

    \begin{enumerate}
        \setcounter{enumi}{1}
        \item Find the line of intersection between $\Pi_1$ and $\Pi_3$.
        \item \begin{enumerate}
            \item Find the condition satisfied by $a$ if the three planes $\Pi_1$, $\Pi_2$ and $\Pi_3$ intersect at one unique point.
            \item Given that all three planes meet in a line $l$, find $a$ and $b$.
            \item Given instead that the three planes have no point in common, what can be said about the values of $a$ and $b$?
        \end{enumerate}
    \end{enumerate}
\end{problem}

\begin{problem}
    The point $A$ and $B$ have position vectors $3\vec i + \vec j$ and $3\vec i + 3\vec j$ respectively. The line $l_1$ and the planes $\Pi_1$ and $\Pi_2$ have equations as follows: \[l_1: \vec r = \oa{OA} + \a \cveciii21{-1}, \quad \Pi_1 : x + 2z = 3, \quad \Pi_2 : \vec r = \l \cveciii110 + \m \cveciii011,\] where $\a$, $\l$ and $\m \in \RR$.

    It is given that the planes $\Pi_1$ and $\Pi_2$ intersect in the line $l_2$ and $B$ lies on $l_2$.
    
    \begin{enumerate}
        \item Find a vector equation of the line $l_2$ and show that the line $l_2$ is parallel to the line $l_1$. Hence, find the shortest distance between the lines $l_1$ and $l_2$.
        \item The plane $\Pi_3$ is parallel to the plane $\Pi_2$ and is equidistant to both point $A$ and the plane $\Pi_2$. Show that the equation of the plane $\Pi_3$ is given by $\vec r \dotp (\vec i - \vec j + \vec k) = 1$. Find the position vector of the foot of perpendicular from the point $A$ to the plane $\Pi_3$.
    \end{enumerate}
\end{problem}

\begin{problem}
    The planes $p_1$, $p_2$ and $p_3$ have equations $x = 1$, $2x + y + az = 5$ and $x + 2y + z = b$, where $a$ and $b$ are real constants. Given that $p_1$ and $p_2$ intersect at the line $l$, show that the vector equation of $l$, in terms of $a$, is $\vec r = \vec i + (3 - \l a)\vec j + \l \vec k$, where $\l$ is a real constant.

    \begin{enumerate}
        \item The acute angle between $l$ and $p_3$ is $60\deg$. Without using a calculator, find the possible values of $a$.
        \item Given that the shortest distance from the origin to $p_3$ is $\sqrt6 / 3$ and without solving for the value of $b$, determine the possible position vectors of the foot of perpendicular from the origin to $p_3$.
        \item What can be said about $a$ and $b$ if $p_1$, $p_2$ and $p_3$ do not have any points in common?
    \end{enumerate}
\end{problem}

\begin{problem}[\chili]
    The points $A$ and $B$ have position vectors $\vec a$ and $\vec b$ respectively. The plane $\pi$, with vector equation $\vec r = \vec b + \l \vec u + \m \vec v$, where $\l$ and $\m$ are real parameters, contains $B$ but not $A$.

    \begin{enumerate}
        \item Show that the perpendicular distance of $A$ from $\pi$ is $p$, where \[p = \frac{\abs{(\vec u \crossp \vec v) \dotp (\vec b - \vec a)}}{\abs{\vec u \crossp \vec v}}.\]
        \item The perpendicular from $A$ to $\pi$ meets $\pi$ at $C$, and $D$ is the point on $AB$ such that $CD$ is perpendicular to $AB$. Show that $AD = p^2 / AB$ and hence, or otherwise, show that the position vector of $D$ is \[\vec a + \bp{\frac{p}{\abs{\vec b - \vec a}}}^2 (\vec b - \vec a).\]
    \end{enumerate}

    In the case where $\vec a = -\vec i + 7\vec j + 8\vec k$, $\vec b = 2\vec i + 7\vec j + 5\vec k$, $\vec u = \vec i - 2\vec j + 2\vec k$ and $\vec v = 3\vec i + 2\vec j + 2\vec k$, find the value of $p$, and show that \[\oa{CD} = \frac{8\sqrt{2}}9 \vec x + \frac49 \vec y,\] where $\vec x$ and $\vec y$ are the unit vectors of $\oa{CB}$ and $\oa{CA}$ respectively.
\end{problem}