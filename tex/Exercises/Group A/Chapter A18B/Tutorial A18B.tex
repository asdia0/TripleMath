\section{Tutorial A18B}

\begin{problem}
    A machine assesses the life of a ball-point pen by measuring the length of a continuous line drawn using the pen. A random sample of 80 pens of brand $A$ have a total writing length of 96.84 km. A random sample of 75 pens of brand $B$ have a total writing length of 93.75 km. Assuming that the standard deviation of the writing length of a single pen is 0.25 km for both brands, test at the 5\% level, whether the mean writing lengths of the two brands differ significantly.
\end{problem}
\begin{solution}
    Let $X_A$ km and $X_B$ km be the total writing length of Brand A and Brand B pens respectively. Let $\m_A = \E{X_A}$ and $\m_B = \E{X_B}$.
    
    Let \nullhyp: $\m_A - \m_B = 0$ and \althyp: $\m_A - \m_B \neq 0$. We perform a two-tail two-sample $z$-test at 5\% significance level. Under \nullhyp, \[\ol{X}_A - \ol{X}_B \sim \Normal{0}{0.25^2 \bp{\frac1{80} + \frac1{75}}}.\] From the sample, \[\ol{x}_A = \frac{96.84}{80} = 1.2105 \quad \tand \quad \ol{x}_B = \frac{93.75}{75} = 1.25.\] Using G.C., the $p$-value is $0.326$, which is greater than the 5\% significance level. Thus, we do not reject \nullhyp{} and conclude there is insufficient evidence at the 5\% significance level that the mean writing lengths of the two brands differ significantly.
\end{solution}

\begin{problem}
    I have two alternative routes to work. The times taken on the 8 randomly chosen occasions that I use route 1 are summarized by $\sum x=182$ and $\sum x^{2}=4202$, while the times taken on the 12 randomly chosen occasions that I take route 2 are summarized by $\sum y=238$ and $\sum y^{2}=5108$, with time being measured in minutes. Determine whether there is significant evidence, at the 5\% level, of a difference in the mean times taken on the two routes. State any assumptions needed.
\end{problem}
\begin{solution}
    Let \nullhyp: $\m_X - \m_Y = 0$ and \althyp: $\m_X - \m_Y \neq 0$. We perform a two-tail two-sample $t$-test at 5\% significance level. Assuming that $X$ and $Y$ are normally distributed and have a common variance, under \nullhyp, \[\frac{\ol{X} - \ol{Y}}{s_p \sqrt{1/8 + 1/12}} \sim \StudentT{8 + 12 - 2} = \StudentT{18}.\] From the sample, \[\ol{x} = \frac18 \bp{182} = 22.75 \quad \tand \quad \ol{y} = \frac1{12} \bp{238} = 19.833.\] Also,
    \begin{align*}
        s_X^2 &= \frac1{8-1}\bs{4202 - \frac18 \bp{182}^2} = 8.7857,\\
        s_Y^2 &= \frac1{12-1} \bs{5108 - \frac1{12} \bp{238}^2} = 35.242.
    \end{align*}
    Thus, the pooled variance is \[s_p^2 = \frac{\bp{8-1} \bp{8.7857} + \bp{12 - 1} \bp{35.242}}{8 + 12 - 2} = 24.953.\] Using G.C., the $p$-value is $0.217$, which is greater than the 5\% significance level. Thus, we do not reject \nullhyp{} and conclude that there is insufficient evidence at the 5\% significance level that there is a difference in the mean times taken on the two routes.
\end{solution}

\begin{problem}
    In an experiment, twelve pairs of plants were positioned close to each other in various different locations in a large greenhouse. One plant in each pair was given fertilizer in April and the other in May. The yields are given below.

    \begin{table}[H]
        \centering
        \begin{tabular}{|c|c|c|c|c|c|c|c|c|c|c|c|c|c|c|c|c|c|}
            \hline Pair & 1 & 2 & 3 & 4 & 5 & 6 & 7 & 8 & 9 & 10 & 11 & 12 \\ \hline
            April & 344 & 307 & 339 & 256 & 398 & 267 & 256 & 407 & 335 & 381 & 300 & 388 \\ \hline
            May & 315 & 289 & 317 & 277 & 363 & 258 & 283 & 385 & 269 & 355 & 275 & 363 \\ \hline
        \end{tabular}
    \end{table}

    Test at 5\% significance level whether the mean yield in April is at least 30 more than the mean yield in May. State any assumptions needed.
\end{problem}
\begin{solution}
    Let $D = \text{April yield} - \text{May yield}$. Let \nullhyp: $\m_D = 30$ and \althyp: $\m_D > 30$. We perform a one-tail paired-sample $t$-test at 5\% significance level. Assuming that $D$ is normally distributed, under \nullhyp, \[\frac{\ol{D} - 30}{S_D / \sqrt{12}} \sim \StudentT{11}.\] From the sample, $\ol{d} = 19.083$ and $s_D = 24.347$. Using G.C. the $p$-value is $0.926$, which is greater than the 5\% significance level. Thus, we do not reject \nullhyp{} and conclude there is insufficient evidence at 5\% significance level that the mean yield in April is at least 30 more than the mean yield in May.
\end{solution}

\begin{problem}
    A school teacher decides to test the effectiveness of using a computer-based lesson to teach trigonometry. The teacher selects and pairs students of equal ability. One student from each pair is randomly chosen and assigned to a control group that receives the standard lesson, while the other student in the pair is then assigned to the experimental group that receives the computer-based lesson. On completion of the course, students in both groups sat for the same test to evaluate their learning outcomes. The test consists of multiple-choice questions, and is administered and marked online. The marks are given in the table below.

    \begin{table}[H]
        \centering
        \begin{tabular}{|l|c|c|c|c|c|c|c|c|c|c|c|c|c|c|c|}
            \hline
            Pair & 1 & 2 & 3 & 4 & 5 & 6 & 7 & 8 & 9 & 10 \\ \hline
            Control & 70 & 65 & 70 & 79 & 72 & 60 & 53 & 50 & 72 & 91 \\ \hline
            Experiment & 89 & 80 & 72 & 72 & 91 & 65 & 60 & 65 & 70 & 88 \\ \hline
        \end{tabular}
    \end{table}

    \begin{enumerate}
        \item Find the largest value of significance level, $\a$, at which it could not be rejected that there is no difference in the two methods. State any assumption(s) necessary for the test to be valid.
        \item At the 10\% significance level it is to be concluded that the experimental group scores higher than the control group by more than $k$ marks. Find the values of $k$.
    \end{enumerate}
\end{problem}
\begin{solution}
    \begin{ppart}
        Let $D = \text{experiment score} - \text{control score}$. Let \nullhyp: $\m_D = 0$ and \althyp: $\m_D \neq 0$. We perform a two-tail paired-sample $t$-test. Assuming that $D$ is normally distributed, under \nullhyp, our test statistic is \[\frac{\ol{D}}{S_D / \sqrt{10}} \sim \StudentT{9}.\] From the sample, $\ol{d} = 7$ and $s_D = 9.5568$. Using G.C., the $p$-value is $0.0457$. Thus, the largest value of $\a$ to not reject \nullhyp{} is $0.0457$.
    \end{ppart}
    \begin{ppart}
        Let \nullhyp: $\m_D = k$ and \althyp: $\m_D > k$. We perform a one-tail paired-sample $t$-test at the 10\% significance level. Assuming that $D$ is normally distributed, under \nullhyp, our test statistic is \[\frac{\ol{D} - k}{S_D / \sqrt{10}} \sim \StudentT{9}.\] The observed test statistic is thus \[\frac{7-k}{9.5568/\sqrt{10}} = 2.3163 - 0.33089 k.\] For \nullhyp{} to be rejected, we require \[2.3163 - 0.33089 k > t_{0.9},\] so \[k < \frac{2.3163 - t_{0.9}}{0.33089} = 2.82.\]
    \end{ppart}
\end{solution}

\begin{problem}
    Two cyclists cycled to work every weekday and recorded the times (in minutes) that they took each day.
    
    The records for one randomly chosen week is as follows, where Cyclist 1 took $x$ minutes and Cyclist 2 took $y$ minutes: \[\sum \bp{x - \ol{x}}^2 = 90.5, \quad \sum y = 160.6, \quad \sum y^2 = 5244.8.\] Assuming that the variances of the times they took each day are the same, find an unbiased estimate for the common variance.

    During another randomly chosen week, the times (in minutes) that they took is recorded as follows: 
    \begin{table}[H]
        \centering
        \begin{tabular}{|c|c|c|c|c|c|}
        \hline
        Cyclist 1 ($x$) & 30.1 & 21.9 & 34.1 & 33.8 & $c$ \\ \hline
        Cyclist 2 ($y$) & 31.7 & 30.5 & 40.1 & 30.2 & 28.1 \\ \hline
        \end{tabular}
    \end{table}
    The missing data is denoted by $c$.

    Based on the second set of data, the hypothesis that the mean time of the two cyclists does not differ is not rejected at 5\% significance level. Find the range of values of $c$. State any assumptions needed for your test to be valid.
\end{problem}
\begin{solution}
    Their individual sample variances are given by \[s_X^2 = \frac1{n - 1} \sum \bp{x - \ol{x}}^2 = \frac1{5 - 1} \bp{90.5} = 22.625\] and \[s_Y^2 = \frac1{n - 1} \bs{\sum y^2 - \frac1{n} \bp{\sum y}^2} = \frac1{5 - 1} \bs{5244.8 - \frac15 \bp{160.6}^2} = 21.582.\] Thus, the pooled variance is \[s_p^2 = \frac{s_X^2 + s_Y^2}{2} = \frac{22.625 + 21.582}{2} = 22.104.\]

    Let \nullhyp: $\m_X - \m_Y = 0$ and \althyp: $\m_X - \m_Y \neq 0$. We perform a two-tail two-sample $t$-test at a 5\% significance level. Assuming the pooled variance is the same, and that $X$ and $Y$ are normally distributed, under \nullhyp, our test statistic is \[T = \frac{\ol{X} - \ol{Y}}{S_p \sqrt{1/5 + 1/5}} \sim \StudentT{5 + 5 - 2} = \StudentT{8}.\] From the sample, \[\ol{x} - \ol{y} = \frac{119.9 + c}{5} - 32.12 = \frac{c}{5} - 8.14.\] Our observed test statistic is thus \[t = \frac{\ol{x} - \ol{y}}{s_p \sqrt{1/5 + 1/5}} = \frac{c/5 - 8.14}{\sqrt{22.104} \sqrt{2/5}} = 0.067261 c - 2.7375.\] For the null hypothesis to not be rejected, we must have \[t_{0.025} < 0.067261 c - 2.7375 < t_{0.975}.\] Solving for $c$, we get \[6.42 = \frac{t_{0.025} + 2.7375}{0.067261} < c < \frac{t_{0.975} + 2.7375}{0.067261} = 74.9.\]
\end{solution}

\begin{problem}
    Due to the differences in environment, the masses of a certain species of small amounts are believed to be greater in Region $A$ than in Region $B$. It is known that the masses in both regions are normally distributed, with masses in Region $A$ having a standard deviation of 0.004 kg and masses in Region $B$ having a standard deviation of 0.09 kg. To test the theory, random samples are taken: 60 animals from Region $A$ had a mean of 3.03 kg and 50 animals from Region $B$ had a mean mass of 3.00 kg. Test at the 1\% significance level, whether the animals of this species in Region $A$ have a greater mean mass than those in Region $B$.

    Will your conclusion be affected if we do not have the information that the masses of the animals are normally distributed? Explain your answer.
\end{problem}
\begin{solution}
    Let $X$ kg and $Y$ kg represent the mass of the animal in Regions $A$ and $B$ respectively. Let \nullhyp: $\m_x - \m_y = 0$ and \althyp: $\m_x - \m_y > 0$. We perform a one-tail two-sample $z$-test at 1\% significance level. Under \nullhyp, \[\ol{X} - \ol{Y} \sim \Normal{0}{\frac{0.04^2}{60} + \frac{0.09^2}{50}}.\] From the sample, $\ol{x} = 3.03$ and $\ol{y} = 3.00$. Using G.C., the $p$-value is 0.0145, which is greater than the 1\% significance level. Thus, we do not reject \nullhyp{} and conclude there is insufficient evidence at 1\% significance level that the animals of this species in Region $A$ have a greater mean mass than those in Region $B$.

    The conclusion will not be affected. This is because the sizes of both samples are large, so the test statistics will be approximately the same by the Central Limit Theorem.
\end{solution}