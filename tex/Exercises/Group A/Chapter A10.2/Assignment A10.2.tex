\section{Assignment A10.2}

\begin{problem}
    On an Argand diagram, mark and label clearly the points $P$ and $Q$ representing the complex numbers $p$ and $q$ respectively, where \[p = \cos\frac\pi4 + \i\sin\frac\pi4, \qquad q = 2\cos\frac\pi4 + 2\i\sin\frac\pi4.\]

    Find the moduli and arguments of the complex numbers $a$, $b$, $c$, $d$ and $e$, where $a = p^4$, $b = q^2$, $c = -ip$, $d = 1/q$, $e = p + p\conj$.

    On your Argand diagram, mark and label the points $A$, $B$, $C$, $D$ and $E$ representing these complex numbers.

    Find the area of triangle $COQ$.

    Find the modulus and argument of $p^{13/3}q^{45/2}$.
\end{problem}
\begin{solution}
    \begin{center}\tikzsetnextfilename{61}
        \begin{tikzpicture}[trim axis left, trim axis right, scale=1.5]
            \begin{axis}[
                domain = 0:10,
                samples = 101,
                axis y line=middle,
                axis x line=middle,
                xtick = {-1, 1.414},
                xticklabels = {-1, $\sqrt2$},
                ytick = {3},
                yticklabels = {$4\i$},
                xmax=2,
                xmin=-2,
                ymin=-1.2,
                ymax=3.5,
                xlabel = {$\Re$},
                ylabel = {$\Im$},
                legend cell align={left},
                legend pos=outer north east,
                after end axis/.code={
                    \path (axis cs:0,0) 
                        node [anchor=north east] {$O$};
                    }
                ]
    
                \coordinate (R) at (10,0);
                \coordinate[label=above:$P$] (P) at (0.707, 0.707);
                \coordinate[label=above:$Q$] (Q) at (1.414, 1.414);
                \coordinate[label=above:$A$] (A) at (-1, 0);
                \coordinate[label=right:$B$] (B) at (0, 3);
                \coordinate[label=below:$C$] (C) at (0.707, -0.707);
                \coordinate[label=below:$D$] (D) at (0.35355, -0.35355);
                \coordinate[label=above:$E$] (E) at (1.414, 0);
                \coordinate (O) at (0, 0);
        
                \draw (O) -- (Q);
                \draw (O) -- (C);
        
                \fill (A) circle[radius=2.5pt];
                \fill (B) circle[radius=2.5pt];
                \fill (C) circle[radius=2.5pt];
                \fill (D) circle[radius=2.5pt];
                \fill (E) circle[radius=2.5pt];
                \fill (P) circle[radius=2.5pt];
                \fill (Q) circle[radius=2.5pt];

                \draw pic [draw, angle radius=7mm, ""] {angle = E--O--P};
                \draw pic [draw, angle radius=9mm, ""] {angle = C--O--E};

                \node at (0.6, -0.3) {$\frac\pi4$};
                \node at (0.5, 0.25) {$\frac\pi4$};

                \node[anchor=south] at ($(O)!0.5!(P)$) {1};
                \node[anchor=south] at ($(Q)!0.5!(P)$) {1};

                \node[anchor=north] at ($(O)!0.5!(D)$) {$\frac12$};
                \node[anchor=north] at ($(C)!0.5!(D)$) {$\frac12$};
            \end{axis}
        \end{tikzpicture}
    \end{center}

    Note that $p = \e^{\i\pi/4}$ and $q = 2\e^{\i\pi/4}$.
    \begin{align*}
        a &= p^4 = \bp{\e^{\i\pi/4}}^4 = \e^{\i\pi},\\
        b &= q^2 = \bp{2\e^{\i\pi/4}}^2 = 4\e^{\i\pi/2},\\
        c &= -\i p = \e^{-\i\pi/2} \e^{\i\pi/4} = \e^{-\i\pi/4},\\
        d &= \frac1q = \frac12 \e^{-\i\pi/4},\\
        e &= p + p\conj = 2 \Re p = 2\cos{\frac\pi4} = \sqrt2.
    \end{align*}

    \[
    \begin{array}{r c @{\hspace*{1.0cm}} c}\toprule
        z & \abs{z} & \arg z \\\cmidrule{1-3}
        a & 1 & \pi\\
        b & 4 & \pi/2\\
        c & 1 & -\pi/4\\
        d & 1/2 & -\pi/4\\
        e & \sqrt2 & 0\\\bottomrule
    \end{array}
    \]

    Since $\angle COQ = \pi/2$, we have $[\triangle COQ] = \frac12(2)(1) = 1$ units$^2$.

    We have \[p^{\frac{13}3}q^{\frac{45}2} = \bp{\e^{\i\frac{\pi}4}}^{\frac{13}3} \bp{2\e^{\i\frac{\pi}4}}^{\frac{45}2} = 2^{\frac{45}2} \e^{\i \frac{161\pi}{24}} = 2^{\frac{45}{2}} \e^{\i \frac{17\pi}{24}}.\] Hence, $\abs{p^{13/3}q^{45/2}} = \e^{45/2}$ and $\arg{p^{13/3}q^{45/2}} = \frac{17}{24}\pi$.
\end{solution}

\begin{problem}
    The complex number $q$ is given by $q = \frac{\e^{\i2\t}}{1 - \e^{\i2\t}}$, where $0 < \t < 2\pi$. In either order,
    \begin{enumerate}
        \item find the real part of $q$,
        \item show that the imaginary part of $q$ is $\frac12 \cot\t$.
    \end{enumerate}
\end{problem}
\begin{solution}
    We have \[q = \frac{\e^{\i2\t}}{1 - \e^{\i2\t}} = \frac{\e^{\i\t}}{\e^{-\i\t} - \e^{\i\t}} = \frac{\cos \t + \i \sin \t}{-2\i \sin\t} = -\frac12 - \frac1{2\i} \cot \t
    = -\frac12 + \frac\i2 \cot \t.\] Hence, $\Re q = -\frac12$ and $\Im q = \frac12 \cot \t$.
\end{solution}

\begin{problem}
    The complex numbers $z$ and $w$ are such that $z = 4\bp{\cos \frac34\pi + \i\sin \frac34 \pi}$ and $w = 1 - \i\sqrt3$. $z\conj$ denotes the conjugate of $z$.

    \begin{enumerate}
        \item Find the modulus $r$ and the argument $\t$ of $w^2 / z\conj$, where $r > 0$ and $-\pi < \t < \pi$.
        \item Given that $\bp{w^2 / z\conj}^n$ is purely imaginary, find the set of values that $n$ can take.
    \end{enumerate}
\end{problem}
\begin{solution}
    \begin{ppart}
        Note that $z = 4\e^{\i3\pi/4}$ and $w = 2\bp{\frac12 - \i\frac{\sqrt3}2} = 2\e^{-\i\pi/3}$. Hence, \[\frac{w^2}{z\conj} = \frac{\bp{2\e^{-\i\frac{\pi}3}}^2}{4\e^{-\i\frac{3\pi}4}} = \frac{4\e^{-\i\frac{2\pi}3}}{4\e^{-\i\frac{3\pi}4}} = \e^{\i\frac{\pi}{12}}.\] Thus, $r = 1$ and $\t = \pi/12$.
    \end{ppart}
    \begin{ppart}
        Note that $\bp{w^2 / z\conj}^n = \bp{\e^{\i\pi/12}}^n = \e^{\i n\pi/12}$. Since $\bp{w^2 / z\conj}^n$ is purely imaginary, we have $\arg \bp{w^2 / z\conj}^n = \pi/2+ \pi k$, where $k \in \ZZ$. Thus, $n\pi/12 = \pi/2 + \pi k$, whence $n =  6 + 12k$. Hence, $\bc{n \in \ZZ: n = 6 + 12k, \, k \in Z}$.
    \end{ppart}
\end{solution}

\begin{problem}
    The complex number $w$ has modulus $\sqrt2$ and argument $\pi/4$ and the complex number $z$ has modulus $\sqrt2$ and argument $5\pi/6$.

    \begin{enumerate}
        \item By first expressing $w$ and $z$ in the form $x + \i y$, find the exact real and imaginary parts of $w + z$.
        \item On the same Argand diagram, sketch the points $P$, $Q$, $R$ representing the complex numbers $z$, $w$, and $z + w$ respectively. State the geometrical shape of the quadrilateral $OPRQ$.
        \item Referring the Argand diagram in part (b), find $\arg{w + z}$ and show that $\tan \frac{11}{24}\pi = \frac{a + \sqrt2}{\sqrt6 + b}$, where $a$ and $b$ are constants to be determined.
    \end{enumerate}
\end{problem}
\clearpage
\begin{solution}
    \begin{ppart}
        Note that \[w = \sqrt2 \e^{\i\pi/4} = \sqrt2 \bp{\cos \frac\pi4 + \i\sin\frac\pi4} = \sqrt2 \bp{\frac1{\sqrt2} + \i \frac1{\sqrt2}} = 1 + \i\] and \[z = \sqrt2 \e^{\i5\pi/6} = \sqrt2 \bp{\cos \frac56 \pi + \i\sin \frac56\pi} = \sqrt2\bp{-\frac{\sqrt3}2 + \i\frac12} = -\frac{\sqrt3}{\sqrt2} + \i\frac1{\sqrt2}.\] Hence, \[w + z = (1 + \i) + \bp{-\frac{\sqrt3}{\sqrt2} + \i\frac1{\sqrt2}} = \bp{1 - \frac{\sqrt3}{\sqrt2}} + \i\bp{1 + \frac1{\sqrt2}}.\]
    \end{ppart}
    \begin{ppart}
        \begin{center}\tikzsetnextfilename{62}
            \begin{tikzpicture}[trim axis left, trim axis right]
                \begin{axis}[
                    domain = 0:10,
                    samples = 101,
                    axis y line=middle,
                    axis x line=middle,
                    xtick = \empty,
                    ytick = \empty,
                    xmax=1.5,
                    xmin=-2.7,
                    ymin=0,
                    ymax=3,
                    xlabel = {$\Re$},
                    ylabel = {$\Im$},
                    legend cell align={left},
                    legend pos=outer north east,
                    after end axis/.code={
                        \path (axis cs:0,0) 
                            node [anchor=north] {$O$};
                        }
                    ]
        
                    \coordinate (R1) at (10,0);
                    \coordinate (L) at (-10, 0);
                    \coordinate[label=left:$P$] (P) at (-2.4495, 0.70711);
                    \coordinate[label=right:$Q$] (Q) at (1, 1);
                    \coordinate[label=above:$R$] (R) at (-1.4495, 1.70711);
                    \coordinate (O) at (0, 0);
            
                    \draw (O) -- (P);
                    \draw (O) -- (Q);
                    \draw (P) -- (R);
                    \draw (R) -- (Q);
                    \draw (O) -- (R);
            
                    \fill (P) circle[radius=2.5pt];
                    \fill (Q) circle[radius=2.5pt];
                    \fill (R) circle[radius=2.5pt];

                    \draw pic [draw, angle radius=8mm, ""] {angle = R1--O--Q};
                    \draw pic [draw, angle radius=12mm, ""] {angle = P--O--L};

                    \node at (0.6, 0.2) {$\frac\pi4$};
                    \node at (-1, 0.14) {$\frac\pi6$};
                \end{axis}
            \end{tikzpicture}
        \end{center}
        $OPRQ$ is a rhombus.
    \end{ppart}
    \begin{ppart}
        Note that $\angle POQ = \pi - \frac\pi6 - \frac\pi4 = \frac7{12} \pi$. Since $\abs{z} = \abs{w}$, we have $OP = OQ$, whence $\angle ROQ = \frac12 \cdot \frac7{12}\pi = \frac7{24}\pi$. Hence, $\arg{w + z} = \frac\pi4 + \frac7{24}\pi = \frac{13}{24}\pi$. Thus, \[\tan{\frac{13}{24}\pi} = \frac{1 + 1/\sqrt2}{1 - \sqrt3/\sqrt2} = \frac{\sqrt2 + 1}{\sqrt2 - \sqrt3} = \frac{2 + \sqrt2}{2 - \sqrt6}\]
        However, $\tan{\frac{13}{24}\pi} = -\tan{\pi - \frac{13}{24}} = -\tan{\frac{11}{24}\pi}$. Hence, \[\tan{\frac{11}{24}\pi} = -\frac{2 + \sqrt2}{2 - \sqrt6} = \frac{2 + \sqrt2}{\sqrt6 - 2},\] whence $a = 2$ and $b = -2$.
    \end{ppart}
\end{solution}

\begin{problem}
    The complex number $z$ is given by $z = 2\bp{\cos\b + \i\sin\b}$ where $0 < \b < \frac\pi2$.

    \begin{enumerate}
        \item Show that $\frac{z}{4 - z^2} = (k\csc\b)\i$, where $k$ is positive real constant to be determined.
        \item State the argument of $\frac{z}{4 - z^2}$, giving your reasons clearly.
        \item Given the complex number $w = -\sqrt3 + \i$, find the three smallest positive integer values of $n$ such that $\bp{\frac{z}{4 - z^2}}(w\conj)^n$ is a real number.
    \end{enumerate}
\end{problem}
\begin{solution}
    \begin{ppart}
        Observe that $z = 2\bp{\cos\b + \i\sin\b} = 2\e^{\i\b}$. Hence, \[\frac{z}{4 - z^2} = \frac{2\e^{\i\b}}{4 - 4\e^{\i2\b}} = \frac12\bp{\frac{1}{\e^{-\i\b} - \e^{\i\b}}} = \frac12\bp{\frac1{-2\i\sin\b}} = \bp{\frac14 \csc \b} \i,\] thus $k = 1/4$.
    \end{ppart}
    \begin{ppart}
        Since $0 < \b < \pi/2$, we know that $\csc \b > 0$. Hence, $\Im{\frac{z}{4 - z^2}} > 0$. Furthermore, $\Re{\frac{z}{4 - z^2}} = 0$. Thus, $\arg{\frac{z}{4 - z^2}} = \pi/2$.
    \end{ppart}
    \begin{ppart}
        Note that $w = -\sqrt3 + \i = 2\bp{-\frac{\sqrt3}2 + \frac12 \i} = 2\e^{-\i5\pi/6}$. Hence, \[            \arg{\bp{\frac{z}{4 - z^2}}(w\conj)^n} = \frac\pi2 + n\bp{-\frac{5\pi}6} = \pi\bp{\frac12 - \frac{5n}{6}}.\]
        For $\bp{\frac{z}{4 - z^2}}(w\conj)^n$ to be a real number, we require $\arg{\bp{\frac{z}{4 - z^2}}(w\conj)^n} = \pi k$, where $k \in \ZZ$. Hence, \[\pi\bp{\frac12 - \frac56 n} = \pi k \implies  \frac12 - \frac56 n = k \implies 3 - 5n = 6k \implies n \equiv 3 \pmod{6}.\] Hence, the three smallest possible values of $n$ are $3$, $9$ and $15$.
    \end{ppart}
\end{solution}