\section{Assignment A15B}

\begin{problem}
    $P$ is a fixed point on the circumference of a circle, centre $O$, and radius $r$. $Q$ is a point on the circumference such that $\angle POQ =\t$, where $\t$ is a random variable with a rectangular distribution in $[0, 2\pi)$. Find the mean and median values for the length of the shorter arc $PQ$. The length of the chord $PQ$ is $X$. Find $\E{X}$ and show that $\Var{X} = 2r^2 \bp{1 - \frac8{\pi^2}}$. Find also $\P{X>r\sqrt3}$.
\end{problem}
\begin{solution}
    Let $L$ be the length of the shorter arc $PQ$. Observe that \[L = \begin{cases}
        r\t, & 0 \leq \t < \pi,\\
        r\bp{2\pi - \t}, & \pi \leq \t < 2\pi.
    \end{cases}\] Note also that $\t$ has probability density function \[f_{\T}(\t) = \begin{cases}
        \frac1{2\pi}, & 0 \leq \t < 2\pi,\\
        0, & \ow.
    \end{cases}\] Thus, \[\E{L} = \int_0^\pi \frac{r \t}{2\pi} \d \t + \int_\pi^{2\pi} \frac{r\bp{2\pi - \t}}{2\pi} \d \t = \frac{r}{2\pi} \evalint{\frac{\t^2}2}0\pi + \frac{r}{2\pi} \evalint{2\pi \t - \frac{\t^2}{2}}0\pi = \frac{\pi r}{2}.\] Now consider the cdf of $L$:
    \begin{gather*}
        F_L(l) = \P{L < l} = \P{r\t < l} + \P{r\bp{2\pi - \t} < l}\\
        = \P{\t < l/r} + \P{\t > 2\pi - l/r} = \frac{l}{2\pi r} + \frac{l}{2\pi r} = \frac{l}{\pi r}.
    \end{gather*}
    Let $m$ be the median of $L$. Then \[F_L(m) = \frac12 \implies \frac{m}{\pi r} = \frac12 \implies m = \frac{\pi r}{2}.\] Thus, both the mean and median values of the shorter arc $PQ$ are $\pi r/2$.

    Note that $X = 2r \sin{\t/2}$. Thus, \[\E{X} = 2r \E{\sin \frac\t2} = 2r \int_0^{2\pi} \sin{\frac\t2} \frac1{2\pi} \d \t = \frac{2r}{\pi} \evalint{-\cos \frac{\t}{2}}0{2\pi} = \frac{4r}{\pi}.\] We also have
    \begin{gather*}
        \E{X^2} = 4r^2 \E{\sin^2 \frac{\t}{2}} = 4r^2 \int_0^{2\pi} \sin[2]{\frac{\t}{2}} \frac1{2\pi} \d \t\\
        = \frac{2r^2}{\pi} \int_0^{2\pi} \frac{1 - \cos \t}{2} \d \t = \frac{r^2}{\pi} \evalint{\t - \sin \t}0{2\pi} = 2r^2.
    \end{gather*}
    Hence, \[\Var{X} = \E{X^2} - \E{X}^2 = 2r^2 - \bp{\frac{4r}{\pi}}^2 = 2r^2 \bp{1 - \frac{8}{\pi^2}}.\]

    Observe that \[2r \sin \frac\t2 = X > r\sqrt3 \implies \sin \frac\t2 > \frac{\sqrt3}{2} \implies \t \in \bp{\frac{2\pi}3, \frac{4\pi}3}.\] Since $\t$ is uniform on $[0, 2\pi)$, \[\P{X > r\sqrt{3}} = \frac{4\pi/3 - 2\pi/3}{2\pi - 0} = \frac13.\]
\end{solution}

\begin{problem}
    The continuous random variable $X$ has the exponential distribution whose probability density function is given by \[f(x) = \begin{cases}
        \m \e^{-\m x }, & x \geq 0,\\
        0, & \ow,
    \end{cases}\] where $\m$ is a positive constant. Television sets are hired out by a rental company. The time in months, $X$, between major repairs has the above exponential distribution with $\m=0.05$.

    \begin{enumerate}
        \item Find, to 3 significant figures, the probability that a television set hired out by the company will not require a major repair for at least a two-year period.
        \item The company agreed to replace any sets for which the time between major repairs is less than $M$ months, where $M$ is a whole number. Given that the company does not want to have to replace more than one set in 5, find the set of possible values of $M$.
    \end{enumerate}
\end{problem}
\begin{solution}
    \begin{ppart}
        \[\P{X > 24} = \int_{24}^\infty 0.05 \e^{-0.05 x} \d x = 0.301 \tosf{3}.\]
    \end{ppart}
    \begin{ppart}
        \[\P{X < M} \leq \frac15 \implies \P{X \geq M} \geq \frac45 \implies \e^{-0.05 M} \geq \frac45.\] Using G.C., $M \in \bc{1, 2, 3, 4}$.
    \end{ppart}
\end{solution}

\begin{problem}
    An examination is marked out of 100. It is taken by a large number of candidates. The mean mark, for all candidates, is $72.1$, and the standard deviation is $15.2$. Give a reason why a normal distribution, with this mean and standard deviation, would not give a good approximation to the distribution of marks.
\end{problem}
\begin{solution}
    Let $X$ be the marks scored by a candidate. Then $X \sim \Normal{72.1}{15.2^2}$. Note that $\P{0 \leq X \leq 100} = 0.967$. Assuming no candidate receives a negative score, $3.32\%$ of candidates are not accounted for under a normal distribution model.
\end{solution}

\begin{problem}
    The weights of boys in a certain age group are normally distributed, with mean 52 kg and standard deviation $\s$ kg. The weights of girls in the same age group are normally distributed, with mean $\m$ kg and standard deviation 5 kg. On average, 1 in 25 randomly chosen boys weighs less than 45 kg; and 2 in 25 randomly chosen girls weigh more than 49 kg.

    \begin{enumerate}
        \item Find the values of $\m$ and $\s$.
        \item Find the probability that the weight of two randomly chosen boys is more than thrice the weight of a randomly chosen girl.
        \item Find the probability that the mean weight of 10 girls chosen is less than 41 kg.
    \end{enumerate}
\end{problem}
\begin{solution}
    Let $B$ kg be the weight of a boy, and let $G$ kg be the weight of a girl. We have $B \sim \Normal{52}{\s^2}$ and $G \sim \Normal{\m}{5^2}$.

    \begin{ppart}
        Since $\P{B < 45} = 1/25$, using G.C., we obtain $\s = 4$. Since $\P{G > 49} = 2/25$, using G.C., we have $\m = 42$.
    \end{ppart}
    \begin{ppart}
        Let $T = B_1 + B_2 - 3G$. Then $T \sim \Normal{2(52) - 3(42)}{2\bp{4^2} + 3^2 \bp{5^2}} = \Normal{-22}{257}$. Thus, the required probability is \[\P{T > 0} = 0.0850 \tosf{3}.\]
    \end{ppart}
    \begin{ppart}
        Let $\ol{G} = \frac1{10} (G_1 + \dots + G_{10})$. Then $\ol{G} \sim \Normal{42}{5^2/10}$. Hence, the required probability is \[\P{\ol{G} < 41} = 0.264 \tosf{3}.\]
    \end{ppart}
\end{solution}