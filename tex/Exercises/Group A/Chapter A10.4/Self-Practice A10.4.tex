\section{Self-Practice A10.4}

\begin{problem}
    If $\arg{z - 2} = 2\pi/3$ and $\abs{z} = 2$, determine $\arg{z}$.
\end{problem}
\begin{solution}
    Let $A(2 + 0\i)$ and $Z(z)$.
    
    \begin{center}\tikzsetnextfilename{498}
        \begin{tikzpicture}[trim axis left, trim axis right]
            \begin{axis}[
                domain = 0:10,
                samples = 101,
                axis y line=middle,
                axis x line=middle,
                xtick = \empty,
                ytick = \empty,
                xmin=-0.5,
                xmax=3,
                ymin=-0.5,
                ymax=3,
                axis equal image,
                xlabel = {$\Re$},
                ylabel = {$\Im$},
                legend cell align={left},
                legend pos=outer north east,
                after end axis/.code={
                    \path (axis cs:0,0) 
                        node [anchor=north east] {$O$};
                    }
                ]

                \coordinate (R) at (10,0);
                \coordinate[label=above right:$Z$] (Z) at (1, 1.73);
                \coordinate[label=below right:$A$] (A) at (2, 0);
                \coordinate (O) at (0, 0);
        
                \draw (O) --node[sloped]{$|$} (Z) -- (A) --node[sloped]{$|$} (O);
                \draw[dashed] (O) circle[radius=2];
        
                \fill (Z) circle[radius=2.5pt];
                \fill (A) circle[radius=2.5pt];
                \draw pic [draw, angle radius=8mm, "$\frac{2\pi}3$"] {angle = R--A--Z};
            \end{axis}
        \end{tikzpicture}
    \end{center}

    Observe that $\angle OAZ = \pi -2\pi/3 = \pi/3$. Since $OA = OZ = 2$ it follows that $\triangle OAZ$ is equilateral, so $\arg{z} = \angle AOZ = \pi/3$.
\end{solution}

\begin{problem}
    $z$ is a complex number such that $\arg{z - 1} = \pi/3$ and $\arg{z - \i} = \pi/6$. By finding the Cartesian equations of the two half-lines, or otherwise, find the value of $\arg{z}$.
\end{problem}
\begin{solution}
    Let $z = x + \i y$, where $x, y \in \RR$. Then \[\arg{z-1} = \arctan \frac{y}{x-1} = \frac\pi3 \implies \frac{y}{x-1} = \sqrt{3} \implies y = \sqrt{3} x - \sqrt3\] and \[\arg{z-\i} = \arctan \frac{y-1}{x} = \frac{\pi}6 \implies \frac{y-1}{x} = \frac1{\sqrt3} \implies y = 1 + \frac1{\sqrt3} x.\]

    Equating the two, we have \[\sqrt{3} x - \sqrt3 = 1 + \frac1{\sqrt3} x \implies x = \frac{1 + \sqrt{3}}{\sqrt3 - 1/\sqrt3} = \frac{\sqrt3 + 3}{2}.\] Thus, \[y = \sqrt{3} \bp{x - 1} = \sqrt{3} \bp{\frac{\sqrt3 + 3}{2} - 1} = \frac{3 + \sqrt3}{2},\] so $x = y$ and \[\arg z = \arctan \frac{y}{x} = \arctan 1 = \frac\pi4.\]
\end{solution}

\begin{problem}
    The complex number $z$ is given by $z = r\e^{\i \t}$, where $r > 0$ and $0 \leq \t \leq \pi/2$.

    \begin{enumerate}
        \item Given that $w = \bp{1 - \i \sqrt3}z$, find $\abs{w}$ in terms of $r$ and $\arg w$ in terms of $\t$.
        \item Given that $r$ has a fixed value, draw an Argand diagram to show the locus of $z$ as $\t$ varies. On the same Argand diagram, show the corresponding locus of $w$. You should identify the modulus and argument of the end-point of each locus.
    \end{enumerate}
\end{problem}
\begin{solution}
    \begin{ppart}
        Note that $1 - \i \sqrt{3} = 2\e^{-\pi/3}$. Thus, \[w = \bp{1 - \sqrt3 \i} z = \bp{2\e^{-\pi/3}}\bp{r \e^{\i \t}} = 2r \e^{\i (\t - \pi/3)}.\] Hence, $\abs{w} = 2r$ and $\arg{w} = \t - \pi/3$.
    \end{ppart}
    \begin{ppart}
        \begin{center}\tikzsetnextfilename{499}
            \begin{tikzpicture}[trim axis left, trim axis right]
                \begin{axis}[
                    domain = 0:10,
                    samples = 101,
                    axis y line=middle,
                    axis x line=middle,
                    xtick = {1},
                    ytick = {1},
                    xticklabels = {$r$},
                    yticklabels = {$r$},
                    xmin=-0.5,
                    xmax=3,
                    ymin=-2,
                    ymax=2,
                    axis equal image,
                    xlabel = {$\Re$},
                    ylabel = {$\Im$},
                    legend cell align={left},
                    legend pos=outer north east,
                    after end axis/.code={
                        \path (axis cs:0,0) 
                            node [anchor=north east] {$O$};
                        }
                    ]

                    \coordinate (R) at (10,0);
                    \coordinate (Z1) at (1, -1.73);
                    \coordinate (Z2) at (1.73, 1);
                    \coordinate (O) at (0, 0);

                    \draw[plotRed, very thick] (1,0) arc (0:90:1);
                    \draw[plotBlue, very thick] (1, -1.73) arc (-60:30:2);

                    \addlegendimage{plotRed, very thick};
                    \addlegendentry{locus of $z$};
                    \addlegendimage{plotBlue, very thick};
                    \addlegendentry{locus of $w$};
            
                    \fill (Z1) circle[radius=2.5pt];
                    \fill (Z2) circle[radius=2.5pt];
                    \fill (1, 0) circle[radius=2.5pt];
                    \fill (0, 1) circle[radius=2.5pt];

                    \draw pic [draw, angle radius=11mm, "${\scriptstyle \pi/6}$"] {angle = R--O--Z2};
                    \draw pic [draw, angle radius=10mm, "${\scriptstyle -\pi/3}$"] {angle = Z1--O--R};

                    \draw[dashed] (O) -- (Z1) node[pos=0.7, anchor=north east] {$2r$};
                    \draw[dashed] (O) -- (Z2) node[pos=0.7, anchor=south east] {$2r$};
                \end{axis}
            \end{tikzpicture}
        \end{center}
    \end{ppart}
\end{solution}

\begin{problem}
    The complex number $z$ satisfies the equation $\abs{z} = \abs{z + 2}$. Show that the real part of $z$ is $-1$. The complex number $z$ also satisfies the equation $\abs{z} = 3$. The two possible values of $z$ are represented by the points $P$ and $Q$ in an Argand diagram. Draw a sketch showing the positions of $P$ and $Q$, and calculate the two possible values of $\arg z$, giving your answers in radians correct to 3 significant figures.

    It is given that $P$ and $Q$ lie on the locus $\abs{z - a} = b$, where $a$ and $b$ are real, and $b > 0$. Give a geometrical description of this locus, and hence find the least possible value of $b$ and the corresponding value of $a$.
\end{problem}
\begin{solution}
    Observe that the locus of $\abs{z} = \abs{z+2}$ is the perpendicular bisector of $(0, 0)$ and $(2, 0)$, which has Cartesian equation $x = 1$, $y \in \RR$. Thus, the real part of $z$ (i.e. $x$) is always 1.

    \begin{center}\tikzsetnextfilename{500}
        \begin{tikzpicture}[trim axis left, trim axis right]
            \begin{axis}[
                domain = 0:10,
                samples = 101,
                axis y line=middle,
                axis x line=middle,
                xtick = {3, -3, 2},
                ytick = {3, -3},
                xmin=-4,
                xmax=4,
                ymin=-4,
                ymax=4,
                axis equal image,
                axis on top,
                xlabel = {$\Re$},
                ylabel = {$\Im$},
                legend cell align={left},
                legend pos=outer north east,
                after end axis/.code={
                    \path (axis cs:0,0) 
                        node [anchor=north east] {$O$};
                    }
                ]

                \coordinate (R) at (10,0);
                \coordinate[label=above right:$P$] (P) at (1, 2.83);
                \coordinate[label=below right:$Q$] (Q) at (1, -2.83);
                \coordinate (O) at (0, 0);
                \coordinate (A) at (1, 0);

                \draw[plotRed, very thick] (O) circle[radius=3];
                \draw[plotBlue, very thick] (1, -4) -- (1, 4);

                \addlegendimage{plotRed, very thick};
                \addlegendentry{$\abs{z} = 3$};
                \addlegendimage{plotBlue, very thick};
                \addlegendentry{$\abs{z} = \abs{z + 2}$};

                \draw (O) -- (P);
                \draw (O) -- (Q);

                \fill (P) circle[radius=2.5pt];
                \fill (Q) circle[radius=2.5pt];

                \draw pic [draw, angle radius=3mm] {right angle = R--A--P};
            \end{axis}
        \end{tikzpicture}
    \end{center}

    From the diagram, \[\arg{z} = \pm \arccos \frac13 = \pm 1.23 \tosf{3}.\]

    The locus of $\abs{z - a} = b$ is a circle of radius $b$ centred at the point representing $a$.

    For $b$ to be at a minimum, $PQ$ must be the diameter of the circle. By the Pythagorean Theorem, \[3^2 = \bp{\frac{PQ}{2}}^2 + 1^2 \implies PQ = 32.\] Thus, \[\min b = \frac{PQ}{2} = \frac{\sqrt{32}}{2} = 2\sqrt{2}.\] The point representing $a$ is then the midpoint of $P$ and $Q$, i.e. $a = 1$.
\end{solution}

\begin{problem}
    The complex number $z$ is given by $z = x + \i y$, where $x > 0$ and $y > 0$. Sketch an Argand diagram, with origin $O$, showing points $P$, $Q$ and $R$ representing $z$, $2\i z$ and $(z + 2\i z)$ respectively. State the size of angle $POQ$, and describe briefly the geometrical relationship between $O$, $P$, $Q$ and $R$.

    \begin{enumerate}
        \item Given that $x = 2y$, show that $R$ lies on the imaginary axis.
        \item Given that $y = 2x$, show that the point representing $z^2$ is collinear with the origin and the point $R$.
        \item Given that $\abs{z} \leq 2$ and $\arctan \frac12 \leq \arg z \leq \arctan 2$, calculate the area of the region in which the point $P$ can lie.
    \end{enumerate}
\end{problem}
\begin{solution}
    \begin{center}\tikzsetnextfilename{501}
        \begin{tikzpicture}[trim axis left, trim axis right]
            \begin{axis}[
                domain = 0:10,
                samples = 101,
                axis y line=middle,
                axis x line=middle,
                xtick = \empty,
                ytick = \empty,
                xmax=2,
                xmin=-3,
                ymin=-0.4,
                ymax=3.4,
                xlabel = {$\Re$},
                ylabel = {$\Im$},
                axis equal image,
                legend cell align={left},
                legend pos=outer north east,
                after end axis/.code={
                    \path (axis cs:0,0) 
                        node [anchor=north east] {$O$};
                    }
                ]

                \coordinate (R) at (10,0);
                \coordinate[label=right:$P(z)$] (Z1) at (1, 1);
                \coordinate[label=left:$Q(2\i z)$] (Z2) at (-2, 2);
                \coordinate[label=above:$R(z + 2\i z)$] (Z3) at (-1, 3);
                \coordinate (O) at (0, 0);
        
                \draw (O) -- (Z1);
                \draw (O) -- (Z2);
                \draw (O) -- (Z3);

                \draw (Z2) -- (Z3) -- (Z1);
        
                \fill (Z1) circle[radius=2.5pt];
                \fill (Z2) circle[radius=2.5pt];
                \fill (Z3) circle[radius=2.5pt];

                \draw pic [draw, angle radius=2mm] {right angle = Z1--O--Z2};
            \end{axis}
        \end{tikzpicture}
    \end{center}

    $\angle POQ = \pi/2$, and $OPQR$ forms a rectangle.

    \begin{ppart}
        Given $x = 2y$, we have \[z + 2\i z = z\bp{1 + 2\i} = \bp{2y + \i y}\bp{1 + 2\i} = 5y\i,\] which is purely imaginary. Hence, $R$ lies on the imaginary axis.
    \end{ppart}
    \begin{ppart}
        Given $y = 2x$, we have \[z = x + 2\i x = x\bp{1 + 2\i} \implies \arg{z^2} = 2\arg{1 + 2\i}.\] Meanwhile, \[\arg{z + 2\i z} = \arg{z} + \arg{1 + 2\i} = \arg{1 + 2\i} + \arg{1 + 2\i} = 2\arg{1 + 2\i}.\] Since $z^2$ and $z + 2\i z$ have identical arguments, the points representing them must be collinear with the origin.
    \end{ppart}
    \clearpage
    \begin{ppart}
        \begin{center}\tikzsetnextfilename{502}
            \begin{tikzpicture}[trim axis left, trim axis right]
                \begin{axis}[
                    domain = 0:10,
                    samples = 101,
                    axis y line=middle,
                    axis x line=middle,
                    xtick = \empty,
                    ytick = \empty,
                    xmax=1.7,
                    xmin=-0.2,
                    ymin=-0.2,
                    ymax=1.7,
                    xlabel = {$\Re$},
                    ylabel = {$\Im$},
                    axis equal image,
                    legend cell align={left},
                    legend pos=outer north east,
                    after end axis/.code={
                        \path (axis cs:0,0) 
                            node [anchor=north east] {$O$};
                        }
                    ]

                    \coordinate (R) at (10,0);
                    \coordinate (Z1) at (1.26, 0.632);
                    \coordinate (Z2) at (0.632, 1.26);
                    \coordinate (O) at (0, 0);
            
                    \draw (O) -- (Z1);
                    \draw (O) -- (Z2);
                    \draw (O) circle[radius=1.41];
            
                    \fill (Z1) circle[radius=2.5pt];
                    \fill (Z2) circle[radius=2.5pt];

                    \node[anchor=north west] at (1.414, 0) {2};
                    \node[anchor=south east] at (0, 1.414) {2};

                    \draw pic [draw, angle radius=10mm, "$\a$"] {angle = Z1--O--Z2};
                \end{axis}
            \end{tikzpicture}
        \end{center}

        From the above figure, we see that $\a = \arctan 2 - \arctan{1/2}$. The area of the region in which $P$ can lie in is thus \[\area = \pi(2)^2 \times \frac{\arctan 2 - \arctan{1/2}}{2\pi} = 1.29 \units[2].\]
    \end{ppart}
\end{solution}

\begin{problem}
    A complex number $z$ satisfies $\abs{z - a} = a$, $a \in \RR^+$.

    \begin{enumerate}
        \item The point $P$ represents the complex number $w$, where $w = 1/z$, in an Argand diagram. Show that the locus of $P$ is a straight line.
        \item Sketch both loci on the same diagram and show that the two loci do not intersect if $0 < a < 1/2$.
        \item For $a = 1/2$, find the range of values of $\arg{z - 1/a}$, giving your answer correct to $0.1\deg$. State the limit of $\arg{z - 1/a}$ when $a$ approaches 0.
    \end{enumerate}
\end{problem}
\begin{solution}
    \begin{ppart}
        We have \[\abs{z - a} = \abs{\frac1w - a} = \abs{\frac{1 - aw}{w}} = a \implies \abs{1 - aw} = \abs{aw} \implies \abs{w} = \abs{\frac{1 - aw}{a}} = \abs{\frac1a - w}.\] Hence, the locus of $P$ is the perpendicular bisector of the origin and $(1/a, 0)$. Equivalently, it is the vertical line passing through $(1/2a, 0)$.
    \end{ppart}
    \clearpage
    \begin{ppart}
        If $0 < a < 1/2$, then $1/a > 2$, so the real part of any point on the locus of $w$ is $1/2a > 1$. The largest real part of any point on the locus of $z$ is $a + a = 2a < 1$. Thus, both loci will not intersect.

        \begin{center}\tikzsetnextfilename{503}
            \begin{tikzpicture}[trim axis left, trim axis right]
                \begin{axis}[
                    domain = 0:10,
                    samples = 101,
                    axis y line=middle,
                    axis x line=middle,
                    xtick = {0.25},
                    ytick = \empty,
                    xticklabels = {$a$},
                    xmin=0,
                    xmax=2.3,
                    ymin=-1,
                    ymax=1,
                    axis on top,
                    xlabel = {$\Re$},
                    ylabel = {$\Im$},
                    axis equal image,
                    legend cell align={left},
                    legend pos=outer north east,
                    after end axis/.code={
                        \path (axis cs:0,0) 
                            node [anchor=east] {$O$};
                        }
                    ]

                    \draw[plotRed, very thick] (0.25, 0) circle[radius=0.25];
                    \draw[plotBlue, very thick] (2, -1) -- (2, 1);
                    \node[anchor=north east] at (2, 0) {$\frac1{2a}$};
                    \node[anchor=north west] at (0.5, 0) {$2a$};

                    \addlegendimage{plotRed, very thick};
                    \addlegendentry{locus of $z$};
                    \addlegendimage{plotBlue, very thick};
                    \addlegendentry{locus of $w$};
                \end{axis}
            \end{tikzpicture}
        \end{center}
    \end{ppart}
    \begin{ppart}
        \begin{center}\tikzsetnextfilename{504}
            \begin{tikzpicture}[trim axis left, trim axis right]
                \begin{axis}[
                    domain = 0:10,
                    samples = 101,
                    axis y line=middle,
                    axis x line=middle,
                    xtick = {0.5, 2},
                    ytick = \empty,
                    xticklabels = {$\frac12$},
                    xmin=0,
                    xmax=2.3,
                    ymin=-0.7,
                    ymax=0.7,
                    axis on top,
                    xlabel = {$\Re$},
                    ylabel = {$\Im$},
                    axis equal image,
                    legend cell align={left},
                    legend pos=outer north east,
                    after end axis/.code={
                        \path (axis cs:0,0) 
                            node [anchor=east] {$O$};
                        }
                    ]

                    \draw[plotRed, very thick] (0.5, 0) circle[radius=0.5];
                    \node[anchor=north west] at (1, 0) {1};

                    \coordinate (A) at (0.691, 0.462);
                    \coordinate (Z) at (2, 0);
                    \coordinate (O) at (0.5, 0);
                    \draw (O) -- (A) -- (Z);

                    \draw pic [draw, angle radius=15mm, "$\t$"] {angle = A--Z--O};
                    \draw pic [draw, angle radius=3mm] {right angle = O--A--Z};
                \end{axis}
            \end{tikzpicture}
        \end{center}

        From the diagram, \[\sin \t = \frac{1/2}{2 - 1/2} \implies \t = \arcsin \frac13.\] Thus, \[160.5\deg = \pi - \t \leq \arg{z-2} \leq \pi + \t = 199.5\deg.\] As $a \to 0$, $\arg{z-1/a} \to \pi$.
    \end{ppart}
\end{solution}

\begin{problem}
    Sketch, on an Argand diagram, the locus representing the complex number $z$ for which \[\abs{z - 4 - 3\i} = 2.\]

    \begin{enumerate}
        \item Given that $a$ is the least possible value of $\abs{z}$, find $a$.
        \item The complex number $p$ is such that \[\abs{p - 4 - 3\i} = 2 \quad \tand \quad \abs{p} = a.\] State the exact value of $\arg p$.
        \item Deduce the greatest value of $\arg{z/p}$, giving your answer correct to 2 decimal places.
    \end{enumerate}
\end{problem}
\clearpage
\begin{solution}
    \begin{center}\tikzsetnextfilename{505}
            \begin{tikzpicture}[trim axis left, trim axis right]
                \begin{axis}[
                    domain = 0:10,
                    samples = 101,
                    axis y line=middle,
                    axis x line=middle,
                    xtick = \empty,
                    ytick = \empty,
                    xticklabels = {$\frac12$},
                    xmin=0,
                    xmax=7,
                    ymin=0,
                    ymax=6,
                    axis on top,
                    xlabel = {$\Re$},
                    ylabel = {$\Im$},
                    axis equal image,
                    legend cell align={left},
                    legend pos=outer north east,
                    after end axis/.code={
                        \path (axis cs:0,0) 
                            node [anchor=east] {$O$};
                        }
                    ]

                    \coordinate[label=right:$\bp{4, 3}$] (A) at (4, 3);
                    \coordinate[label=left:$P$] (P) at (2.4, 1.8);
                    \coordinate (Z) at (2.23, 3.93);
                    \coordinate (O) at (0, 0);

                    \draw (O) -- (A) --node[pos=0.5, anchor=south west] {2} (Z) -- (O);

                    \draw[plotRed, very thick] (A) circle[radius=2];

                    \fill (P) circle[radius=2.5pt];
                    \fill (A) circle[radius=2.5pt];
                    \fill (Z) circle[radius=2.5pt];

                    \addlegendimage{plotRed, very thick};
                    \addlegendentry{locus of $z$};

                    \draw pic [draw, angle radius=3mm] {right angle = O--Z--A};
                    \draw pic [draw, angle radius=15mm, "$\t$"] {angle = A--O--Z};
                \end{axis}
            \end{tikzpicture}
        \end{center}

    \begin{ppart}
        Clearly, $a = \sqrt{4^2 + 3^2} - 2 = 3$.
    \end{ppart}
    \begin{ppart}
        Clearly, $\arg p = \arctan{3/4}$.
    \end{ppart}
    \begin{ppart}
        Observe that \[\max \arg \frac{z}{p} = \t = \arcsin \frac25  = 23.58\deg.\]
    \end{ppart}
\end{solution}

\begin{problem}[\chili]
    On an Argand diagram, the point $U$ represents the complex number $z$, and the points $V$ and $W$ represent the complex numbers $z^2$ and $z^2 + 1$ respectively.

    \begin{enumerate}
        \item \begin{enumerate}
            \item Given that $\arg{z}= \a$, where $\pi/4 < \a < \pi/2$, so that $U$ lies on the half-line $L_1$ with equation $y = x \tan \a$ for $x > 0$, show that $V$ lies on the half-line $L_2$ with equation $y = x \tan 2\a$ for $x < 0$. Find the equation of the locus $L_3$ of $W$.
            \item The points $E$ and $F$ represent the values of $z$ for which $W$ coincides with $U$. Find the value of $\a$ for which the common point of $L_1$ and $L_3$ is either $E$ or $F$.
        \end{enumerate}
        \item Given instead that $\abs{z} = k$, where $k > 0$, so that $U$ lies on a circle $C$, show that $W$ lies on a circle $C'$, and find its centre and radius. Find the value of $k$ for which the common points of $C$ and $C'$ are $E$ and $F$.
    \end{enumerate}
\end{problem}
\begin{solution}
    \begin{ppart}
        \begin{psubpart}
            Note that $\arg z^2 = 2\arg z = 2\a$. Let $z^2 = x + \i y$, where $x, y \in \RR$. Then $\arg{z^2} = \arctan{y/x}$. Equating the two yields \[\arctan \frac{y}{x} = 2\a \implies y = x \tan{2\a}.\] Note that $\arg{z^2} = 2\a \in (\pi/2, \pi)$, so $x = \Re{z^2} < 0$.

            $L_3$ is precisely $L_2$ shifted one unit in the positive real axis. Hence, the equation of $L_3$ is $y = (x-1) \tan{2\a}$.
        \end{psubpart}
        \begin{psubpart}
            Since $W$ coincides with $U$, we have $z = z^2 + 1$. Solving, we get \[z = \frac{1 + \sqrt{3} \i}{2}.\] Note that we reject the negative branch since $\arg z \in (\pi/4, \pi/2)$ implies $\Im z > 0$. Thus, \[\a = \arg z = \arctan \frac{\sqrt3}{1} = \frac\pi3.\]
        \end{psubpart}
    \end{ppart}
    \begin{ppart}
        Observe that \[\abs{\bp{z^2 + 1} - 1} = \abs{z^2} = k^2,\] so $W$ lies on a circle with radius $k^2$ and centre $(1, 0)$. For $E$ and $F$ to lie on $C$, we require \[k = \abs{\frac{1 \pm \sqrt{3} \i}{2}} = 1.\]
    \end{ppart}
\end{solution}