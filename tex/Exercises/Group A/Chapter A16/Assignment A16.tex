\section{Assignment A16}

\begin{problem}
    Alice receives ``like'' notifications from her Facebook friends at random, with an average of one ``like'' received over two days. Taking a year as 52 weeks, find the probabilities that in a year,
    \begin{enumerate}
        \item there are not more than 2 weeks in which she receives 6 ``like'' notices in a week,
        \item the mean number of ``like'' notices received per week is at least 4 by the use of a suitable approximation.
    \end{enumerate}
\end{problem}
\begin{solution}
    Let $L$ be the number of ``likes'' received in a week. Then $L \sim \Po{\frac72}$.

    \begin{ppart}
        Note that $\P{L = 6} = 0.077098 \tosf{5}$. Let $W$ be the number of weeks in a year in which Alice receives 6 ``likes''. Then $W \sim \Binom{52}{0.077098}$. The required probability is hence $\P{W \leq 2} = 0.225 \tosf{3}$.
    \end{ppart}
    \begin{ppart}
        Let $\ol{L} = \frac1{52} \bp{L_1 + \dots + L_{52}}$. Since the sample size (52) is large, by the Central Limit Theorem, $\ol{L} \sim \Normal{\frac72}{\frac1{52} \frac72}$ approximately. The required probability is hence $\P{\ol{L} \geq 4} = 0.0270 \tosf{3}$.
    \end{ppart}
\end{solution}

\begin{problem}
    The mass of doughnuts produced in a doughnut factory is found to have mean 150 g and standard deviation 60 g.

    \begin{enumerate}
        \item Find the probability that the mean mass of a random sample of 200 doughnuts is between 145 g and 160 g. Give a reason why it is not necessary to assume that the mass of doughnuts is normally distributed.
        \item Find the least value of $n$ such that the probability that the mean mass of a sample of $n$ doughnuts is greater than 140 g is greater than $0.8$.
    \end{enumerate}
\end{problem}
\begin{solution}
    Let $M$ g be the mass of a doughnut.

    \begin{ppart}
        Let $\ol{M}_{k} = \frac1{200} \bp{M_1 + \dots + M_{k}}$. Since the sample size (200) is large, by the Central Limit Theorem, $\ol{M}_{200} \sim \Normal{150}{60^2/200}$ approximately. Hence, the required probability is \[\P{145 < \ol{M}_{200} < 160} = 0.871 \tosf{3}.\]

        The Central Limit Theorem applies to any distribution, so long as $n$ is large. Hence, $M$ need not be normally distributed.
    \end{ppart}
    \begin{ppart}
        Suppose $n$ is large. Then by the Central Limit Theorem, $\ol{M}_n \sim \Normal{150}{60^2/n}$ approximately. Consider $\P{\ol{M}_n > 140} > 0.8$. Using G.C., the least $n$ is 26, which is large.
    \end{ppart}
\end{solution}

\begin{problem}
    A pharmaceutical company created a new drug to treat a particular illness. The patients experienced weight loss due to the side effects of the new drug. A random sample of 50 individuals was selected and the weight loss by each individual, $x$ kg, is recorded and summarized as follows: \[\sum \bp{x - 5} = 120 \quad \tand \quad \sum \bp{x - 5}^2 = 2500.\]

    \begin{enumerate}
        \item Describe how the company can obtain the random sample of 50 individuals.
        \item Calculate unbiased estimates of $\m$, the population mean, and $\s^2$, the population variance.
        \item Estimate the probability that the mean weight loss by fifty randomly chosen patients who took the new drug is greater than 7 kg.
    \end{enumerate}
\end{problem}
\begin{solution}
    \begin{ppart}
        Assign each patient a unique positive integer. Using a random number generator, obtain 50 distinct positive integers. The patients assigned to these numbers are then sampled.
    \end{ppart}
    \begin{ppart}
        Note that \[\sum \bp{x - 5} = 120 \implies \sum x = 120 + 50(5) = 370\] and \[\sum \bp{x - 5}^2 = \sum \bp{x^2 - 10x + 25} = 2500 \implies \sum x^2 = 4950.\] Hence, \[\ol{x} = \frac1n \sum x = 7.4 \quad \tand \quad s^2 = \frac1{n-1} \bs{\sum x^2 - \frac1n \bp{\sum x}^2} = 45.152 \tosf{5}.\] Let $\ol{W}$ kg be the mean weight loss by 50 randomly chosen patients. Since the sample size (50) is large, by the Central Limit Theorem, $\ol{W} \sim \Normal{7.4}{45.152}$. Hence, the required probability is $\P{\ol{W} \geq 7} = 0.524 \tosf{3}$.
    \end{ppart}
\end{solution}