\section{Tutorial A15A}

\begin{problem}
    The continuous random variable $X$ has probability density function given by \[f(x) = \begin{cases}
        \frac{k}{1 + x^2}, & -1 \leq x \leq 1,\\
        0, & \ow.
    \end{cases}\]
    Find the value of $k$ and determine
    \begin{enumerate}
        \item the mode of $X$;
        \item the expectation (mean) and variance of $X$;
        \item $F$, the cumulative density function of $X$ and the median of $X$.
    \end{enumerate}
\end{problem}
\begin{solution}
    Since the sum of probabilities is 1, we have \[1 = \int_{-\infty}^{\infty} f(x) \d x = \int_{-1}^1 \frac{k}{1 + x^2} \d x = k \evalint{\arctan x}{-1}{1} = \frac\pi2 k,\] whence $k = 2/\pi$.

    \begin{ppart}
        From the graph of $y = f(x)$, it is clear that $f(x)$ attains a maximum at $x = 0$, whence the mode of $X$ is $0$.
    \end{ppart}
    \begin{ppart}
        Since $f(x)$ is symmetric about $x = 0$, we clearly have $\E{X} = 0$. Hence, \[\Var{X} = \E{X^2} = \int_{-\infty}^\infty x^2 f(x) \d x = \frac2\pi \int_{-1}^1 \frac{x^2}{1 + x^2} \d x = 0.273 \tosf{3}.\]
    \end{ppart}
    \begin{ppart}
        Note that \[\int_{-1}^{x} f(t) \d t = \frac2\pi \int_{-1}^x \frac{\d t}{1 + t^2} = \frac2\pi \evalint{\arctan t}{-1}{x} = \frac2\pi \arctan x + \frac12.\] Hence, $F(x)$, the cdf of $X$, is given by \[F(x) = \begin{cases}
            0, & x < -1,\\
            \frac2\pi \arctan x + \frac12, & -1 \leq x \leq 1,\\
            1, & x > 1.
        \end{cases}\]
        Since $f(x)$ is symmetric about $x = 0$, the median of $X$ is 0.
    \end{ppart}
\end{solution}

\clearpage
\begin{problem}
    $X$ is a continuous random variable, taking values in the interval $0 < x \leq 1$, whose probability density function is given by $f(x) = 2(1 -x)$. Calculate the expectations of $X$, $2X + 1$, and of $X^3$.
\end{problem}
\begin{solution}
    We have
    \begin{align*}
        \E{X} &= \int_{-\infty}^\infty xf(x) \d x = \int_0^1 2x(1-x) \d x = 0.333 \tosf{3}.\\
        \E{2X+1} &= \int_{-\infty}^\infty (2x+1)f(x) \d x = \int_0^1 2(2x+1)(1-x) \d x = 1.67 \tosf{3}.\\
        \E{X^3} &= \int_{-\infty}^\infty x^3 f(x) \d x = \int_0^1 2x^3 (1-x) \d x = 0.1.
    \end{align*}
\end{solution}

\begin{problem}
    The continuous random variable $X$ has cumulative distribution function $F$ given by \[F(x) = \begin{cases}
        0, & x \leq 0,\\
        \sqrt{x}, & 0 < x < 1, \\
        1, & x \geq 1.
    \end{cases}\]

    Find the median of $X$. The probability density function of $X$ is $f$. Write down an expression for $f(x)$ for $0 < x < 1$. Hence,
    \begin{enumerate}
        \item show that $\E{X} = \frac13$;
        \item find $\Var{X}$.
    \end{enumerate}

    Show that the mean of $\sqrt{X}$ and the median of $\sqrt{X}$ are equal.
\end{problem}
\begin{solution}
    Let $m$ be the median of $X$. Then \[F(m) = \sqrt{m} = \frac12 \implies m = \frac14.\] Hence, 1/4 is the median of $X$.

    Differentiating the cdf, we see that the pdf of $X$ is given by \[f(x) = \begin{cases}
        \frac1{2\sqrt{x}}, & 0 < x < 1,\\
        0, & \ow.
    \end{cases}\]
    
    \begin{ppart}
        We have \[\E{X} = \int_{-\infty}^{\infty} x f(x) \d x = \int_0^1 \frac{x}{2\sqrt{x}} \d x = \frac12 \evalint{\frac23 x^{3/2}}01 = \frac13.\]
    \end{ppart}
    \begin{ppart}
        Note that \[\E{X^2} = \int_{-\infty}^{\infty} x^2 f(x) \d x = \int_0^1 \frac{x^2}{2\sqrt{x}} \d x = \frac12 \evalint{\frac25 x^{2/5}}01 = \frac15.\] Hence, \[\Var{X} = \E{X^2} - \E{X}^2 = \frac15 - \bp{\frac13}^2 = \frac4{45}.\]
    \end{ppart}

    Note that \[\E{\sqrt{X}} = \int_{-\infty}^{\infty} \sqrt{x} f(x) \d x = \int_0^1 \frac{\sqrt{x}}{2\sqrt{x}} \d x = \frac12.\] Let $s$ be the median of $\sqrt{X}$. Then \[\frac12 = \P{\sqrt{X} < s} = \P{X < s^2} = \sqrt{s^2} = s.\] Thus, both the mean and median of $\sqrt{X}$ are equal to 1/2.
\end{solution}

\begin{problem}
    The continuous random variable $X$ has probability density function given by \[f(x) = \begin{cases}
        kx, & 0 \leq x \leq 1,\\
        kx^2, & 1 < x \leq 2,\\
        0, & \ow.
    \end{cases}\]
    
    \begin{enumerate}
        \item Show that $k = 6/17$.
        \item Find the cumulative distribution function of $X$.
        \item Find, correct to two decimal places, the median, $m$, of $X$.
        \item Find, correct to two decimal places, $\P{\abs{X - m} < 0.75}$.
    \end{enumerate}
\end{problem}
\begin{solution}
    \begin{ppart}
        Since the probabilities sum to 1, we have \[1 = \int_{-\infty}^\infty f(x) \d x = k \int_0^1 x \d x + k \int_1^2 x^2 \d x = k \evalint{\frac{x^2}2}01 + k\evalint{\frac{x^3}{3}}12 = \frac{17}{6} k,\] whence $k = 6/17$.
    \end{ppart}
    \begin{ppart}
        Let $F$ be the cdf of $X$. For $0 \leq x \leq 1$, \[F(x) = \int_0^x \frac{6}{17} t \d t = \frac3{17} x^2.\] For $1 < x \leq 2$, \[F(x) = F(1) + \int_1^x \frac6{17} t^2 \d t = \frac3{17} + \frac{2}{17} \bp{x^3 - 1} = \frac1{17} \bp{2x^3 + 1}.\] Thus, $F$ is given by \[F(x) = \begin{cases}
            0, & x < 0,\\
            \frac3{17} x^2, & 0 \leq x \leq 1,\\
            \frac1{17} \bp{2x^3 + 1}, & 1 < x \leq 2,\\
            1, & x > 2.
        \end{cases}\]
    \end{ppart}
    \begin{ppart}
        By inspection, $1 < m \leq 2$. Hence, \[\frac12 = F(m) = \frac{1}{17} \bp{2m^3 + 1} \implies m = 1.55 \todp{2}.\]
    \end{ppart}
    \begin{ppart}
        Note that $\abs{X - m} < 0.75$ is equivalent to $0.8 < X < 2.3$. Hence, \[\P{\abs{X - m} < 0.75} = \P{0.8 < X < 2.3} = F(2.3) - F(0.8) = 0.89 \todp{2}.\]
    \end{ppart}
\end{solution}

\begin{problem}
    The continuous random variable $X$ has probability density function defined by \[f(x) = \begin{cases}
        0, & x < 0,\\
        kx, & 0 \leq x \leq 2,\\
        \frac{16k}{x^3}, & \ow.
    \end{cases}\]

    Calculate the value of $k$. Find the median value and the expectation of $X$.
    
    Prove that the standard deviation of $X$ is infinite.
    
    Find the value of $a$ such that $\P{X>a}=0.005$.
    
    Find the cumulative distribution function of $X$ and sketch its graph.
\end{problem}
\begin{solution}
    Since the sum of probabilities is 1, we have \[1 = \int_{-\infty}^\infty f(x) \d x = \int_0^2 kx \d x + \int_2^\infty \frac{16k}{x^3} \d x = \evalint{\frac{kx^2}2}02 + \evalint{\frac{-8k}{x^{2}}}2\infty = 4k.\] Hence, $k = 1/4$.

    By inspection, the median of $X$ is 2. The expectation of $X$ is given by \[\E{X} = \int_{-\infty}^\infty x f(x) \d x = \int_0^2 \frac{x^2}4 \d x + \int_2^\infty \frac{4}{x^2} \d x = 2.67 \tosf{3}.\]

    Observe that $\E{X^2}$ diverges to $\infty$: \[\E{X^2} = \int_{-\infty}^\infty x^2 f(x) \d x = \int_0^2 \frac{x^3}4 \d x + \int_2^\infty \frac{4}{x} \d x > \int_2^\infty \frac{1}{x} \d x \to \infty.\] Hence, the standard deviation $\s$ of $X$ also diverges to $\infty$: \[\s = \sqrt{\Var{X}} = \sqrt{\E{X^2} - \E{X}^2} > \sqrt{\E{X^2}} \to \infty.\]

    Let $F$ be the cdf of $X$. We have $F(x) = 0$ for $x < 0$. For $0 \leq x \leq 2$, \[F(x) = F(0) + \int_0^x \frac{t}{4} \d t = 0 + \evalint{\frac{t^2}{8}}0x = \frac{x^2}{8}.\] For $x > 2$, \[F(x) = F(2) + \int_2^x \frac4{t^3} \d t = \frac{2^2}{8} + \evalint{-\frac{2}{t^2}}2x = 1 - \frac2{x^2}.\] Hence, \[F(x) = \begin{cases}
        0, & x < 0,\\
        \frac{x^2}8, & 0 \leq x \leq 2,\\
        1 - \frac2{x^2}, & x > 2.
    \end{cases}\]
    
    \begin{figure}[H]\tikzsetnextfilename{407}
    \centering
    \begin{tikzpicture}[trim axis left, trim axis right]
        \begin{axis}[
            domain = -2:8,
            ymin = 0,
            ymax = 1.2,
            samples = 101,
            axis y line=middle,
            axis x line=middle,
            xtick = {2},
            ytick = {1, 0.5},
            xlabel = {$x$},
            ylabel = {$F(x)$},
            legend cell align={left},
            legend pos=outer north east,
            after end axis/.code={
                \path (axis cs:0,0) 
                node [anchor=north] {$O$};
                }
            ]
            \addplot[black, very thick, domain=-2:0] {0};
            \addplot[black, very thick, domain=0:2] {x^2 / 8};
            \addplot[black, very thick, domain=2:8] {1 - 2/x^2};
            \addplot[dashed] {1};
            \draw[dashed] (2, 0) -- (2, 0.5) -- (0, 0.5);
        \end{axis}
    \end{tikzpicture}
    \end{figure}
    
    Consider $\P{X > a} = 0.005$, which is equivalent to $\P{X \leq a} = 0.995$. By inspection, $a > 2$. Hence, \[\P{X \leq a} = 1 - \frac{2}{a^2} = 0.995 \implies a = 20.\]
\end{solution}

\begin{problem}
    The continuous random variable $X$ has probability density function given by \[f(x) = \begin{cases}
        cx^2, & 0 \leq x \leq 2,\\
        2c(4-x), & 2 < x \leq 4,\\
        0, & \ow,
    \end{cases}\] where $c$ is a constant.

    \begin{enumerate}
        \item Show that $c=0.15$.
        \item Find the mean of $X$.
        \item Find the lower quartile of $X$.
        \item Find the probability that a single observation of $X$ lies between the lower quartile and the mean.
        \item Three independent observations of $X$ are taken. Find the probability that one of the observations is greater than the mean and the other two are less than the median value of $X$.
    \end{enumerate}
\end{problem}
\begin{solution}
    \begin{ppart}
        Since the sum of probabilities is 1, we have \[1 = \int_{-\infty}^\infty f(x) \d x = \int_0^2 cx^2 \d x + \int_2^4 2c(4-x) \d x = c\evalint{\frac{x^3}{3}}02 + 2c \evalint{4x - \frac{x^2}{2}}24 = \frac{20}3 c,\] whence $c = 3/20 = 0.15$.
    \end{ppart}
    \begin{ppart}
        We have \[\E{X} = \int_{-\infty}^\infty x f(x) \d x = \int_0^2 \frac3{20} x^3 \d x + \int_2^4 \frac{3}{10} x(4-x) \d x = 2.2.\]
    \end{ppart}
    \begin{ppart}
        Let $F$ be the cdf of $X$. Clearly, $F(x) = 0$ for $x < 0$. For $0 \leq x \leq 2$, \[F(x) = F(0) + \int_0^x \frac3{20} t^2 \d t = \frac3{20} \evalint{\frac{t^3}{3}}0x = \frac{x^3}{20}.\] For $2 < x \leq 4$, \[F(x) = F(2) + \int_2^x \frac3{10} (4-t) \d t = \frac{2^3}{20} + \frac3{10} \evalint{4t - \frac{t^2}2}2x = \frac{-3x^2 + 24x - 28}{20}.\] Finally, $F(x) = 1$ for $x > 4$. Thus, \[F(x) = \begin{cases}
            0, & x < 0,\\
            \frac{1}{20} x^3, & 0 \leq x \leq 2,\\
            \frac1{20} (-3x^2 + 24x - 28), & 2 < x \leq 4,\\
            1, & x > 4.
        \end{cases}\]

        Let $l$ be the lower quartile of $X$. Then $\P{X < l} = \frac14$. By inspection, $0 \leq l \leq 2$. Hence, \[\P{X < l} = \frac{l^3}{20} = \frac14 \implies l = \sqrt[3]{5}.\]
    \end{ppart}
    \begin{ppart}
        We have \[\P{\sqrt[3]{5} < X < 2.2} = F(2.2) - F\of{\sqrt[3]{5}} = 0.264 \tosf{3}.\]
    \end{ppart}
    \begin{ppart}
        By definition, the probability that an observation of $X$ is less than the median value is 1/2. Hence, the required probability is simply \[\comb{3}{1} \P{X > 2.2} \bp{\frac12}^2 = \comb{3}{1} \bs{1 - \P{X < 2.2}} \bp{\frac12}^2 = 0.365 \tosf{3}.\]
    \end{ppart}
\end{solution}

\begin{problem}
    The cumulative distribution function of a continuous random variable $X$ is given by \[F(x) = \begin{cases}
        0, & x < -2,\\
        k\bp{4x - \frac{x^3}3 + \frac{16}{3}}, & -2 \leq x \leq 2,\\
        1, & x > 2.
    \end{cases}\]

    Find
    \begin{enumerate}
        \item the value of $k$;
        \item the probability density function for $X$;
        \item the mean and variance of $X$.
    \end{enumerate}
\end{problem}
\begin{solution}
    \begin{ppart}
        Since $F$ is continuous, we have $F(2) = F\of{2^+}$, i.e. \[k \bp{4(2) - \frac{2^3}3 + \frac{16}3} = 1 \implies k = \frac3{32}.\]
    \end{ppart}
    \begin{ppart}
        Note that \[\der{}{x} \bs{\frac3{32}\bp{4x - \frac{x^3}3 + \frac{16}{3}}} = \frac{3}{32} \bp{4 - x^2}.\] Hence, \[f(x) = \begin{cases}
            \frac3{32} (4 - x^2), & -2 \leq x \leq 2,\\
            0, & \ow.
        \end{cases}\]
    \end{ppart}
    \begin{ppart}
        Observe that $x f(x)$ is an odd function (symmetric about $x = 0$). Hence, $\E{X} = 0$. Thus, \[\Var{X} = \E{X^2} = \int_{-\infty}^\infty x^2 f(x) \d x = \int_{-2}^2 \frac3{32} x^2 \bp{4 - x^2} \d x = 0.8.\]
    \end{ppart}
\end{solution}

\begin{problem}
    A continuous random variable $X$ has distribution function given by \[F(x) = \begin{cases}
        0, & x < 0,\\
        \frac1{12} x^2, & 0 \leq x \leq 3,\\
        -\frac14 x^2 + 2x - 3, & 3 < x \leq 4,\\
        1, & x > 4.
    \end{cases}\]

    \begin{enumerate}
        \item Show that the median is $\sqrt{6}$.
        \item Find the corresponding probability density function, and show that the mean is $\frac{7}{3}$.
        \item Find the value of $k$ such that $\P{X < k} = \P{3 < X < 4}$.
    \end{enumerate}
\end{problem}
\begin{solution}
    \begin{ppart}
        Let $m$ be the median of $X$. By inspection, $0 \leq m \leq 3$. Thus, \[\P{X < m} = \frac1{12} m^2 = \frac12 \implies m = \sqrt{6}.\]
    \end{ppart}
    \begin{ppart}
        Differentiating $F$, we get \[f(x) = \begin{cases}
            \frac{x}6, & 0 \leq x \leq 3,\\
            -\frac{x}2 + 2, & 3 < x \leq 4,\\
            0, & \ow.
        \end{cases}\]
        Hence, \[\E{X} = \int_{-\infty}^{\infty} x f(x) \d x = \int_0^3 \frac{x^2}6 \d x + \int_3^4 x\bp{-\frac{x}{2} + 2} \d x = \evalint{\frac{x^3}{18}}03 + \evalint{-\frac{x^3}6 + x^2}34 = \frac73.\]
    \end{ppart}
    \begin{ppart}
        Note that \[\P{3 < X < 4} = F(4) - F(3) = 1 - \frac{3^2}{12} = \frac14.\] Hence, by inspection, $0 \leq k \leq 3$. Thus, \[\P{X < k} = \frac{k^2}{12} = \frac14 = \P{3 < X < 4} \implies k = \sqrt{3}.\]
    \end{ppart}
\end{solution}

\begin{problem}
    The continuous random variable $X$ has probability density function given by \[f(x) = \begin{cases}
        \frac{k}{(x+1)^4}, & x \geq 0,\\
        0, & x < 0,
    \end{cases}\] where $k$ is a constant.

    \begin{enumerate}
        \item Show that $k=3$, and find the cumulative distribution function. Find also the value of $x$ such that $\P{X \leq x}= 7/8$.
        \item Find $\E{X + 1}$, and deduce that $\E{X} = 1/2$.
        \item By considering $\Var{X+1}$, or otherwise, find $\Var{X}$.
    \end{enumerate}
\end{problem}
\begin{solution}
    \begin{ppart}
        Note that \[\int \frac1{(x+1)^4} \d x = -\frac1{3 (x+1)^3}.\] Since the probabilities sum to 1, \[1 = \int_{-\infty}^{\infty} f(x) \d x = k \int_0^\infty \frac{1}{(x+1)^4} \d x = k \evalint{-\frac{1}{3(x+1)^3}}0\infty = \frac{k}3 \implies k = 3.\]

        Let $F$ be the cdf of $X$. Clearly, $F(x) = 0$ for $x < 0$. For $x \geq 0$, we have \[F(x) = F(0) + \int_0^x \frac3{(t+1)^4} \d t = 0 + 3 \evalint{-\frac1{3(t+1)^3}}0x = 1 - \frac{1}{(x+1)^3}.\] Thus, \[F(x) = \begin{cases}
            0, & x < 0,\\
            1 - \frac1{(x+1)^3}, & x \geq 0.
        \end{cases}\]

        Consider $\P{X \leq x} = 7/8$: \[\P{X \leq x} = 1 - \frac1{(x+1)^3} = \frac78 \implies x = 1.\]
    \end{ppart}
    \begin{ppart}
        We have \[\E{X + 1} = \int_{-\infty}^\infty (x+1) f(x) \d x = 3\int_0^\infty \frac{1}{(x+1)^3} \d x = 3\evalint{-\frac1{2(x+1)^2}}0\infty = \frac32.\] Thus, \[\E{X} = \E{(X+1) - 1} = \E{X + 1} - \E{1} = \frac32 - 1 = \frac12.\]
    \end{ppart}
    \begin{ppart}
        Consider $\E{(X+1)^2}$: \[\E{(X+1)^2} = \int_{-\infty}^\infty (x+1)^2 f(x) \d x = 3 \int_0^\infty \frac1{(x+1)^2} \d x = 3\evalint{-\frac1{x+1}}0\infty = 3.\] Thus, \[\Var{X} = \Var{X + 1} = \E{(X+1)^2} - \E{X+1}^2 = 3 - \bp{\frac32}^2 = \frac34.\]
    \end{ppart}
\end{solution}

\begin{problem}
    The probability that a randomly chosen flight from Stanston Airport is delayed by more than $x$ hours is $\frac{1}{100} (x-10)^2$, $x\in\RR$, $0\leq x\leq10$. No flights leave early, and none is delayed for more than $10$ hours. The delay, in hours, for a randomly chosen flight is denoted by $X$.

    \begin{enumerate}
        \item Find the median, $m$, of $X$, correct to three significant figures.
        \item Find the cumulative distribution function, $F$, of $X$ and sketch the graph of $F$.
        \item Find the probability distribution function, $f$, of $X$ and sketch the graph of $f$.
        \item Show that $\E{X} = 10/3$.
    \end{enumerate}

    A random sample of 2 flights is taken. Find the probability that both flights are delayed by more than $m$ hours, where $m$ is the median of $X$.
\end{problem}
\begin{solution}
    \begin{ppart}
        Note that \[\P{X > x} = \begin{cases}
            \frac1{100} (x-10)^2, & 0 \leq x \leq 10,\\
            0, & \ow.
        \end{cases}\] Thus, \[\frac12 = \P{X > m} = \frac1{100} (m-10)^2 \implies m = 2.93 \tosf{3}.\] Note that we reject $m = 17.1$ since $0 \leq m \leq 10$.
    \end{ppart}
    \begin{ppart}
        We have \[F(x) = \P{X \leq x} = 1 - \P{X > x} = \begin{cases}
            0, & x < 0,\\
            1 - \frac1{100} (x-10)^2, & 0 \leq x \leq 10,\\
            1, & x > 10.
        \end{cases}\]

        \begin{figure}[H]\tikzsetnextfilename{408}
        \centering
        \begin{tikzpicture}[trim axis left, trim axis right]
            \begin{axis}[
                domain = -2:12,
                ymin=0,
                ymax=1.2,
                samples = 101,
                axis y line=middle,
                axis x line=middle,
                xtick = {2.93, 10},
                xticklabels = {$m$, 10},
                ytick = {1, 0.5},
                xlabel = {$x$},
                ylabel = {$F(x)$},
                legend cell align={left},
                legend pos=outer north east,
                after end axis/.code={
                    \path (axis cs:0,0) 
                    node [anchor=north] {$O$};
                    }
                ]
                \addplot[black, very thick, domain=-2:0] {0};
                \addplot[black, very thick, domain=0:10] {1 - 0.01 * (x-10)^2};
                \addplot[black, very thick, domain=10:12] {1};
                \draw[dashed] (2.93, 0) -- (2.93, 0.5) -- (0, 0.5);
                \draw[dashed] (10, 0) -- (10, 1);
                \addplot[black, dashed] {1};
            \end{axis}
        \end{tikzpicture}
        \end{figure}
    \end{ppart}
    \begin{ppart}
        Differentiating $F$, we get \[f(x) = \begin{cases}
            -\frac1{50} (x-10), & 0 \leq x \leq 10,\\
            0, & \ow.
        \end{cases}\]

        \begin{figure}[H]\tikzsetnextfilename{409}
        \centering
        \begin{tikzpicture}[trim axis left, trim axis right]
            \begin{axis}[
                ymin = -0.03,
                ymax = 0.3,
                samples = 101,
                axis y line=middle,
                axis x line=middle,
                xtick = {10},
                ytick = {0.2},
                xlabel = {$x$},
                ylabel = {$f(x)$},
                legend cell align={left},
                legend pos=outer north east,
                after end axis/.code={
                    \path (axis cs:0,0) 
                    node [anchor=north east] {$O$};
                    }
                ]
                \addplot[black, very thick, domain=0:10] {-0.02 * (x - 10)};
                \addplot[black, very thick, domain=-2:0] {0};
                \addplot[black, very thick, domain=10:12] {0};
                \fill (0, 0.20) circle[radius=2.5pt];
                \draw (0, 0) circle[radius=2.5pt];
            \end{axis}
        \end{tikzpicture}
        \end{figure}
    \end{ppart}
    \begin{ppart}
        We have \[\E{X} = \int_{-\infty}^\infty x f(x) \d x = -\frac1{50} \int_0^{10} x(x-10) \d x = -\frac1{50} \evalint{\frac{x^3}{3} - 5x^2}0{10} = \frac{10}{3}.\]
    \end{ppart}

    The required probability is \[\P{X > m}^2 = \bp{\frac12}^2 = \frac14.\]
\end{solution}

\begin{problem}
    The continuous random variable $X$ has a probability density function \[f(x) = \begin{cases}
        \frac2\pi, & 0 \leq x \leq \frac\pi2,\\
        0, & \ow.
    \end{cases}\]
    
    The random variable $Y$ is defined by $Y=\cos X.$ Find $\E{Y},$ and show that $\Var{Y} = \frac12 - \frac4{\pi^2}$. Find the median of $Y$.
\end{problem}
\begin{solution}
    We have \[\E{Y} = \E{\cos X} = \frac2\pi \int_0^{\pi/2} \cos{x} \d x = \frac2\pi \evalint{\sin x}0{\pi/2} = \frac2\pi.\] Similarly,
    \begin{gather*}
        \E{Y^2} = \E{\cos^2 X} = \frac2\pi \int_0^{\pi/2} \cos[2]{x} \d x = \frac2\pi \int_0^{\pi/2} \frac{1 + \cos{2x}}{2} \d x\\
        = \frac1\pi \evalint{x + \frac{\sin 2x}2}0{\pi/2} = \frac12.
    \end{gather*}
    Thus, \[\Var{Y} = \E{Y^2} - \E{Y}^2 = \frac12 - \bp{\frac2\pi}^2 = \frac12 - \frac4{\pi^2}.\]

    Let $m$ be the median of $Y$. We have \[\frac12 = \P{Y < m} = \P{\cos x < m} = \P{x > \arccos m} = 1 - \frac2\pi \arccos m,\] whence $m = \cos{\pi/4} = 1/\sqrt2$.
\end{solution}

\begin{problem}
    A random variable $X$ has probability density function \[f(x) = \begin{cases}
        \frac2{x^2}, & 1 \leq x \leq 2,\\
        0, & \ow,
    \end{cases}\] and the random variable $Y$ is defined by $Y=4X^{3}$. Find
    \begin{enumerate}
        \item the mean and variance of $Y$;
        \item $\P{10 < Y < 20}$;
        \item the median of $Y$.
    \end{enumerate}
\end{problem}
\begin{solution}
    \begin{ppart}
        We have \[\E{Y} = \E{4X^3} = \int_1^2 8x \d x = \evalint{4x^2}12 = 12\] and \[\E{Y^2} = \E{16X^6} = \int_1^2 32x^4 \d x = \evalint{\frac{32}{5} x^5}12 = \frac{992}{5}.\] Thus, \[\Var{Y} = \E{Y^2} - \E{Y}^2 = \frac{992}{5} - 12^2 = \frac{272}{5}.\]
    \end{ppart}
    \begin{ppart}
        Note that \[F(x) = \begin{cases}
            0, & x < 1,\\
            2-\frac{2}{x}, & 1 \leq x \leq 2,\\
            1, & x > 2.
        \end{cases}\] Thus,
        \begin{gather*}
            \P{10 < Y < 20} = \P{10 < 4X^3 < 20} = \P{\sqrt[3]{5/2} < X < \sqrt[3]{5}}\\
            = F\of{\sqrt[3]{5}} - F\of{\sqrt[3]{5/2}} = 0.304 \tosf{3}.
        \end{gather*}
    \end{ppart}
    \begin{ppart}
        Let $m$ be the median of $Y$. We have \[\frac12 = \P{Y < m} = \P{4X^3 < m} = \P{X < \sqrt[3]{m/4}} = 2 - \frac2{\sqrt[3]{m/4}}.\] Solving, we get $m = 9.48 \tosf{3}$.
    \end{ppart}
\end{solution}