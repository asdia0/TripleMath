\section{Tutorial B8}

\begin{problem}
    Write down the integral for the area of the shaded region for each of the figure below and use the GC to evaluate it, to 3 significant figures.

    \begin{enumerate}
        \item \begin{center}\tikzsetnextfilename{305}
            \begin{tikzpicture}[trim axis left, trim axis right]
                \begin{axis}[
                    axis on top,
                    domain = -1.3:1.3,
                    samples = 101,
                    axis y line=middle,
                    axis x line=middle,
                    xtick = \empty,
                    ytick = \empty,
                    xlabel = {$x$},
                    ylabel = {$y$},
                    legend cell align={left},
                    legend pos=outer north east,
                    after end axis/.code={
                        \path (axis cs:0,0) 
                            node [anchor=north east] {$O$};
                        }
                    ]
                    \addplot[plotRed, name path=f1] {x^3 + 2*x^2 - x - 2};
                    \addlegendentry{$y = x^3 + 2x^2 - x -2$};
                    \addplot[thin, name path=null] {0};
                    \addplot[color=black!20] fill between[of=null and f1,soft clip={domain=-1:1}];
                \end{axis}
            \end{tikzpicture}
        \end{center}
        \item \begin{center}\tikzsetnextfilename{306}
            \begin{tikzpicture}[trim axis left, trim axis right]
                \pgfdeclarelayer{pre main}
                \pgfsetlayers{pre main,main}
                \begin{axis}[
                    axis on top,
                    domain = -2.5:2.5,
                    samples = 101,
                    axis y line=middle,
                    axis x line=middle,
                    xtick = {2},
                    ytick = {4},
                    xlabel = {$x$},
                    ylabel = {$y$},
                    legend cell align={left},
                    legend pos=outer north east,
                    after end axis/.code={
                        \path (axis cs:0,0) 
                            node [anchor=north east] {$O$};
                        }
                    ]
                    \pgfonlayer{pre main}
                    \fill[black!20] (0,0) rectangle (2, 4);
                    \endpgfonlayer
                    \addplot[plotRed, name path=f1] {x^2};
                    \addlegendentry{$y = x^2$};
                    \addplot[thin, name path=null] {0};
                    \addplot[color=white] fill between[of=null and f1,soft clip={domain=0:2}];
                    \draw[dotted] (2, 0) -- (2, 4);
                \end{axis}
            \end{tikzpicture}
        \end{center}
        \item \begin{center}\tikzsetnextfilename{307}
            \begin{tikzpicture}[trim axis left, trim axis right]
                \pgfdeclarelayer{pre main}
                \pgfsetlayers{pre main,main}
                \begin{axis}[
                    axis on top,
                    domain = -1:2.5,
                    samples = 101,
                    axis y line=middle,
                    axis x line=middle,
                    xtick = {2},
                    ytick = \empty,
                    xlabel = {$x$},
                    ylabel = {$y$},
                    legend cell align={left},
                    legend pos=outer north east,
                    after end axis/.code={
                        \path (axis cs:0,0) 
                            node [anchor=north east] {$O$};
                        }
                    ]
                    \pgfonlayer{pre main}
                    \fill[black!20] (0,0) rectangle (2, e^4);
                    \endpgfonlayer
                    \addplot[plotRed, name path=f1] {e^(2*x)};
                    \addlegendentry{$y = \e^{2x}$};
                    \addplot[thin, name path=null] {0};
                    \addplot[color=white] fill between[of=null and f1,soft clip={domain=0:2}];
                    \draw[dotted] (2, 0) -- (2, e^4);
                \end{axis}
            \end{tikzpicture}
        \end{center}
        \item \begin{center}\tikzsetnextfilename{308}
            \begin{tikzpicture}[trim axis left, trim axis right]
                \pgfdeclarelayer{pre main}
                \pgfsetlayers{pre main,main}
                \begin{axis}[
                    axis on top,
                    domain = -pi:pi,
                    samples = 101,
                    axis y line=middle,
                    axis x line=middle,
                    xtick = {-pi/2, pi/2},
                    xticklabels={$-\frac\pi2$, $\frac\pi2$},
                    ytick = \empty,
                    xlabel = {$x$},
                    ylabel = {$y$},
                    ymax=2.5,
                    ymin=-2.5,
                    legend cell align={left},
                    legend pos=outer north east,
                    after end axis/.code={
                        \path (axis cs:0,0) 
                            node [anchor=north east] {$O$};
                        }
                    ]
                    \pgfonlayer{pre main}
                    \fill[black!20] (-pi/2, -2) rectangle (0, 0);
                    \fill[black!20] (pi/2, 2) rectangle (0, 0);
                    \endpgfonlayer
                    \addplot[plotRed, name path=f1] {2*sin(\x r)};
                    \addlegendentry{$y = 2\sin x$};
                    \addplot[thin, name path=null] {0};
                    \addplot[color=white] fill between[of=null and f1,soft clip={domain=-pi/2:pi/2}];
                    \draw[dotted] (-pi/2, 0) -- (-pi/2, -2);
                    \draw[dotted] (pi/2, 0) -- (pi/2, 2);
                \end{axis}
            \end{tikzpicture}
        \end{center}
        \item \begin{center}\tikzsetnextfilename{309}
            \begin{tikzpicture}[trim axis left, trim axis right]
                \begin{axis}[
                    axis on top,
                    domain = -2.5:2.5,
                    samples = 101,
                    axis y line=middle,
                    axis x line=middle,
                    xtick = {-2, 2},
                    xticklabels={$-2$, $2$},
                    ytick = \empty,
                    xlabel = {$x$},
                    ylabel = {$y$},
                    legend cell align={left},
                    legend pos=outer north east,
                    after end axis/.code={
                        \path (axis cs:0,0) 
                            node [anchor=north east] {$O$};
                        }
                    ]
                    \addplot[plotRed, name path=f1] {x^2};
                    \addplot[plotBlue, name path=f2] {8-x^2};
                    \addlegendentry{$y = x^2$};
                    \addlegendentry{$y = 8-x^2$};
                    \addplot[thin, name path=null] {0};
                    \addplot[color=black!20] fill between[of=null and f2,soft clip={domain=-2:2}];
                    \addplot[color=white] fill between[of=null and f1,soft clip={domain=-2:2}];
                    \draw[dotted] (-2, 0) -- (-2, 4);
                    \draw[dotted] (2, 0) -- (2, 4);
                \end{axis}
            \end{tikzpicture}
        \end{center}
        \item \begin{center}\tikzsetnextfilename{310}
            \begin{tikzpicture}[trim axis left, trim axis right]
                \begin{axis}[
                    axis on top,
                    domain = -0.5:2.5,
                    samples = 101,
                    axis y line=middle,
                    axis x line=middle,
                    xtick = {1},
                    ytick = \empty,
                    xlabel = {$x$},
                    ylabel = {$y$},
                    ymax=5,
                    legend cell align={left},
                    legend pos=outer north east,
                    after end axis/.code={
                        \path (axis cs:0,0) 
                            node [anchor=north east] {$O$};
                        }
                    ]
                    \addplot[plotRed, name path=f1] {4-x^2};
                    \addplot[plotBlue, name path=f2] {3*x};
                    \addlegendentry{$y = 4-x^2$};
                    \addlegendentry{$y = 3x$};
                    \addplot[thin, name path=null] {0};
                    \addplot[color=black!20] fill between[of=null and f1,soft clip={domain=0:1}];
                    \addplot[color=white] fill between[of=null and f2,soft clip={domain=0:1}];
                    \draw[dotted] (1, 0) -- (1, 3);
                \end{axis}
            \end{tikzpicture}
        \end{center}
        \item \begin{center}\tikzsetnextfilename{311}
            \begin{tikzpicture}[trim axis left, trim axis right]
                \begin{axis}[
                    axis on top,
                    domain = -2:2,
                    samples = 101,
                    axis y line=middle,
                    axis x line=middle,
                    xtick = {-1, 1},
                    ytick = \empty,
                    xlabel = {$x$},
                    ylabel = {$y$},
                    legend cell align={left},
                    legend pos=outer north east,
                    after end axis/.code={
                        \path (axis cs:0,0) 
                            node [anchor=north east] {$O$};
                        }
                    ]
                    \addplot[plotRed, name path=f1] {9-x^2};
                    \addplot[plotBlue, name path=f2] {e^x};
                    \addlegendentry{$y = 9-x^2$};
                    \addlegendentry{$y = \e^x$};
                    \addplot[thin, name path=null] {0};
                    \addplot[color=black!20] fill between[of=null and f1,soft clip={domain=-1:1}];
                    \addplot[color=white] fill between[of=null and f2,soft clip={domain=-1:1}];
                    \draw[dotted] (-1, 0) -- (-1, 8);
                    \draw[dotted] (1, 0) -- (1, 8);
                \end{axis}
            \end{tikzpicture}
        \end{center}
        \item \begin{center}\tikzsetnextfilename{312}
            \begin{tikzpicture}[trim axis left, trim axis right]
                \begin{axis}[
                    axis on top,
                    domain = 0:1.1,
                    samples = 101,
                    axis y line=middle,
                    axis x line=middle,
                    xtick = {1},
                    ytick = {1},
                    xlabel = {$x$},
                    ylabel = {$y$},
                    legend cell align={left},
                    legend pos=outer north east,
                    after end axis/.code={
                        \path (axis cs:0,0) 
                            node [anchor=north] {$O$};
                        }
                    ]
                    \addplot[plotRed, name path=f2] {sqrt(x)};
                    \addplot[plotBlue, name path=f1] {x^2};
                    \addlegendentry{$y = \sqrt{x}$};
                    \addlegendentry{$y = x^2$};
                    \addplot[thin, name path=null] {0};
                    \addplot[color=black!20] fill between[of=null and f2,soft clip={domain=0:1}];
                    \addplot[color=white] fill between[of=null and f1,soft clip={domain=0:1}];
                    \draw[dotted] (1, 0) -- (1, 1);
                \end{axis}
            \end{tikzpicture}
        \end{center}
    \end{enumerate}
\end{problem}
\begin{solution}
    \begin{ppart}
        \[\area = -\int_{-1}^1 \bp{x^3 + 2x^2 - x - 2} \d x = 2.67 \units[2] \tosf{3}.\]
    \end{ppart}
    \begin{ppart}
        Note that $y = x^2 \implies x = \sqrt{y}$. \[\area = \int_0^4 \sqrt{y} \d y = 5.33 \units[2] \tosf{3}.\]
    \end{ppart}
    \begin{ppart}
        Note that $y = \e^{2x} \implies x = \frac12 \ln y$. Also, when $x = 0$, we have $y = 1$. Further, when $x = 2$, we have $y = \e^4$. Thus, \[\area = \int_0^{\e^4} \frac12 \ln y \d y = 82.4 \units[2] \tosf{3}.\]
    \end{ppart}
    \begin{ppart}
        Note that when $x = \pi/2$, we have $y = 2$. Thus, \[\area = 2 \int_0^2 \arcsin \frac{y}2 \d y = 2.28 \units[2] \tosf{3}.\]
    \end{ppart}
    \begin{ppart}
        \[\area = \int_{-2}^2 \bs{\bp{8-x^2} - x^2} \d x = 21.3 \units[2] \tosf{3}.\]
    \end{ppart}
    \begin{ppart}
        \[\area = \int_0^1 \bs{\bp{4-x^2}-3x} \d x = 2.17 \units[2] \tosf{3}.\]
    \end{ppart}
    \begin{ppart}
        \[\area = \int_{-1}^1 \bs{\bp{9-x^2} - \e^x} \d x = 15.0 \units[2] \tosf{3}.\]
    \end{ppart}
    \begin{ppart}
        \[\area = \int_0^1 \bp{\sqrt{x} - x^2} \d x = 0.333 \units[2] \tosf{3}.\]
    \end{ppart}
\end{solution}

\begin{problem}
    \begin{enumerate}
        \item Write down the integral for the volume of the solid generated when the shaded region is rotated about the $x$-axis through $2\pi$ for questions 1(a), (e), (f) and (h) using the disc method and use the GC to evaluate it.
        \item Write down the integral for the volume of the solid generated when the shaded region is rotated about the $y$-axis through $2\pi$ for questions 1(b), (d) and (f) using the disc method and use the GC to evaluate it.
    \end{enumerate}
\end{problem}
\begin{solution}
    \begin{ppart}
        \begin{psubpart}
            \[\volume = \pi \int_{-1}^1 \bp{x^3 + 2x^2 - x - 2}^2 \d x = 13.9 \units[3] \tosf{3}.\]
        \end{psubpart}
        \begin{psubpart}
            \[\volume = \pi \int_{-2}^2 \bs{\bp{8-x^2}^2 - x^2} \d x = 536 \units[3] \tosf{3}.\]
        \end{psubpart}
        \begin{psubpart}
            \[\volume = \pi \int_0^1 \bs{\bp{4-x^2}^2 - \bp{3x}^2} \d x = 33.1 \units[3] \tosf{3}.\]
        \end{psubpart}
        \begin{psubpart}
            \[\volume = \pi \int_0^1 \bs{\bp{\sqrt x}^2 - \bp{x^2}^2} \d x = 0.942 \units[3] \tosf{3}.\]
        \end{psubpart}
    \end{ppart}
    \begin{ppart}
        \begin{psubpart}
            \[\volume = \pi \int_0^4 \bp{\sqrt{y}}^2 \d y = 25.1 \units[3] \tosf{3}.\]
        \end{psubpart}
        \begin{psubpart}
            \[\volume = 2\pi \int_0^2 \arcsin^2 \frac{y}2 \d y = 5.87 \units[3] \tosf{3}.\]
        \end{psubpart}
        \begin{psubpart}
            \[\volume = \pi \int_3^4 (4-y) \d y + \frac{\pi\bp{1^2}\bp{3}}3 = 4.71 \units[3] \tosf{3}.\]
        \end{psubpart}
    \end{ppart}
\end{solution}

\begin{problem}
    \begin{enumerate}
        \item Write down the integral for the volume of the solid generated when the shaded region is rotated about the $x$-axis through $2\pi$ for questions 1(e), (f) and (h) using the shell method and use the GC to evaluate it.
        \item Write down the integral for the volume of the solid generated when the shaded region is rotated about the $y$-axis through $2\pi$ for questions 1(b), (d) and (f) using the shell method and use the GC to evaluate it.
    \end{enumerate}
\end{problem}
\begin{solution}
    \begin{ppart}
        \begin{psubpart}
            Note that $y = x^2 \implies x = \sqrt{y}$ and $y = 8 - x^2 \implies x = \sqrt{8 - y}$ for $x > 0$. Thus, \[\volume = 2\bp{2\pi \int_0^4 \sqrt{y}\cdot y \d y + 2\pi \int_4^8 \sqrt{8-y}\cdot y \d y} = 536 \units[3] \tosf{3}.\]
        \end{psubpart}
        \begin{psubpart}
            Note that $y = 3x \implies x = y/3$ and $y = 4 - x^2 \implies x = \sqrt{4-y}$ for $x > 0$. Thus, \[\volume = 2\pi\int_0^3 \frac13 y \cdot y \d y + 2\pi\int_3^4 \sqrt{4-y}\cdot y \d y = 33.1 \units[3] \tosf{3}.\]
        \end{psubpart}
        \begin{psubpart}
            Note that $y = \sqrt{x} \implies x = y^2$ and $y = x^2 \implies x = \sqrt{y}$ for $x > 0$. Thus, \[\volume = 2\pi\int_0^1 \bp{\sqrt{y} - y^2}y \d y = 0.942 \units[3] \tosf{3}.\]
        \end{psubpart}
    \end{ppart}
    \begin{ppart}
        \begin{psubpart}
            \[\volume = 2\pi\int_0^2 x\cdot x^2 \d x = 25.1 \units[3] \tosf{3}.\]
        \end{psubpart}
        \begin{psubpart}
            \[\volume = 2 \cdot 2\pi\int_0^{\pi/2} x\bp{2 - 2\sin x} \d x = 5.87 \units[3] \tosf{3}.\]
        \end{psubpart}
        \begin{psubpart}
            \[\volume = 2\pi\int_0^1 x\bs{\bp{4-x^2} - 3x} \d x = 4.71 \units[3] \tosf{3}.\]
        \end{psubpart}
    \end{ppart}
\end{solution}

\begin{problem}
    Calculate the area enclosed by the petals of the curve $r = \sin2\t$ where $r \geq 0$.
\end{problem}
\begin{solution}
    Note that $r \geq 0 \implies \sin2\t \geq 0 \implies r \in \left[0, \frac\pi2\right] \cup \left[\pi, \frac{3\pi}2\right]$. Thus, \[\area = 2 \cdot \frac12 \int_0^{\pi/2} \sin^2 2\t \d \t = \int_0^{\pi/2} \frac{1 - \cos 4\t}{2} \d \t = \frac12 \evalint{\t - \frac{\sin 4\t}{4}}0{\pi/2}= \frac{\pi}4 \units[2].\]
\end{solution}

\begin{problem}
    The finite region $A$ is bounded by the curve $y = x^2$ and a minor arc of the circle $x^2 + y^2 = 12$.

    \begin{enumerate}
        \item Find the numerical value of the area of $A$, correct to 2 decimal places.
        \item Find the exact volume of the solid obtained when $A$ is rotated about the $x$-axis through $2\pi$ radians.
        \item Find the exact volume of the solid obtained when $A$ is rotated about the $y$-axis through $\pi$ radians.
    \end{enumerate}
\end{problem}
\begin{solution}
    \begin{center}\tikzsetnextfilename{313}
        \begin{tikzpicture}[trim axis left, trim axis right]
            \begin{axis}[
                axis on top,
                domain = -sqrt(12):sqrt(12),
                samples = 101,
                axis y line=middle,
                axis x line=middle,
                xtick = \empty,
                ytick = \empty,
                xlabel = {$x$},
                ylabel = {$y$},
                ymax=1.75*sqrt(12),
                ymin=0,
                legend cell align={left},
                legend pos=outer north east,
                after end axis/.code={
                    \path (axis cs:0,0) 
                        node [anchor=north] {$O$};
                    }
                ]
                \addplot[plotRed, name path=f1] {x^2};
    
                \addlegendentry{$y = x^2$};
    
                \addplot[plotBlue, name path=f2] {sqrt(12 - x^2)};
    
                \addlegendentry{$x^2 + y^2 = 12$};

                \addplot[color=black!20] fill between[of=f1 and f2,soft clip={domain=0:1.73}];

                \addplot[color=black!20] fill between[of=f2 and f1,soft clip={domain=-1.73:0}];

                \node[anchor=west] at (0, 1.73) {$A$};
            \end{axis}
        \end{tikzpicture}
    \end{center}

    \begin{ppart}
        Consider the intersections between $y = x^2$ and $x^2 + y^2 = 12$.
        \begin{gather*}
            x^2 + y^2 = x^2 + \bp{x^2}^2 = 12 \implies x^4 + x^2 - 12 = (x^2 - 3)(x^2 + 4) = 0 \\
            \implies \bp{x - \sqrt3}\bp{x+\sqrt3}\bp{x^2 + 4} = 0.
        \end{gather*}
        Hence, the two curves intersect at $x = -\sqrt3$ and $x = \sqrt3$. Note that $x^2 + 4 = 0$ has no solution since $x^2 + 4 > 0$. Also note that $x^2 + y^2 = 12 \implies y = \sqrt{12 - x^2}$ for $y > 0$. Thus, \[\area = 2\int_0^{\sqrt3} \bp{\sqrt{12-x^2} - x^2} \d x = 8.02 \units[2] \tosf{3}.\]
    \end{ppart}
    \begin{ppart}
        Note that $x^2 + y^2 = 12 \implies y^2 = 12 - x^2$.
        \begin{align*}
            \volume &= 2\pi\int_0^{\sqrt3} \bs{\bp{12 - x^2} - \bp{x^2}^2} \d x = 2\pi \evalint{12x - \frac{x^3}3- \frac{x^5}5}{0}{\sqrt3} = \frac{92\sqrt{3}\pi}5 \units[3].
        \end{align*}
    \end{ppart}
    \begin{ppart}
        Note that when the curves intersect at $x = \sqrt{3}$, we have $y = 3$. Furthermore, when $x = 0$, we have $y = \sqrt{12}$. Also note that $x^2 + y^2 = 12 \implies x^2 = 12 - y^2$. \[\volume = \pi\int_0^{3} y \d y + \pi \int_{3}^{\sqrt{12}} \bp{12 - y^2} \d y = \pi\bp{16\sqrt{3} - \frac{45}2} \units[3].\]
    \end{ppart}
\end{solution}

\begin{problem}
    \begin{center}\tikzsetnextfilename{314}
        \begin{tikzpicture}[trim axis left, trim axis right]
            \begin{axis}[
                domain = -1:2,
                samples = 101,
                axis y line=middle,
                axis x line=middle,
                xtick = {0.2, 0.4, 0.6, 1.4, 1.6},
                xticklabels = {$\frac1n$, $\frac2n$, $\frac3n$, $\frac{n-2}n$, $\quad\frac{n-1}n$},
                ytick = \empty,
                xlabel = {$x$},
                ylabel = {$y$},
                legend cell align={left},
                legend pos=outer north east,
                after end axis/.code={
                    \path (axis cs:0,0.5) 
                        node [anchor=north] {$O$};
                    }
                ]
                \addplot[plotRed, name path=f1] {2^x};
    
                \addlegendentry{$y = 2^x$};

                \draw (0, 0) -- (0, 2);
                \draw (0.2, 0) -- (0.2, 1);
                \draw (0, 1) -- (0.2, 1);

                \draw (0.2, 0) -- (0.2, 2^0.2);
                \draw (0.4, 0) -- (0.4, 2^0.2);
                \draw (0.2, 2^0.2) -- (0.4, 2^0.2);

                \draw (0.4, 0) -- (0.4, 2^0.4);
                \draw (0.6, 0) -- (0.6, 2^0.4);
                \draw (0.4, 2^0.4) -- (0.6, 2^0.4);

                \draw (1.4, 0) -- (1.4, 2^1.4);
                \draw (1.6, 0) -- (1.6, 2^1.4);
                \draw (1.4, 2^1.4) -- (1.6, 2^1.4);

                \draw (1.6, 0) -- (1.6, 2^1.6);
                \draw (1.8, 0) -- (1.8, 2^1.6);
                \draw (1.6, 2^1.6) -- (1.8, 2^1.6);
            \end{axis}
        \end{tikzpicture}
    \end{center}

    \begin{enumerate}
        \item The graph of $y = 2^x$, for $0 \leq x \leq 1$ is shown in the diagram. Rectangles, each of width $\frac1n$, are drawn under the curve. Given that $\sum\limits_{k = 0}^n x^k = \frac{1-x^{n+1}}{1-x}$, show that the total area $A$ of all $n$ rectangles is given by $\frac1n \bp{\frac1{2^{\frac1n} - 1}}$.
        \item Find the limit of $A$ in exact form as $n \to \infty$.
    \end{enumerate}

    Let $V$ be the volume of all $n$ rectangles rotated about the $x$-axis.

    \begin{enumerate}
        \setcounter{enumi}{2}
        \item Find $V$ in terms of $n$.
        \item State the limit of $V$ in exact form as $n \to \infty$.
    \end{enumerate}
\end{problem}
\begin{solution}
    \begin{ppart}
        \[A = \sum_{k=0}^{n-1} \frac{2^{k/n}}n = \frac1n \sum_{k=0}^{n-1} \bp{2^{1/n}}^k = \frac1n \cdot \frac{1 - \bp{2^{1/n}}^{n}}{1 - 2^{1/n}} = \frac1n \bp{\frac{1 -2}{1 - 2^{1/n}}} = \frac1n \bp{\frac{1}{2^{1/n} - 1}}.\]
    \end{ppart}
    \begin{ppart}
        \[\lim_{n \to \infty} A = \lim_{n \to \infty} \frac{1/n}{2^{1/n} - 1} = \lim_{m \to 0} \frac{m}{2^m - 1} = \lim_{m \to 0} \frac{1}{\ln 2 \cdot 2^m} = \frac1{\ln 2}.\]
    \end{ppart}
    \begin{ppart}
        \begin{gather*}
            V = \pi \sum_{k=0}^{n-1} \frac1n \bp{2^{k/n}}^2 = \frac\pi{n} \sum_{k=0}^{n-1} \bp{2^{2/n}}^k = \frac\pi{n} \bp{\frac{1 - \bp{2^{2/n}}^{n}}{1 - 2^{2/n}}}\\
            = \frac\pi{n} \bp{\frac{1 - 4}{1 - 2^{2/n}}} = \frac{3\pi}{n\bp{4^{1/n} - 1}}.
        \end{gather*}
    \end{ppart}
    \begin{ppart}
        \begin{gather*}
            \lim_{n \to \infty} V = \lim_{n \to \infty} \frac{3\pi}{n\bp{4^{1/n} - 1}} = 3\pi \lim_{n \to \infty} \frac{1/n}{4^{1/n} - 1} = 3\pi \lim_{m \to 0} \frac{m}{4^m - 1} \\
            = 3\pi \lim_{m \to 0} \frac{1}{4^m \ln 4} = 3\pi \bp{\frac1{\ln 4}} = \frac{3\pi}{2\ln2}.
        \end{gather*}
    \end{ppart}
\end{solution}

\begin{problem}
    $O$ is the origin and $A$ is the point on the curve $y = \tan x$ where $x = \pi/3$.

    \begin{enumerate}
        \item Calculate the area of the region $R$ enclosed by the arc $OA$, the $x$-axis and the line $x = \pi/3$, giving your answer in an exact form.
        \item The region $S$ is enclosed by the arc $OA$, the $y$-axis and the line $y = \sqrt3$. Find the volume of the solid of revolution formed when $S$ is rotated through 360$\deg$ about the $x$-axis, giving your answer in an exact form.
        \item Find $\int_0^{\sqrt3} \arctan y \d y$ in exact form.
    \end{enumerate}
\end{problem}
\begin{solution}
    \begin{center}\tikzsetnextfilename{315}
        \begin{tikzpicture}[trim axis left, trim axis right]
            \begin{axis}[
                axis on top,
                domain = 0:1.2,
                samples = 101,
                axis y line=middle,
                axis x line=middle,
                xtick = {pi/3},
                ytick = {1.73},
                xticklabels = {$\frac\pi3$},
                yticklabels = {$\sqrt 3$},
                xlabel = {$x$},
                ylabel = {$y$},
                ymin=0,
                legend cell align={left},
                legend pos=outer north east,
                after end axis/.code={
                    \path (axis cs:0,0) 
                        node [anchor=north east] {$O$};
                    }
                ]
                \addplot[plotRed, name path=f1] {tan(\x r)};
    
                \addlegendentry{$y = \tan x$};
                \addplot[dotted, name path=f2] {1.73};
                \addplot[dotted, name path=null] {0};
    
                \addplot[color=black!20] fill between[of=null and f1,soft clip={domain=0:pi/3}];

                \addplot[color=black!10] fill between[of=f2 and f1,soft clip={domain=0:pi/3}];

                \node at (0.3, 1) {$S$};
                \node at (0.8, 0.5) {$R$};

                \fill (pi/3, 1.73) circle[radius=2.5 pt] node[anchor=south east] {$A$};
            \end{axis}
        \end{tikzpicture}
    \end{center}

    \begin{ppart}
        \[[R] = \int_0^{\pi/3} \tan x \d x = \evalint{\ln \sec x}{0}{\pi/3} = \ln 2 \units[2].\]
    \end{ppart}
    \begin{ppart}
        \begin{gather*}
            \volume = \pi \int_0^{\pi/3} \bs{\bp{\sqrt3}^2 - \tan^2 x} \d x = \pi \int_0^{\pi/3} \bp{3 - \sec^2 x + 1} \d x \\
            = \pi\evalint{4x - \tan x}{0}{\pi/3} = \bp{\frac{4\pi^2}{3} - \sqrt{3}\pi} \units[3].
        \end{gather*}
    \end{ppart}
    \begin{ppart}
        Observe that $\int_0^{\sqrt3} \arctan y \d y = [S] = [R \cup S] - [R] = (\pi/3) \cdot \sqrt{3} - \ln 2$.
    \end{ppart}
\end{solution}

\begin{problem}
    A portion of the curve $ay = x^2$, where $a$ is a positive constant, is rotated about the vertical axis $Oy$ to form the curved surface of an open bowl. The bowl has a horizontal circular base of radius $r$ and a horizontal circular rim of radius $3r$.

    \begin{enumerate}
        \item Prove that the depth of the bowl is $\frac{8r^2}a$.
        \item Find the volume of the bowl in terms of $r$ and $a$.
        \item Given that the volume of the bowl is $\frac{\pi a^3}{10}$, find the depth of the bowl in terms of $a$ only.
    \end{enumerate}
\end{problem}
\begin{solution}
    Note that $ay = x^2 \implies y = \frac{x^2}a$.
    \begin{center}\tikzsetnextfilename{316}
        \begin{tikzpicture}[trim axis left, trim axis right]
            \begin{axis}[
                axis on top,
                domain = 0:3.5,
                samples = 101,
                axis y line=middle,
                axis x line=middle,
                xtick = {1, 3},
                ytick = {1, 9},
                xticklabels = {$r$, $3r$},
                yticklabels = {$\frac{r^2}a$, $\frac{(3r)^2}a$},
                xlabel = {$x$},
                ylabel = {$y$},
                ymin=0,
                legend cell align={left},
                legend pos=outer north east,
                after end axis/.code={
                    \path (axis cs:0,0) 
                        node [anchor=north east] {$O$};
                    }
                ]
                \addplot[plotRed, name path=f1] {x^2};
    
                \addlegendentry{$y = x^2/a$};
                \addplot[dotted, name path=f2] {1};
                \addplot[dotted, name path=f3] {9};
    
                \addplot[color=black!20] fill between[of=f2 and f3, soft clip={domain=0:1.01}];

                \addplot[color=black!20] fill between[of=f3 and f1,soft clip={domain=1:3}];
            \end{axis}
        \end{tikzpicture}
    \end{center}

    \begin{ppart}
        \[\text{Depth of bowl} = \frac{(3r)^2}a - \frac{r^2}a = \frac{8r^2}a \units.\]
    \end{ppart}
    \begin{ppart}
        \[\volume = \pi\int_{r^2/a}^{9r^2/a} ay \d y = \pi \evalint{\frac{a}2 y^2}{r^2/a}{9r^2/a} = \frac{a\pi}2 \cdot \frac{80r^4}{a^2} = \frac{40\pi r^4}{a} \units[3].\]
    \end{ppart}
    \begin{ppart}
        \[\frac{40\pi r^4}{a} = \frac{\pi a^3}{10} \implies 400r^4 = a^4 \implies 20r^2 = a^2 \implies r^2 = \frac1{20} a^2.\] Hence, the depth of the bowl is \[\frac{8}{a} \bp{\frac1{20} a^2} = \frac25 a \units.\]
    \end{ppart}
\end{solution}

\clearpage
\begin{problem}
    The diagram shows the region $R$ bounded by part of the curve $C$ with equation $y = 3 - x^2$, the $y$-axis and the line $y = 2$, lying in the first quadrant.

    \begin{center}\tikzsetnextfilename{317}
        \begin{tikzpicture}[trim axis left, trim axis right]
            \begin{axis}[
                domain = 0:2,
                samples = 101,
                ymax=4,
                xmin=-1,
                axis y line=middle,
                axis x line=middle,
                xtick = \empty,
                ytick = {2, 3},
                xlabel = {$x$},
                ylabel = {$y$},
                legend cell align={left},
                legend pos=outer north east,
                after end axis/.code={
                    \path (axis cs:0,0) 
                        node [anchor=north east] {$O$};
                    }
                ]
                \addplot[plotRed, name path=quad] {3-x^2};
                \addlegendentry{$C$};

                \addplot[dotted, name path=line] {2};

                \addplot[color=black!20] fill between[of=quad and line, soft clip={domain=0:1}];

                \node at (0.3, 2.5) {$R$};
            \end{axis}
        \end{tikzpicture}
    \end{center}

    Write down the equation of the curve obtained when $C$ is translated by 2 units in the negative $y$-direction.

    Hence, or otherwise, show that the volume of the solid formed when $R$ is rotated completely about the line $y = 2$ is given by $\pi \int_0^1 \bp{1 - 2x^2 + x^4} \d x$ and evaluate this integral exactly.
\end{problem}
\begin{solution}
    Clearly, $C: y = 1-x^2$.

    Note that $3 - x^2 = 2 \implies x = \pm 1$, whence $x = 1$ since $x > 0$.
    \begin{gather*}
        \volume = \pi\int_0^1 \bp{1 - x^2}^2 \d x = \pi\int_0^1 \bp{1 - 2x^2 + x^4} \d x\\
        = \pi \evalint{x - \frac23 x^3 + \frac15 x^5}01 = \frac{8}{15} \pi \units[3].
    \end{gather*}
\end{solution}

\begin{problem}
    The diagram below shows a region $R$ bounded by the curve $(y + 5)^2 = x-3$ and the line $y = x - 10$. Find the volume of solid formed when $R$ is rotated four right angles about 
    \begin{enumerate}
        \item the $y$-axis, and
        \item the $x$-axis.
    \end{enumerate}

    \begin{center}\tikzsetnextfilename{318}
        \begin{tikzpicture}[trim axis left, trim axis right]
            \begin{axis}[
                domain = 0:30,
                samples = 151,
                ymax=4,
                axis y line=middle,
                axis x line=middle,
                xtick = \empty,
                ytick = \empty,
                xlabel = {$x$},
                ylabel = {$y$},
                legend cell align={left},
                legend pos=outer north east,
                after end axis/.code={
                    \path (axis cs:0,0) 
                        node [anchor=east] {$O$};
                    }
                ]
                \addplot[plotRed, name path=quad1] {sqrt(x-3)-5};
                \addlegendentry{$(y+5)^2 = x - 3$};

                \addplot[plotBlue, name path=line] {x-10};
                \addlegendentry{$y = x - 10$};

                \addplot[plotRed, name path=quad2] {-sqrt(x-3)-5};

                \addplot[color=black!20] fill between[of=quad1 and quad2, soft clip={domain=3:4}];

                \addplot[color=black!20] fill between[of=quad1 and line, soft clip={domain=4: 7}];

                \node at (4.2, -4.6) {$R$};
            \end{axis}
        \end{tikzpicture}
    \end{center}
\end{problem}
\clearpage
\begin{solution}
    \begin{ppart}
        Consider the intersections between $(y+5)^2 = x-3$ and $y = x-10$. \[(y+5)^2 = (x-5)^2 = x-3 \implies  x^2 - 11x + 28 = (x-4)(x-7) = 0.\] Hence, $x = 4$ and $x = 7$, whence $y = -6$ and $y = -3$. Thus, the two curves intersect at $(4, -6)$ and $(7, -3)$.

        Note that $(y+5)^2 = x - 3 \implies x = 3 + (y+5)^2$ and $y = x - 10 \implies x = y + 10$. \[\volume = \pi \int_{-6}^{-3} \bs{\bp{y+10}^2 - \bp{3 + (y+5)^2}^2} \d y = 130 \units[3] \tosf{3}.\]
    \end{ppart}
    \begin{ppart}
        Note that
        \[(y+5)^2 = x - 3 \implies \begin{cases}
                y = -5+\sqrt{x-3},\qquad y \geq -5\\
                y = -5-\sqrt{x-3}, \qquad y < -5
        \end{cases}\] Thus,
        \begin{align*}
            \volume &= \pi\int_3^4 \bs{\bp{-5-\sqrt{x-3}}^2 - \bp{-5+\sqrt{x+3}}^2} \d x\\
            &\quad + \pi\int_4^7 \bs{\bp{x-10}^2 - \bp{-5+\sqrt{x-3}}^2} \d x = 127 \units[3] \tosf{3}.
        \end{align*}
    \end{ppart}
\end{solution}

\begin{problem}
    The curve $C$ is defined by the following pair of parametric equations. \[x = t - \frac1{t^2}, \, y = 2 - t^2,\qquad t > 0.\]

    Find the area of the finite region $R$ enclosed by the curve $C$ and the axes as well as the volume of solid obtained when $R$ is rotated about the $x$-axis through 4 right-angles.
\end{problem}
\begin{solution}
    \begin{center}\tikzsetnextfilename{319}
        \begin{tikzpicture}[trim axis left, trim axis right]
            \begin{axis}[
                axis on top,
                domain = 0.95:1.5,
                samples = 151,
                axis y line=middle,
                axis x line=middle,
                xtick = \empty,
                ytick = \empty,
                xlabel = {$x$},
                ylabel = {$y$},
                legend cell align={left},
                legend pos=outer north east,
                after end axis/.code={
                    \path (axis cs:0,0) 
                        node [anchor=north east] {$O$};
                    }
                ]
                \addplot[plotRed, name path=C] ({x - 1/x^2}, {2 - x^2});
                \addlegendentry{$C$};

                \addplot[thin, name path=null, domain=0:0.914] {0};

                \addplot[color=black!20] fill between[of=C and null, soft clip={domain=0:0.914}];

                \node at (0.3, 0.4) {$R$};
            \end{axis}
        \end{tikzpicture}
    \end{center}

    Note that when $x = 0$, we have $ t = 1$. Also note that when $y = 0$, we have $t = \sqrt2$, whence $x = \sqrt2 - 1/2$. Thus,
    \begin{gather*}
        [R] = \int_0^{\sqrt2 - 1/2} y \d x = \int_1^{\sqrt2} \bp{2 - t^2} \der{x}{t} \d t \\
        = \int_1^{\sqrt2} \bp{2 - t^2}\bp{1 + \frac2{t^3}} \d t = 0.526 \units[2] \tosf{3}.
    \end{gather*}

    Also,
    \begin{gather*}
        \volume = \pi \int_0^{\sqrt2 - \frac12} y^2 \d x = \pi\int_1^{\sqrt2} \bp{2 - t^2}\der{x}{t} \d t \\
        = \pi\int_1^{\sqrt2} \bp{2 - t^2}\bp{1 + \frac2{t^3}} \d t = 1.19 \units[3] \tosf{3}.
    \end{gather*}
\end{solution}

\begin{problem}
    Find the area enclosed by the ellipse $x = a \cos t, \, y = b \sin t$, where $a$ and $b$ are positive constants. Find also the volume of solid obtained when the region enclosed by the ellipse is rotated through $\pi$ radians about the $x$-axis.
\end{problem}
\begin{solution}
    By symmetry, we only need to consider the area of the ellipse in the first quadrant. Note that $x = 0 \implies t = \pi/2$ and $x = a \implies t = 0$. Hence,
    \begin{gather*}
        \area = 4\int_0^a y \d x = 4\int_{\pi/2}^0 y \cdot \der{x}{t} \d t = 4\int_{\pi/2}^0 (b\sin t)(-a\sin t) \d t = 4ab\int_0^{\pi/2} \sin^2 t \d t\\
        = 4ab\int_0^{\pi/2} \frac{1 - \cos 2t}2 \d t = 2ab \evalint{t - \frac{\sin 2t}2}{0}{\pi/2} = \pi ab \units[2].
    \end{gather*}

    Also,
    \begin{gather*}
        \volume = 2\pi \int_0^a y^2 \d x = 2\pi\int_{\pi/2}^0 y^2 \cdot \der{x}{t} \d t = 2\pi\int_{\pi/2}^0 (b\sin t)^2 (-a\sin t) \d t\\
        = 2\pi a b^2 \int_0^{\pi/2} \sin^3 t \d t = 2\pi a b^2 \int_0^{\pi/2} \frac{3\sin t - \sin 3t}{4} \d t \\
        = \frac12 \pi a b^2 \evalint{-3\cos t + \frac13 \cos 3t}{0}{\pi/2} = \frac{4\pi}3 ab^2 \units[3].
    \end{gather*}
\end{solution}

\begin{problem}
    Find the polar equation of the curve $C$ with equation $x^5 + y^5 = 5bx^2y^2$, where $b$ is a positive constant. Sketch the part of the curve $C$ where $0 \leq \t \leq \frac\pi2$. Show, using polar coordinates, that the area $A$ of the region enclosed by this part of the curve is given by
    \[
        A = \frac{25b^2}2 \int_0^{\frac\pi2} \frac{\sin^4\t\cos^4\t}{(\cos^5\t + \sin^5\t)^2} \d \t
    \]
    By differentiating $\frac1{1 + \tan^5\t}$ with respect to $\t$, or otherwise, find the exact value of $A$ in terms of $b$.
\end{problem}
\begin{solution}
    \begin{gather*}
        x^5 + y^5 = 5bx^2y^2 \implies (r\cos \t)^5 + (r\sin\t)^5 = 5b(r\cos\t)^2(r\sin\t)^2\\
        \implies r\bp{\cos^5\t + \sin^5\t} = 5b\cos^2\t\sin^2\t \implies r = \frac{5b\cos^2\t\sin^2\t}{\cos^5\t + \sin^5\t}.
    \end{gather*}

    \begin{center}\tikzsetnextfilename{320}
        \begin{tikzpicture}[trim axis left, trim axis right]
            \begin{axis}[
                samples = 150,
                axis y line=middle,
                axis x line=middle,
                xtick = \empty,
                ytick = \empty,
                xlabel = {$\t=0$},
                ylabel = {$\t = \frac\pi2$},
                legend cell align={left},
                legend pos=outer north east,
                after end axis/.code={
                    \path (axis cs:0,0) 
                        node [anchor=north east] {$O$};
                    }
                ]
                \addplot[color=plotRed,data cs=polarrad, domain=0:pi/2] {(5*cos(\x r)^2*sin(\x r)^2)/(cos(\x r)^5 + sin(\x r)^5)};
    
                \addlegendentry{$C$};
            \end{axis}
        \end{tikzpicture}
    \end{center}
    We have
    \begin{gather*}
        A = \frac12 \int_0^{\pi/2} \bp{\frac{5b\cos^2\t\sin^2\t}{\cos^5\t + \sin^5\t}}^2 \d \t = \frac12 \int_0^{\pi/2} \frac{25b^2\cos^4\t\sin^4\t}{\bp{\cos^5\t + \sin^5\t}^2} \d \t \\
        = \frac{25b^2}2 \int_0^{\pi/2} \frac{\cos^4\t\sin^4\t}{\bp{\cos^5\t + \sin^5\t}^2} \d \t.
    \end{gather*}

    Note that \[\der{}{\t} \frac1{1 + \tan^5\t} = -\frac{5\tan^4\t \sec^2\t}{\bp{1 + \tan^5\t}^2} = -5 \bp{\frac{\cos^{10}\t}{\cos^5\t + \sin^5\t}}\bp{\frac{\sin^4\t}{\cos^6\t}} = -\frac{5\cos^4\t\sin^4\t}{\cos^5\t + \sin^5\t}.\] Hence, \[A = \frac{-5b^2}2 \int_0^{\pi/2} -\frac{5\cos^4\t\sin^4\t}{\bp{\cos^5\t + \sin^5\t}^2} \d \t = -\frac{5b^2}2 \evalint{\frac{1}{1 + \tan^5\t}}{0}{\pi/2} = \frac{5b^2}2.\]
\end{solution}

\begin{problem}
    The polar equation of a curve is given by $r = \e^\t$ where $0 \leq \t \leq \pi/2$. Cartesian axes are taken at the pole $O$. Express $x$ and $y$ in terms of $\t$ and hence find the Cartesian equation of the tangent at $\bp{\e^{\pi/2}, \pi/2}$. The region $R$ is bounded by the polar curve, tangent and the $x$-axis. Find the exact area of the region $R$.
\end{problem}
\begin{solution}
    We have $x = \e^\t \cos \t$ and $y = \e^\t \sin \t$. Thus, \[\der{y}{x} = \frac{\derx{y}{\t}}{\derx{x}{\t}} = \frac{e^\t \cos\t + e^\t\sin\t}{-e^\t \sin\t + e^\t \cos\t} = \frac{\cos\t + \sin\t}{\cos\t - \sin\t}.\] Hence, \[\evalder{\der{y}{x}}{\t = \pi/2} = \frac{\cos(\pi/2) + \sin(\pi/2) }{\cos(\pi/2) - \sin(\pi/2) } = -1.\]

    When $\t = \pi/2$, we have $x = 0$ and $y = \e^{\pi/2}$. Hence, the tangent is given by \[y - e^{\pi/2} = -(x - 0) \implies y = -x + e^{\pi/2}.\] Thus, \[[R] = \frac12 \bp{\e^{\pi/2}}\bp{\e^{\pi/2}} - \frac12 \int_0^{\pi/2} \bp{e^\t}^2 \d \t = \frac{\e^\pi}2 - \frac12 \evalint{\frac{\e^{2\t}}2}0{\pi/2} = \frac{e^\pi + 1}4 \units[2].\]
\end{solution}

\begin{problem}
    \begin{center}\tikzsetnextfilename{321}
        \begin{tikzpicture}[trim axis left, trim axis right]
            \begin{axis}[
                domain = 0:13,
                samples = 101,
                axis y line=middle,
                axis x line=middle,
                xtick = \empty,
                ytick = \empty,
                xlabel = {$t$},
                ylabel = {$v$},
                legend cell align={left},
                legend pos=outer north east,
                after end axis/.code={
                    \path (axis cs:0,0) 
                        node [anchor=north east] {$O$};
                    }
                ]
                \addplot[plotRed] {x * (x-10)};
            \end{axis}
        \end{tikzpicture}
    \end{center}
    The diagram shows the velocity-time graph of a particle moving in a straight line. The equation of the curve shown is $v = t(t - 10)$ where $t$ seconds is the time and $v$ ms$^{-1}$ is the velocity. The particle starts at a point $A$ on the line when $t = 0$.

    Calculate
    \begin{enumerate}
        \item the distance travelled by the particle before coming to instantaneous rest, and
        \item the time at which the particle returns to $A$.
    \end{enumerate}
\end{problem}
\begin{solution}
    \begin{ppart}
        For instantaneous rest, $v = 0$. Hence, $t(t-10) = 0$, whence $t = 10$. Note that we reject $t = 0$ since $t > 0$. The distance travelled by the particle before coming to instantaneous rest is hence \[-\int_0^{10} v \d t = -\int_0^{10} t(t-10) \d t = -\int_0^{10} \bp{t^2 - 10t} \d t = -\evalint{\frac{t^3}3 - \frac{10t^2}2}{0}{10} = \frac{500}3 \text{ m}.\]
    \end{ppart}
    \begin{ppart}
        When the particle returns to $A$, $s = 0$. Let the time at which the particle returns to $A$ be $t_0$. \[\int_0^{t_0} v \d t = \int_0^{t_0} t(t-10) \d t = \evalint{\frac{t_0^3}3 - \frac{10t_0^2}2}{0}{t_0} = \frac13 t_0^3 - 5t_0^2 = \frac13t_0^2 \bp{t_0 - 15} = 0.\] Thus, $t_0 = 15$. Note that we reject $t_0 = 0$ since $t_0 > 0$. It hence takes the particle 15 seconds to return to $A$.
    \end{ppart}
\end{solution}