\chapter*{Preface}

\section*{About this Book}

This book is a collection of notes and exercises based on the mathematics courses offered at Dunman High School\footnote{It must be stated that these notes are unofficial and are obviously not endorsed by the school.}. The scope of this book follows that of the 2025 H2 Mathematics (\href{https://www.seab.gov.sg/files/A%20Level%20Syllabus%20Sch%20Cddts/2025/8865_y25_sy.pdf}{9758}), H2 Further Mathematics (\href{https://www.seab.gov.sg/files/A%20Level%20Syllabus%20Sch%20Cddts/2025/9649_y25_sy.pdf}{9649}) and H3 Mathematics (\href{https://www.seab.gov.sg/files/A%20Level%20Syllabus%20Sch%20Cddts/2025/9820_y25_sy.pdf}{9820}) syllabi for the Singapore-Cambridge A-Level examinations.

\section*{Notation}

All definitions, results, recipes (methods) and examples are colour-coded green, blue, purple and red respectively. 

Challenging exercises are marked with a ``\chili'' symbol.

The area of a polygon $A_1 A_2 \dots A_n$ is notated $[A_1 A_2 \dots A_n]$. In particular, the area of a triangle $ABC$ is notated $[\triangle ABC]$.

For formatting reasons, an inline column vector is notated as $\cveciiix{x}{y}{z}$.

Let $n$ be a positive integer. Then $[n]$ represents the set $\bc{1, 2, \dots, n}$.

\section*{Contributing}

The source code for this book is available on GitHub at \href{https://github.com/asdia0/TripleMath}{\texttt{asdia0/TripleMath}}. Contributions are more than welcome.