\documentclass{echw}

\title{Assignment B8\\Applications of Integration I}
\author{Eytan Chong}
\date{2024-06-26}

\begin{document}
    \problem{}
        The diagram shows the region $R$, which is bounded by the axes and the part of the curve $y^2 = 4a(a-x)$ lying in the first quadrant.

        Find, in terms of $a$, the volume, $V_x$, of the solid formed when $R$ is rotated completely about the $x$-axis.

        The volume of the solid formed when $R$ is rotated completely about the $y$-axis is $V_y$. Show that $V_y = \dfrac8{15} V_x$.

        The region $S$, lying in the first quadrant, is bounded by the curve $y^2 = 4a(a-x)$ and the lines $x = a$ and $y = 2a$. Find, in terms of $a$, the volume of the solid formed when $S$ is rotated completely about the $y$-axis.

    \solution

    \problem{}
        The region bounded by the axes and the curve $y = \cos x$ from $x = 0$ to $x = \dfrac12 \pi$ is divided into two parts, of areas $A_1$ and $A_2$, by the curve $y = \sin x$.

        \begin{enumerate}
            \item Prove that $A_2 = \sqrt2 A_1$.
            \item Prove Find the volume of the solid obtained when the region with area $A_2$ is rotated about the $y$-axis through $2\pi$ radians. Give your answer in exact form.
        \end{enumerate}

    \solution
        \part

        \part

    \problem{}
        A curve has parametric equations
        \[
            x = \cos^2 t, \, y = \sin^3 t, \, 0 \leq t \leq \dfrac12 \pi
        \]
        
        \begin{enumerate}
            \item Sketch the curve.
            \item Show that the area under the curve for $0 \leq t \leq \dfrac12 \pi$ is $2\displaystyle\int_0^{\pi/2} \cos t \sin^4 t \d t$, and find the exact value of the area.
            \item Find the volume of the solid obtained when the region in (b) is rotated about the $y$-axis through $2\pi$ radians.
        \end{enumerate}

    \solution
        \part

        \part

        \part

    \problem{}
        \begin{enumerate}
            \item Given that $f$ is a continuous function, explain, with the aid of a sketch, why the value of
            \[
                \lim_{n \to \infty} \dfrac1n \bs{f\of{\dfrac1n} + f\of{\dfrac2n} + \ldots + f\of{\dfrac{n}{n}}}
            \]
            is $\displaystyle\int_0^1 f(x) \d x$.
            \item Hence, evaluate $\displaystyle \lim_{n \to \infty} \dfrac1n \bp{\dfrac{\sqrt[3]1 + \sqrt[3]2 + \ldots + \sqrt[3]n}{\sqrt[3]n}}$.
        \end{enumerate}

    \solution
        \part

        \part

    \problem{}
        The function $f$ satisfies $f'(x) > 0$ for $a \leq x \leq b$, and $g$ is the inverse of $f$. By making a suitable change of variable, prove that
        \[
            \int_a^b f(x) \d x = b\b - a\a - \int_\a^\b g(y) \d y
        \]
        where $\a = f(a)$ and $\b = f(b)$. Interpret this formula geometrically by means of a sketch where $\a$ and $a$ are positive. Verify this result for the case where $f(x) = e^{2x}$, $a = 0$, $b = 1$.

        Prove similarly and interpret geometrically the formula
        \[
            2\pi \int_a^b xf(x) \d x = \pi(b^2\b - a^2\a) - \pi\int_\a^\b \bs{g(y)}^2 \d y
        \]

    \solution
\end{document}