\documentclass{echw}

\title{Tutorial B12\\First Order Differential Equations}
\author{Eytan Chong}
\date{2024-08-06}

\begin{document}
    \problem{}
        Given that $y = 1$ when $x = 1$, find the particular solution of the differential equation $\der{y}{x} = \dfrac{y^2}{x}$.

    \solution
        \begin{alignat*}{2}
            && \der{y}{x} &= \dfrac{y^2}{x}\\
            \implies&& \dfrac{1}{y^2} \der{y}{x} &= \dfrac{1}{x}\\
            \implies&& \int \dfrac{1}{y^2} \der{y}{x} \d x &= \int \dfrac{1}{x} \d x\\
            \implies&& \int \dfrac{1}{y^2} \d y &= \int \dfrac{1}{x} \d x\\
            \implies&& -\dfrac{1}{y} &= \ln \abs{x} + C_1\\
            \implies&& y &= -\dfrac1{\ln \abs{x} + C_1}\\
            && &= \dfrac{1}{C - \ln \abs{x}}
        \end{alignat*}
        At $x = 1$, $y = 1$, $1 = \dfrac{1}{C - \ln \abs{1}} \implies C = 1$.
        \boxt{$y = \dfrac1{1 - \ln \abs{x}}$}

    \problem{}
        Two variables $x$ and $t$ are connected by the differential equation $\der{x}{t} = \dfrac{kx}{10-x}$, where $0 < x < 10$ and where $k$ is a constant. It is given that $x = 1$ when $t = 0$ and that $x = 2$ when $t = 1$. Find the value of $t$ when $x = 5$, given your answer to three s.f.

    \solution
        \begin{alignat*}{2}
            && \der{x}{t} &= \dfrac{kx}{10-x}\\
            \implies&& \dfrac{10-x}{x} \der{x}{t} &= k\\
            \implies&& \int \dfrac{10-x}{x} \der{x}{t} \d t &= k \int \d t\\
            \implies&& \int \dfrac{10-x}{x} \d x &= k \int \d t\\
            \implies&& \int \bp{\dfrac{10}{x} - 1} \d x &= kt + C\\
            \implies&& 10\ln x - x &= kt + C
        \end{alignat*}
        At $t = 0$, $-x = 1$, $C = -1$.

        At $t = 1$, $x = 2$, $10 \ln 2 - 2 = k - 1 \implies k = \ln 2 - 1$.

        When $x = 5$, $10\ln 5 - 5 = (\ln 2 - 1)t - 1 \implies t = \dfrac{10\ln 5 - 4}{10 \ln 2 - 1} = 2.04 \tosf{3}$.

        \boxt{When $x = 5$, $t = 2.04$}

    \problem{}
        Use the substitution $y = u - 2x$ to find the general solution of the differential equation $\der{y}{x} = -\dfrac{8x + 4y + 1}{4x+2y+1}$.

    \solution
        Note that $y = u - 2x \implies u = y + 2x$. Also, $\der{y}{x} = \der{u}{x} - 2$.

        \begin{alignat*}{2}
            && \der{y}{x} &= -\dfrac{8x + 4y + 1}{4x+2y+1}\\
            \implies&& \der{u}{x} - 2 &= -\dfrac{8x + 4(u - 2x) + 1}{4x+2(u - 2x)+1}\\
            && &= -\dfrac{4u + 1}{2u + 1}\\
            \implies&& \der{u}{x} &= 2 - \dfrac{4u+1}{2u+1}\\
            && &= \dfrac{4u+2}{2u+1} - \dfrac{4u+1}{2u+1}\\
            && &= \dfrac1{2u+1}\\
            \implies&& (2u+1)\der{u}{x} &= 1\\
            \implies&& \int (2u+1)\der{u}{x} \d x &= \int \d x\\
            \implies&& \int (2u+1) \d u &= \int \d x\\
            \implies&& u^2 + u &= x + C\\
            \implies&& (y + 2x)^2 + (y + 2x) &= x + C\\
            \implies&& (y + 2x)^2 + y + x &= C
        \end{alignat*}

        \boxt{$(y + 2x)^2 + y + x = C$}

    \problem{}
        By using the substitution $z = ye^{2x}$, find the general solution of the differential equation $\der{y}{x} + 2y = xe^{-2x}$.

        Find the particular solution of the differential equation given that $\der{y}{x} = 1$ when $x = 0$.

    \solution
        Note that $z = ye^{2x} \implies \der{z}{x} = \der{y}{x} e^{2x} + 2y e^{2x} = \der{y}{x} e^{2x} + 2z$. Hence, $\der{y}{x} = \der{z}{x} e^{-2x} - 2y$.
        \begin{alignat*}{2}
            && \der{y}{x} + 2y &= xe^{-2x}\\
            \implies&& \der{z}{x} e^{-2x} - 2y + 2y &= xe^{-2x}\\
            \implies&& \der{z}{x} &= x\\
            \implies&& z &= \dfrac{x^2}{2} + C\\
            \implies&& ye^{2x} &= \dfrac{x^2}{2} + C\\
            \implies&& y &= \dfrac{x^2}{2e^{2x}} + \dfrac{C}{e^{2x}}
        \end{alignat*}

        \boxt{$y = \dfrac{x^2}{2e^{2x}} + \dfrac{C}{e^{2x}}$}
        
        At $x = 0$, $\der{y}{x} = 1$. Hence, $1 + 2y = 0 \implies y = -\dfrac12$. Thus, $C = -\dfrac12$.

        \boxt{$y = \dfrac{x^2 - 1}{2e^{2x}}$}

    \problem{}
        Find the general solution of the differential equation $\der{y}{x} = 6xy^3$.

        Find its particular solution given that $y = 0.5$ when $x = 0$.

        Determine the interval of validity for the particular solution.

    \solution
        \begin{alignat*}{2}
            && \der{y}{x} &= 6xy^3\\
            \implies&& \dfrac{1}{y^3} \der{y}{x} &= 6x\\
            \implies&& \int \dfrac{1}{y^3} \der{y}{x} \d x &= 6 \int x \d  x\\
            \implies&& \int \dfrac{1}{y^3} \d y &= 6 \int x \d  x\\
            \implies&& -\dfrac12 \dfrac{1}{y^2} &= 3x^2 + C_1 \\
            \implies&& \dfrac{1}{y^2} &= C-6x^2\\
            \implies&& y^2 &= \dfrac{1}{C - 6x^2}
        \end{alignat*}

        \boxt{$y^2 = \dfrac{1}{C - 6x^2}$}

        At $x = 0$, $y = 0.5$, $\dfrac14 = \dfrac{1}{C} \implies C = 4$.
        \boxt{$y^2 = \dfrac{1}{4 - 6x^2}$}
        Observe that we require $4 - 6x^2 > 0$, whence $-\sqrt{\dfrac23} < x < \sqrt{\dfrac23}$.

        \boxt{The interval of validity is $\bp{-\sqrt{\dfrac23}, \sqrt{\dfrac23}}$.}

    \problem{}
        \begin{enumerate}
            \item Find the general solution of the differential equation $\der{y}{x} = \dfrac{3x}{x^2 + 1}$.
            \item What can you say about the gradient of every solution as $x \to \pm \infty$?
            \item Find the particular solution of the differential equation for which $y = 2$ when $x = 0$. Hence sketch the graph of this solution.
        \end{enumerate}

    \solution
        \part
            \begin{alignat*}{2}
                && \der{y}{x} &= \dfrac{3x}{x^2 + 1} \\
                && &= \dfrac32 \dfrac{2x}{x^2 + 1}\\
                \implies&& y &= \dfrac32 \ln{x^2 + 1} + C
            \end{alignat*}

            \boxt{$y = \dfrac32 \ln{x^2 + 1} + C$}

        \part
            As $x \to \pm \infty$, $\dfrac{3x}{x^2 + 1} \to 0^+$. Hence, the gradient of every solution approaches 0 from above.

        \part
            When $x = 0$ and $y = 2$, $C = 2$.

            \boxt{$y = \dfrac32 \ln{x^2 + 1} + 2$}

            \begin{center}
                \begin{tikzpicture}[trim axis left, trim axis right]
                    \begin{axis}[
                        domain = -20:20,
                        samples = 101,
                        axis y line=middle,
                        axis x line=middle,
                        xtick = \empty,
                        ytick = {2},
                        xlabel = {$x$},
                        ylabel = {$y$},
                        ymin=0,
                        legend cell align={left},
                        legend pos=outer north east,
                        after end axis/.code={
                            \path (axis cs:0,0) 
                                node [anchor=north] {$O$};
                            }
                        ]
                        \addplot[plotRed] {1.5 * ln(x^2 + 1) + 2};
            
                        \addlegendentry{$y = \frac32 \ln{x^2 + 1} + 2$};
                    \end{axis}
                \end{tikzpicture}
            \end{center}

    \problem{}
        The variables $x$, $y$ and $z$ are connected by the following differential equations.
        \begin{align}
            \der{z}{x} &= 3-2z \label{E7-A}\\
            \der{y}{x} &= z \label{E7-B}
        \end{align}
        \begin{enumerate}
            \item Given that $z < \dfrac32$, solve equation~\ref{E7-A} to find $z$ in terms of $x$.
            \item Hence find $y$ in terms of $x$.
            \item Use the result in part (b) to show that \[\der[2]{y}{x} = a\der{y}{x} + b\] for constants $a$ and $b$ to be determined.
            \item The curve of the solution in part (b) passes through the points $(0, 1)$ and $\bp{2, 3+e^{-4}}$. Sketch this curve, indicating its axial intercept and asymptote (if any).
        \end{enumerate}
    
    \solution
        \part
            \begin{alignat*}{2}
                &&\der{z}{x} &= 3-2z\\
                \implies&& \dfrac{1}{3-2z} \der{z}{x} &= 1\\
                \implies&& \int \dfrac{1}{3-2z} \der{z}{x} \d x &= \int \d x\\
                \implies&& \int \dfrac{1}{3-2z} \d z &= \int \d x\\
                \implies&& -\dfrac12 \ln{3-2z} &= x + C_1\\
                \implies&&\ln{3-2z} &= C_2-2x\\
                \implies&&3 - 2z &= C_3 e^{-2x}\\
                \implies&&z &= \dfrac32 - C_4 e^{-2x}
            \end{alignat*}
            \boxt{$z = \dfrac32 - A e^{-2x}$, $A > 0$}

        \part
            \begin{alignat*}{2}
                &&\der{y}{x} &= \dfrac32 - C_4e^{-2x}\\
                \implies&& y &= \int \bp{\dfrac32 - C_4e^{-2x}} \d x\\
                && &= \dfrac32 x - \dfrac{C_4}{2} e^{-2x} + C_6
            \end{alignat*}
            \boxt{$y = \dfrac32 x - \dfrac{A}{2} e^{-2x} + B$}

        \part
            \begin{alignat*}{2}
                &&\der{y}{x} &= \dfrac32 + C_4e^{-2x}\\
                \implies&&\der[2]{y}{x} &= -2C_4 e^{-2x}\\
                && &= -2\bp{\der{y}{x} - \dfrac32}\\
                && &= -2\der{y}{x} + 3
            \end{alignat*}
            \boxt{$a = -2$, $b = 3$}

        \part
            At $(0, 1)$, we obtain $1 = -\dfrac{A}{2} + B$.

            At $\bp{2, 3 + e^{-4}}$, we obtain $3 + e^{-4} = 3 - \dfrac{A}{2} e^{-4} + B \implies 1 = -\dfrac{A}{2} + Be^{4}$.

            Hence, $B = Be^4$, whence $B = 0$ and $A = -2$. The curve thus has equation $y = \dfrac32 x + e^{-2x}$.

            \begin{center}
                \begin{tikzpicture}[trim axis left, trim axis right]
                    \begin{axis}[
                        domain = -0.5:1.4,
                        samples = 101,
                        axis y line=middle,
                        axis x line=middle,
                        xtick = \empty,
                        ytick = {1},
                        xlabel = {$x$},
                        ylabel = {$y$},
                        ymin=0,
                        legend cell align={left},
                        legend pos=outer north east,
                        after end axis/.code={
                            \path (axis cs:0,0) 
                                node [anchor=north] {$O$};
                            }
                        ]
                        \addplot[plotRed, name path=f1, unbounded coords = jump] {1.5 * x + e^(-2*x)};
            
                        \addlegendentry{$y = \frac32 x + e^{-2x}$};

                        \addplot[dotted] {1.5 * x};
                        \node[rotate=50, below] at (1, 1.5) {$y = \frac32 x$};
                    \end{axis}
                \end{tikzpicture}
            \end{center}
    
    \problem{}
        A bottle containing liquid is taken from a refrigerator and placed in a room where the termperature is a constant 20\,$\deg$C. As the liquid warms up, the rate of increase of its temperature $\t$\,$\deg$C after time $t$ minutes is proportional to the temperature difference $(20-\t)$\,$\deg$C. Initially the temperature of the liquid is $10$\,$\deg$C and the rate of increase of the temperature is 1\,$\deg$C per minute. By setting up and solving a differential equation, show that $\t = 20 - 10e^{-t/10}$.

        Find the time it takes the liquid to reach a temperature of $15$\,$\deg$C, and state what happens to $\t$ for large values of $t$. Sketch a graph of $\t$ against $t$.
    
    \solution
        Since $\der{\t}{t} \propto (20 - \t)$, we have $\der{\t}{t} = k(20 - \t)$, where $k$ is a constant. We now solve for $\t$.
        \begin{alignat*}{2}
            &&\der{\t}{t} &= k(20 - \t)\\
            \implies&& \dfrac1{20 - \t} \der{\t}{t} &= k\\
            \implies&& \int \dfrac1{20 - \t} \der{\t}{t} \d t &= k \int \d t\\
            \implies&& \int \dfrac1{20 - \t} \d \t &= k \int \d t\\
            \implies&& - \ln{20 - \t} &= kt + C_1\\
            \implies&& \ln{20-\t} &= C_2 - kt\\
            \implies&& 20 - \t &= C e^{-kt}\\
            \implies&& \t &= 20 - Ce^{-kt}
        \end{alignat*}

        At $t = 0$, $\t = 10$. Hence, $10 = 20 - C \implies C = 10$. We also have $\evalder{\der{\t}{t}}{0} = 1$. Hence, $1 = k\bs{20 - (20 - 10e^0)} = 10k \implies k = \dfrac1{10}$. Thus, \[\t = 20 - 10e^{-t/10}.\]

        Consider $\t = 15$.
        \begin{alignat*}{2}
            && 15 &= 20 - 10e^{-t/10}\\
            \implies&& e^{-t/10} &= \dfrac12\\
            \implies&& -\dfrac{t}{10} &= -\ln 2\\
            \implies&& t &= 10\ln 2
        \end{alignat*}
        \boxt{It takes $10\ln2$ minutes for the liquid to reach a temperature of 15\,$\deg$C.}

        \boxt{As $t \to \infty$, $\t \to 20$.}

        \begin{center}
            \begin{tikzpicture}[trim axis left, trim axis right]
                \begin{axis}[
                    domain = 0:50,
                    samples = 101,
                    axis y line=middle,
                    axis x line=middle,
                    xtick = \empty,
                    ytick = {10, 20},
                    xlabel = {$t$},
                    ylabel = {$\t$},
                    ymax=25,
                    ymin=0,
                    legend cell align={left},
                    legend pos=outer north east,
                    after end axis/.code={
                        \path (axis cs:0,0) 
                            node [anchor=north east] {$O$};
                        }
                    ]
                    \addplot[plotRed, name path=f1, unbounded coords = jump] {20 - 10*e^(-0.1 * x)};
        
                    \addlegendentry{$\t = 20 - 10e^{-t/10}$};

                    \addplot[dotted]{20};
                \end{axis}
            \end{tikzpicture}
        \end{center}
    
    \problem{}
        \begin{enumerate}
            \item Find $\displaystyle\int \dfrac1{100-v^2} \d x$.
            \item A stone is dropped from a stationary balloon. It leaves the balloon with zero speed, and $t$ seconds later its speed $v$ metres per second satisfies the differential equation \[\der{v}{t} = 10 - 0.1v^2.\]
            \begin{enumerate}
                \item Find $t$ in terms of $v$. Hence find the exact tiem the stone takes to reach a speed of 5 metres per second.
                \item Find the speed of the stone after 1 second.
                \item What happens to the speed of the stone for large values of $t$?
            \end{enumerate}
        \end{enumerate}

    \solution
        \part
            \begin{align*}
                \int \dfrac1{100 - v^2} \d v &= \dfrac{1}{2(10)} \ln{\dfrac{10 + v}{10 - v}} + C\\
                &= \dfrac{1}{20} \ln{\dfrac{10 + v}{10 - v}} + C
            \end{align*}

            \boxt{$\displaystyle\int \dfrac1{100 - v^2} \d v = \dfrac{1}{20} \ln{\dfrac{10 + v}{10 - v}} + C$}

        \part
            \subpart
                \begin{alignat*}{2}
                    && \der{v}{t} &= 10 - 0.1v^2\\
                    && &= \dfrac1{10}(100 - v^2)\\
                    \implies&& \dfrac1{100 - v^2} \der{v}{t} &= \dfrac1{10}\\
                    \implies&& \int \dfrac1{100 - v^2} \der{v}{t} \d t &= \dfrac1{10} \int \d t\\
                    \implies&& \int \dfrac1{100 - v^2} \d v &= \dfrac1{10} \int \d t\\
                    \implies&& \dfrac1{20} \ln{\dfrac{10 + v}{10 - v}} + C &= \dfrac1{10}t\\
                    \implies&& t &= \dfrac1{2} \ln{\dfrac{10 + v}{10 - v}} + C \\
                \end{alignat*}
                At $t = 0$, $v = 0$. Hence, $C = 0$.
                \boxt{$t = \dfrac1{2} \ln{\dfrac{10 + v}{10 - v}}$}

            \subpart

                Consider $t = 1$.
                \begin{alignat*}{2}
                    && \dfrac1{2} \ln{\dfrac{10 + v}{10 - v}} &= 1\\
                    \implies&& \ln{\dfrac{10 + v}{10 - v}} &= 2\\
                    \implies&&\dfrac{10 + v}{10 - v} &= e^2\\
                    \implies&&10 + v &= e^2 (10 - v)\\
                    \implies&& v(1 + e^2) &= 10(e^2 - 1)\\
                    \implies&& v &= \dfrac{10(e^2 - 1)}{e^2 + 1}
                \end{alignat*}

                \boxt{After 1 second, the speed of the stone is $\dfrac{10(e^2 - 1)}{e^2 + 1}$ m/s.}

            \subpart

                As $t \to \infty$, we have $\ln{\dfrac{10 + v}{10 - v}} \to \infty \implies \dfrac{10 + v}{10 - v} \to \infty$. Thus, $v \to 10^-$.
                \boxt{For large values of $t$, the speed of the stone approaches 10 m/s.}

    \problem{}
        Two scientists are investigating the change of a certain population of an animal species of size $n$ thousand at time $t$ years. It is known that due to its inability to reproduce effectively, the species is unable to replace itself in the long run.
        \begin{enumerate}
            \item One scientist suggests that $n$ and $t$ are related by the differential equation $\der[2]{n}{t} = 10 - 6t$. Given that $n = 100$ when $t = 0$, show that the general solution of this differential equation is $n = 5t^2 - t^3 + Ct + 100$, where $C$ is a constant. Sketch the solution curve of the particular solution when $C = 0$, stating the axial intercepts clearly.
            \item The other scientist suggests that $n$ and $t$ are related by the differential equation $\der{n}{t} = 3 - 0.02n$. Find $n$ in terms of $t$, given again that $n = 100$ when $t = 0$. Explain in simple terms what will eventually happen to the population using this model.
        \end{enumerate}
        Which is a more appropriate model in modeling the population of the animal species?

    \solution
        \part
            \begin{alignat*}{2}
                &&\der[2]{n}{t} &= 10 - 6t\\
                \implies&&\der{n}{t} &= \int (10 - 6t) \d \t\\
                && &= 10t - 3t^2 + C\\
                \implies&&n &= \int \bp{10t - 3t^2 + C} \d t\\
                && &= 5t^2 - t^3 + C t + C'
            \end{alignat*}
            When $t = 0$ and $n = 100$, we have $C' = 100$. Thus, \[n = 5t^2 - t^3 + Ct + 100.\]
            
            When $C = 0$, $n = 5t^2 - t^3 + 100$.

            \begin{center}
                \begin{tikzpicture}[trim axis left, trim axis right]
                    \begin{axis}[
                        domain = 0:7.1,
                        samples = 101,
                        axis y line=middle,
                        axis x line=middle,
                        xtick = {7.03},
                        ytick = {100},
                        xlabel = {$n$},
                        ylabel = {$t$},
                        ymax = 150,
                        legend cell align={left},
                        legend pos=outer north east,
                        after end axis/.code={
                            \path (axis cs:0,0) 
                                node [anchor=north east] {$O$};
                            }
                        ]
                        \addplot[plotRed, name path=f1, unbounded coords = jump] {5*x^2 - x^3 + 100};
            
                        \addlegendentry{$n = 5t^2 - t^3 + 100$};
                    \end{axis}
                \end{tikzpicture}
            \end{center}
        
        \part
            \begin{alignat*}{2}
                && \der{n}{t} &= 3 - 0.02n\\
                && &= \dfrac{150 - n}{50}\\
                \implies&& \dfrac{1}{150-n} \der{n}{t} &= \dfrac1{50} \\
                \implies&& \int \dfrac{1}{150-n} \der{n}{t} \d t &= \dfrac1{50} \int \d t\\
                \implies&& \int \dfrac{1}{150-n} \d n &= \dfrac1{50} \int \d t\\
                \implies&& -\ln{150 - n} &= \dfrac1{50} t + C_1\\
                \implies&&\ln{150 - n} &= C_2 - \dfrac1{50} t\\
                \implies&& 150 - n &= C e^{-t/50}\\
                \implies&& n &= 150 - Ce^{-t/50}
            \end{alignat*}
            When $t = 0$ and $n = 100$, we have $C = 50$.
            \boxt{$n = 150 - 50e^{-t/50}$}
            As $t \to \infty$, $n \to 100$. Hence, the population will decrease before plateauing at 100 thousand.

        The second model is more appropriate, as it account for the fact that the species will eventually go extinct ($n = 0$) due to the fact that they cannot replace itself in the long run.

    \problem{}
        A rectangular tank has a horizontal base. Water is flowing into the tank at a constant rate, and flows out at a rate which is proportional to the depth of water in the tank. At time $t$ seconds, the depth of water in the tank is $x$ metres. If the depth is $0.5$ m, it remains at this constant value. Show that $\der{x}{t} = -k(2x-1)$, where $k$ is a positive constant. When $t = 0$, the depth of water in the tank is $0.75$ m and is decreasing at a rate of $0.01$ m s$^{-1}$. Find the time at which the depth of water is $0.55$ m.
        
    \solution
        Let $k_i$ m/s be the rate at which water is flowing into the tank. Note that $k_i \geq 0$. Let the rate at which water is flowing out of the tank be $k_o x$ m/s. Then $\derx{x}{t} = k_i - k_o x$. At $x = 0.5$, the volume of water in the tank is constant, i.e. $\evalder{\derx{x}{t}}{0.5} = 0$. This gives $k_i - 0.5 k_o = 0$, whence $k_0 = 2k_i$. Thus, $\derx{x}{t} = k_i - 2k_i x = -k_i (2x - 1)$. Renaming $k_i$ as $k$, we have \[\der{x}{t} = -k(2x - 1)\] as desired.

        We now solve for $t$.
        \begin{alignat*}{2}
            &&\der{x}{t} &= -k(2x - 1)\\
            \implies&& \dfrac{1}{2x-1} \der{x}{t} &= -k\\
            \implies&& \int \dfrac{1}{2x-1} \der{x}{t} \d t &= -k \int \d t\\
            \implies&& \int \dfrac{1}{2x-1} \d x &= -k \int \d t\\
            \implies&& \dfrac12 \ln{2x-1} + C_1 &= -kt \\
            \implies&& \ln{2x-1} + C_2 &= -2kt\\
            \implies&& t &= -\dfrac{1}{2k}(\ln{2x-1} + C_2)
        \end{alignat*}

        At $t = 0$, we have $x = 0.75$. This gives $0 = \ln{2 \cdot 0.75 - 1} + C_2$, whence $C_2 = \ln2$. We also have $\evalder{\derx{x}{t}}{0} = -0.01$. We thus obtain $-0.01 = -k(2 \cdot 0.75 -1)$, whence $k = 0.02$. Thus, \[t = -\dfrac{1}{0.04}(\ln{2x-1} + \ln2) = -25 \ln{4x-2}.\] Hence, when $x = 0.55$, we have $t = -25\ln 0.2 = 25\ln5$.

        \boxt{The depth of the water is $0.55$ m when $t = 25\ln5$ s.}

    \problem{}
        In a model of mortgage repayment, the sum of money owned to the Building Society is denoted by $x$ and the time is denoted by $t$. Both $x$ and $t$ are taken to be continuous variables. The sum of money owned to the Building Society increases, due to interest, at a rate proportional to the sum of money owed. Money is also repaid at a constant rate $r$.

        When $x = a$, interest and repayment balance. Show that, for $x > 0$, $\der{x}{t} = \dfrac{r}{a} (x - a)$.

        Given that, when $t = 0$, $x = A$, find $x$ in terms of $t$, $r$, $a$ and $A$.

        On a single, clearly labelled sketch, show the graph of $x$ against $t$ in the two cases:
        \begin{enumerate}
            \item $A > a$.
            \item $A < a$.
        \end{enumerate}

        State the circumstances under which the loan is repaid in a finite time $T$ and show that, in this case, $T = \dfrac{a}{r} \ln \dfrac{a}{a - A}$.
    
    \solution
        Let the rate at which money is owned to the Building Society be $kx$. Then $\der{x}{t} = kx - r$. At $x = a$, interest and repayment balance, i.e. $\evalder{\derx{x}{t}}{a} = 0$. This gives $ka - r = 0 \implies k = r/a$. Thus, \[\der{x}{t} = \dfrac{r}{a} x - r = \dfrac{r}{a} (x - a).\]

        We now solve for $x$.
        \begin{alignat*}{2}
            &&\der{x}{t} = \dfrac{r}{a} (x - a)\\
            \implies&&\dfrac{1}{x-a} \der{x}{t} &= \dfrac{r}{a}\\
            \implies&&\int \dfrac{1}{x-a} \der{x}{t} \d t &= \dfrac{r}{a} \int \d t\\
            \implies&&\int \dfrac{1}{x-a} \d x &= \dfrac{r}{a} \int \d t\\
            \implies&& \ln{x-a} &= \dfrac{r}{a} t + C_1\\
            \implies&& x-a &= Ce^{rt/a}\\
            \implies&& x &= Ce^{rt/a} + a
        \end{alignat*}

        When $t = 0$, we have $x = A$. This gives $A = C + a$, whence $C = A - a$. Thus,
        \boxt{$x = (A-a)e^{rt/a} + a$}

        \begin{center}
            \begin{tikzpicture}[trim axis left, trim axis right]
                \begin{axis}[
                    domain = 0:4,
                    samples = 101,
                    axis y line=middle,
                    axis x line=middle,
                    xtick = {2 * ln 3},
                    xticklabels = {$T$},
                    ytick = {4},
                    yticklabels = {$A$},
                    xlabel = {$t$},
                    ylabel = {$x$},
                    legend cell align={left},
                    legend pos=outer north east,
                    after end axis/.code={
                        \path (axis cs:0,0) 
                            node [anchor=east] {$O$};
                        }
                    ]
                    \addplot[plotRed] {2*e^(x/2) + 2};
        
                    \addlegendentry{$A>a$};

                    \addplot[plotBlue] {-2*e^(x/2) + 6};
        
                    \addlegendentry{$A<a$};
                \end{axis}
            \end{tikzpicture}
        \end{center}

        For the loan to be repaid in finite time, $A < a$. At time $T$, the loan has been repaid, i.e. $x = 0$. Note that $C_1 = \ln C = \ln{A-a}$. Hence,
        \begin{alignat*}{2}
            &&\dfrac{r}{a} T + \ln{A-a} &= \ln{0-a}\\
            \implies&& \dfrac{r}{a} T &= \ln a - \ln{A-a}\\
            && &= \ln \dfrac{a}{A-a}\\
            \implies&& T &= \dfrac{a}{r} \ln \dfrac{a}{A-a}
        \end{alignat*}

\end{document}