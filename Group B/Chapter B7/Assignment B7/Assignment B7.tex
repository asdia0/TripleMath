\documentclass{echw}

\title{Assignment B7\\Integration Techniques}
\author{Eytan Chong}
\date{2024-05-07}

\begin{document}
    \problem{}
        \begin{enumerate}
            \item Find $\displaystyle\int \dfrac{6x^3 + 2}{x^2 + 1} \d x$.
            \item Evaluate $\displaystyle\int_2^4 x\ln x \d x$ exactly.
        \end{enumerate}

    \solution
        \part
            \begin{align*}
                \int \dfrac{6x^3 + 2}{x^2 + 1} \d x &= 6\int \dfrac{x^3}{x^2 + 1} \d x + 2 \int \dfrac1{x^2 + 1} \d x\\
                &= 6\int \dfrac{x^3}{x^2 + 1} \d x +2\arctan x + C\\
                &= 6\int \dfrac{x^2}{x^2 + 1}\cdot x \d x +2\arctan x + C \usub{u &= x^2 + 1\\\d u &= 2x \d x}\\
                &= 6 \cdot \dfrac12 \int \dfrac{u-1}{u} \d u + 2\arctan x + C\\
                &= 3 \int \left(1 - \dfrac1u\right) \d u + 2\arctan x + C\\
                &= 3\left(u - \ln \abs{u}\right) + 2\arctan x + C\\
                &= 3\left(x^2 + 1\right) - \ln{x^2 + 1} + 2\arctan x + C\\
                &= 3x^2 - \ln\left(x^2 + 1\right) + 2\arctan x + C
            \end{align*}

            \boxt{$\displaystyle\int \dfrac{6x^3 + 2}{x^2 + 1} \d x = 3x^2 - \ln\left(x^2 + 1\right) + 2\arctan x + C$}

        \part
            Note that $\der{}{x} x\ln x = x \cdot \dfrac1x + \ln x \cdot 1 = 1 + \ln x$.

            \begin{equation*}
                \begin{array}{r c @{\hspace*{1.0cm}} c}\toprule
                    & D & I \\\cmidrule{1-3}
                    + & x\ln x & 1 \\
                    - & 1 + \ln x & x \\\bottomrule
                \end{array}
            \end{equation*}

            Let $I = \int x\ln x \d x$.
            {\allowdisplaybreaks
            \begin{alignat*}{2}
                &&I &= \int x\ln x \d x\\
                && &= x^2\ln x - \int \left(x + x\ln x\right) \d x\\
                && &= x^2\ln x - \dfrac12 x^2 - I\\
                \implies&&2I &= x^2 \ln x - \dfrac12 x^2 + C\\
                \implies&&I &= \dfrac12 x^2 \ln x - \dfrac14 x^2 + C
            \end{alignat*}}

            Evaluating $I$ from $x = 2$ to $4$,
            \begin{align*}
                \int_2^4 x\ln x \d x &= \eval{\dfrac12 x^2 \ln x - \dfrac14 x^2}24\\
                &= 14 \ln 2 - 3
            \end{align*}

            \boxt{$\displaystyle\int_2^4 x\ln x \d x = 14 \ln 2 - 3$}

    \problem{}
        \begin{enumerate}
            \item Use the derivative of $\cos \t$ to show that $\der{}{\t} \sec \t = \sec \t \tan \t$.
            \item Use the substitution $x = \sec \t - 1$ to find the exact value of $\displaystyle\int_{\sqrt2 - 1}^1 \dfrac1{(x+1)\sqrt{x^2 + 2x}} \d x$.
        \end{enumerate}

    \solution
        \part
            \begin{align*}
                \der{}{\t} \sec \t &= \der{}{\t} \dfrac1{\cos\t}\\
                &= \dfrac1{\cos^2\t} \cdot \der{}{\t} \cos\t\\
                &= \dfrac1{\cos^2\t}\cdot\sin\t\\
                &= \dfrac1{\cos\t} \cdot \dfrac{\sin\t}{\cos\t}\\
                &= \sec\t\tan\t
            \end{align*}

        \part
            Consider the substitution $x = \sec \t - 1 \implies \d x = \sec\t\tan\t \d \t$. When $x = 1$, we have $\t = \dfrac\pi3$. When $x = \sqrt2 - 1$, we have $\t = \dfrac\pi4$. Also note that $x + 1 = \sec\t$. Consider $x^2 + 2x$.
            \begin{align*}
                x^2 + 2x &= (\sec \t - 1)^2 + 2(\sec \t - 1)\\
                &= \sec^2\t - 2\sec\t + 1 + 2\sec\t - 2\\
                &= \sec^2\t - 1\\
                &= \tan^2\t
            \end{align*}

            Hence, $\sqrt{x^2 + 2x} = \tan\t$. Hence,
            \begin{align*}
                \int_{\sqrt2 - 1}^1 \dfrac1{(x+1)\sqrt{x^2 + 2x}} \d x &= \int_{\tfrac\pi4}^{\tfrac\pi3} \dfrac{\sec\t\tan\t}{\sec\t\tan\t} \d\t\\
                &= \int_{\tfrac\pi4}^{\tfrac\pi3} \d \t\\
                &= \dfrac\pi3 - \dfrac\pi4\\
                &= \dfrac\pi{12}
            \end{align*}

            \boxt{$\displaystyle\int_{\sqrt2 - 1}^1 \dfrac1{(x+1)\sqrt{x^2 + 2x}} \d x = \dfrac\pi{12}$}

    \problem{}
        The expression $\dfrac{x^2}{9-x^2}$ can be written in the form $A + \dfrac{B}{3 - x} + \dfrac{C}{3 + x}$.
        
        \begin{enumerate}
            \item Find the values of constants $A$, $B$ and $C$.
            \item Show that $\displaystyle\int_0^2 \dfrac{x^2}{9-x^2} \d x = \dfrac32 \ln 5 - 2$.
            \item Hence, find the value of $\displaystyle\int_0^2 \ln{9-x^2} \d x$, giving your answer in terms of $\ln 5$.
        \end{enumerate}

    \solution
        \part
            \begin{align*}
                \dfrac{x^2}{9-x^2} &= \dfrac{-\left(9 - x^2\right) + 9}{9 - x^2}\\
                &= -1 + \dfrac9{9 - x^2}\\
                &= -1 + \dfrac9{(3-x)(3+x)}\\
                &= -1 + \dfrac{9/6}{3-x} + \dfrac{9/6}{3 + x}\\
                &= -1 + \dfrac{3/2}{3-x} + \dfrac{3/2}{3 + x}
            \end{align*}

            \boxt{$A = -1$, $B = \dfrac32$, $C = \dfrac32$}

        \part
            \begin{align*}
                \int_0^2 \dfrac{x^2}{9-x^2} \d x &= \int_0^2 \left(-1 + \dfrac{3/2}{3-x} + \dfrac{3/2}{3 + x}\right) \d x\\
                &= \eval{-x -\dfrac32 \ln{3 - x} + \dfrac32 \ln{3 + x}}02\\
                &= \dfrac32\ln5 - 2
            \end{align*}

        \part
            \begin{equation*}
                \begin{array}{r c @{\hspace*{1.0cm}} c}\toprule
                    & D & I \\\cmidrule{1-3}
                    + & \ln\left(9 - x^2\right) & 1 \\
                    - & -\dfrac{2x}{9-x^2} & x \\\bottomrule
                \end{array}
            \end{equation*}

            \begin{align*}
                \int_0^2 \ln\left(9 - x^2\right) \d x &= \eval{x\ln\left(9 - x^2\right)}02 + 2\int_0^2 \dfrac{x^2}{9-x^2} \d x\\
                &= 2\ln5 + 2\left(\dfrac32 \ln5 - 2\right)\\
                &= 5\ln5 - 4
            \end{align*}

            \boxt{$\displaystyle\int_0^2 \ln\left(9 - x^2\right) \d x = 5\ln5 - 4$}

\end{document}