\documentclass{echw}

\title{Assignment B14\\Euler Method and Improved Euler Method}
\author{Eytan Chong}
\date{2024-08-23}

\begin{document}
    \problem{}
        \begin{enumerate}
            \item Explain why the Euler method will fail for the initial-value problem \[\der{y}{x} = y \cos \sqrt x, \qquad y(0) = 0,\] where $y = y(x)$ satisfies that differential equation and is not a constant.
            \item Suppose the initial condition for the problem in part (a) is now $y(0) = 10$. Use the improved Euler method with a step size of $0.1$ to find, to three decimal places, an estimate for $y(0.1)$.
            \item Solve the differential equation \[\der{y}{x} = \dfrac12 y (2-x), \qquad y > 0, \quad y(0) = 10,\] expressing $y$ in terms of $x$, and simplifying your answer as far as possible.
            \item Explain why the solution found in part (c) will give a reasonable estimate for $y(0.1)$ in part (b).
        \end{enumerate}

    \solution
        \part
            By the Euler method, \[y_1 = y_0 + \D x (y_0 \cos \sqrt{x_0}) = 0 + \D x (0 \cos 0) = 0.\] It follows that $y_n = 0$ for all $n \in \N$, whence $y$ is the zero function. However, because $y$ is not a constant function, $y$ cannot be the zero function, a contradiction. Hence, the Euler method fails.

        \part
            Let $\D x = 0.1$, $y_0 = 10$ and $x_n = n \D x$.
            \begin{align*}
                \wt{y_1} &= y_0 + \D x (y_0 \cos \sqrt{x_0}) = 11\\
                y_1 &= y_0 + \dfrac12 \D x \bs{y_0 \cos \sqrt{x_0} + \wt{y_1} \cos \sqrt{x_1}} = 11.023 \todp{3}
            \end{align*}
            \boxt{$y(0.1) \approx 11.023$}

        \part
            {\allowdisplaybreaks
            \begin{alignat*}{2}
                &&\der{y}{x} &= \dfrac12 y (2-x)\\
                \implies&& \dfrac1y \der{y}{x} &= \dfrac12 (2-x)\\
                \implies&& \int \dfrac1y \der{y}{x} \d x&=\int \dfrac12 (2-x) \d x\\
                \implies&& \int \dfrac1y \d y &=\int \dfrac12 (2-x) \d x\\
                \implies&& \ln y &= \dfrac12 \bs{2x - \dfrac12 x^2} + C_1\\
                && &= x - \dfrac14 x^2 + C_1\\
                \implies&& y &= C\exp{x - \dfrac14 x^2}
            \end{alignat*}}
            Since $y(0) = 10$, we have $C = 10$. Thus,
            \boxt{$y = 10\exp{x - \dfrac14 x^2}$}

        \part
            For small $x$, we have that $\cos \sqrt{x} \approx 1 - \dfrac12 (\sqrt x)^2 = \dfrac12 (2 - x)$. Thus, \[y \cos \sqrt x \approx \dfrac12 y (2-x),\] whence the two differential equations and thus their solutions are approximately equal. Since $x = 0.1$ is small, the solution found in part (c) will give a reasonable estimate for $y(0.1)$ in part (b).
    
    \problem{}
        A particle moves along a straight line which passes through a fixed point $O$. It is acted on by two resistive forces, one of which is proportional to its displacement $x$ from $O$ while the other is proportional to its speed $v$. As a result, the motion of the particle is governed by the differential equation \[v \der{v}{x} = -7x - 24v.\] Given that $v = 121$ when $x = 0$, estimate the value of $v$ when $x = 1$ using
        \begin{enumerate}
            \item one iteration of the Euler method,
            \item one iteration of the improved Euler method.
        \end{enumerate}
        Hence, explain why $v$ is approximately a linear function of $x$ for $0 \leq x \leq 1$.

        By considering the values of $\dfrac{x}{v}$ for $0 \leq x \leq 1$, use the given differential equatoin to find an expression for this linear function.

    \solution
        Rewriting the given differential equation, we obtain \[\der{v}{x} = -\dfrac{7x}{v} - 24.\] Let $f(x, v) = -\dfrac{7x}{v} - 24$, $\D x = 1$, $v_0 = 121$, and $x_n = n \D x$.
        \part
            By the Euler method,
            \[v_1 = v_0 + \D x f(x_0, v_0) = 97\]
            \boxt{$y(1) \approx 97$}

        \part
            By the improved Euler method,
            \begin{align*}
                \wt{v_1} &= v_0 + \D x f(x_0, v_0) = 97\\
                v_1 &= v_0 + \dfrac12 \D x \bs{f(x_0, v_0) + f(x_1, \wt{v_1})} = 96.964
            \end{align*}
            \boxt{$y(1) \approx 96.964$}

            \dash

            The gradient of $v$ at $x = 0$ is $f(x_0, v_0) = -24$, which is very close to the gradient of $v$ at $x=1$, $f(x_1, \wt{v_1}) = -24.072$. Since the gradient of $v$ is approximately constant for $0 \leq x \leq 1$, we have that $v$ is approximately a linear function on that interval.

            \dash

            Observe that for $0 \leq x \leq 1$, $x/v \approx 0$ since $x \in [0, 1]$, while $v \geq 96$. Thus, $\derx{v}{x} \approx -24$, whence $v = -24 x + C$. Since $v = 121$ when $x = 0$, we have $\boxed{v \approx -24x + 121}$.

    \problem{}
        The function $y = y(x)$ satisfies \[\der{y}{x} = \dfrac15 (\tan x + x^3 y).\] The value of $y(h)$ is to be found, where $h$ is a small positive number, and $y(0) = 0$.
        \begin{enumerate}
            \item Use one step of the improved Euler method to find an alternative approximation to $y(h)$ in terms of $h$.
            \item It can be shown that $y = y(x)$ satisfies \[y(h) = e^{0.05h^4} \int_0^h \dfrac{\tan x}{5} e^{-0.05x^4} \d x.\] Assume that $h$ is small and hence find another approximation to $y(h)$ in terms of $h$.
            \item Discuss the relative merits of these two methods employed to obtain these approximations.
        \end{enumerate}

    \solution
        \part
            Let $f(x, y) = \dfrac15 (\tan x + x^3 y)$, $\D x = h$, $y_0 = 0$ and $x_n = n \D x$. By the improved Euler method,
            \begin{align*}
                \wt{y_1} &= y_0 + \D x f(x_0, y_0) = 0\\
                y_1 &= y_0 + \dfrac12 \D x \bs{f(x_0, y_0) + f(x_1, \wt{y_1})}\\
                &= 0 + \dfrac12 h \bs{0 + \dfrac15(\tan h + 0)}\\
                &= \dfrac{h \tan h}{10} 
            \end{align*}
            \boxt{$y(h) \approx \dfrac{h\tan h}{10}$}

        \part
            Since $h$ is small, we have that $e^{0.05h^4} \approx 1$. Furthermore, since we are integrating over the interval $x \in [0, h]$, the integrand $\dfrac{\tan x}{5} e^{-0.05x^4}$ can likewise be approximated by $\dfrac{x}{5}$. Our integral hence transforms to
            \begin{align*}
                y(h) &= e^{0.05h^4} \int_0^h \dfrac{\tan x}{5} e^{-0.05x^4} \d x\\
                &\approx \int_0^h \dfrac{x}{5} \d x\\
                &= \boxed{\dfrac{h^2}{10}}
            \end{align*}

        \part
            The improved Euler method involves more steps, while the approximation in (b) is more direct.

\end{document}